%!TEX root = ../main.tex

\chapter{Introduction}
\label{chap:intro}

Raman spectroscopy is a versatile analytical technique for the non-destructive characterization of molecules based on inelastic light scattering. Two experiments will be performed to demonstrate the capabilities of Raman spectroscopy.

Firstly, the ethanol content of an unknown mixture of ethanol and water is to be determined. Different known mixtures will be used to establish a correlation between the measured signals and the ethanol concentration. In a second experiment, the isosbestic point as well as the temperature curve of water being subjected to changes in temperature.

Before the setup, execution and results of the experiment are presented, the underlying physical principles of Raman spectroscopy are explained briefly. The assignment, description of the equipment and procedure and further details about the Lab Course are described in the given handbook \autocite{brauerApplicationRamanSpectroscopy2022}.