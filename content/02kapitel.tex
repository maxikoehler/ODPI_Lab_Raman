%!TEX root = ../main.tex

%%%%%%%%%%%%%%%%%%%%%%%%%%%%%%%
%%%%%%%%%%%%%%%%%%%%%%%%%%%%%%%
\chapter{Theoretical basics}
\label{chap:theoretical}

The following theoretical basics are summarized from the standard literature in optics \autocite{bornPrinciplesOpticsElectromagnetic1999,hechtOptik2005,lipsonOptik1997,niedrigOptikWellenUnd2004} and more specifically Raman application \autocite{herzbergMolecularSpectraMolecular2013,schraderInfraredRamanSpectroscopy1995}. 

%%%%%%%%%%%%%%%%%%%%%%%%%%%%%%%
\section{Molecule - light interactions}

%%%%%%%%%%%%%%%%%%%%%%%%%%%%%%%
\section{Scattering effects}

%%%%%%%%%%%%%%%%%%%%%%%%%%%%%%%
\section{Measurement of different phisical properties - RAMAN spectroscopy}

%%%%%%%%%%%%%%%%%%%%%%%%%%%%%%%
%%%%%%%%%%%%%%%%%%%%%%%%%%%%%%%
\chapter{Experimental setup}
\label{chap:experimental}

%%%%%%%%%%%%%%%%%%%%%%%%%%%%%%%
\section{Used equipment}

%%%%%%%%%%%%%%%%%%%%%%%%%%%%%%%
\section{Measurement setup and preparations}

%%%%%%%%%%%%%%%%%%%%%%%%%%%%%%%
\section{Expectations}

%%%%%%%%%%%%%%%%%%%%%%%%%%%%%%%
\section{Execution}

%%%%%%%%%%%%%%%%%%%%%%%%%%%%%%%
%%%%%%%%%%%%%%%%%%%%%%%%%%%%%%%
\chapter{Results}
\label{chap:results}

%%%%%%%%%%%%%%%%%%%%%%%%%%%%%%%
\section{Data presentation and preparation}

\subsection*{Species determination}
\label{subsec:spec-prep}

\subsection*{Temperature calculation}
\label{subsec:temp-prep}

The raw data of the intesity spectrum over the Raman shift can be obtained in \autoref{fig:plot-temp-raw}. 

\begin{figure}[!htb]
    \centering
    %% Creator: Matplotlib, PGF backend
%%
%% To include the figure in your LaTeX document, write
%%   \input{<filename>.pgf}
%%
%% Make sure the required packages are loaded in your preamble
%%   \usepackage{pgf}
%%
%% Also ensure that all the required font packages are loaded; for instance,
%% the lmodern package is sometimes necessary when using math font.
%%   \usepackage{lmodern}
%%
%% Figures using additional raster images can only be included by \input if
%% they are in the same directory as the main LaTeX file. For loading figures
%% from other directories you can use the `import` package
%%   \usepackage{import}
%%
%% and then include the figures with
%%   \import{<path to file>}{<filename>.pgf}
%%
%% Matplotlib used the following preamble
%%   
%%   \usepackage{fontspec}
%%   \setmainfont{Charter.ttc}[Path=\detokenize{/System/Library/Fonts/Supplemental/}]
%%   \setsansfont{DejaVuSans.ttf}[Path=\detokenize{/opt/homebrew/lib/python3.10/site-packages/matplotlib/mpl-data/fonts/ttf/}]
%%   \setmonofont{DejaVuSansMono.ttf}[Path=\detokenize{/opt/homebrew/lib/python3.10/site-packages/matplotlib/mpl-data/fonts/ttf/}]
%%   \makeatletter\@ifpackageloaded{underscore}{}{\usepackage[strings]{underscore}}\makeatother
%%
\begingroup%
\makeatletter%
\begin{pgfpicture}%
\pgfpathrectangle{\pgfpointorigin}{\pgfqpoint{6.400000in}{4.800000in}}%
\pgfusepath{use as bounding box, clip}%
\begin{pgfscope}%
\pgfsetbuttcap%
\pgfsetmiterjoin%
\definecolor{currentfill}{rgb}{1.000000,1.000000,1.000000}%
\pgfsetfillcolor{currentfill}%
\pgfsetlinewidth{0.000000pt}%
\definecolor{currentstroke}{rgb}{1.000000,1.000000,1.000000}%
\pgfsetstrokecolor{currentstroke}%
\pgfsetdash{}{0pt}%
\pgfpathmoveto{\pgfqpoint{0.000000in}{0.000000in}}%
\pgfpathlineto{\pgfqpoint{6.400000in}{0.000000in}}%
\pgfpathlineto{\pgfqpoint{6.400000in}{4.800000in}}%
\pgfpathlineto{\pgfqpoint{0.000000in}{4.800000in}}%
\pgfpathlineto{\pgfqpoint{0.000000in}{0.000000in}}%
\pgfpathclose%
\pgfusepath{fill}%
\end{pgfscope}%
\begin{pgfscope}%
\pgfsetbuttcap%
\pgfsetmiterjoin%
\definecolor{currentfill}{rgb}{1.000000,1.000000,1.000000}%
\pgfsetfillcolor{currentfill}%
\pgfsetlinewidth{0.000000pt}%
\definecolor{currentstroke}{rgb}{0.000000,0.000000,0.000000}%
\pgfsetstrokecolor{currentstroke}%
\pgfsetstrokeopacity{0.000000}%
\pgfsetdash{}{0pt}%
\pgfpathmoveto{\pgfqpoint{0.800000in}{0.528000in}}%
\pgfpathlineto{\pgfqpoint{5.760000in}{0.528000in}}%
\pgfpathlineto{\pgfqpoint{5.760000in}{4.224000in}}%
\pgfpathlineto{\pgfqpoint{0.800000in}{4.224000in}}%
\pgfpathlineto{\pgfqpoint{0.800000in}{0.528000in}}%
\pgfpathclose%
\pgfusepath{fill}%
\end{pgfscope}%
\begin{pgfscope}%
\pgfpathrectangle{\pgfqpoint{0.800000in}{0.528000in}}{\pgfqpoint{4.960000in}{3.696000in}}%
\pgfusepath{clip}%
\pgfsetrectcap%
\pgfsetroundjoin%
\pgfsetlinewidth{0.803000pt}%
\definecolor{currentstroke}{rgb}{0.690196,0.690196,0.690196}%
\pgfsetstrokecolor{currentstroke}%
\pgfsetdash{}{0pt}%
\pgfpathmoveto{\pgfqpoint{1.013437in}{0.528000in}}%
\pgfpathlineto{\pgfqpoint{1.013437in}{4.224000in}}%
\pgfusepath{stroke}%
\end{pgfscope}%
\begin{pgfscope}%
\pgfsetbuttcap%
\pgfsetroundjoin%
\definecolor{currentfill}{rgb}{0.000000,0.000000,0.000000}%
\pgfsetfillcolor{currentfill}%
\pgfsetlinewidth{0.803000pt}%
\definecolor{currentstroke}{rgb}{0.000000,0.000000,0.000000}%
\pgfsetstrokecolor{currentstroke}%
\pgfsetdash{}{0pt}%
\pgfsys@defobject{currentmarker}{\pgfqpoint{0.000000in}{-0.048611in}}{\pgfqpoint{0.000000in}{0.000000in}}{%
\pgfpathmoveto{\pgfqpoint{0.000000in}{0.000000in}}%
\pgfpathlineto{\pgfqpoint{0.000000in}{-0.048611in}}%
\pgfusepath{stroke,fill}%
}%
\begin{pgfscope}%
\pgfsys@transformshift{1.013437in}{0.528000in}%
\pgfsys@useobject{currentmarker}{}%
\end{pgfscope}%
\end{pgfscope}%
\begin{pgfscope}%
\definecolor{textcolor}{rgb}{0.000000,0.000000,0.000000}%
\pgfsetstrokecolor{textcolor}%
\pgfsetfillcolor{textcolor}%
\pgftext[x=1.013437in,y=0.430778in,,top]{\color{textcolor}\rmfamily\fontsize{12.000000}{14.400000}\selectfont \(\displaystyle {2600}\)}%
\end{pgfscope}%
\begin{pgfscope}%
\pgfpathrectangle{\pgfqpoint{0.800000in}{0.528000in}}{\pgfqpoint{4.960000in}{3.696000in}}%
\pgfusepath{clip}%
\pgfsetrectcap%
\pgfsetroundjoin%
\pgfsetlinewidth{0.803000pt}%
\definecolor{currentstroke}{rgb}{0.690196,0.690196,0.690196}%
\pgfsetstrokecolor{currentstroke}%
\pgfsetdash{}{0pt}%
\pgfpathmoveto{\pgfqpoint{1.768092in}{0.528000in}}%
\pgfpathlineto{\pgfqpoint{1.768092in}{4.224000in}}%
\pgfusepath{stroke}%
\end{pgfscope}%
\begin{pgfscope}%
\pgfsetbuttcap%
\pgfsetroundjoin%
\definecolor{currentfill}{rgb}{0.000000,0.000000,0.000000}%
\pgfsetfillcolor{currentfill}%
\pgfsetlinewidth{0.803000pt}%
\definecolor{currentstroke}{rgb}{0.000000,0.000000,0.000000}%
\pgfsetstrokecolor{currentstroke}%
\pgfsetdash{}{0pt}%
\pgfsys@defobject{currentmarker}{\pgfqpoint{0.000000in}{-0.048611in}}{\pgfqpoint{0.000000in}{0.000000in}}{%
\pgfpathmoveto{\pgfqpoint{0.000000in}{0.000000in}}%
\pgfpathlineto{\pgfqpoint{0.000000in}{-0.048611in}}%
\pgfusepath{stroke,fill}%
}%
\begin{pgfscope}%
\pgfsys@transformshift{1.768092in}{0.528000in}%
\pgfsys@useobject{currentmarker}{}%
\end{pgfscope}%
\end{pgfscope}%
\begin{pgfscope}%
\definecolor{textcolor}{rgb}{0.000000,0.000000,0.000000}%
\pgfsetstrokecolor{textcolor}%
\pgfsetfillcolor{textcolor}%
\pgftext[x=1.768092in,y=0.430778in,,top]{\color{textcolor}\rmfamily\fontsize{12.000000}{14.400000}\selectfont \(\displaystyle {2800}\)}%
\end{pgfscope}%
\begin{pgfscope}%
\pgfpathrectangle{\pgfqpoint{0.800000in}{0.528000in}}{\pgfqpoint{4.960000in}{3.696000in}}%
\pgfusepath{clip}%
\pgfsetrectcap%
\pgfsetroundjoin%
\pgfsetlinewidth{0.803000pt}%
\definecolor{currentstroke}{rgb}{0.690196,0.690196,0.690196}%
\pgfsetstrokecolor{currentstroke}%
\pgfsetdash{}{0pt}%
\pgfpathmoveto{\pgfqpoint{2.522746in}{0.528000in}}%
\pgfpathlineto{\pgfqpoint{2.522746in}{4.224000in}}%
\pgfusepath{stroke}%
\end{pgfscope}%
\begin{pgfscope}%
\pgfsetbuttcap%
\pgfsetroundjoin%
\definecolor{currentfill}{rgb}{0.000000,0.000000,0.000000}%
\pgfsetfillcolor{currentfill}%
\pgfsetlinewidth{0.803000pt}%
\definecolor{currentstroke}{rgb}{0.000000,0.000000,0.000000}%
\pgfsetstrokecolor{currentstroke}%
\pgfsetdash{}{0pt}%
\pgfsys@defobject{currentmarker}{\pgfqpoint{0.000000in}{-0.048611in}}{\pgfqpoint{0.000000in}{0.000000in}}{%
\pgfpathmoveto{\pgfqpoint{0.000000in}{0.000000in}}%
\pgfpathlineto{\pgfqpoint{0.000000in}{-0.048611in}}%
\pgfusepath{stroke,fill}%
}%
\begin{pgfscope}%
\pgfsys@transformshift{2.522746in}{0.528000in}%
\pgfsys@useobject{currentmarker}{}%
\end{pgfscope}%
\end{pgfscope}%
\begin{pgfscope}%
\definecolor{textcolor}{rgb}{0.000000,0.000000,0.000000}%
\pgfsetstrokecolor{textcolor}%
\pgfsetfillcolor{textcolor}%
\pgftext[x=2.522746in,y=0.430778in,,top]{\color{textcolor}\rmfamily\fontsize{12.000000}{14.400000}\selectfont \(\displaystyle {3000}\)}%
\end{pgfscope}%
\begin{pgfscope}%
\pgfpathrectangle{\pgfqpoint{0.800000in}{0.528000in}}{\pgfqpoint{4.960000in}{3.696000in}}%
\pgfusepath{clip}%
\pgfsetrectcap%
\pgfsetroundjoin%
\pgfsetlinewidth{0.803000pt}%
\definecolor{currentstroke}{rgb}{0.690196,0.690196,0.690196}%
\pgfsetstrokecolor{currentstroke}%
\pgfsetdash{}{0pt}%
\pgfpathmoveto{\pgfqpoint{3.277401in}{0.528000in}}%
\pgfpathlineto{\pgfqpoint{3.277401in}{4.224000in}}%
\pgfusepath{stroke}%
\end{pgfscope}%
\begin{pgfscope}%
\pgfsetbuttcap%
\pgfsetroundjoin%
\definecolor{currentfill}{rgb}{0.000000,0.000000,0.000000}%
\pgfsetfillcolor{currentfill}%
\pgfsetlinewidth{0.803000pt}%
\definecolor{currentstroke}{rgb}{0.000000,0.000000,0.000000}%
\pgfsetstrokecolor{currentstroke}%
\pgfsetdash{}{0pt}%
\pgfsys@defobject{currentmarker}{\pgfqpoint{0.000000in}{-0.048611in}}{\pgfqpoint{0.000000in}{0.000000in}}{%
\pgfpathmoveto{\pgfqpoint{0.000000in}{0.000000in}}%
\pgfpathlineto{\pgfqpoint{0.000000in}{-0.048611in}}%
\pgfusepath{stroke,fill}%
}%
\begin{pgfscope}%
\pgfsys@transformshift{3.277401in}{0.528000in}%
\pgfsys@useobject{currentmarker}{}%
\end{pgfscope}%
\end{pgfscope}%
\begin{pgfscope}%
\definecolor{textcolor}{rgb}{0.000000,0.000000,0.000000}%
\pgfsetstrokecolor{textcolor}%
\pgfsetfillcolor{textcolor}%
\pgftext[x=3.277401in,y=0.430778in,,top]{\color{textcolor}\rmfamily\fontsize{12.000000}{14.400000}\selectfont \(\displaystyle {3200}\)}%
\end{pgfscope}%
\begin{pgfscope}%
\pgfpathrectangle{\pgfqpoint{0.800000in}{0.528000in}}{\pgfqpoint{4.960000in}{3.696000in}}%
\pgfusepath{clip}%
\pgfsetrectcap%
\pgfsetroundjoin%
\pgfsetlinewidth{0.803000pt}%
\definecolor{currentstroke}{rgb}{0.690196,0.690196,0.690196}%
\pgfsetstrokecolor{currentstroke}%
\pgfsetdash{}{0pt}%
\pgfpathmoveto{\pgfqpoint{4.032056in}{0.528000in}}%
\pgfpathlineto{\pgfqpoint{4.032056in}{4.224000in}}%
\pgfusepath{stroke}%
\end{pgfscope}%
\begin{pgfscope}%
\pgfsetbuttcap%
\pgfsetroundjoin%
\definecolor{currentfill}{rgb}{0.000000,0.000000,0.000000}%
\pgfsetfillcolor{currentfill}%
\pgfsetlinewidth{0.803000pt}%
\definecolor{currentstroke}{rgb}{0.000000,0.000000,0.000000}%
\pgfsetstrokecolor{currentstroke}%
\pgfsetdash{}{0pt}%
\pgfsys@defobject{currentmarker}{\pgfqpoint{0.000000in}{-0.048611in}}{\pgfqpoint{0.000000in}{0.000000in}}{%
\pgfpathmoveto{\pgfqpoint{0.000000in}{0.000000in}}%
\pgfpathlineto{\pgfqpoint{0.000000in}{-0.048611in}}%
\pgfusepath{stroke,fill}%
}%
\begin{pgfscope}%
\pgfsys@transformshift{4.032056in}{0.528000in}%
\pgfsys@useobject{currentmarker}{}%
\end{pgfscope}%
\end{pgfscope}%
\begin{pgfscope}%
\definecolor{textcolor}{rgb}{0.000000,0.000000,0.000000}%
\pgfsetstrokecolor{textcolor}%
\pgfsetfillcolor{textcolor}%
\pgftext[x=4.032056in,y=0.430778in,,top]{\color{textcolor}\rmfamily\fontsize{12.000000}{14.400000}\selectfont \(\displaystyle {3400}\)}%
\end{pgfscope}%
\begin{pgfscope}%
\pgfpathrectangle{\pgfqpoint{0.800000in}{0.528000in}}{\pgfqpoint{4.960000in}{3.696000in}}%
\pgfusepath{clip}%
\pgfsetrectcap%
\pgfsetroundjoin%
\pgfsetlinewidth{0.803000pt}%
\definecolor{currentstroke}{rgb}{0.690196,0.690196,0.690196}%
\pgfsetstrokecolor{currentstroke}%
\pgfsetdash{}{0pt}%
\pgfpathmoveto{\pgfqpoint{4.786711in}{0.528000in}}%
\pgfpathlineto{\pgfqpoint{4.786711in}{4.224000in}}%
\pgfusepath{stroke}%
\end{pgfscope}%
\begin{pgfscope}%
\pgfsetbuttcap%
\pgfsetroundjoin%
\definecolor{currentfill}{rgb}{0.000000,0.000000,0.000000}%
\pgfsetfillcolor{currentfill}%
\pgfsetlinewidth{0.803000pt}%
\definecolor{currentstroke}{rgb}{0.000000,0.000000,0.000000}%
\pgfsetstrokecolor{currentstroke}%
\pgfsetdash{}{0pt}%
\pgfsys@defobject{currentmarker}{\pgfqpoint{0.000000in}{-0.048611in}}{\pgfqpoint{0.000000in}{0.000000in}}{%
\pgfpathmoveto{\pgfqpoint{0.000000in}{0.000000in}}%
\pgfpathlineto{\pgfqpoint{0.000000in}{-0.048611in}}%
\pgfusepath{stroke,fill}%
}%
\begin{pgfscope}%
\pgfsys@transformshift{4.786711in}{0.528000in}%
\pgfsys@useobject{currentmarker}{}%
\end{pgfscope}%
\end{pgfscope}%
\begin{pgfscope}%
\definecolor{textcolor}{rgb}{0.000000,0.000000,0.000000}%
\pgfsetstrokecolor{textcolor}%
\pgfsetfillcolor{textcolor}%
\pgftext[x=4.786711in,y=0.430778in,,top]{\color{textcolor}\rmfamily\fontsize{12.000000}{14.400000}\selectfont \(\displaystyle {3600}\)}%
\end{pgfscope}%
\begin{pgfscope}%
\pgfpathrectangle{\pgfqpoint{0.800000in}{0.528000in}}{\pgfqpoint{4.960000in}{3.696000in}}%
\pgfusepath{clip}%
\pgfsetrectcap%
\pgfsetroundjoin%
\pgfsetlinewidth{0.803000pt}%
\definecolor{currentstroke}{rgb}{0.690196,0.690196,0.690196}%
\pgfsetstrokecolor{currentstroke}%
\pgfsetdash{}{0pt}%
\pgfpathmoveto{\pgfqpoint{5.541366in}{0.528000in}}%
\pgfpathlineto{\pgfqpoint{5.541366in}{4.224000in}}%
\pgfusepath{stroke}%
\end{pgfscope}%
\begin{pgfscope}%
\pgfsetbuttcap%
\pgfsetroundjoin%
\definecolor{currentfill}{rgb}{0.000000,0.000000,0.000000}%
\pgfsetfillcolor{currentfill}%
\pgfsetlinewidth{0.803000pt}%
\definecolor{currentstroke}{rgb}{0.000000,0.000000,0.000000}%
\pgfsetstrokecolor{currentstroke}%
\pgfsetdash{}{0pt}%
\pgfsys@defobject{currentmarker}{\pgfqpoint{0.000000in}{-0.048611in}}{\pgfqpoint{0.000000in}{0.000000in}}{%
\pgfpathmoveto{\pgfqpoint{0.000000in}{0.000000in}}%
\pgfpathlineto{\pgfqpoint{0.000000in}{-0.048611in}}%
\pgfusepath{stroke,fill}%
}%
\begin{pgfscope}%
\pgfsys@transformshift{5.541366in}{0.528000in}%
\pgfsys@useobject{currentmarker}{}%
\end{pgfscope}%
\end{pgfscope}%
\begin{pgfscope}%
\definecolor{textcolor}{rgb}{0.000000,0.000000,0.000000}%
\pgfsetstrokecolor{textcolor}%
\pgfsetfillcolor{textcolor}%
\pgftext[x=5.541366in,y=0.430778in,,top]{\color{textcolor}\rmfamily\fontsize{12.000000}{14.400000}\selectfont \(\displaystyle {3800}\)}%
\end{pgfscope}%
\begin{pgfscope}%
\definecolor{textcolor}{rgb}{0.000000,0.000000,0.000000}%
\pgfsetstrokecolor{textcolor}%
\pgfsetfillcolor{textcolor}%
\pgftext[x=3.280000in,y=0.216287in,,top]{\color{textcolor}\rmfamily\fontsize{12.000000}{14.400000}\selectfont Raman shift \(\displaystyle \Delta v\) in \(\displaystyle \mathrm{cm}^\mathrm{-1}\)}%
\end{pgfscope}%
\begin{pgfscope}%
\pgfpathrectangle{\pgfqpoint{0.800000in}{0.528000in}}{\pgfqpoint{4.960000in}{3.696000in}}%
\pgfusepath{clip}%
\pgfsetrectcap%
\pgfsetroundjoin%
\pgfsetlinewidth{0.803000pt}%
\definecolor{currentstroke}{rgb}{0.690196,0.690196,0.690196}%
\pgfsetstrokecolor{currentstroke}%
\pgfsetdash{}{0pt}%
\pgfpathmoveto{\pgfqpoint{0.800000in}{0.698956in}}%
\pgfpathlineto{\pgfqpoint{5.760000in}{0.698956in}}%
\pgfusepath{stroke}%
\end{pgfscope}%
\begin{pgfscope}%
\pgfsetbuttcap%
\pgfsetroundjoin%
\definecolor{currentfill}{rgb}{0.000000,0.000000,0.000000}%
\pgfsetfillcolor{currentfill}%
\pgfsetlinewidth{0.803000pt}%
\definecolor{currentstroke}{rgb}{0.000000,0.000000,0.000000}%
\pgfsetstrokecolor{currentstroke}%
\pgfsetdash{}{0pt}%
\pgfsys@defobject{currentmarker}{\pgfqpoint{-0.048611in}{0.000000in}}{\pgfqpoint{-0.000000in}{0.000000in}}{%
\pgfpathmoveto{\pgfqpoint{-0.000000in}{0.000000in}}%
\pgfpathlineto{\pgfqpoint{-0.048611in}{0.000000in}}%
\pgfusepath{stroke,fill}%
}%
\begin{pgfscope}%
\pgfsys@transformshift{0.800000in}{0.698956in}%
\pgfsys@useobject{currentmarker}{}%
\end{pgfscope}%
\end{pgfscope}%
\begin{pgfscope}%
\definecolor{textcolor}{rgb}{0.000000,0.000000,0.000000}%
\pgfsetstrokecolor{textcolor}%
\pgfsetfillcolor{textcolor}%
\pgftext[x=0.412657in, y=0.637636in, left, base]{\color{textcolor}\rmfamily\fontsize{12.000000}{14.400000}\selectfont \(\displaystyle {0.00}\)}%
\end{pgfscope}%
\begin{pgfscope}%
\pgfpathrectangle{\pgfqpoint{0.800000in}{0.528000in}}{\pgfqpoint{4.960000in}{3.696000in}}%
\pgfusepath{clip}%
\pgfsetrectcap%
\pgfsetroundjoin%
\pgfsetlinewidth{0.803000pt}%
\definecolor{currentstroke}{rgb}{0.690196,0.690196,0.690196}%
\pgfsetstrokecolor{currentstroke}%
\pgfsetdash{}{0pt}%
\pgfpathmoveto{\pgfqpoint{0.800000in}{1.286196in}}%
\pgfpathlineto{\pgfqpoint{5.760000in}{1.286196in}}%
\pgfusepath{stroke}%
\end{pgfscope}%
\begin{pgfscope}%
\pgfsetbuttcap%
\pgfsetroundjoin%
\definecolor{currentfill}{rgb}{0.000000,0.000000,0.000000}%
\pgfsetfillcolor{currentfill}%
\pgfsetlinewidth{0.803000pt}%
\definecolor{currentstroke}{rgb}{0.000000,0.000000,0.000000}%
\pgfsetstrokecolor{currentstroke}%
\pgfsetdash{}{0pt}%
\pgfsys@defobject{currentmarker}{\pgfqpoint{-0.048611in}{0.000000in}}{\pgfqpoint{-0.000000in}{0.000000in}}{%
\pgfpathmoveto{\pgfqpoint{-0.000000in}{0.000000in}}%
\pgfpathlineto{\pgfqpoint{-0.048611in}{0.000000in}}%
\pgfusepath{stroke,fill}%
}%
\begin{pgfscope}%
\pgfsys@transformshift{0.800000in}{1.286196in}%
\pgfsys@useobject{currentmarker}{}%
\end{pgfscope}%
\end{pgfscope}%
\begin{pgfscope}%
\definecolor{textcolor}{rgb}{0.000000,0.000000,0.000000}%
\pgfsetstrokecolor{textcolor}%
\pgfsetfillcolor{textcolor}%
\pgftext[x=0.412657in, y=1.224876in, left, base]{\color{textcolor}\rmfamily\fontsize{12.000000}{14.400000}\selectfont \(\displaystyle {0.05}\)}%
\end{pgfscope}%
\begin{pgfscope}%
\pgfpathrectangle{\pgfqpoint{0.800000in}{0.528000in}}{\pgfqpoint{4.960000in}{3.696000in}}%
\pgfusepath{clip}%
\pgfsetrectcap%
\pgfsetroundjoin%
\pgfsetlinewidth{0.803000pt}%
\definecolor{currentstroke}{rgb}{0.690196,0.690196,0.690196}%
\pgfsetstrokecolor{currentstroke}%
\pgfsetdash{}{0pt}%
\pgfpathmoveto{\pgfqpoint{0.800000in}{1.873436in}}%
\pgfpathlineto{\pgfqpoint{5.760000in}{1.873436in}}%
\pgfusepath{stroke}%
\end{pgfscope}%
\begin{pgfscope}%
\pgfsetbuttcap%
\pgfsetroundjoin%
\definecolor{currentfill}{rgb}{0.000000,0.000000,0.000000}%
\pgfsetfillcolor{currentfill}%
\pgfsetlinewidth{0.803000pt}%
\definecolor{currentstroke}{rgb}{0.000000,0.000000,0.000000}%
\pgfsetstrokecolor{currentstroke}%
\pgfsetdash{}{0pt}%
\pgfsys@defobject{currentmarker}{\pgfqpoint{-0.048611in}{0.000000in}}{\pgfqpoint{-0.000000in}{0.000000in}}{%
\pgfpathmoveto{\pgfqpoint{-0.000000in}{0.000000in}}%
\pgfpathlineto{\pgfqpoint{-0.048611in}{0.000000in}}%
\pgfusepath{stroke,fill}%
}%
\begin{pgfscope}%
\pgfsys@transformshift{0.800000in}{1.873436in}%
\pgfsys@useobject{currentmarker}{}%
\end{pgfscope}%
\end{pgfscope}%
\begin{pgfscope}%
\definecolor{textcolor}{rgb}{0.000000,0.000000,0.000000}%
\pgfsetstrokecolor{textcolor}%
\pgfsetfillcolor{textcolor}%
\pgftext[x=0.412657in, y=1.812116in, left, base]{\color{textcolor}\rmfamily\fontsize{12.000000}{14.400000}\selectfont \(\displaystyle {0.10}\)}%
\end{pgfscope}%
\begin{pgfscope}%
\pgfpathrectangle{\pgfqpoint{0.800000in}{0.528000in}}{\pgfqpoint{4.960000in}{3.696000in}}%
\pgfusepath{clip}%
\pgfsetrectcap%
\pgfsetroundjoin%
\pgfsetlinewidth{0.803000pt}%
\definecolor{currentstroke}{rgb}{0.690196,0.690196,0.690196}%
\pgfsetstrokecolor{currentstroke}%
\pgfsetdash{}{0pt}%
\pgfpathmoveto{\pgfqpoint{0.800000in}{2.460676in}}%
\pgfpathlineto{\pgfqpoint{5.760000in}{2.460676in}}%
\pgfusepath{stroke}%
\end{pgfscope}%
\begin{pgfscope}%
\pgfsetbuttcap%
\pgfsetroundjoin%
\definecolor{currentfill}{rgb}{0.000000,0.000000,0.000000}%
\pgfsetfillcolor{currentfill}%
\pgfsetlinewidth{0.803000pt}%
\definecolor{currentstroke}{rgb}{0.000000,0.000000,0.000000}%
\pgfsetstrokecolor{currentstroke}%
\pgfsetdash{}{0pt}%
\pgfsys@defobject{currentmarker}{\pgfqpoint{-0.048611in}{0.000000in}}{\pgfqpoint{-0.000000in}{0.000000in}}{%
\pgfpathmoveto{\pgfqpoint{-0.000000in}{0.000000in}}%
\pgfpathlineto{\pgfqpoint{-0.048611in}{0.000000in}}%
\pgfusepath{stroke,fill}%
}%
\begin{pgfscope}%
\pgfsys@transformshift{0.800000in}{2.460676in}%
\pgfsys@useobject{currentmarker}{}%
\end{pgfscope}%
\end{pgfscope}%
\begin{pgfscope}%
\definecolor{textcolor}{rgb}{0.000000,0.000000,0.000000}%
\pgfsetstrokecolor{textcolor}%
\pgfsetfillcolor{textcolor}%
\pgftext[x=0.412657in, y=2.399356in, left, base]{\color{textcolor}\rmfamily\fontsize{12.000000}{14.400000}\selectfont \(\displaystyle {0.15}\)}%
\end{pgfscope}%
\begin{pgfscope}%
\pgfpathrectangle{\pgfqpoint{0.800000in}{0.528000in}}{\pgfqpoint{4.960000in}{3.696000in}}%
\pgfusepath{clip}%
\pgfsetrectcap%
\pgfsetroundjoin%
\pgfsetlinewidth{0.803000pt}%
\definecolor{currentstroke}{rgb}{0.690196,0.690196,0.690196}%
\pgfsetstrokecolor{currentstroke}%
\pgfsetdash{}{0pt}%
\pgfpathmoveto{\pgfqpoint{0.800000in}{3.047916in}}%
\pgfpathlineto{\pgfqpoint{5.760000in}{3.047916in}}%
\pgfusepath{stroke}%
\end{pgfscope}%
\begin{pgfscope}%
\pgfsetbuttcap%
\pgfsetroundjoin%
\definecolor{currentfill}{rgb}{0.000000,0.000000,0.000000}%
\pgfsetfillcolor{currentfill}%
\pgfsetlinewidth{0.803000pt}%
\definecolor{currentstroke}{rgb}{0.000000,0.000000,0.000000}%
\pgfsetstrokecolor{currentstroke}%
\pgfsetdash{}{0pt}%
\pgfsys@defobject{currentmarker}{\pgfqpoint{-0.048611in}{0.000000in}}{\pgfqpoint{-0.000000in}{0.000000in}}{%
\pgfpathmoveto{\pgfqpoint{-0.000000in}{0.000000in}}%
\pgfpathlineto{\pgfqpoint{-0.048611in}{0.000000in}}%
\pgfusepath{stroke,fill}%
}%
\begin{pgfscope}%
\pgfsys@transformshift{0.800000in}{3.047916in}%
\pgfsys@useobject{currentmarker}{}%
\end{pgfscope}%
\end{pgfscope}%
\begin{pgfscope}%
\definecolor{textcolor}{rgb}{0.000000,0.000000,0.000000}%
\pgfsetstrokecolor{textcolor}%
\pgfsetfillcolor{textcolor}%
\pgftext[x=0.412657in, y=2.986596in, left, base]{\color{textcolor}\rmfamily\fontsize{12.000000}{14.400000}\selectfont \(\displaystyle {0.20}\)}%
\end{pgfscope}%
\begin{pgfscope}%
\pgfpathrectangle{\pgfqpoint{0.800000in}{0.528000in}}{\pgfqpoint{4.960000in}{3.696000in}}%
\pgfusepath{clip}%
\pgfsetrectcap%
\pgfsetroundjoin%
\pgfsetlinewidth{0.803000pt}%
\definecolor{currentstroke}{rgb}{0.690196,0.690196,0.690196}%
\pgfsetstrokecolor{currentstroke}%
\pgfsetdash{}{0pt}%
\pgfpathmoveto{\pgfqpoint{0.800000in}{3.635156in}}%
\pgfpathlineto{\pgfqpoint{5.760000in}{3.635156in}}%
\pgfusepath{stroke}%
\end{pgfscope}%
\begin{pgfscope}%
\pgfsetbuttcap%
\pgfsetroundjoin%
\definecolor{currentfill}{rgb}{0.000000,0.000000,0.000000}%
\pgfsetfillcolor{currentfill}%
\pgfsetlinewidth{0.803000pt}%
\definecolor{currentstroke}{rgb}{0.000000,0.000000,0.000000}%
\pgfsetstrokecolor{currentstroke}%
\pgfsetdash{}{0pt}%
\pgfsys@defobject{currentmarker}{\pgfqpoint{-0.048611in}{0.000000in}}{\pgfqpoint{-0.000000in}{0.000000in}}{%
\pgfpathmoveto{\pgfqpoint{-0.000000in}{0.000000in}}%
\pgfpathlineto{\pgfqpoint{-0.048611in}{0.000000in}}%
\pgfusepath{stroke,fill}%
}%
\begin{pgfscope}%
\pgfsys@transformshift{0.800000in}{3.635156in}%
\pgfsys@useobject{currentmarker}{}%
\end{pgfscope}%
\end{pgfscope}%
\begin{pgfscope}%
\definecolor{textcolor}{rgb}{0.000000,0.000000,0.000000}%
\pgfsetstrokecolor{textcolor}%
\pgfsetfillcolor{textcolor}%
\pgftext[x=0.412657in, y=3.573836in, left, base]{\color{textcolor}\rmfamily\fontsize{12.000000}{14.400000}\selectfont \(\displaystyle {0.25}\)}%
\end{pgfscope}%
\begin{pgfscope}%
\pgfpathrectangle{\pgfqpoint{0.800000in}{0.528000in}}{\pgfqpoint{4.960000in}{3.696000in}}%
\pgfusepath{clip}%
\pgfsetrectcap%
\pgfsetroundjoin%
\pgfsetlinewidth{0.803000pt}%
\definecolor{currentstroke}{rgb}{0.690196,0.690196,0.690196}%
\pgfsetstrokecolor{currentstroke}%
\pgfsetdash{}{0pt}%
\pgfpathmoveto{\pgfqpoint{0.800000in}{4.222396in}}%
\pgfpathlineto{\pgfqpoint{5.760000in}{4.222396in}}%
\pgfusepath{stroke}%
\end{pgfscope}%
\begin{pgfscope}%
\pgfsetbuttcap%
\pgfsetroundjoin%
\definecolor{currentfill}{rgb}{0.000000,0.000000,0.000000}%
\pgfsetfillcolor{currentfill}%
\pgfsetlinewidth{0.803000pt}%
\definecolor{currentstroke}{rgb}{0.000000,0.000000,0.000000}%
\pgfsetstrokecolor{currentstroke}%
\pgfsetdash{}{0pt}%
\pgfsys@defobject{currentmarker}{\pgfqpoint{-0.048611in}{0.000000in}}{\pgfqpoint{-0.000000in}{0.000000in}}{%
\pgfpathmoveto{\pgfqpoint{-0.000000in}{0.000000in}}%
\pgfpathlineto{\pgfqpoint{-0.048611in}{0.000000in}}%
\pgfusepath{stroke,fill}%
}%
\begin{pgfscope}%
\pgfsys@transformshift{0.800000in}{4.222396in}%
\pgfsys@useobject{currentmarker}{}%
\end{pgfscope}%
\end{pgfscope}%
\begin{pgfscope}%
\definecolor{textcolor}{rgb}{0.000000,0.000000,0.000000}%
\pgfsetstrokecolor{textcolor}%
\pgfsetfillcolor{textcolor}%
\pgftext[x=0.412657in, y=4.161076in, left, base]{\color{textcolor}\rmfamily\fontsize{12.000000}{14.400000}\selectfont \(\displaystyle {0.30}\)}%
\end{pgfscope}%
\begin{pgfscope}%
\definecolor{textcolor}{rgb}{0.000000,0.000000,0.000000}%
\pgfsetstrokecolor{textcolor}%
\pgfsetfillcolor{textcolor}%
\pgftext[x=0.357102in,y=2.376000in,,bottom,rotate=90.000000]{\color{textcolor}\rmfamily\fontsize{12.000000}{14.400000}\selectfont Signal intensity in \(\displaystyle 10^6\)}%
\end{pgfscope}%
\begin{pgfscope}%
\pgfpathrectangle{\pgfqpoint{0.800000in}{0.528000in}}{\pgfqpoint{4.960000in}{3.696000in}}%
\pgfusepath{clip}%
\pgfsetrectcap%
\pgfsetroundjoin%
\pgfsetlinewidth{1.505625pt}%
\definecolor{currentstroke}{rgb}{0.121569,0.466667,0.705882}%
\pgfsetstrokecolor{currentstroke}%
\pgfsetdash{}{0pt}%
\pgfpathmoveto{\pgfqpoint{1.025455in}{0.706952in}}%
\pgfpathlineto{\pgfqpoint{1.042271in}{0.705071in}}%
\pgfpathlineto{\pgfqpoint{1.059079in}{0.706257in}}%
\pgfpathlineto{\pgfqpoint{1.076865in}{0.704730in}}%
\pgfpathlineto{\pgfqpoint{1.111420in}{0.703853in}}%
\pgfpathlineto{\pgfqpoint{1.299303in}{0.700818in}}%
\pgfpathlineto{\pgfqpoint{1.315970in}{0.698935in}}%
\pgfpathlineto{\pgfqpoint{1.333607in}{0.699942in}}%
\pgfpathlineto{\pgfqpoint{1.366893in}{0.697764in}}%
\pgfpathlineto{\pgfqpoint{1.384500in}{0.698689in}}%
\pgfpathlineto{\pgfqpoint{1.417731in}{0.697862in}}%
\pgfpathlineto{\pgfqpoint{1.435309in}{0.698999in}}%
\pgfpathlineto{\pgfqpoint{1.451901in}{0.697293in}}%
\pgfpathlineto{\pgfqpoint{1.486033in}{0.698274in}}%
\pgfpathlineto{\pgfqpoint{1.502597in}{0.697743in}}%
\pgfpathlineto{\pgfqpoint{1.519152in}{0.698821in}}%
\pgfpathlineto{\pgfqpoint{1.587226in}{0.697567in}}%
\pgfpathlineto{\pgfqpoint{1.603735in}{0.699338in}}%
\pgfpathlineto{\pgfqpoint{1.637696in}{0.699864in}}%
\pgfpathlineto{\pgfqpoint{1.670651in}{0.702600in}}%
\pgfpathlineto{\pgfqpoint{1.704537in}{0.702636in}}%
\pgfpathlineto{\pgfqpoint{1.720982in}{0.703754in}}%
\pgfpathlineto{\pgfqpoint{1.737418in}{0.703157in}}%
\pgfpathlineto{\pgfqpoint{1.754811in}{0.705509in}}%
\pgfpathlineto{\pgfqpoint{1.771229in}{0.705467in}}%
\pgfpathlineto{\pgfqpoint{1.805003in}{0.709800in}}%
\pgfpathlineto{\pgfqpoint{1.821393in}{0.709866in}}%
\pgfpathlineto{\pgfqpoint{1.854148in}{0.714100in}}%
\pgfpathlineto{\pgfqpoint{1.887828in}{0.718962in}}%
\pgfpathlineto{\pgfqpoint{1.904174in}{0.721958in}}%
\pgfpathlineto{\pgfqpoint{1.937799in}{0.725619in}}%
\pgfpathlineto{\pgfqpoint{1.970427in}{0.732773in}}%
\pgfpathlineto{\pgfqpoint{2.020261in}{0.741991in}}%
\pgfpathlineto{\pgfqpoint{2.052800in}{0.749057in}}%
\pgfpathlineto{\pgfqpoint{2.152112in}{0.774814in}}%
\pgfpathlineto{\pgfqpoint{2.184508in}{0.787245in}}%
\pgfpathlineto{\pgfqpoint{2.200693in}{0.790699in}}%
\pgfpathlineto{\pgfqpoint{2.250146in}{0.809494in}}%
\pgfpathlineto{\pgfqpoint{2.266296in}{0.818384in}}%
\pgfpathlineto{\pgfqpoint{2.315641in}{0.841003in}}%
\pgfpathlineto{\pgfqpoint{2.380046in}{0.881351in}}%
\pgfpathlineto{\pgfqpoint{2.429203in}{0.915973in}}%
\pgfpathlineto{\pgfqpoint{2.493364in}{0.972596in}}%
\pgfpathlineto{\pgfqpoint{2.542336in}{1.020468in}}%
\pgfpathlineto{\pgfqpoint{2.558328in}{1.038874in}}%
\pgfpathlineto{\pgfqpoint{2.574312in}{1.060349in}}%
\pgfpathlineto{\pgfqpoint{2.606254in}{1.099537in}}%
\pgfpathlineto{\pgfqpoint{2.655040in}{1.168178in}}%
\pgfpathlineto{\pgfqpoint{2.686896in}{1.220233in}}%
\pgfpathlineto{\pgfqpoint{2.702810in}{1.250917in}}%
\pgfpathlineto{\pgfqpoint{2.718717in}{1.276905in}}%
\pgfpathlineto{\pgfqpoint{2.734615in}{1.307219in}}%
\pgfpathlineto{\pgfqpoint{2.750504in}{1.343044in}}%
\pgfpathlineto{\pgfqpoint{2.782257in}{1.408631in}}%
\pgfpathlineto{\pgfqpoint{2.813977in}{1.480385in}}%
\pgfpathlineto{\pgfqpoint{2.846594in}{1.558963in}}%
\pgfpathlineto{\pgfqpoint{2.878244in}{1.642956in}}%
\pgfpathlineto{\pgfqpoint{2.909861in}{1.734349in}}%
\pgfpathlineto{\pgfqpoint{2.925657in}{1.781318in}}%
\pgfpathlineto{\pgfqpoint{2.988756in}{1.994466in}}%
\pgfpathlineto{\pgfqpoint{3.051720in}{2.225540in}}%
\pgfpathlineto{\pgfqpoint{3.114551in}{2.464943in}}%
\pgfpathlineto{\pgfqpoint{3.130238in}{2.526845in}}%
\pgfpathlineto{\pgfqpoint{3.161586in}{2.641263in}}%
\pgfpathlineto{\pgfqpoint{3.193822in}{2.749809in}}%
\pgfpathlineto{\pgfqpoint{3.209467in}{2.806829in}}%
\pgfpathlineto{\pgfqpoint{3.256351in}{2.951107in}}%
\pgfpathlineto{\pgfqpoint{3.302243in}{3.068305in}}%
\pgfpathlineto{\pgfqpoint{3.317831in}{3.098029in}}%
\pgfpathlineto{\pgfqpoint{3.333410in}{3.132712in}}%
\pgfpathlineto{\pgfqpoint{3.348980in}{3.160938in}}%
\pgfpathlineto{\pgfqpoint{3.395643in}{3.231390in}}%
\pgfpathlineto{\pgfqpoint{3.442231in}{3.278686in}}%
\pgfpathlineto{\pgfqpoint{3.457744in}{3.292629in}}%
\pgfpathlineto{\pgfqpoint{3.473249in}{3.301090in}}%
\pgfpathlineto{\pgfqpoint{3.488746in}{3.312996in}}%
\pgfpathlineto{\pgfqpoint{3.504234in}{3.313090in}}%
\pgfpathlineto{\pgfqpoint{3.519715in}{3.317990in}}%
\pgfpathlineto{\pgfqpoint{3.550651in}{3.323314in}}%
\pgfpathlineto{\pgfqpoint{3.566107in}{3.325852in}}%
\pgfpathlineto{\pgfqpoint{3.581554in}{3.322438in}}%
\pgfpathlineto{\pgfqpoint{3.596994in}{3.326497in}}%
\pgfpathlineto{\pgfqpoint{3.612425in}{3.320693in}}%
\pgfpathlineto{\pgfqpoint{3.626942in}{3.322421in}}%
\pgfpathlineto{\pgfqpoint{3.642357in}{3.319619in}}%
\pgfpathlineto{\pgfqpoint{3.657765in}{3.322610in}}%
\pgfpathlineto{\pgfqpoint{3.673164in}{3.322256in}}%
\pgfpathlineto{\pgfqpoint{3.688555in}{3.323985in}}%
\pgfpathlineto{\pgfqpoint{3.719313in}{3.338117in}}%
\pgfpathlineto{\pgfqpoint{3.734680in}{3.347865in}}%
\pgfpathlineto{\pgfqpoint{3.750039in}{3.359343in}}%
\pgfpathlineto{\pgfqpoint{3.765390in}{3.374817in}}%
\pgfpathlineto{\pgfqpoint{3.779830in}{3.393219in}}%
\pgfpathlineto{\pgfqpoint{3.795166in}{3.409874in}}%
\pgfpathlineto{\pgfqpoint{3.825812in}{3.455760in}}%
\pgfpathlineto{\pgfqpoint{3.841123in}{3.480294in}}%
\pgfpathlineto{\pgfqpoint{3.856426in}{3.502059in}}%
\pgfpathlineto{\pgfqpoint{3.871721in}{3.529382in}}%
\pgfpathlineto{\pgfqpoint{3.901388in}{3.588032in}}%
\pgfpathlineto{\pgfqpoint{3.931922in}{3.653859in}}%
\pgfpathlineto{\pgfqpoint{3.962425in}{3.720544in}}%
\pgfpathlineto{\pgfqpoint{3.977664in}{3.748923in}}%
\pgfpathlineto{\pgfqpoint{4.007224in}{3.817835in}}%
\pgfpathlineto{\pgfqpoint{4.022440in}{3.849841in}}%
\pgfpathlineto{\pgfqpoint{4.037648in}{3.879020in}}%
\pgfpathlineto{\pgfqpoint{4.052848in}{3.912865in}}%
\pgfpathlineto{\pgfqpoint{4.083224in}{3.964720in}}%
\pgfpathlineto{\pgfqpoint{4.097507in}{3.980358in}}%
\pgfpathlineto{\pgfqpoint{4.112676in}{3.999300in}}%
\pgfpathlineto{\pgfqpoint{4.127836in}{4.015038in}}%
\pgfpathlineto{\pgfqpoint{4.142989in}{4.024261in}}%
\pgfpathlineto{\pgfqpoint{4.158134in}{4.035520in}}%
\pgfpathlineto{\pgfqpoint{4.172381in}{4.041320in}}%
\pgfpathlineto{\pgfqpoint{4.187510in}{4.044351in}}%
\pgfpathlineto{\pgfqpoint{4.202632in}{4.045812in}}%
\pgfpathlineto{\pgfqpoint{4.232852in}{4.039947in}}%
\pgfpathlineto{\pgfqpoint{4.247062in}{4.031940in}}%
\pgfpathlineto{\pgfqpoint{4.292310in}{3.997061in}}%
\pgfpathlineto{\pgfqpoint{4.307377in}{3.975137in}}%
\pgfpathlineto{\pgfqpoint{4.321550in}{3.948460in}}%
\pgfpathlineto{\pgfqpoint{4.336602in}{3.924228in}}%
\pgfpathlineto{\pgfqpoint{4.351646in}{3.895972in}}%
\pgfpathlineto{\pgfqpoint{4.366682in}{3.862527in}}%
\pgfpathlineto{\pgfqpoint{4.380827in}{3.825375in}}%
\pgfpathlineto{\pgfqpoint{4.395847in}{3.793299in}}%
\pgfpathlineto{\pgfqpoint{4.410861in}{3.756928in}}%
\pgfpathlineto{\pgfqpoint{4.425866in}{3.716508in}}%
\pgfpathlineto{\pgfqpoint{4.439981in}{3.673367in}}%
\pgfpathlineto{\pgfqpoint{4.454971in}{3.631920in}}%
\pgfpathlineto{\pgfqpoint{4.484928in}{3.537514in}}%
\pgfpathlineto{\pgfqpoint{4.513975in}{3.441653in}}%
\pgfpathlineto{\pgfqpoint{4.572857in}{3.228407in}}%
\pgfpathlineto{\pgfqpoint{4.704248in}{2.780126in}}%
\pgfpathlineto{\pgfqpoint{4.748784in}{2.641289in}}%
\pgfpathlineto{\pgfqpoint{4.777565in}{2.548199in}}%
\pgfpathlineto{\pgfqpoint{4.792381in}{2.504877in}}%
\pgfpathlineto{\pgfqpoint{4.806317in}{2.458293in}}%
\pgfpathlineto{\pgfqpoint{4.821118in}{2.415506in}}%
\pgfpathlineto{\pgfqpoint{4.849826in}{2.326113in}}%
\pgfpathlineto{\pgfqpoint{4.879374in}{2.241408in}}%
\pgfpathlineto{\pgfqpoint{4.893269in}{2.198120in}}%
\pgfpathlineto{\pgfqpoint{4.908024in}{2.157338in}}%
\pgfpathlineto{\pgfqpoint{4.922772in}{2.111602in}}%
\pgfpathlineto{\pgfqpoint{4.966105in}{1.958019in}}%
\pgfpathlineto{\pgfqpoint{5.023203in}{1.719572in}}%
\pgfpathlineto{\pgfqpoint{5.037891in}{1.654531in}}%
\pgfpathlineto{\pgfqpoint{5.051709in}{1.585123in}}%
\pgfpathlineto{\pgfqpoint{5.094846in}{1.389007in}}%
\pgfpathlineto{\pgfqpoint{5.123281in}{1.268415in}}%
\pgfpathlineto{\pgfqpoint{5.137918in}{1.214202in}}%
\pgfpathlineto{\pgfqpoint{5.166310in}{1.118035in}}%
\pgfpathlineto{\pgfqpoint{5.194674in}{1.042176in}}%
\pgfpathlineto{\pgfqpoint{5.209274in}{1.009273in}}%
\pgfpathlineto{\pgfqpoint{5.237595in}{0.952801in}}%
\pgfpathlineto{\pgfqpoint{5.265888in}{0.908087in}}%
\pgfpathlineto{\pgfqpoint{5.308702in}{0.854995in}}%
\pgfpathlineto{\pgfqpoint{5.336924in}{0.827191in}}%
\pgfpathlineto{\pgfqpoint{5.365118in}{0.805755in}}%
\pgfpathlineto{\pgfqpoint{5.393284in}{0.786657in}}%
\pgfpathlineto{\pgfqpoint{5.464003in}{0.754274in}}%
\pgfpathlineto{\pgfqpoint{5.478466in}{0.750091in}}%
\pgfpathlineto{\pgfqpoint{5.492071in}{0.744583in}}%
\pgfpathlineto{\pgfqpoint{5.534545in}{0.732861in}}%
\pgfpathlineto{\pgfqpoint{5.534545in}{0.732861in}}%
\pgfusepath{stroke}%
\end{pgfscope}%
\begin{pgfscope}%
\pgfpathrectangle{\pgfqpoint{0.800000in}{0.528000in}}{\pgfqpoint{4.960000in}{3.696000in}}%
\pgfusepath{clip}%
\pgfsetrectcap%
\pgfsetroundjoin%
\pgfsetlinewidth{1.505625pt}%
\definecolor{currentstroke}{rgb}{1.000000,0.498039,0.054902}%
\pgfsetstrokecolor{currentstroke}%
\pgfsetdash{}{0pt}%
\pgfpathmoveto{\pgfqpoint{1.025455in}{0.707756in}}%
\pgfpathlineto{\pgfqpoint{1.042271in}{0.706640in}}%
\pgfpathlineto{\pgfqpoint{1.076865in}{0.707058in}}%
\pgfpathlineto{\pgfqpoint{1.093654in}{0.705785in}}%
\pgfpathlineto{\pgfqpoint{1.111420in}{0.705697in}}%
\pgfpathlineto{\pgfqpoint{1.128190in}{0.704234in}}%
\pgfpathlineto{\pgfqpoint{1.144950in}{0.705529in}}%
\pgfpathlineto{\pgfqpoint{1.179428in}{0.703663in}}%
\pgfpathlineto{\pgfqpoint{1.213867in}{0.703635in}}%
\pgfpathlineto{\pgfqpoint{1.230580in}{0.702772in}}%
\pgfpathlineto{\pgfqpoint{1.248266in}{0.703130in}}%
\pgfpathlineto{\pgfqpoint{1.264961in}{0.701782in}}%
\pgfpathlineto{\pgfqpoint{1.299303in}{0.701923in}}%
\pgfpathlineto{\pgfqpoint{1.401120in}{0.698021in}}%
\pgfpathlineto{\pgfqpoint{1.468484in}{0.698329in}}%
\pgfpathlineto{\pgfqpoint{1.519152in}{0.696798in}}%
\pgfpathlineto{\pgfqpoint{1.536672in}{0.697751in}}%
\pgfpathlineto{\pgfqpoint{1.553208in}{0.696000in}}%
\pgfpathlineto{\pgfqpoint{1.569736in}{0.697312in}}%
\pgfpathlineto{\pgfqpoint{1.603735in}{0.697368in}}%
\pgfpathlineto{\pgfqpoint{1.688082in}{0.698484in}}%
\pgfpathlineto{\pgfqpoint{1.737418in}{0.698306in}}%
\pgfpathlineto{\pgfqpoint{1.771229in}{0.701056in}}%
\pgfpathlineto{\pgfqpoint{1.787638in}{0.700504in}}%
\pgfpathlineto{\pgfqpoint{1.904174in}{0.710378in}}%
\pgfpathlineto{\pgfqpoint{1.920511in}{0.710530in}}%
\pgfpathlineto{\pgfqpoint{1.954117in}{0.714595in}}%
\pgfpathlineto{\pgfqpoint{2.020261in}{0.723285in}}%
\pgfpathlineto{\pgfqpoint{2.052800in}{0.727241in}}%
\pgfpathlineto{\pgfqpoint{2.070012in}{0.731123in}}%
\pgfpathlineto{\pgfqpoint{2.134947in}{0.740814in}}%
\pgfpathlineto{\pgfqpoint{2.152112in}{0.745665in}}%
\pgfpathlineto{\pgfqpoint{2.168314in}{0.747852in}}%
\pgfpathlineto{\pgfqpoint{2.266296in}{0.770603in}}%
\pgfpathlineto{\pgfqpoint{2.282437in}{0.775856in}}%
\pgfpathlineto{\pgfqpoint{2.315641in}{0.784958in}}%
\pgfpathlineto{\pgfqpoint{2.396126in}{0.815869in}}%
\pgfpathlineto{\pgfqpoint{2.413142in}{0.824534in}}%
\pgfpathlineto{\pgfqpoint{2.429203in}{0.830981in}}%
\pgfpathlineto{\pgfqpoint{2.493364in}{0.863916in}}%
\pgfpathlineto{\pgfqpoint{2.526335in}{0.882996in}}%
\pgfpathlineto{\pgfqpoint{2.574312in}{0.914996in}}%
\pgfpathlineto{\pgfqpoint{2.622212in}{0.949431in}}%
\pgfpathlineto{\pgfqpoint{2.670972in}{0.990823in}}%
\pgfpathlineto{\pgfqpoint{2.734615in}{1.054041in}}%
\pgfpathlineto{\pgfqpoint{2.750504in}{1.071988in}}%
\pgfpathlineto{\pgfqpoint{2.782257in}{1.111005in}}%
\pgfpathlineto{\pgfqpoint{2.830756in}{1.171300in}}%
\pgfpathlineto{\pgfqpoint{2.878244in}{1.240798in}}%
\pgfpathlineto{\pgfqpoint{2.909861in}{1.292542in}}%
\pgfpathlineto{\pgfqpoint{2.941444in}{1.346908in}}%
\pgfpathlineto{\pgfqpoint{2.972994in}{1.409967in}}%
\pgfpathlineto{\pgfqpoint{2.988756in}{1.440650in}}%
\pgfpathlineto{\pgfqpoint{3.035992in}{1.546134in}}%
\pgfpathlineto{\pgfqpoint{3.051720in}{1.582081in}}%
\pgfpathlineto{\pgfqpoint{3.067441in}{1.622906in}}%
\pgfpathlineto{\pgfqpoint{3.083152in}{1.660458in}}%
\pgfpathlineto{\pgfqpoint{3.145916in}{1.827636in}}%
\pgfpathlineto{\pgfqpoint{3.161586in}{1.872241in}}%
\pgfpathlineto{\pgfqpoint{3.178169in}{1.915546in}}%
\pgfpathlineto{\pgfqpoint{3.240731in}{2.097210in}}%
\pgfpathlineto{\pgfqpoint{3.256351in}{2.139488in}}%
\pgfpathlineto{\pgfqpoint{3.286648in}{2.224959in}}%
\pgfpathlineto{\pgfqpoint{3.333410in}{2.349087in}}%
\pgfpathlineto{\pgfqpoint{3.348980in}{2.385949in}}%
\pgfpathlineto{\pgfqpoint{3.380097in}{2.452344in}}%
\pgfpathlineto{\pgfqpoint{3.426710in}{2.541939in}}%
\pgfpathlineto{\pgfqpoint{3.442231in}{2.567313in}}%
\pgfpathlineto{\pgfqpoint{3.457744in}{2.589849in}}%
\pgfpathlineto{\pgfqpoint{3.473249in}{2.615254in}}%
\pgfpathlineto{\pgfqpoint{3.504234in}{2.648882in}}%
\pgfpathlineto{\pgfqpoint{3.535187in}{2.679929in}}%
\pgfpathlineto{\pgfqpoint{3.566107in}{2.704555in}}%
\pgfpathlineto{\pgfqpoint{3.581554in}{2.719322in}}%
\pgfpathlineto{\pgfqpoint{3.612425in}{2.739461in}}%
\pgfpathlineto{\pgfqpoint{3.626942in}{2.751469in}}%
\pgfpathlineto{\pgfqpoint{3.642357in}{2.760461in}}%
\pgfpathlineto{\pgfqpoint{3.657765in}{2.771654in}}%
\pgfpathlineto{\pgfqpoint{3.703938in}{2.817858in}}%
\pgfpathlineto{\pgfqpoint{3.734680in}{2.857003in}}%
\pgfpathlineto{\pgfqpoint{3.765390in}{2.907027in}}%
\pgfpathlineto{\pgfqpoint{3.810493in}{2.998108in}}%
\pgfpathlineto{\pgfqpoint{3.825812in}{3.031153in}}%
\pgfpathlineto{\pgfqpoint{3.887007in}{3.182450in}}%
\pgfpathlineto{\pgfqpoint{3.931922in}{3.318055in}}%
\pgfpathlineto{\pgfqpoint{3.947178in}{3.360270in}}%
\pgfpathlineto{\pgfqpoint{4.052848in}{3.685989in}}%
\pgfpathlineto{\pgfqpoint{4.083224in}{3.767587in}}%
\pgfpathlineto{\pgfqpoint{4.127836in}{3.866295in}}%
\pgfpathlineto{\pgfqpoint{4.142989in}{3.895362in}}%
\pgfpathlineto{\pgfqpoint{4.158134in}{3.917808in}}%
\pgfpathlineto{\pgfqpoint{4.172381in}{3.943019in}}%
\pgfpathlineto{\pgfqpoint{4.187510in}{3.961927in}}%
\pgfpathlineto{\pgfqpoint{4.202632in}{3.983469in}}%
\pgfpathlineto{\pgfqpoint{4.217746in}{4.001152in}}%
\pgfpathlineto{\pgfqpoint{4.232852in}{4.014775in}}%
\pgfpathlineto{\pgfqpoint{4.247062in}{4.029836in}}%
\pgfpathlineto{\pgfqpoint{4.262152in}{4.038863in}}%
\pgfpathlineto{\pgfqpoint{4.277235in}{4.051102in}}%
\pgfpathlineto{\pgfqpoint{4.307377in}{4.056000in}}%
\pgfpathlineto{\pgfqpoint{4.321550in}{4.055355in}}%
\pgfpathlineto{\pgfqpoint{4.336602in}{4.050709in}}%
\pgfpathlineto{\pgfqpoint{4.366682in}{4.037131in}}%
\pgfpathlineto{\pgfqpoint{4.395847in}{4.006646in}}%
\pgfpathlineto{\pgfqpoint{4.410861in}{3.993823in}}%
\pgfpathlineto{\pgfqpoint{4.425866in}{3.975367in}}%
\pgfpathlineto{\pgfqpoint{4.454971in}{3.930728in}}%
\pgfpathlineto{\pgfqpoint{4.484928in}{3.873481in}}%
\pgfpathlineto{\pgfqpoint{4.513975in}{3.807547in}}%
\pgfpathlineto{\pgfqpoint{4.557928in}{3.687345in}}%
\pgfpathlineto{\pgfqpoint{4.587778in}{3.599868in}}%
\pgfpathlineto{\pgfqpoint{4.631619in}{3.467575in}}%
\pgfpathlineto{\pgfqpoint{4.690262in}{3.273792in}}%
\pgfpathlineto{\pgfqpoint{4.719101in}{3.172029in}}%
\pgfpathlineto{\pgfqpoint{4.733947in}{3.122623in}}%
\pgfpathlineto{\pgfqpoint{4.879374in}{2.578462in}}%
\pgfpathlineto{\pgfqpoint{4.908024in}{2.468124in}}%
\pgfpathlineto{\pgfqpoint{4.922772in}{2.409720in}}%
\pgfpathlineto{\pgfqpoint{4.951379in}{2.284516in}}%
\pgfpathlineto{\pgfqpoint{4.994668in}{2.071223in}}%
\pgfpathlineto{\pgfqpoint{5.008507in}{1.995276in}}%
\pgfpathlineto{\pgfqpoint{5.023203in}{1.921495in}}%
\pgfpathlineto{\pgfqpoint{5.066383in}{1.686707in}}%
\pgfpathlineto{\pgfqpoint{5.081049in}{1.613895in}}%
\pgfpathlineto{\pgfqpoint{5.094846in}{1.537318in}}%
\pgfpathlineto{\pgfqpoint{5.123281in}{1.399833in}}%
\pgfpathlineto{\pgfqpoint{5.137918in}{1.334874in}}%
\pgfpathlineto{\pgfqpoint{5.166310in}{1.223819in}}%
\pgfpathlineto{\pgfqpoint{5.194674in}{1.130329in}}%
\pgfpathlineto{\pgfqpoint{5.223868in}{1.053735in}}%
\pgfpathlineto{\pgfqpoint{5.237595in}{1.019721in}}%
\pgfpathlineto{\pgfqpoint{5.265888in}{0.961900in}}%
\pgfpathlineto{\pgfqpoint{5.294153in}{0.915431in}}%
\pgfpathlineto{\pgfqpoint{5.308702in}{0.894171in}}%
\pgfpathlineto{\pgfqpoint{5.336924in}{0.860005in}}%
\pgfpathlineto{\pgfqpoint{5.365118in}{0.832187in}}%
\pgfpathlineto{\pgfqpoint{5.393284in}{0.809181in}}%
\pgfpathlineto{\pgfqpoint{5.421423in}{0.790268in}}%
\pgfpathlineto{\pgfqpoint{5.464003in}{0.767033in}}%
\pgfpathlineto{\pgfqpoint{5.534545in}{0.741295in}}%
\pgfpathlineto{\pgfqpoint{5.534545in}{0.741295in}}%
\pgfusepath{stroke}%
\end{pgfscope}%
\begin{pgfscope}%
\pgfpathrectangle{\pgfqpoint{0.800000in}{0.528000in}}{\pgfqpoint{4.960000in}{3.696000in}}%
\pgfusepath{clip}%
\pgfsetbuttcap%
\pgfsetroundjoin%
\pgfsetlinewidth{1.505625pt}%
\definecolor{currentstroke}{rgb}{0.501961,0.000000,0.501961}%
\pgfsetstrokecolor{currentstroke}%
\pgfsetdash{{5.550000pt}{2.400000pt}}{0.000000pt}%
\pgfpathmoveto{\pgfqpoint{3.994324in}{0.528000in}}%
\pgfpathlineto{\pgfqpoint{3.994324in}{4.224000in}}%
\pgfusepath{stroke}%
\end{pgfscope}%
\begin{pgfscope}%
\pgfsetrectcap%
\pgfsetmiterjoin%
\pgfsetlinewidth{0.803000pt}%
\definecolor{currentstroke}{rgb}{0.000000,0.000000,0.000000}%
\pgfsetstrokecolor{currentstroke}%
\pgfsetdash{}{0pt}%
\pgfpathmoveto{\pgfqpoint{0.800000in}{0.528000in}}%
\pgfpathlineto{\pgfqpoint{0.800000in}{4.224000in}}%
\pgfusepath{stroke}%
\end{pgfscope}%
\begin{pgfscope}%
\pgfsetrectcap%
\pgfsetmiterjoin%
\pgfsetlinewidth{0.803000pt}%
\definecolor{currentstroke}{rgb}{0.000000,0.000000,0.000000}%
\pgfsetstrokecolor{currentstroke}%
\pgfsetdash{}{0pt}%
\pgfpathmoveto{\pgfqpoint{5.760000in}{0.528000in}}%
\pgfpathlineto{\pgfqpoint{5.760000in}{4.224000in}}%
\pgfusepath{stroke}%
\end{pgfscope}%
\begin{pgfscope}%
\pgfsetrectcap%
\pgfsetmiterjoin%
\pgfsetlinewidth{0.803000pt}%
\definecolor{currentstroke}{rgb}{0.000000,0.000000,0.000000}%
\pgfsetstrokecolor{currentstroke}%
\pgfsetdash{}{0pt}%
\pgfpathmoveto{\pgfqpoint{0.800000in}{0.528000in}}%
\pgfpathlineto{\pgfqpoint{5.760000in}{0.528000in}}%
\pgfusepath{stroke}%
\end{pgfscope}%
\begin{pgfscope}%
\pgfsetrectcap%
\pgfsetmiterjoin%
\pgfsetlinewidth{0.803000pt}%
\definecolor{currentstroke}{rgb}{0.000000,0.000000,0.000000}%
\pgfsetstrokecolor{currentstroke}%
\pgfsetdash{}{0pt}%
\pgfpathmoveto{\pgfqpoint{0.800000in}{4.224000in}}%
\pgfpathlineto{\pgfqpoint{5.760000in}{4.224000in}}%
\pgfusepath{stroke}%
\end{pgfscope}%
\begin{pgfscope}%
\pgfsetbuttcap%
\pgfsetmiterjoin%
\definecolor{currentfill}{rgb}{1.000000,1.000000,1.000000}%
\pgfsetfillcolor{currentfill}%
\pgfsetfillopacity{0.800000}%
\pgfsetlinewidth{1.003750pt}%
\definecolor{currentstroke}{rgb}{0.800000,0.800000,0.800000}%
\pgfsetstrokecolor{currentstroke}%
\pgfsetstrokeopacity{0.800000}%
\pgfsetdash{}{0pt}%
\pgfpathmoveto{\pgfqpoint{0.916667in}{3.363860in}}%
\pgfpathlineto{\pgfqpoint{2.559619in}{3.363860in}}%
\pgfpathquadraticcurveto{\pgfqpoint{2.592952in}{3.363860in}}{\pgfqpoint{2.592952in}{3.397194in}}%
\pgfpathlineto{\pgfqpoint{2.592952in}{4.107333in}}%
\pgfpathquadraticcurveto{\pgfqpoint{2.592952in}{4.140667in}}{\pgfqpoint{2.559619in}{4.140667in}}%
\pgfpathlineto{\pgfqpoint{0.916667in}{4.140667in}}%
\pgfpathquadraticcurveto{\pgfqpoint{0.883333in}{4.140667in}}{\pgfqpoint{0.883333in}{4.107333in}}%
\pgfpathlineto{\pgfqpoint{0.883333in}{3.397194in}}%
\pgfpathquadraticcurveto{\pgfqpoint{0.883333in}{3.363860in}}{\pgfqpoint{0.916667in}{3.363860in}}%
\pgfpathlineto{\pgfqpoint{0.916667in}{3.363860in}}%
\pgfpathclose%
\pgfusepath{stroke,fill}%
\end{pgfscope}%
\begin{pgfscope}%
\pgfsetrectcap%
\pgfsetroundjoin%
\pgfsetlinewidth{1.505625pt}%
\definecolor{currentstroke}{rgb}{0.121569,0.466667,0.705882}%
\pgfsetstrokecolor{currentstroke}%
\pgfsetdash{}{0pt}%
\pgfpathmoveto{\pgfqpoint{0.950000in}{4.009693in}}%
\pgfpathlineto{\pgfqpoint{1.116667in}{4.009693in}}%
\pgfpathlineto{\pgfqpoint{1.283333in}{4.009693in}}%
\pgfusepath{stroke}%
\end{pgfscope}%
\begin{pgfscope}%
\definecolor{textcolor}{rgb}{0.000000,0.000000,0.000000}%
\pgfsetstrokecolor{textcolor}%
\pgfsetfillcolor{textcolor}%
\pgftext[x=1.416667in,y=3.951360in,left,base]{\color{textcolor}\rmfamily\fontsize{12.000000}{14.400000}\selectfont \(\displaystyle T_\mathrm{K}=293,35~\mathrm{K}\)}%
\end{pgfscope}%
\begin{pgfscope}%
\pgfsetrectcap%
\pgfsetroundjoin%
\pgfsetlinewidth{1.505625pt}%
\definecolor{currentstroke}{rgb}{1.000000,0.498039,0.054902}%
\pgfsetstrokecolor{currentstroke}%
\pgfsetdash{}{0pt}%
\pgfpathmoveto{\pgfqpoint{0.950000in}{3.767425in}}%
\pgfpathlineto{\pgfqpoint{1.116667in}{3.767425in}}%
\pgfpathlineto{\pgfqpoint{1.283333in}{3.767425in}}%
\pgfusepath{stroke}%
\end{pgfscope}%
\begin{pgfscope}%
\definecolor{textcolor}{rgb}{0.000000,0.000000,0.000000}%
\pgfsetstrokecolor{textcolor}%
\pgfsetfillcolor{textcolor}%
\pgftext[x=1.416667in,y=3.709091in,left,base]{\color{textcolor}\rmfamily\fontsize{12.000000}{14.400000}\selectfont \(\displaystyle T_\mathrm{K}=346,65~\mathrm{K}\)}%
\end{pgfscope}%
\begin{pgfscope}%
\pgfsetbuttcap%
\pgfsetroundjoin%
\pgfsetlinewidth{1.505625pt}%
\definecolor{currentstroke}{rgb}{0.501961,0.000000,0.501961}%
\pgfsetstrokecolor{currentstroke}%
\pgfsetdash{{5.550000pt}{2.400000pt}}{0.000000pt}%
\pgfpathmoveto{\pgfqpoint{0.950000in}{3.525156in}}%
\pgfpathlineto{\pgfqpoint{1.116667in}{3.525156in}}%
\pgfpathlineto{\pgfqpoint{1.283333in}{3.525156in}}%
\pgfusepath{stroke}%
\end{pgfscope}%
\begin{pgfscope}%
\definecolor{textcolor}{rgb}{0.000000,0.000000,0.000000}%
\pgfsetstrokecolor{textcolor}%
\pgfsetfillcolor{textcolor}%
\pgftext[x=1.416667in,y=3.466822in,left,base]{\color{textcolor}\rmfamily\fontsize{12.000000}{14.400000}\selectfont isosbestic point}%
\end{pgfscope}%
\end{pgfpicture}%
\makeatother%
\endgroup%

    \caption[Absolute Raman shift intensities for the lowest and highest temperatures]{Plot of the absolute scattering intensities over the Raman shift $\Delta \tilde{v}$; Plot at the highest recorded temperature and the lowest recorded temperature for simplification purposes}
    \label{fig:plot-temp-raw}
\end{figure}

%%%%%%%%%%%%%%%%%%%%%%%%%%%%%%%
\section{Evaluation and error discussion}
\label{sec:eval}

\subsection*{Species determination}
\label{subsec:spec-eval}

\commenting{
    Text and data from JP.

    Plots need to be integrated.
}

\subsection*{Temperature calculation}
\label{subsec:temp-eval}

The plot of the change of scattered intensities over the Raman shift is illustrated in \autoref{fig:plot-temp}. The isosbestic point can be determined at roughly $3390 \mathrm{cm^{-1}}$, via the lowest standard deviation of all the normalized data sets. The areas can be interpreted as the integration of the curve within the boundaries $[3390~\mathrm{cm^{-1}},~3800~\mathrm{cm^{-1}}]$ for $A_\mathrm{right}$, and $[2600~\mathrm{cm^{-1}},~3390~\mathrm{cm^{-1}}]$ for $A_\mathrm{left}$. Due to the already integrating behavior of the CCD sensor, it is calculated as the sum of all data points within the presented ranges.

\begin{figure}[!htb]
    \centering
    % \includegraphics[width=\textwidth]{02kapitel/temp-shift.png}
    %% Creator: Matplotlib, PGF backend
%%
%% To include the figure in your LaTeX document, write
%%   \input{<filename>.pgf}
%%
%% Make sure the required packages are loaded in your preamble
%%   \usepackage{pgf}
%%
%% Also ensure that all the required font packages are loaded; for instance,
%% the lmodern package is sometimes necessary when using math font.
%%   \usepackage{lmodern}
%%
%% Figures using additional raster images can only be included by \input if
%% they are in the same directory as the main LaTeX file. For loading figures
%% from other directories you can use the `import` package
%%   \usepackage{import}
%%
%% and then include the figures with
%%   \import{<path to file>}{<filename>.pgf}
%%
%% Matplotlib used the following preamble
%%   
%%   \usepackage{fontspec}
%%   \setmainfont{Charter.ttc}[Path=\detokenize{/System/Library/Fonts/Supplemental/}]
%%   \setsansfont{DejaVuSans.ttf}[Path=\detokenize{/opt/homebrew/lib/python3.10/site-packages/matplotlib/mpl-data/fonts/ttf/}]
%%   \setmonofont{DejaVuSansMono.ttf}[Path=\detokenize{/opt/homebrew/lib/python3.10/site-packages/matplotlib/mpl-data/fonts/ttf/}]
%%   \makeatletter\@ifpackageloaded{underscore}{}{\usepackage[strings]{underscore}}\makeatother
%%
\begingroup%
\makeatletter%
\begin{pgfpicture}%
\pgfpathrectangle{\pgfpointorigin}{\pgfqpoint{6.400000in}{4.800000in}}%
\pgfusepath{use as bounding box, clip}%
\begin{pgfscope}%
\pgfsetbuttcap%
\pgfsetmiterjoin%
\definecolor{currentfill}{rgb}{1.000000,1.000000,1.000000}%
\pgfsetfillcolor{currentfill}%
\pgfsetlinewidth{0.000000pt}%
\definecolor{currentstroke}{rgb}{1.000000,1.000000,1.000000}%
\pgfsetstrokecolor{currentstroke}%
\pgfsetdash{}{0pt}%
\pgfpathmoveto{\pgfqpoint{0.000000in}{0.000000in}}%
\pgfpathlineto{\pgfqpoint{6.400000in}{0.000000in}}%
\pgfpathlineto{\pgfqpoint{6.400000in}{4.800000in}}%
\pgfpathlineto{\pgfqpoint{0.000000in}{4.800000in}}%
\pgfpathlineto{\pgfqpoint{0.000000in}{0.000000in}}%
\pgfpathclose%
\pgfusepath{fill}%
\end{pgfscope}%
\begin{pgfscope}%
\pgfsetbuttcap%
\pgfsetmiterjoin%
\definecolor{currentfill}{rgb}{1.000000,1.000000,1.000000}%
\pgfsetfillcolor{currentfill}%
\pgfsetlinewidth{0.000000pt}%
\definecolor{currentstroke}{rgb}{0.000000,0.000000,0.000000}%
\pgfsetstrokecolor{currentstroke}%
\pgfsetstrokeopacity{0.000000}%
\pgfsetdash{}{0pt}%
\pgfpathmoveto{\pgfqpoint{0.800000in}{0.528000in}}%
\pgfpathlineto{\pgfqpoint{5.760000in}{0.528000in}}%
\pgfpathlineto{\pgfqpoint{5.760000in}{4.224000in}}%
\pgfpathlineto{\pgfqpoint{0.800000in}{4.224000in}}%
\pgfpathlineto{\pgfqpoint{0.800000in}{0.528000in}}%
\pgfpathclose%
\pgfusepath{fill}%
\end{pgfscope}%
\begin{pgfscope}%
\pgfpathrectangle{\pgfqpoint{0.800000in}{0.528000in}}{\pgfqpoint{4.960000in}{3.696000in}}%
\pgfusepath{clip}%
\pgfsetrectcap%
\pgfsetroundjoin%
\pgfsetlinewidth{0.803000pt}%
\definecolor{currentstroke}{rgb}{0.690196,0.690196,0.690196}%
\pgfsetstrokecolor{currentstroke}%
\pgfsetdash{}{0pt}%
\pgfpathmoveto{\pgfqpoint{1.013437in}{0.528000in}}%
\pgfpathlineto{\pgfqpoint{1.013437in}{4.224000in}}%
\pgfusepath{stroke}%
\end{pgfscope}%
\begin{pgfscope}%
\pgfsetbuttcap%
\pgfsetroundjoin%
\definecolor{currentfill}{rgb}{0.000000,0.000000,0.000000}%
\pgfsetfillcolor{currentfill}%
\pgfsetlinewidth{0.803000pt}%
\definecolor{currentstroke}{rgb}{0.000000,0.000000,0.000000}%
\pgfsetstrokecolor{currentstroke}%
\pgfsetdash{}{0pt}%
\pgfsys@defobject{currentmarker}{\pgfqpoint{0.000000in}{-0.048611in}}{\pgfqpoint{0.000000in}{0.000000in}}{%
\pgfpathmoveto{\pgfqpoint{0.000000in}{0.000000in}}%
\pgfpathlineto{\pgfqpoint{0.000000in}{-0.048611in}}%
\pgfusepath{stroke,fill}%
}%
\begin{pgfscope}%
\pgfsys@transformshift{1.013437in}{0.528000in}%
\pgfsys@useobject{currentmarker}{}%
\end{pgfscope}%
\end{pgfscope}%
\begin{pgfscope}%
\definecolor{textcolor}{rgb}{0.000000,0.000000,0.000000}%
\pgfsetstrokecolor{textcolor}%
\pgfsetfillcolor{textcolor}%
\pgftext[x=1.013437in,y=0.430778in,,top]{\color{textcolor}\rmfamily\fontsize{12.000000}{14.400000}\selectfont \(\displaystyle {2600}\)}%
\end{pgfscope}%
\begin{pgfscope}%
\pgfpathrectangle{\pgfqpoint{0.800000in}{0.528000in}}{\pgfqpoint{4.960000in}{3.696000in}}%
\pgfusepath{clip}%
\pgfsetrectcap%
\pgfsetroundjoin%
\pgfsetlinewidth{0.803000pt}%
\definecolor{currentstroke}{rgb}{0.690196,0.690196,0.690196}%
\pgfsetstrokecolor{currentstroke}%
\pgfsetdash{}{0pt}%
\pgfpathmoveto{\pgfqpoint{1.768092in}{0.528000in}}%
\pgfpathlineto{\pgfqpoint{1.768092in}{4.224000in}}%
\pgfusepath{stroke}%
\end{pgfscope}%
\begin{pgfscope}%
\pgfsetbuttcap%
\pgfsetroundjoin%
\definecolor{currentfill}{rgb}{0.000000,0.000000,0.000000}%
\pgfsetfillcolor{currentfill}%
\pgfsetlinewidth{0.803000pt}%
\definecolor{currentstroke}{rgb}{0.000000,0.000000,0.000000}%
\pgfsetstrokecolor{currentstroke}%
\pgfsetdash{}{0pt}%
\pgfsys@defobject{currentmarker}{\pgfqpoint{0.000000in}{-0.048611in}}{\pgfqpoint{0.000000in}{0.000000in}}{%
\pgfpathmoveto{\pgfqpoint{0.000000in}{0.000000in}}%
\pgfpathlineto{\pgfqpoint{0.000000in}{-0.048611in}}%
\pgfusepath{stroke,fill}%
}%
\begin{pgfscope}%
\pgfsys@transformshift{1.768092in}{0.528000in}%
\pgfsys@useobject{currentmarker}{}%
\end{pgfscope}%
\end{pgfscope}%
\begin{pgfscope}%
\definecolor{textcolor}{rgb}{0.000000,0.000000,0.000000}%
\pgfsetstrokecolor{textcolor}%
\pgfsetfillcolor{textcolor}%
\pgftext[x=1.768092in,y=0.430778in,,top]{\color{textcolor}\rmfamily\fontsize{12.000000}{14.400000}\selectfont \(\displaystyle {2800}\)}%
\end{pgfscope}%
\begin{pgfscope}%
\pgfpathrectangle{\pgfqpoint{0.800000in}{0.528000in}}{\pgfqpoint{4.960000in}{3.696000in}}%
\pgfusepath{clip}%
\pgfsetrectcap%
\pgfsetroundjoin%
\pgfsetlinewidth{0.803000pt}%
\definecolor{currentstroke}{rgb}{0.690196,0.690196,0.690196}%
\pgfsetstrokecolor{currentstroke}%
\pgfsetdash{}{0pt}%
\pgfpathmoveto{\pgfqpoint{2.522746in}{0.528000in}}%
\pgfpathlineto{\pgfqpoint{2.522746in}{4.224000in}}%
\pgfusepath{stroke}%
\end{pgfscope}%
\begin{pgfscope}%
\pgfsetbuttcap%
\pgfsetroundjoin%
\definecolor{currentfill}{rgb}{0.000000,0.000000,0.000000}%
\pgfsetfillcolor{currentfill}%
\pgfsetlinewidth{0.803000pt}%
\definecolor{currentstroke}{rgb}{0.000000,0.000000,0.000000}%
\pgfsetstrokecolor{currentstroke}%
\pgfsetdash{}{0pt}%
\pgfsys@defobject{currentmarker}{\pgfqpoint{0.000000in}{-0.048611in}}{\pgfqpoint{0.000000in}{0.000000in}}{%
\pgfpathmoveto{\pgfqpoint{0.000000in}{0.000000in}}%
\pgfpathlineto{\pgfqpoint{0.000000in}{-0.048611in}}%
\pgfusepath{stroke,fill}%
}%
\begin{pgfscope}%
\pgfsys@transformshift{2.522746in}{0.528000in}%
\pgfsys@useobject{currentmarker}{}%
\end{pgfscope}%
\end{pgfscope}%
\begin{pgfscope}%
\definecolor{textcolor}{rgb}{0.000000,0.000000,0.000000}%
\pgfsetstrokecolor{textcolor}%
\pgfsetfillcolor{textcolor}%
\pgftext[x=2.522746in,y=0.430778in,,top]{\color{textcolor}\rmfamily\fontsize{12.000000}{14.400000}\selectfont \(\displaystyle {3000}\)}%
\end{pgfscope}%
\begin{pgfscope}%
\pgfpathrectangle{\pgfqpoint{0.800000in}{0.528000in}}{\pgfqpoint{4.960000in}{3.696000in}}%
\pgfusepath{clip}%
\pgfsetrectcap%
\pgfsetroundjoin%
\pgfsetlinewidth{0.803000pt}%
\definecolor{currentstroke}{rgb}{0.690196,0.690196,0.690196}%
\pgfsetstrokecolor{currentstroke}%
\pgfsetdash{}{0pt}%
\pgfpathmoveto{\pgfqpoint{3.277401in}{0.528000in}}%
\pgfpathlineto{\pgfqpoint{3.277401in}{4.224000in}}%
\pgfusepath{stroke}%
\end{pgfscope}%
\begin{pgfscope}%
\pgfsetbuttcap%
\pgfsetroundjoin%
\definecolor{currentfill}{rgb}{0.000000,0.000000,0.000000}%
\pgfsetfillcolor{currentfill}%
\pgfsetlinewidth{0.803000pt}%
\definecolor{currentstroke}{rgb}{0.000000,0.000000,0.000000}%
\pgfsetstrokecolor{currentstroke}%
\pgfsetdash{}{0pt}%
\pgfsys@defobject{currentmarker}{\pgfqpoint{0.000000in}{-0.048611in}}{\pgfqpoint{0.000000in}{0.000000in}}{%
\pgfpathmoveto{\pgfqpoint{0.000000in}{0.000000in}}%
\pgfpathlineto{\pgfqpoint{0.000000in}{-0.048611in}}%
\pgfusepath{stroke,fill}%
}%
\begin{pgfscope}%
\pgfsys@transformshift{3.277401in}{0.528000in}%
\pgfsys@useobject{currentmarker}{}%
\end{pgfscope}%
\end{pgfscope}%
\begin{pgfscope}%
\definecolor{textcolor}{rgb}{0.000000,0.000000,0.000000}%
\pgfsetstrokecolor{textcolor}%
\pgfsetfillcolor{textcolor}%
\pgftext[x=3.277401in,y=0.430778in,,top]{\color{textcolor}\rmfamily\fontsize{12.000000}{14.400000}\selectfont \(\displaystyle {3200}\)}%
\end{pgfscope}%
\begin{pgfscope}%
\pgfpathrectangle{\pgfqpoint{0.800000in}{0.528000in}}{\pgfqpoint{4.960000in}{3.696000in}}%
\pgfusepath{clip}%
\pgfsetrectcap%
\pgfsetroundjoin%
\pgfsetlinewidth{0.803000pt}%
\definecolor{currentstroke}{rgb}{0.690196,0.690196,0.690196}%
\pgfsetstrokecolor{currentstroke}%
\pgfsetdash{}{0pt}%
\pgfpathmoveto{\pgfqpoint{4.032056in}{0.528000in}}%
\pgfpathlineto{\pgfqpoint{4.032056in}{4.224000in}}%
\pgfusepath{stroke}%
\end{pgfscope}%
\begin{pgfscope}%
\pgfsetbuttcap%
\pgfsetroundjoin%
\definecolor{currentfill}{rgb}{0.000000,0.000000,0.000000}%
\pgfsetfillcolor{currentfill}%
\pgfsetlinewidth{0.803000pt}%
\definecolor{currentstroke}{rgb}{0.000000,0.000000,0.000000}%
\pgfsetstrokecolor{currentstroke}%
\pgfsetdash{}{0pt}%
\pgfsys@defobject{currentmarker}{\pgfqpoint{0.000000in}{-0.048611in}}{\pgfqpoint{0.000000in}{0.000000in}}{%
\pgfpathmoveto{\pgfqpoint{0.000000in}{0.000000in}}%
\pgfpathlineto{\pgfqpoint{0.000000in}{-0.048611in}}%
\pgfusepath{stroke,fill}%
}%
\begin{pgfscope}%
\pgfsys@transformshift{4.032056in}{0.528000in}%
\pgfsys@useobject{currentmarker}{}%
\end{pgfscope}%
\end{pgfscope}%
\begin{pgfscope}%
\definecolor{textcolor}{rgb}{0.000000,0.000000,0.000000}%
\pgfsetstrokecolor{textcolor}%
\pgfsetfillcolor{textcolor}%
\pgftext[x=4.032056in,y=0.430778in,,top]{\color{textcolor}\rmfamily\fontsize{12.000000}{14.400000}\selectfont \(\displaystyle {3400}\)}%
\end{pgfscope}%
\begin{pgfscope}%
\pgfpathrectangle{\pgfqpoint{0.800000in}{0.528000in}}{\pgfqpoint{4.960000in}{3.696000in}}%
\pgfusepath{clip}%
\pgfsetrectcap%
\pgfsetroundjoin%
\pgfsetlinewidth{0.803000pt}%
\definecolor{currentstroke}{rgb}{0.690196,0.690196,0.690196}%
\pgfsetstrokecolor{currentstroke}%
\pgfsetdash{}{0pt}%
\pgfpathmoveto{\pgfqpoint{4.786711in}{0.528000in}}%
\pgfpathlineto{\pgfqpoint{4.786711in}{4.224000in}}%
\pgfusepath{stroke}%
\end{pgfscope}%
\begin{pgfscope}%
\pgfsetbuttcap%
\pgfsetroundjoin%
\definecolor{currentfill}{rgb}{0.000000,0.000000,0.000000}%
\pgfsetfillcolor{currentfill}%
\pgfsetlinewidth{0.803000pt}%
\definecolor{currentstroke}{rgb}{0.000000,0.000000,0.000000}%
\pgfsetstrokecolor{currentstroke}%
\pgfsetdash{}{0pt}%
\pgfsys@defobject{currentmarker}{\pgfqpoint{0.000000in}{-0.048611in}}{\pgfqpoint{0.000000in}{0.000000in}}{%
\pgfpathmoveto{\pgfqpoint{0.000000in}{0.000000in}}%
\pgfpathlineto{\pgfqpoint{0.000000in}{-0.048611in}}%
\pgfusepath{stroke,fill}%
}%
\begin{pgfscope}%
\pgfsys@transformshift{4.786711in}{0.528000in}%
\pgfsys@useobject{currentmarker}{}%
\end{pgfscope}%
\end{pgfscope}%
\begin{pgfscope}%
\definecolor{textcolor}{rgb}{0.000000,0.000000,0.000000}%
\pgfsetstrokecolor{textcolor}%
\pgfsetfillcolor{textcolor}%
\pgftext[x=4.786711in,y=0.430778in,,top]{\color{textcolor}\rmfamily\fontsize{12.000000}{14.400000}\selectfont \(\displaystyle {3600}\)}%
\end{pgfscope}%
\begin{pgfscope}%
\pgfpathrectangle{\pgfqpoint{0.800000in}{0.528000in}}{\pgfqpoint{4.960000in}{3.696000in}}%
\pgfusepath{clip}%
\pgfsetrectcap%
\pgfsetroundjoin%
\pgfsetlinewidth{0.803000pt}%
\definecolor{currentstroke}{rgb}{0.690196,0.690196,0.690196}%
\pgfsetstrokecolor{currentstroke}%
\pgfsetdash{}{0pt}%
\pgfpathmoveto{\pgfqpoint{5.541366in}{0.528000in}}%
\pgfpathlineto{\pgfqpoint{5.541366in}{4.224000in}}%
\pgfusepath{stroke}%
\end{pgfscope}%
\begin{pgfscope}%
\pgfsetbuttcap%
\pgfsetroundjoin%
\definecolor{currentfill}{rgb}{0.000000,0.000000,0.000000}%
\pgfsetfillcolor{currentfill}%
\pgfsetlinewidth{0.803000pt}%
\definecolor{currentstroke}{rgb}{0.000000,0.000000,0.000000}%
\pgfsetstrokecolor{currentstroke}%
\pgfsetdash{}{0pt}%
\pgfsys@defobject{currentmarker}{\pgfqpoint{0.000000in}{-0.048611in}}{\pgfqpoint{0.000000in}{0.000000in}}{%
\pgfpathmoveto{\pgfqpoint{0.000000in}{0.000000in}}%
\pgfpathlineto{\pgfqpoint{0.000000in}{-0.048611in}}%
\pgfusepath{stroke,fill}%
}%
\begin{pgfscope}%
\pgfsys@transformshift{5.541366in}{0.528000in}%
\pgfsys@useobject{currentmarker}{}%
\end{pgfscope}%
\end{pgfscope}%
\begin{pgfscope}%
\definecolor{textcolor}{rgb}{0.000000,0.000000,0.000000}%
\pgfsetstrokecolor{textcolor}%
\pgfsetfillcolor{textcolor}%
\pgftext[x=5.541366in,y=0.430778in,,top]{\color{textcolor}\rmfamily\fontsize{12.000000}{14.400000}\selectfont \(\displaystyle {3800}\)}%
\end{pgfscope}%
\begin{pgfscope}%
\definecolor{textcolor}{rgb}{0.000000,0.000000,0.000000}%
\pgfsetstrokecolor{textcolor}%
\pgfsetfillcolor{textcolor}%
\pgftext[x=3.280000in,y=0.216287in,,top]{\color{textcolor}\rmfamily\fontsize{12.000000}{14.400000}\selectfont Raman shift \(\displaystyle \Delta v\) in \(\displaystyle \mathrm{cm}^\mathrm{-1}\)}%
\end{pgfscope}%
\begin{pgfscope}%
\pgfpathrectangle{\pgfqpoint{0.800000in}{0.528000in}}{\pgfqpoint{4.960000in}{3.696000in}}%
\pgfusepath{clip}%
\pgfsetrectcap%
\pgfsetroundjoin%
\pgfsetlinewidth{0.803000pt}%
\definecolor{currentstroke}{rgb}{0.690196,0.690196,0.690196}%
\pgfsetstrokecolor{currentstroke}%
\pgfsetdash{}{0pt}%
\pgfpathmoveto{\pgfqpoint{0.800000in}{0.698956in}}%
\pgfpathlineto{\pgfqpoint{5.760000in}{0.698956in}}%
\pgfusepath{stroke}%
\end{pgfscope}%
\begin{pgfscope}%
\pgfsetbuttcap%
\pgfsetroundjoin%
\definecolor{currentfill}{rgb}{0.000000,0.000000,0.000000}%
\pgfsetfillcolor{currentfill}%
\pgfsetlinewidth{0.803000pt}%
\definecolor{currentstroke}{rgb}{0.000000,0.000000,0.000000}%
\pgfsetstrokecolor{currentstroke}%
\pgfsetdash{}{0pt}%
\pgfsys@defobject{currentmarker}{\pgfqpoint{-0.048611in}{0.000000in}}{\pgfqpoint{-0.000000in}{0.000000in}}{%
\pgfpathmoveto{\pgfqpoint{-0.000000in}{0.000000in}}%
\pgfpathlineto{\pgfqpoint{-0.048611in}{0.000000in}}%
\pgfusepath{stroke,fill}%
}%
\begin{pgfscope}%
\pgfsys@transformshift{0.800000in}{0.698956in}%
\pgfsys@useobject{currentmarker}{}%
\end{pgfscope}%
\end{pgfscope}%
\begin{pgfscope}%
\definecolor{textcolor}{rgb}{0.000000,0.000000,0.000000}%
\pgfsetstrokecolor{textcolor}%
\pgfsetfillcolor{textcolor}%
\pgftext[x=0.331061in, y=0.637636in, left, base]{\color{textcolor}\rmfamily\fontsize{12.000000}{14.400000}\selectfont \(\displaystyle {0.000}\)}%
\end{pgfscope}%
\begin{pgfscope}%
\pgfpathrectangle{\pgfqpoint{0.800000in}{0.528000in}}{\pgfqpoint{4.960000in}{3.696000in}}%
\pgfusepath{clip}%
\pgfsetrectcap%
\pgfsetroundjoin%
\pgfsetlinewidth{0.803000pt}%
\definecolor{currentstroke}{rgb}{0.690196,0.690196,0.690196}%
\pgfsetstrokecolor{currentstroke}%
\pgfsetdash{}{0pt}%
\pgfpathmoveto{\pgfqpoint{0.800000in}{1.409346in}}%
\pgfpathlineto{\pgfqpoint{5.760000in}{1.409346in}}%
\pgfusepath{stroke}%
\end{pgfscope}%
\begin{pgfscope}%
\pgfsetbuttcap%
\pgfsetroundjoin%
\definecolor{currentfill}{rgb}{0.000000,0.000000,0.000000}%
\pgfsetfillcolor{currentfill}%
\pgfsetlinewidth{0.803000pt}%
\definecolor{currentstroke}{rgb}{0.000000,0.000000,0.000000}%
\pgfsetstrokecolor{currentstroke}%
\pgfsetdash{}{0pt}%
\pgfsys@defobject{currentmarker}{\pgfqpoint{-0.048611in}{0.000000in}}{\pgfqpoint{-0.000000in}{0.000000in}}{%
\pgfpathmoveto{\pgfqpoint{-0.000000in}{0.000000in}}%
\pgfpathlineto{\pgfqpoint{-0.048611in}{0.000000in}}%
\pgfusepath{stroke,fill}%
}%
\begin{pgfscope}%
\pgfsys@transformshift{0.800000in}{1.409346in}%
\pgfsys@useobject{currentmarker}{}%
\end{pgfscope}%
\end{pgfscope}%
\begin{pgfscope}%
\definecolor{textcolor}{rgb}{0.000000,0.000000,0.000000}%
\pgfsetstrokecolor{textcolor}%
\pgfsetfillcolor{textcolor}%
\pgftext[x=0.331061in, y=1.348026in, left, base]{\color{textcolor}\rmfamily\fontsize{12.000000}{14.400000}\selectfont \(\displaystyle {0.002}\)}%
\end{pgfscope}%
\begin{pgfscope}%
\pgfpathrectangle{\pgfqpoint{0.800000in}{0.528000in}}{\pgfqpoint{4.960000in}{3.696000in}}%
\pgfusepath{clip}%
\pgfsetrectcap%
\pgfsetroundjoin%
\pgfsetlinewidth{0.803000pt}%
\definecolor{currentstroke}{rgb}{0.690196,0.690196,0.690196}%
\pgfsetstrokecolor{currentstroke}%
\pgfsetdash{}{0pt}%
\pgfpathmoveto{\pgfqpoint{0.800000in}{2.119737in}}%
\pgfpathlineto{\pgfqpoint{5.760000in}{2.119737in}}%
\pgfusepath{stroke}%
\end{pgfscope}%
\begin{pgfscope}%
\pgfsetbuttcap%
\pgfsetroundjoin%
\definecolor{currentfill}{rgb}{0.000000,0.000000,0.000000}%
\pgfsetfillcolor{currentfill}%
\pgfsetlinewidth{0.803000pt}%
\definecolor{currentstroke}{rgb}{0.000000,0.000000,0.000000}%
\pgfsetstrokecolor{currentstroke}%
\pgfsetdash{}{0pt}%
\pgfsys@defobject{currentmarker}{\pgfqpoint{-0.048611in}{0.000000in}}{\pgfqpoint{-0.000000in}{0.000000in}}{%
\pgfpathmoveto{\pgfqpoint{-0.000000in}{0.000000in}}%
\pgfpathlineto{\pgfqpoint{-0.048611in}{0.000000in}}%
\pgfusepath{stroke,fill}%
}%
\begin{pgfscope}%
\pgfsys@transformshift{0.800000in}{2.119737in}%
\pgfsys@useobject{currentmarker}{}%
\end{pgfscope}%
\end{pgfscope}%
\begin{pgfscope}%
\definecolor{textcolor}{rgb}{0.000000,0.000000,0.000000}%
\pgfsetstrokecolor{textcolor}%
\pgfsetfillcolor{textcolor}%
\pgftext[x=0.331061in, y=2.058417in, left, base]{\color{textcolor}\rmfamily\fontsize{12.000000}{14.400000}\selectfont \(\displaystyle {0.004}\)}%
\end{pgfscope}%
\begin{pgfscope}%
\pgfpathrectangle{\pgfqpoint{0.800000in}{0.528000in}}{\pgfqpoint{4.960000in}{3.696000in}}%
\pgfusepath{clip}%
\pgfsetrectcap%
\pgfsetroundjoin%
\pgfsetlinewidth{0.803000pt}%
\definecolor{currentstroke}{rgb}{0.690196,0.690196,0.690196}%
\pgfsetstrokecolor{currentstroke}%
\pgfsetdash{}{0pt}%
\pgfpathmoveto{\pgfqpoint{0.800000in}{2.830128in}}%
\pgfpathlineto{\pgfqpoint{5.760000in}{2.830128in}}%
\pgfusepath{stroke}%
\end{pgfscope}%
\begin{pgfscope}%
\pgfsetbuttcap%
\pgfsetroundjoin%
\definecolor{currentfill}{rgb}{0.000000,0.000000,0.000000}%
\pgfsetfillcolor{currentfill}%
\pgfsetlinewidth{0.803000pt}%
\definecolor{currentstroke}{rgb}{0.000000,0.000000,0.000000}%
\pgfsetstrokecolor{currentstroke}%
\pgfsetdash{}{0pt}%
\pgfsys@defobject{currentmarker}{\pgfqpoint{-0.048611in}{0.000000in}}{\pgfqpoint{-0.000000in}{0.000000in}}{%
\pgfpathmoveto{\pgfqpoint{-0.000000in}{0.000000in}}%
\pgfpathlineto{\pgfqpoint{-0.048611in}{0.000000in}}%
\pgfusepath{stroke,fill}%
}%
\begin{pgfscope}%
\pgfsys@transformshift{0.800000in}{2.830128in}%
\pgfsys@useobject{currentmarker}{}%
\end{pgfscope}%
\end{pgfscope}%
\begin{pgfscope}%
\definecolor{textcolor}{rgb}{0.000000,0.000000,0.000000}%
\pgfsetstrokecolor{textcolor}%
\pgfsetfillcolor{textcolor}%
\pgftext[x=0.331061in, y=2.768808in, left, base]{\color{textcolor}\rmfamily\fontsize{12.000000}{14.400000}\selectfont \(\displaystyle {0.006}\)}%
\end{pgfscope}%
\begin{pgfscope}%
\pgfpathrectangle{\pgfqpoint{0.800000in}{0.528000in}}{\pgfqpoint{4.960000in}{3.696000in}}%
\pgfusepath{clip}%
\pgfsetrectcap%
\pgfsetroundjoin%
\pgfsetlinewidth{0.803000pt}%
\definecolor{currentstroke}{rgb}{0.690196,0.690196,0.690196}%
\pgfsetstrokecolor{currentstroke}%
\pgfsetdash{}{0pt}%
\pgfpathmoveto{\pgfqpoint{0.800000in}{3.540519in}}%
\pgfpathlineto{\pgfqpoint{5.760000in}{3.540519in}}%
\pgfusepath{stroke}%
\end{pgfscope}%
\begin{pgfscope}%
\pgfsetbuttcap%
\pgfsetroundjoin%
\definecolor{currentfill}{rgb}{0.000000,0.000000,0.000000}%
\pgfsetfillcolor{currentfill}%
\pgfsetlinewidth{0.803000pt}%
\definecolor{currentstroke}{rgb}{0.000000,0.000000,0.000000}%
\pgfsetstrokecolor{currentstroke}%
\pgfsetdash{}{0pt}%
\pgfsys@defobject{currentmarker}{\pgfqpoint{-0.048611in}{0.000000in}}{\pgfqpoint{-0.000000in}{0.000000in}}{%
\pgfpathmoveto{\pgfqpoint{-0.000000in}{0.000000in}}%
\pgfpathlineto{\pgfqpoint{-0.048611in}{0.000000in}}%
\pgfusepath{stroke,fill}%
}%
\begin{pgfscope}%
\pgfsys@transformshift{0.800000in}{3.540519in}%
\pgfsys@useobject{currentmarker}{}%
\end{pgfscope}%
\end{pgfscope}%
\begin{pgfscope}%
\definecolor{textcolor}{rgb}{0.000000,0.000000,0.000000}%
\pgfsetstrokecolor{textcolor}%
\pgfsetfillcolor{textcolor}%
\pgftext[x=0.331061in, y=3.479199in, left, base]{\color{textcolor}\rmfamily\fontsize{12.000000}{14.400000}\selectfont \(\displaystyle {0.008}\)}%
\end{pgfscope}%
\begin{pgfscope}%
\definecolor{textcolor}{rgb}{0.000000,0.000000,0.000000}%
\pgfsetstrokecolor{textcolor}%
\pgfsetfillcolor{textcolor}%
\pgftext[x=0.275505in,y=2.376000in,,bottom,rotate=90.000000]{\color{textcolor}\rmfamily\fontsize{12.000000}{14.400000}\selectfont Normalized signal intensity}%
\end{pgfscope}%
\begin{pgfscope}%
\pgfpathrectangle{\pgfqpoint{0.800000in}{0.528000in}}{\pgfqpoint{4.960000in}{3.696000in}}%
\pgfusepath{clip}%
\pgfsetrectcap%
\pgfsetroundjoin%
\pgfsetlinewidth{1.505625pt}%
\definecolor{currentstroke}{rgb}{0.121569,0.466667,0.705882}%
\pgfsetstrokecolor{currentstroke}%
\pgfsetdash{}{0pt}%
\pgfpathmoveto{\pgfqpoint{1.025455in}{0.706217in}}%
\pgfpathlineto{\pgfqpoint{1.042271in}{0.704509in}}%
\pgfpathlineto{\pgfqpoint{1.059079in}{0.705586in}}%
\pgfpathlineto{\pgfqpoint{1.076865in}{0.704200in}}%
\pgfpathlineto{\pgfqpoint{1.111420in}{0.703403in}}%
\pgfpathlineto{\pgfqpoint{1.299303in}{0.700647in}}%
\pgfpathlineto{\pgfqpoint{1.315970in}{0.698937in}}%
\pgfpathlineto{\pgfqpoint{1.333607in}{0.699852in}}%
\pgfpathlineto{\pgfqpoint{1.366893in}{0.697874in}}%
\pgfpathlineto{\pgfqpoint{1.384500in}{0.698714in}}%
\pgfpathlineto{\pgfqpoint{1.417731in}{0.697963in}}%
\pgfpathlineto{\pgfqpoint{1.435309in}{0.698995in}}%
\pgfpathlineto{\pgfqpoint{1.451901in}{0.697446in}}%
\pgfpathlineto{\pgfqpoint{1.486033in}{0.698337in}}%
\pgfpathlineto{\pgfqpoint{1.502597in}{0.697855in}}%
\pgfpathlineto{\pgfqpoint{1.519152in}{0.698834in}}%
\pgfpathlineto{\pgfqpoint{1.587226in}{0.697694in}}%
\pgfpathlineto{\pgfqpoint{1.603735in}{0.699303in}}%
\pgfpathlineto{\pgfqpoint{1.637696in}{0.699780in}}%
\pgfpathlineto{\pgfqpoint{1.688082in}{0.702126in}}%
\pgfpathlineto{\pgfqpoint{1.737418in}{0.702771in}}%
\pgfpathlineto{\pgfqpoint{1.754811in}{0.704907in}}%
\pgfpathlineto{\pgfqpoint{1.771229in}{0.704868in}}%
\pgfpathlineto{\pgfqpoint{1.805003in}{0.708803in}}%
\pgfpathlineto{\pgfqpoint{1.821393in}{0.708863in}}%
\pgfpathlineto{\pgfqpoint{1.854148in}{0.712708in}}%
\pgfpathlineto{\pgfqpoint{1.887828in}{0.717123in}}%
\pgfpathlineto{\pgfqpoint{1.904174in}{0.719844in}}%
\pgfpathlineto{\pgfqpoint{1.937799in}{0.723168in}}%
\pgfpathlineto{\pgfqpoint{1.970427in}{0.729665in}}%
\pgfpathlineto{\pgfqpoint{2.036535in}{0.740982in}}%
\pgfpathlineto{\pgfqpoint{2.102497in}{0.755942in}}%
\pgfpathlineto{\pgfqpoint{2.134947in}{0.764456in}}%
\pgfpathlineto{\pgfqpoint{2.152112in}{0.767842in}}%
\pgfpathlineto{\pgfqpoint{2.184508in}{0.779131in}}%
\pgfpathlineto{\pgfqpoint{2.200693in}{0.782267in}}%
\pgfpathlineto{\pgfqpoint{2.250146in}{0.799336in}}%
\pgfpathlineto{\pgfqpoint{2.282437in}{0.814288in}}%
\pgfpathlineto{\pgfqpoint{2.298569in}{0.820307in}}%
\pgfpathlineto{\pgfqpoint{2.315641in}{0.827949in}}%
\pgfpathlineto{\pgfqpoint{2.380046in}{0.864589in}}%
\pgfpathlineto{\pgfqpoint{2.429203in}{0.896029in}}%
\pgfpathlineto{\pgfqpoint{2.493364in}{0.947449in}}%
\pgfpathlineto{\pgfqpoint{2.542336in}{0.990921in}}%
\pgfpathlineto{\pgfqpoint{2.558328in}{1.007635in}}%
\pgfpathlineto{\pgfqpoint{2.574312in}{1.027136in}}%
\pgfpathlineto{\pgfqpoint{2.606254in}{1.062723in}}%
\pgfpathlineto{\pgfqpoint{2.655040in}{1.125056in}}%
\pgfpathlineto{\pgfqpoint{2.686896in}{1.172328in}}%
\pgfpathlineto{\pgfqpoint{2.702810in}{1.200191in}}%
\pgfpathlineto{\pgfqpoint{2.718717in}{1.223791in}}%
\pgfpathlineto{\pgfqpoint{2.734615in}{1.251319in}}%
\pgfpathlineto{\pgfqpoint{2.750504in}{1.283852in}}%
\pgfpathlineto{\pgfqpoint{2.782257in}{1.343411in}}%
\pgfpathlineto{\pgfqpoint{2.813977in}{1.408571in}}%
\pgfpathlineto{\pgfqpoint{2.846594in}{1.479928in}}%
\pgfpathlineto{\pgfqpoint{2.878244in}{1.556202in}}%
\pgfpathlineto{\pgfqpoint{2.909861in}{1.639195in}}%
\pgfpathlineto{\pgfqpoint{2.925657in}{1.681848in}}%
\pgfpathlineto{\pgfqpoint{2.988756in}{1.875408in}}%
\pgfpathlineto{\pgfqpoint{3.051720in}{2.085246in}}%
\pgfpathlineto{\pgfqpoint{3.114551in}{2.302648in}}%
\pgfpathlineto{\pgfqpoint{3.130238in}{2.358861in}}%
\pgfpathlineto{\pgfqpoint{3.161586in}{2.462764in}}%
\pgfpathlineto{\pgfqpoint{3.193822in}{2.561335in}}%
\pgfpathlineto{\pgfqpoint{3.209467in}{2.613115in}}%
\pgfpathlineto{\pgfqpoint{3.256351in}{2.744133in}}%
\pgfpathlineto{\pgfqpoint{3.302243in}{2.850561in}}%
\pgfpathlineto{\pgfqpoint{3.317831in}{2.877553in}}%
\pgfpathlineto{\pgfqpoint{3.333410in}{2.909048in}}%
\pgfpathlineto{\pgfqpoint{3.348980in}{2.934681in}}%
\pgfpathlineto{\pgfqpoint{3.395643in}{2.998658in}}%
\pgfpathlineto{\pgfqpoint{3.442231in}{3.041607in}}%
\pgfpathlineto{\pgfqpoint{3.457744in}{3.054269in}}%
\pgfpathlineto{\pgfqpoint{3.473249in}{3.061953in}}%
\pgfpathlineto{\pgfqpoint{3.488746in}{3.072765in}}%
\pgfpathlineto{\pgfqpoint{3.504234in}{3.072850in}}%
\pgfpathlineto{\pgfqpoint{3.519715in}{3.077300in}}%
\pgfpathlineto{\pgfqpoint{3.550651in}{3.082135in}}%
\pgfpathlineto{\pgfqpoint{3.566107in}{3.084439in}}%
\pgfpathlineto{\pgfqpoint{3.581554in}{3.081339in}}%
\pgfpathlineto{\pgfqpoint{3.596994in}{3.085024in}}%
\pgfpathlineto{\pgfqpoint{3.612425in}{3.079754in}}%
\pgfpathlineto{\pgfqpoint{3.626942in}{3.081323in}}%
\pgfpathlineto{\pgfqpoint{3.642357in}{3.078779in}}%
\pgfpathlineto{\pgfqpoint{3.657765in}{3.081494in}}%
\pgfpathlineto{\pgfqpoint{3.673164in}{3.081174in}}%
\pgfpathlineto{\pgfqpoint{3.688555in}{3.082744in}}%
\pgfpathlineto{\pgfqpoint{3.719313in}{3.095577in}}%
\pgfpathlineto{\pgfqpoint{3.734680in}{3.104429in}}%
\pgfpathlineto{\pgfqpoint{3.750039in}{3.114852in}}%
\pgfpathlineto{\pgfqpoint{3.765390in}{3.128904in}}%
\pgfpathlineto{\pgfqpoint{3.779830in}{3.145615in}}%
\pgfpathlineto{\pgfqpoint{3.795166in}{3.160739in}}%
\pgfpathlineto{\pgfqpoint{3.825812in}{3.202409in}}%
\pgfpathlineto{\pgfqpoint{3.841123in}{3.224688in}}%
\pgfpathlineto{\pgfqpoint{3.856426in}{3.244452in}}%
\pgfpathlineto{\pgfqpoint{3.887007in}{3.296181in}}%
\pgfpathlineto{\pgfqpoint{3.901388in}{3.322525in}}%
\pgfpathlineto{\pgfqpoint{3.931922in}{3.382302in}}%
\pgfpathlineto{\pgfqpoint{3.962425in}{3.442859in}}%
\pgfpathlineto{\pgfqpoint{3.977664in}{3.468629in}}%
\pgfpathlineto{\pgfqpoint{4.007224in}{3.531208in}}%
\pgfpathlineto{\pgfqpoint{4.022440in}{3.560273in}}%
\pgfpathlineto{\pgfqpoint{4.037648in}{3.586771in}}%
\pgfpathlineto{\pgfqpoint{4.052848in}{3.617505in}}%
\pgfpathlineto{\pgfqpoint{4.083224in}{3.664595in}}%
\pgfpathlineto{\pgfqpoint{4.097507in}{3.678796in}}%
\pgfpathlineto{\pgfqpoint{4.112676in}{3.695996in}}%
\pgfpathlineto{\pgfqpoint{4.127836in}{3.710289in}}%
\pgfpathlineto{\pgfqpoint{4.142989in}{3.718664in}}%
\pgfpathlineto{\pgfqpoint{4.158134in}{3.728888in}}%
\pgfpathlineto{\pgfqpoint{4.172381in}{3.734155in}}%
\pgfpathlineto{\pgfqpoint{4.187510in}{3.736908in}}%
\pgfpathlineto{\pgfqpoint{4.202632in}{3.738235in}}%
\pgfpathlineto{\pgfqpoint{4.232852in}{3.732909in}}%
\pgfpathlineto{\pgfqpoint{4.247062in}{3.725637in}}%
\pgfpathlineto{\pgfqpoint{4.292310in}{3.693964in}}%
\pgfpathlineto{\pgfqpoint{4.307377in}{3.674054in}}%
\pgfpathlineto{\pgfqpoint{4.321550in}{3.649829in}}%
\pgfpathlineto{\pgfqpoint{4.336602in}{3.627824in}}%
\pgfpathlineto{\pgfqpoint{4.351646in}{3.602165in}}%
\pgfpathlineto{\pgfqpoint{4.366682in}{3.571794in}}%
\pgfpathlineto{\pgfqpoint{4.380827in}{3.538055in}}%
\pgfpathlineto{\pgfqpoint{4.395847in}{3.508927in}}%
\pgfpathlineto{\pgfqpoint{4.410861in}{3.475899in}}%
\pgfpathlineto{\pgfqpoint{4.425866in}{3.439194in}}%
\pgfpathlineto{\pgfqpoint{4.439981in}{3.400017in}}%
\pgfpathlineto{\pgfqpoint{4.454971in}{3.362379in}}%
\pgfpathlineto{\pgfqpoint{4.484928in}{3.276649in}}%
\pgfpathlineto{\pgfqpoint{4.513975in}{3.189598in}}%
\pgfpathlineto{\pgfqpoint{4.572857in}{2.995949in}}%
\pgfpathlineto{\pgfqpoint{4.704248in}{2.588866in}}%
\pgfpathlineto{\pgfqpoint{4.748784in}{2.462788in}}%
\pgfpathlineto{\pgfqpoint{4.777565in}{2.378253in}}%
\pgfpathlineto{\pgfqpoint{4.792381in}{2.338912in}}%
\pgfpathlineto{\pgfqpoint{4.806317in}{2.296609in}}%
\pgfpathlineto{\pgfqpoint{4.821118in}{2.257754in}}%
\pgfpathlineto{\pgfqpoint{4.849826in}{2.176577in}}%
\pgfpathlineto{\pgfqpoint{4.879374in}{2.099656in}}%
\pgfpathlineto{\pgfqpoint{4.893269in}{2.060346in}}%
\pgfpathlineto{\pgfqpoint{4.908024in}{2.023312in}}%
\pgfpathlineto{\pgfqpoint{4.922772in}{1.981779in}}%
\pgfpathlineto{\pgfqpoint{4.966105in}{1.842310in}}%
\pgfpathlineto{\pgfqpoint{5.023203in}{1.625777in}}%
\pgfpathlineto{\pgfqpoint{5.037891in}{1.566713in}}%
\pgfpathlineto{\pgfqpoint{5.051709in}{1.503684in}}%
\pgfpathlineto{\pgfqpoint{5.094846in}{1.325591in}}%
\pgfpathlineto{\pgfqpoint{5.123281in}{1.216081in}}%
\pgfpathlineto{\pgfqpoint{5.137918in}{1.166850in}}%
\pgfpathlineto{\pgfqpoint{5.166310in}{1.079521in}}%
\pgfpathlineto{\pgfqpoint{5.194674in}{1.010634in}}%
\pgfpathlineto{\pgfqpoint{5.209274in}{0.980755in}}%
\pgfpathlineto{\pgfqpoint{5.237595in}{0.929473in}}%
\pgfpathlineto{\pgfqpoint{5.265888in}{0.888868in}}%
\pgfpathlineto{\pgfqpoint{5.308702in}{0.840654in}}%
\pgfpathlineto{\pgfqpoint{5.336924in}{0.815406in}}%
\pgfpathlineto{\pgfqpoint{5.365118in}{0.795940in}}%
\pgfpathlineto{\pgfqpoint{5.393284in}{0.778598in}}%
\pgfpathlineto{\pgfqpoint{5.464003in}{0.749191in}}%
\pgfpathlineto{\pgfqpoint{5.478466in}{0.745391in}}%
\pgfpathlineto{\pgfqpoint{5.492071in}{0.740390in}}%
\pgfpathlineto{\pgfqpoint{5.534545in}{0.729745in}}%
\pgfpathlineto{\pgfqpoint{5.534545in}{0.729745in}}%
\pgfusepath{stroke}%
\end{pgfscope}%
\begin{pgfscope}%
\pgfpathrectangle{\pgfqpoint{0.800000in}{0.528000in}}{\pgfqpoint{4.960000in}{3.696000in}}%
\pgfusepath{clip}%
\pgfsetrectcap%
\pgfsetroundjoin%
\pgfsetlinewidth{1.505625pt}%
\definecolor{currentstroke}{rgb}{1.000000,0.498039,0.054902}%
\pgfsetstrokecolor{currentstroke}%
\pgfsetdash{}{0pt}%
\pgfpathmoveto{\pgfqpoint{1.025455in}{0.707756in}}%
\pgfpathlineto{\pgfqpoint{1.042271in}{0.706640in}}%
\pgfpathlineto{\pgfqpoint{1.076865in}{0.707058in}}%
\pgfpathlineto{\pgfqpoint{1.093654in}{0.705785in}}%
\pgfpathlineto{\pgfqpoint{1.111420in}{0.705697in}}%
\pgfpathlineto{\pgfqpoint{1.128190in}{0.704234in}}%
\pgfpathlineto{\pgfqpoint{1.144950in}{0.705529in}}%
\pgfpathlineto{\pgfqpoint{1.179428in}{0.703663in}}%
\pgfpathlineto{\pgfqpoint{1.213867in}{0.703635in}}%
\pgfpathlineto{\pgfqpoint{1.230580in}{0.702772in}}%
\pgfpathlineto{\pgfqpoint{1.248266in}{0.703130in}}%
\pgfpathlineto{\pgfqpoint{1.264961in}{0.701782in}}%
\pgfpathlineto{\pgfqpoint{1.299303in}{0.701923in}}%
\pgfpathlineto{\pgfqpoint{1.401120in}{0.698021in}}%
\pgfpathlineto{\pgfqpoint{1.468484in}{0.698329in}}%
\pgfpathlineto{\pgfqpoint{1.519152in}{0.696798in}}%
\pgfpathlineto{\pgfqpoint{1.536672in}{0.697751in}}%
\pgfpathlineto{\pgfqpoint{1.553208in}{0.696000in}}%
\pgfpathlineto{\pgfqpoint{1.569736in}{0.697312in}}%
\pgfpathlineto{\pgfqpoint{1.603735in}{0.697368in}}%
\pgfpathlineto{\pgfqpoint{1.688082in}{0.698484in}}%
\pgfpathlineto{\pgfqpoint{1.737418in}{0.698306in}}%
\pgfpathlineto{\pgfqpoint{1.771229in}{0.701056in}}%
\pgfpathlineto{\pgfqpoint{1.787638in}{0.700504in}}%
\pgfpathlineto{\pgfqpoint{1.904174in}{0.710378in}}%
\pgfpathlineto{\pgfqpoint{1.920511in}{0.710530in}}%
\pgfpathlineto{\pgfqpoint{1.954117in}{0.714595in}}%
\pgfpathlineto{\pgfqpoint{2.020261in}{0.723285in}}%
\pgfpathlineto{\pgfqpoint{2.052800in}{0.727241in}}%
\pgfpathlineto{\pgfqpoint{2.070012in}{0.731123in}}%
\pgfpathlineto{\pgfqpoint{2.134947in}{0.740814in}}%
\pgfpathlineto{\pgfqpoint{2.152112in}{0.745666in}}%
\pgfpathlineto{\pgfqpoint{2.168314in}{0.747852in}}%
\pgfpathlineto{\pgfqpoint{2.266296in}{0.770604in}}%
\pgfpathlineto{\pgfqpoint{2.282437in}{0.775856in}}%
\pgfpathlineto{\pgfqpoint{2.315641in}{0.784958in}}%
\pgfpathlineto{\pgfqpoint{2.396126in}{0.815869in}}%
\pgfpathlineto{\pgfqpoint{2.413142in}{0.824534in}}%
\pgfpathlineto{\pgfqpoint{2.429203in}{0.830981in}}%
\pgfpathlineto{\pgfqpoint{2.493364in}{0.863916in}}%
\pgfpathlineto{\pgfqpoint{2.526335in}{0.882995in}}%
\pgfpathlineto{\pgfqpoint{2.574312in}{0.914996in}}%
\pgfpathlineto{\pgfqpoint{2.622212in}{0.949431in}}%
\pgfpathlineto{\pgfqpoint{2.670972in}{0.990823in}}%
\pgfpathlineto{\pgfqpoint{2.734615in}{1.054041in}}%
\pgfpathlineto{\pgfqpoint{2.750504in}{1.071988in}}%
\pgfpathlineto{\pgfqpoint{2.782257in}{1.111005in}}%
\pgfpathlineto{\pgfqpoint{2.830756in}{1.171300in}}%
\pgfpathlineto{\pgfqpoint{2.878244in}{1.240798in}}%
\pgfpathlineto{\pgfqpoint{2.909861in}{1.292543in}}%
\pgfpathlineto{\pgfqpoint{2.941444in}{1.346908in}}%
\pgfpathlineto{\pgfqpoint{2.972994in}{1.409967in}}%
\pgfpathlineto{\pgfqpoint{2.988756in}{1.440650in}}%
\pgfpathlineto{\pgfqpoint{3.035992in}{1.546134in}}%
\pgfpathlineto{\pgfqpoint{3.051720in}{1.582081in}}%
\pgfpathlineto{\pgfqpoint{3.067441in}{1.622906in}}%
\pgfpathlineto{\pgfqpoint{3.083152in}{1.660458in}}%
\pgfpathlineto{\pgfqpoint{3.145916in}{1.827637in}}%
\pgfpathlineto{\pgfqpoint{3.161586in}{1.872241in}}%
\pgfpathlineto{\pgfqpoint{3.178169in}{1.915546in}}%
\pgfpathlineto{\pgfqpoint{3.240731in}{2.097210in}}%
\pgfpathlineto{\pgfqpoint{3.256351in}{2.139488in}}%
\pgfpathlineto{\pgfqpoint{3.286648in}{2.224959in}}%
\pgfpathlineto{\pgfqpoint{3.333410in}{2.349088in}}%
\pgfpathlineto{\pgfqpoint{3.348980in}{2.385949in}}%
\pgfpathlineto{\pgfqpoint{3.380097in}{2.452344in}}%
\pgfpathlineto{\pgfqpoint{3.426710in}{2.541939in}}%
\pgfpathlineto{\pgfqpoint{3.442231in}{2.567313in}}%
\pgfpathlineto{\pgfqpoint{3.457744in}{2.589849in}}%
\pgfpathlineto{\pgfqpoint{3.473249in}{2.615254in}}%
\pgfpathlineto{\pgfqpoint{3.504234in}{2.648882in}}%
\pgfpathlineto{\pgfqpoint{3.535187in}{2.679929in}}%
\pgfpathlineto{\pgfqpoint{3.566107in}{2.704555in}}%
\pgfpathlineto{\pgfqpoint{3.581554in}{2.719322in}}%
\pgfpathlineto{\pgfqpoint{3.612425in}{2.739462in}}%
\pgfpathlineto{\pgfqpoint{3.626942in}{2.751469in}}%
\pgfpathlineto{\pgfqpoint{3.642357in}{2.760461in}}%
\pgfpathlineto{\pgfqpoint{3.657765in}{2.771654in}}%
\pgfpathlineto{\pgfqpoint{3.703938in}{2.817858in}}%
\pgfpathlineto{\pgfqpoint{3.734680in}{2.857003in}}%
\pgfpathlineto{\pgfqpoint{3.765390in}{2.907027in}}%
\pgfpathlineto{\pgfqpoint{3.810493in}{2.998108in}}%
\pgfpathlineto{\pgfqpoint{3.825812in}{3.031154in}}%
\pgfpathlineto{\pgfqpoint{3.887007in}{3.182449in}}%
\pgfpathlineto{\pgfqpoint{3.931922in}{3.318055in}}%
\pgfpathlineto{\pgfqpoint{3.947178in}{3.360270in}}%
\pgfpathlineto{\pgfqpoint{4.052848in}{3.685990in}}%
\pgfpathlineto{\pgfqpoint{4.083224in}{3.767587in}}%
\pgfpathlineto{\pgfqpoint{4.127836in}{3.866295in}}%
\pgfpathlineto{\pgfqpoint{4.142989in}{3.895362in}}%
\pgfpathlineto{\pgfqpoint{4.158134in}{3.917808in}}%
\pgfpathlineto{\pgfqpoint{4.172381in}{3.943019in}}%
\pgfpathlineto{\pgfqpoint{4.187510in}{3.961927in}}%
\pgfpathlineto{\pgfqpoint{4.202632in}{3.983469in}}%
\pgfpathlineto{\pgfqpoint{4.217746in}{4.001153in}}%
\pgfpathlineto{\pgfqpoint{4.232852in}{4.014776in}}%
\pgfpathlineto{\pgfqpoint{4.247062in}{4.029836in}}%
\pgfpathlineto{\pgfqpoint{4.262152in}{4.038864in}}%
\pgfpathlineto{\pgfqpoint{4.277235in}{4.051102in}}%
\pgfpathlineto{\pgfqpoint{4.307377in}{4.056000in}}%
\pgfpathlineto{\pgfqpoint{4.321550in}{4.055355in}}%
\pgfpathlineto{\pgfqpoint{4.336602in}{4.050709in}}%
\pgfpathlineto{\pgfqpoint{4.366682in}{4.037131in}}%
\pgfpathlineto{\pgfqpoint{4.395847in}{4.006646in}}%
\pgfpathlineto{\pgfqpoint{4.410861in}{3.993823in}}%
\pgfpathlineto{\pgfqpoint{4.425866in}{3.975367in}}%
\pgfpathlineto{\pgfqpoint{4.454971in}{3.930729in}}%
\pgfpathlineto{\pgfqpoint{4.484928in}{3.873481in}}%
\pgfpathlineto{\pgfqpoint{4.513975in}{3.807548in}}%
\pgfpathlineto{\pgfqpoint{4.557928in}{3.687346in}}%
\pgfpathlineto{\pgfqpoint{4.587778in}{3.599868in}}%
\pgfpathlineto{\pgfqpoint{4.631619in}{3.467575in}}%
\pgfpathlineto{\pgfqpoint{4.690262in}{3.273792in}}%
\pgfpathlineto{\pgfqpoint{4.719101in}{3.172029in}}%
\pgfpathlineto{\pgfqpoint{4.733947in}{3.122623in}}%
\pgfpathlineto{\pgfqpoint{4.879374in}{2.578462in}}%
\pgfpathlineto{\pgfqpoint{4.908024in}{2.468124in}}%
\pgfpathlineto{\pgfqpoint{4.922772in}{2.409720in}}%
\pgfpathlineto{\pgfqpoint{4.951379in}{2.284517in}}%
\pgfpathlineto{\pgfqpoint{4.994668in}{2.071224in}}%
\pgfpathlineto{\pgfqpoint{5.008507in}{1.995276in}}%
\pgfpathlineto{\pgfqpoint{5.023203in}{1.921495in}}%
\pgfpathlineto{\pgfqpoint{5.066383in}{1.686707in}}%
\pgfpathlineto{\pgfqpoint{5.081049in}{1.613895in}}%
\pgfpathlineto{\pgfqpoint{5.094846in}{1.537318in}}%
\pgfpathlineto{\pgfqpoint{5.123281in}{1.399833in}}%
\pgfpathlineto{\pgfqpoint{5.137918in}{1.334874in}}%
\pgfpathlineto{\pgfqpoint{5.166310in}{1.223819in}}%
\pgfpathlineto{\pgfqpoint{5.194674in}{1.130329in}}%
\pgfpathlineto{\pgfqpoint{5.223868in}{1.053735in}}%
\pgfpathlineto{\pgfqpoint{5.237595in}{1.019721in}}%
\pgfpathlineto{\pgfqpoint{5.265888in}{0.961900in}}%
\pgfpathlineto{\pgfqpoint{5.294153in}{0.915431in}}%
\pgfpathlineto{\pgfqpoint{5.308702in}{0.894171in}}%
\pgfpathlineto{\pgfqpoint{5.336924in}{0.860005in}}%
\pgfpathlineto{\pgfqpoint{5.365118in}{0.832187in}}%
\pgfpathlineto{\pgfqpoint{5.393284in}{0.809181in}}%
\pgfpathlineto{\pgfqpoint{5.421423in}{0.790269in}}%
\pgfpathlineto{\pgfqpoint{5.464003in}{0.767033in}}%
\pgfpathlineto{\pgfqpoint{5.534545in}{0.741295in}}%
\pgfpathlineto{\pgfqpoint{5.534545in}{0.741295in}}%
\pgfusepath{stroke}%
\end{pgfscope}%
\begin{pgfscope}%
\pgfpathrectangle{\pgfqpoint{0.800000in}{0.528000in}}{\pgfqpoint{4.960000in}{3.696000in}}%
\pgfusepath{clip}%
\pgfsetbuttcap%
\pgfsetroundjoin%
\pgfsetlinewidth{1.505625pt}%
\definecolor{currentstroke}{rgb}{0.501961,0.000000,0.501961}%
\pgfsetstrokecolor{currentstroke}%
\pgfsetdash{{5.550000pt}{2.400000pt}}{0.000000pt}%
\pgfpathmoveto{\pgfqpoint{3.994324in}{0.528000in}}%
\pgfpathlineto{\pgfqpoint{3.994324in}{4.224000in}}%
\pgfusepath{stroke}%
\end{pgfscope}%
\begin{pgfscope}%
\pgfsetrectcap%
\pgfsetmiterjoin%
\pgfsetlinewidth{0.803000pt}%
\definecolor{currentstroke}{rgb}{0.000000,0.000000,0.000000}%
\pgfsetstrokecolor{currentstroke}%
\pgfsetdash{}{0pt}%
\pgfpathmoveto{\pgfqpoint{0.800000in}{0.528000in}}%
\pgfpathlineto{\pgfqpoint{0.800000in}{4.224000in}}%
\pgfusepath{stroke}%
\end{pgfscope}%
\begin{pgfscope}%
\pgfsetrectcap%
\pgfsetmiterjoin%
\pgfsetlinewidth{0.803000pt}%
\definecolor{currentstroke}{rgb}{0.000000,0.000000,0.000000}%
\pgfsetstrokecolor{currentstroke}%
\pgfsetdash{}{0pt}%
\pgfpathmoveto{\pgfqpoint{5.760000in}{0.528000in}}%
\pgfpathlineto{\pgfqpoint{5.760000in}{4.224000in}}%
\pgfusepath{stroke}%
\end{pgfscope}%
\begin{pgfscope}%
\pgfsetrectcap%
\pgfsetmiterjoin%
\pgfsetlinewidth{0.803000pt}%
\definecolor{currentstroke}{rgb}{0.000000,0.000000,0.000000}%
\pgfsetstrokecolor{currentstroke}%
\pgfsetdash{}{0pt}%
\pgfpathmoveto{\pgfqpoint{0.800000in}{0.528000in}}%
\pgfpathlineto{\pgfqpoint{5.760000in}{0.528000in}}%
\pgfusepath{stroke}%
\end{pgfscope}%
\begin{pgfscope}%
\pgfsetrectcap%
\pgfsetmiterjoin%
\pgfsetlinewidth{0.803000pt}%
\definecolor{currentstroke}{rgb}{0.000000,0.000000,0.000000}%
\pgfsetstrokecolor{currentstroke}%
\pgfsetdash{}{0pt}%
\pgfpathmoveto{\pgfqpoint{0.800000in}{4.224000in}}%
\pgfpathlineto{\pgfqpoint{5.760000in}{4.224000in}}%
\pgfusepath{stroke}%
\end{pgfscope}%
\begin{pgfscope}%
\pgfsetbuttcap%
\pgfsetmiterjoin%
\definecolor{currentfill}{rgb}{1.000000,1.000000,1.000000}%
\pgfsetfillcolor{currentfill}%
\pgfsetfillopacity{0.800000}%
\pgfsetlinewidth{1.003750pt}%
\definecolor{currentstroke}{rgb}{0.800000,0.800000,0.800000}%
\pgfsetstrokecolor{currentstroke}%
\pgfsetstrokeopacity{0.800000}%
\pgfsetdash{}{0pt}%
\pgfpathmoveto{\pgfqpoint{0.916667in}{3.363860in}}%
\pgfpathlineto{\pgfqpoint{2.559619in}{3.363860in}}%
\pgfpathquadraticcurveto{\pgfqpoint{2.592952in}{3.363860in}}{\pgfqpoint{2.592952in}{3.397194in}}%
\pgfpathlineto{\pgfqpoint{2.592952in}{4.107333in}}%
\pgfpathquadraticcurveto{\pgfqpoint{2.592952in}{4.140667in}}{\pgfqpoint{2.559619in}{4.140667in}}%
\pgfpathlineto{\pgfqpoint{0.916667in}{4.140667in}}%
\pgfpathquadraticcurveto{\pgfqpoint{0.883333in}{4.140667in}}{\pgfqpoint{0.883333in}{4.107333in}}%
\pgfpathlineto{\pgfqpoint{0.883333in}{3.397194in}}%
\pgfpathquadraticcurveto{\pgfqpoint{0.883333in}{3.363860in}}{\pgfqpoint{0.916667in}{3.363860in}}%
\pgfpathlineto{\pgfqpoint{0.916667in}{3.363860in}}%
\pgfpathclose%
\pgfusepath{stroke,fill}%
\end{pgfscope}%
\begin{pgfscope}%
\pgfsetrectcap%
\pgfsetroundjoin%
\pgfsetlinewidth{1.505625pt}%
\definecolor{currentstroke}{rgb}{0.121569,0.466667,0.705882}%
\pgfsetstrokecolor{currentstroke}%
\pgfsetdash{}{0pt}%
\pgfpathmoveto{\pgfqpoint{0.950000in}{4.009693in}}%
\pgfpathlineto{\pgfqpoint{1.116667in}{4.009693in}}%
\pgfpathlineto{\pgfqpoint{1.283333in}{4.009693in}}%
\pgfusepath{stroke}%
\end{pgfscope}%
\begin{pgfscope}%
\definecolor{textcolor}{rgb}{0.000000,0.000000,0.000000}%
\pgfsetstrokecolor{textcolor}%
\pgfsetfillcolor{textcolor}%
\pgftext[x=1.416667in,y=3.951360in,left,base]{\color{textcolor}\rmfamily\fontsize{12.000000}{14.400000}\selectfont \(\displaystyle T_\mathrm{K}=293,35~\mathrm{K}\)}%
\end{pgfscope}%
\begin{pgfscope}%
\pgfsetrectcap%
\pgfsetroundjoin%
\pgfsetlinewidth{1.505625pt}%
\definecolor{currentstroke}{rgb}{1.000000,0.498039,0.054902}%
\pgfsetstrokecolor{currentstroke}%
\pgfsetdash{}{0pt}%
\pgfpathmoveto{\pgfqpoint{0.950000in}{3.767425in}}%
\pgfpathlineto{\pgfqpoint{1.116667in}{3.767425in}}%
\pgfpathlineto{\pgfqpoint{1.283333in}{3.767425in}}%
\pgfusepath{stroke}%
\end{pgfscope}%
\begin{pgfscope}%
\definecolor{textcolor}{rgb}{0.000000,0.000000,0.000000}%
\pgfsetstrokecolor{textcolor}%
\pgfsetfillcolor{textcolor}%
\pgftext[x=1.416667in,y=3.709091in,left,base]{\color{textcolor}\rmfamily\fontsize{12.000000}{14.400000}\selectfont \(\displaystyle T_\mathrm{K}=346,65~\mathrm{K}\)}%
\end{pgfscope}%
\begin{pgfscope}%
\pgfsetbuttcap%
\pgfsetroundjoin%
\pgfsetlinewidth{1.505625pt}%
\definecolor{currentstroke}{rgb}{0.501961,0.000000,0.501961}%
\pgfsetstrokecolor{currentstroke}%
\pgfsetdash{{5.550000pt}{2.400000pt}}{0.000000pt}%
\pgfpathmoveto{\pgfqpoint{0.950000in}{3.525156in}}%
\pgfpathlineto{\pgfqpoint{1.116667in}{3.525156in}}%
\pgfpathlineto{\pgfqpoint{1.283333in}{3.525156in}}%
\pgfusepath{stroke}%
\end{pgfscope}%
\begin{pgfscope}%
\definecolor{textcolor}{rgb}{0.000000,0.000000,0.000000}%
\pgfsetstrokecolor{textcolor}%
\pgfsetfillcolor{textcolor}%
\pgftext[x=1.416667in,y=3.466822in,left,base]{\color{textcolor}\rmfamily\fontsize{12.000000}{14.400000}\selectfont isosbestic point}%
\end{pgfscope}%
\end{pgfpicture}%
\makeatother%
\endgroup%

    \caption[Area-normalized Raman shift intensities for the lowest and highest temperatures]{Plot of the area-normalized scattering intensities over the Raman shift $\Delta \tilde{v}$; Plot at the highest recorded temperature and the lowest recorded temperature for simplification purposes}
    \label{fig:plot-temp}
\end{figure}

With the record of a temperature calibration curve, a function with the form
\begin{align}
    T(x)=42.1 \cdot x^3 - 200.5 \cdot x^2 + 412.4 \cdot x -233.7
\end{align}
can be fitted to calculate the temperature of the liquid through putting in the area ratios $x$. The in between results and the final temperature for the highest and the lowest temperature curve are shown in \autoref{tab:temp}. Compared to the measurement with the thermocouple K, there is a mean error of $3.46~\mathrm{K}$, and a mean squared error of $12.83~\mathrm{K}$ over all measurements.

\begin{table}[!htb]
    \centering
    \small
    \caption[Temperature calculation results comparison]{Calculation results for temperature determination through Raman spectroscopy; comparing with the temperature value of the thermocouple K; only for the highest and the lowest temperature due to simplification}
    \label{tab:temp}
    \vspace{12pt}
    \begin{tabular}{|l|r|r|}
        \hline % \cline{2-3}
        \rowcolor{lightgray}                & highest temperature   & lowest temperature \\ \hline \hline % \multicolumn{1}{l|}{\cellcolor{white}}
        area ratio                          & 1.51281               & 1.01273 \\ \hline
        determined $\mathrm{T}$ Raman       & 77.08 $\mathrm{K}$    & 22.04 $\mathrm{K}$ \\ \hline
        measured $\mathrm{T}$ thermocouple  & 73.5 $\mathrm{K}$     & 20.2 $\mathrm{K}$\\ \hline
        error                               & 3.58 $\mathrm{K}$     & 1.84 $\mathrm{K}$ \\ \hline        
    \end{tabular}
\end{table}

\begin{figure}[!htb]
    \centering
    %% Creator: Matplotlib, PGF backend
%%
%% To include the figure in your LaTeX document, write
%%   \input{<filename>.pgf}
%%
%% Make sure the required packages are loaded in your preamble
%%   \usepackage{pgf}
%%
%% Also ensure that all the required font packages are loaded; for instance,
%% the lmodern package is sometimes necessary when using math font.
%%   \usepackage{lmodern}
%%
%% Figures using additional raster images can only be included by \input if
%% they are in the same directory as the main LaTeX file. For loading figures
%% from other directories you can use the `import` package
%%   \usepackage{import}
%%
%% and then include the figures with
%%   \import{<path to file>}{<filename>.pgf}
%%
%% Matplotlib used the following preamble
%%   
%%   \makeatletter\@ifpackageloaded{underscore}{}{\usepackage[strings]{underscore}}\makeatother
%%
\begingroup%
\makeatletter%
\begin{pgfpicture}%
\pgfpathrectangle{\pgfpointorigin}{\pgfqpoint{6.000000in}{4.000000in}}%
\pgfusepath{use as bounding box, clip}%
\begin{pgfscope}%
\pgfsetbuttcap%
\pgfsetmiterjoin%
\definecolor{currentfill}{rgb}{1.000000,1.000000,1.000000}%
\pgfsetfillcolor{currentfill}%
\pgfsetlinewidth{0.000000pt}%
\definecolor{currentstroke}{rgb}{1.000000,1.000000,1.000000}%
\pgfsetstrokecolor{currentstroke}%
\pgfsetdash{}{0pt}%
\pgfpathmoveto{\pgfqpoint{0.000000in}{0.000000in}}%
\pgfpathlineto{\pgfqpoint{6.000000in}{0.000000in}}%
\pgfpathlineto{\pgfqpoint{6.000000in}{4.000000in}}%
\pgfpathlineto{\pgfqpoint{0.000000in}{4.000000in}}%
\pgfpathlineto{\pgfqpoint{0.000000in}{0.000000in}}%
\pgfpathclose%
\pgfusepath{fill}%
\end{pgfscope}%
\begin{pgfscope}%
\pgfsetbuttcap%
\pgfsetmiterjoin%
\definecolor{currentfill}{rgb}{1.000000,1.000000,1.000000}%
\pgfsetfillcolor{currentfill}%
\pgfsetlinewidth{0.000000pt}%
\definecolor{currentstroke}{rgb}{0.000000,0.000000,0.000000}%
\pgfsetstrokecolor{currentstroke}%
\pgfsetstrokeopacity{0.000000}%
\pgfsetdash{}{0pt}%
\pgfpathmoveto{\pgfqpoint{0.750000in}{0.500000in}}%
\pgfpathlineto{\pgfqpoint{5.400000in}{0.500000in}}%
\pgfpathlineto{\pgfqpoint{5.400000in}{3.520000in}}%
\pgfpathlineto{\pgfqpoint{0.750000in}{3.520000in}}%
\pgfpathlineto{\pgfqpoint{0.750000in}{0.500000in}}%
\pgfpathclose%
\pgfusepath{fill}%
\end{pgfscope}%
\begin{pgfscope}%
\pgfpathrectangle{\pgfqpoint{0.750000in}{0.500000in}}{\pgfqpoint{4.650000in}{3.020000in}}%
\pgfusepath{clip}%
\pgfsetrectcap%
\pgfsetroundjoin%
\pgfsetlinewidth{0.803000pt}%
\definecolor{currentstroke}{rgb}{0.690196,0.690196,0.690196}%
\pgfsetstrokecolor{currentstroke}%
\pgfsetdash{}{0pt}%
\pgfpathmoveto{\pgfqpoint{0.961364in}{0.500000in}}%
\pgfpathlineto{\pgfqpoint{0.961364in}{3.520000in}}%
\pgfusepath{stroke}%
\end{pgfscope}%
\begin{pgfscope}%
\pgfsetbuttcap%
\pgfsetroundjoin%
\definecolor{currentfill}{rgb}{0.000000,0.000000,0.000000}%
\pgfsetfillcolor{currentfill}%
\pgfsetlinewidth{0.803000pt}%
\definecolor{currentstroke}{rgb}{0.000000,0.000000,0.000000}%
\pgfsetstrokecolor{currentstroke}%
\pgfsetdash{}{0pt}%
\pgfsys@defobject{currentmarker}{\pgfqpoint{0.000000in}{-0.048611in}}{\pgfqpoint{0.000000in}{0.000000in}}{%
\pgfpathmoveto{\pgfqpoint{0.000000in}{0.000000in}}%
\pgfpathlineto{\pgfqpoint{0.000000in}{-0.048611in}}%
\pgfusepath{stroke,fill}%
}%
\begin{pgfscope}%
\pgfsys@transformshift{0.961364in}{0.500000in}%
\pgfsys@useobject{currentmarker}{}%
\end{pgfscope}%
\end{pgfscope}%
\begin{pgfscope}%
\definecolor{textcolor}{rgb}{0.000000,0.000000,0.000000}%
\pgfsetstrokecolor{textcolor}%
\pgfsetfillcolor{textcolor}%
\pgftext[x=0.961364in,y=0.402778in,,top]{\color{textcolor}\rmfamily\fontsize{12.000000}{14.400000}\selectfont \(\displaystyle {0}\)}%
\end{pgfscope}%
\begin{pgfscope}%
\pgfpathrectangle{\pgfqpoint{0.750000in}{0.500000in}}{\pgfqpoint{4.650000in}{3.020000in}}%
\pgfusepath{clip}%
\pgfsetrectcap%
\pgfsetroundjoin%
\pgfsetlinewidth{0.803000pt}%
\definecolor{currentstroke}{rgb}{0.690196,0.690196,0.690196}%
\pgfsetstrokecolor{currentstroke}%
\pgfsetdash{}{0pt}%
\pgfpathmoveto{\pgfqpoint{1.637727in}{0.500000in}}%
\pgfpathlineto{\pgfqpoint{1.637727in}{3.520000in}}%
\pgfusepath{stroke}%
\end{pgfscope}%
\begin{pgfscope}%
\pgfsetbuttcap%
\pgfsetroundjoin%
\definecolor{currentfill}{rgb}{0.000000,0.000000,0.000000}%
\pgfsetfillcolor{currentfill}%
\pgfsetlinewidth{0.803000pt}%
\definecolor{currentstroke}{rgb}{0.000000,0.000000,0.000000}%
\pgfsetstrokecolor{currentstroke}%
\pgfsetdash{}{0pt}%
\pgfsys@defobject{currentmarker}{\pgfqpoint{0.000000in}{-0.048611in}}{\pgfqpoint{0.000000in}{0.000000in}}{%
\pgfpathmoveto{\pgfqpoint{0.000000in}{0.000000in}}%
\pgfpathlineto{\pgfqpoint{0.000000in}{-0.048611in}}%
\pgfusepath{stroke,fill}%
}%
\begin{pgfscope}%
\pgfsys@transformshift{1.637727in}{0.500000in}%
\pgfsys@useobject{currentmarker}{}%
\end{pgfscope}%
\end{pgfscope}%
\begin{pgfscope}%
\definecolor{textcolor}{rgb}{0.000000,0.000000,0.000000}%
\pgfsetstrokecolor{textcolor}%
\pgfsetfillcolor{textcolor}%
\pgftext[x=1.637727in,y=0.402778in,,top]{\color{textcolor}\rmfamily\fontsize{12.000000}{14.400000}\selectfont \(\displaystyle {200}\)}%
\end{pgfscope}%
\begin{pgfscope}%
\pgfpathrectangle{\pgfqpoint{0.750000in}{0.500000in}}{\pgfqpoint{4.650000in}{3.020000in}}%
\pgfusepath{clip}%
\pgfsetrectcap%
\pgfsetroundjoin%
\pgfsetlinewidth{0.803000pt}%
\definecolor{currentstroke}{rgb}{0.690196,0.690196,0.690196}%
\pgfsetstrokecolor{currentstroke}%
\pgfsetdash{}{0pt}%
\pgfpathmoveto{\pgfqpoint{2.314091in}{0.500000in}}%
\pgfpathlineto{\pgfqpoint{2.314091in}{3.520000in}}%
\pgfusepath{stroke}%
\end{pgfscope}%
\begin{pgfscope}%
\pgfsetbuttcap%
\pgfsetroundjoin%
\definecolor{currentfill}{rgb}{0.000000,0.000000,0.000000}%
\pgfsetfillcolor{currentfill}%
\pgfsetlinewidth{0.803000pt}%
\definecolor{currentstroke}{rgb}{0.000000,0.000000,0.000000}%
\pgfsetstrokecolor{currentstroke}%
\pgfsetdash{}{0pt}%
\pgfsys@defobject{currentmarker}{\pgfqpoint{0.000000in}{-0.048611in}}{\pgfqpoint{0.000000in}{0.000000in}}{%
\pgfpathmoveto{\pgfqpoint{0.000000in}{0.000000in}}%
\pgfpathlineto{\pgfqpoint{0.000000in}{-0.048611in}}%
\pgfusepath{stroke,fill}%
}%
\begin{pgfscope}%
\pgfsys@transformshift{2.314091in}{0.500000in}%
\pgfsys@useobject{currentmarker}{}%
\end{pgfscope}%
\end{pgfscope}%
\begin{pgfscope}%
\definecolor{textcolor}{rgb}{0.000000,0.000000,0.000000}%
\pgfsetstrokecolor{textcolor}%
\pgfsetfillcolor{textcolor}%
\pgftext[x=2.314091in,y=0.402778in,,top]{\color{textcolor}\rmfamily\fontsize{12.000000}{14.400000}\selectfont \(\displaystyle {400}\)}%
\end{pgfscope}%
\begin{pgfscope}%
\pgfpathrectangle{\pgfqpoint{0.750000in}{0.500000in}}{\pgfqpoint{4.650000in}{3.020000in}}%
\pgfusepath{clip}%
\pgfsetrectcap%
\pgfsetroundjoin%
\pgfsetlinewidth{0.803000pt}%
\definecolor{currentstroke}{rgb}{0.690196,0.690196,0.690196}%
\pgfsetstrokecolor{currentstroke}%
\pgfsetdash{}{0pt}%
\pgfpathmoveto{\pgfqpoint{2.990455in}{0.500000in}}%
\pgfpathlineto{\pgfqpoint{2.990455in}{3.520000in}}%
\pgfusepath{stroke}%
\end{pgfscope}%
\begin{pgfscope}%
\pgfsetbuttcap%
\pgfsetroundjoin%
\definecolor{currentfill}{rgb}{0.000000,0.000000,0.000000}%
\pgfsetfillcolor{currentfill}%
\pgfsetlinewidth{0.803000pt}%
\definecolor{currentstroke}{rgb}{0.000000,0.000000,0.000000}%
\pgfsetstrokecolor{currentstroke}%
\pgfsetdash{}{0pt}%
\pgfsys@defobject{currentmarker}{\pgfqpoint{0.000000in}{-0.048611in}}{\pgfqpoint{0.000000in}{0.000000in}}{%
\pgfpathmoveto{\pgfqpoint{0.000000in}{0.000000in}}%
\pgfpathlineto{\pgfqpoint{0.000000in}{-0.048611in}}%
\pgfusepath{stroke,fill}%
}%
\begin{pgfscope}%
\pgfsys@transformshift{2.990455in}{0.500000in}%
\pgfsys@useobject{currentmarker}{}%
\end{pgfscope}%
\end{pgfscope}%
\begin{pgfscope}%
\definecolor{textcolor}{rgb}{0.000000,0.000000,0.000000}%
\pgfsetstrokecolor{textcolor}%
\pgfsetfillcolor{textcolor}%
\pgftext[x=2.990455in,y=0.402778in,,top]{\color{textcolor}\rmfamily\fontsize{12.000000}{14.400000}\selectfont \(\displaystyle {600}\)}%
\end{pgfscope}%
\begin{pgfscope}%
\pgfpathrectangle{\pgfqpoint{0.750000in}{0.500000in}}{\pgfqpoint{4.650000in}{3.020000in}}%
\pgfusepath{clip}%
\pgfsetrectcap%
\pgfsetroundjoin%
\pgfsetlinewidth{0.803000pt}%
\definecolor{currentstroke}{rgb}{0.690196,0.690196,0.690196}%
\pgfsetstrokecolor{currentstroke}%
\pgfsetdash{}{0pt}%
\pgfpathmoveto{\pgfqpoint{3.666818in}{0.500000in}}%
\pgfpathlineto{\pgfqpoint{3.666818in}{3.520000in}}%
\pgfusepath{stroke}%
\end{pgfscope}%
\begin{pgfscope}%
\pgfsetbuttcap%
\pgfsetroundjoin%
\definecolor{currentfill}{rgb}{0.000000,0.000000,0.000000}%
\pgfsetfillcolor{currentfill}%
\pgfsetlinewidth{0.803000pt}%
\definecolor{currentstroke}{rgb}{0.000000,0.000000,0.000000}%
\pgfsetstrokecolor{currentstroke}%
\pgfsetdash{}{0pt}%
\pgfsys@defobject{currentmarker}{\pgfqpoint{0.000000in}{-0.048611in}}{\pgfqpoint{0.000000in}{0.000000in}}{%
\pgfpathmoveto{\pgfqpoint{0.000000in}{0.000000in}}%
\pgfpathlineto{\pgfqpoint{0.000000in}{-0.048611in}}%
\pgfusepath{stroke,fill}%
}%
\begin{pgfscope}%
\pgfsys@transformshift{3.666818in}{0.500000in}%
\pgfsys@useobject{currentmarker}{}%
\end{pgfscope}%
\end{pgfscope}%
\begin{pgfscope}%
\definecolor{textcolor}{rgb}{0.000000,0.000000,0.000000}%
\pgfsetstrokecolor{textcolor}%
\pgfsetfillcolor{textcolor}%
\pgftext[x=3.666818in,y=0.402778in,,top]{\color{textcolor}\rmfamily\fontsize{12.000000}{14.400000}\selectfont \(\displaystyle {800}\)}%
\end{pgfscope}%
\begin{pgfscope}%
\pgfpathrectangle{\pgfqpoint{0.750000in}{0.500000in}}{\pgfqpoint{4.650000in}{3.020000in}}%
\pgfusepath{clip}%
\pgfsetrectcap%
\pgfsetroundjoin%
\pgfsetlinewidth{0.803000pt}%
\definecolor{currentstroke}{rgb}{0.690196,0.690196,0.690196}%
\pgfsetstrokecolor{currentstroke}%
\pgfsetdash{}{0pt}%
\pgfpathmoveto{\pgfqpoint{4.343182in}{0.500000in}}%
\pgfpathlineto{\pgfqpoint{4.343182in}{3.520000in}}%
\pgfusepath{stroke}%
\end{pgfscope}%
\begin{pgfscope}%
\pgfsetbuttcap%
\pgfsetroundjoin%
\definecolor{currentfill}{rgb}{0.000000,0.000000,0.000000}%
\pgfsetfillcolor{currentfill}%
\pgfsetlinewidth{0.803000pt}%
\definecolor{currentstroke}{rgb}{0.000000,0.000000,0.000000}%
\pgfsetstrokecolor{currentstroke}%
\pgfsetdash{}{0pt}%
\pgfsys@defobject{currentmarker}{\pgfqpoint{0.000000in}{-0.048611in}}{\pgfqpoint{0.000000in}{0.000000in}}{%
\pgfpathmoveto{\pgfqpoint{0.000000in}{0.000000in}}%
\pgfpathlineto{\pgfqpoint{0.000000in}{-0.048611in}}%
\pgfusepath{stroke,fill}%
}%
\begin{pgfscope}%
\pgfsys@transformshift{4.343182in}{0.500000in}%
\pgfsys@useobject{currentmarker}{}%
\end{pgfscope}%
\end{pgfscope}%
\begin{pgfscope}%
\definecolor{textcolor}{rgb}{0.000000,0.000000,0.000000}%
\pgfsetstrokecolor{textcolor}%
\pgfsetfillcolor{textcolor}%
\pgftext[x=4.343182in,y=0.402778in,,top]{\color{textcolor}\rmfamily\fontsize{12.000000}{14.400000}\selectfont \(\displaystyle {1000}\)}%
\end{pgfscope}%
\begin{pgfscope}%
\pgfpathrectangle{\pgfqpoint{0.750000in}{0.500000in}}{\pgfqpoint{4.650000in}{3.020000in}}%
\pgfusepath{clip}%
\pgfsetrectcap%
\pgfsetroundjoin%
\pgfsetlinewidth{0.803000pt}%
\definecolor{currentstroke}{rgb}{0.690196,0.690196,0.690196}%
\pgfsetstrokecolor{currentstroke}%
\pgfsetdash{}{0pt}%
\pgfpathmoveto{\pgfqpoint{5.019545in}{0.500000in}}%
\pgfpathlineto{\pgfqpoint{5.019545in}{3.520000in}}%
\pgfusepath{stroke}%
\end{pgfscope}%
\begin{pgfscope}%
\pgfsetbuttcap%
\pgfsetroundjoin%
\definecolor{currentfill}{rgb}{0.000000,0.000000,0.000000}%
\pgfsetfillcolor{currentfill}%
\pgfsetlinewidth{0.803000pt}%
\definecolor{currentstroke}{rgb}{0.000000,0.000000,0.000000}%
\pgfsetstrokecolor{currentstroke}%
\pgfsetdash{}{0pt}%
\pgfsys@defobject{currentmarker}{\pgfqpoint{0.000000in}{-0.048611in}}{\pgfqpoint{0.000000in}{0.000000in}}{%
\pgfpathmoveto{\pgfqpoint{0.000000in}{0.000000in}}%
\pgfpathlineto{\pgfqpoint{0.000000in}{-0.048611in}}%
\pgfusepath{stroke,fill}%
}%
\begin{pgfscope}%
\pgfsys@transformshift{5.019545in}{0.500000in}%
\pgfsys@useobject{currentmarker}{}%
\end{pgfscope}%
\end{pgfscope}%
\begin{pgfscope}%
\definecolor{textcolor}{rgb}{0.000000,0.000000,0.000000}%
\pgfsetstrokecolor{textcolor}%
\pgfsetfillcolor{textcolor}%
\pgftext[x=5.019545in,y=0.402778in,,top]{\color{textcolor}\rmfamily\fontsize{12.000000}{14.400000}\selectfont \(\displaystyle {1200}\)}%
\end{pgfscope}%
\begin{pgfscope}%
\definecolor{textcolor}{rgb}{0.000000,0.000000,0.000000}%
\pgfsetstrokecolor{textcolor}%
\pgfsetfillcolor{textcolor}%
\pgftext[x=3.075000in,y=0.199075in,,top]{\color{textcolor}\rmfamily\fontsize{12.000000}{14.400000}\selectfont Time in \(\displaystyle \mathrm{s}\)}%
\end{pgfscope}%
\begin{pgfscope}%
\pgfpathrectangle{\pgfqpoint{0.750000in}{0.500000in}}{\pgfqpoint{4.650000in}{3.020000in}}%
\pgfusepath{clip}%
\pgfsetrectcap%
\pgfsetroundjoin%
\pgfsetlinewidth{0.803000pt}%
\definecolor{currentstroke}{rgb}{0.690196,0.690196,0.690196}%
\pgfsetstrokecolor{currentstroke}%
\pgfsetdash{}{0pt}%
\pgfpathmoveto{\pgfqpoint{0.750000in}{0.632473in}}%
\pgfpathlineto{\pgfqpoint{5.400000in}{0.632473in}}%
\pgfusepath{stroke}%
\end{pgfscope}%
\begin{pgfscope}%
\pgfsetbuttcap%
\pgfsetroundjoin%
\definecolor{currentfill}{rgb}{0.000000,0.000000,0.000000}%
\pgfsetfillcolor{currentfill}%
\pgfsetlinewidth{0.803000pt}%
\definecolor{currentstroke}{rgb}{0.000000,0.000000,0.000000}%
\pgfsetstrokecolor{currentstroke}%
\pgfsetdash{}{0pt}%
\pgfsys@defobject{currentmarker}{\pgfqpoint{-0.048611in}{0.000000in}}{\pgfqpoint{-0.000000in}{0.000000in}}{%
\pgfpathmoveto{\pgfqpoint{-0.000000in}{0.000000in}}%
\pgfpathlineto{\pgfqpoint{-0.048611in}{0.000000in}}%
\pgfusepath{stroke,fill}%
}%
\begin{pgfscope}%
\pgfsys@transformshift{0.750000in}{0.632473in}%
\pgfsys@useobject{currentmarker}{}%
\end{pgfscope}%
\end{pgfscope}%
\begin{pgfscope}%
\definecolor{textcolor}{rgb}{0.000000,0.000000,0.000000}%
\pgfsetstrokecolor{textcolor}%
\pgfsetfillcolor{textcolor}%
\pgftext[x=0.489585in, y=0.574603in, left, base]{\color{textcolor}\rmfamily\fontsize{12.000000}{14.400000}\selectfont \(\displaystyle {20}\)}%
\end{pgfscope}%
\begin{pgfscope}%
\pgfpathrectangle{\pgfqpoint{0.750000in}{0.500000in}}{\pgfqpoint{4.650000in}{3.020000in}}%
\pgfusepath{clip}%
\pgfsetrectcap%
\pgfsetroundjoin%
\pgfsetlinewidth{0.803000pt}%
\definecolor{currentstroke}{rgb}{0.690196,0.690196,0.690196}%
\pgfsetstrokecolor{currentstroke}%
\pgfsetdash{}{0pt}%
\pgfpathmoveto{\pgfqpoint{0.750000in}{1.112448in}}%
\pgfpathlineto{\pgfqpoint{5.400000in}{1.112448in}}%
\pgfusepath{stroke}%
\end{pgfscope}%
\begin{pgfscope}%
\pgfsetbuttcap%
\pgfsetroundjoin%
\definecolor{currentfill}{rgb}{0.000000,0.000000,0.000000}%
\pgfsetfillcolor{currentfill}%
\pgfsetlinewidth{0.803000pt}%
\definecolor{currentstroke}{rgb}{0.000000,0.000000,0.000000}%
\pgfsetstrokecolor{currentstroke}%
\pgfsetdash{}{0pt}%
\pgfsys@defobject{currentmarker}{\pgfqpoint{-0.048611in}{0.000000in}}{\pgfqpoint{-0.000000in}{0.000000in}}{%
\pgfpathmoveto{\pgfqpoint{-0.000000in}{0.000000in}}%
\pgfpathlineto{\pgfqpoint{-0.048611in}{0.000000in}}%
\pgfusepath{stroke,fill}%
}%
\begin{pgfscope}%
\pgfsys@transformshift{0.750000in}{1.112448in}%
\pgfsys@useobject{currentmarker}{}%
\end{pgfscope}%
\end{pgfscope}%
\begin{pgfscope}%
\definecolor{textcolor}{rgb}{0.000000,0.000000,0.000000}%
\pgfsetstrokecolor{textcolor}%
\pgfsetfillcolor{textcolor}%
\pgftext[x=0.489585in, y=1.054577in, left, base]{\color{textcolor}\rmfamily\fontsize{12.000000}{14.400000}\selectfont \(\displaystyle {30}\)}%
\end{pgfscope}%
\begin{pgfscope}%
\pgfpathrectangle{\pgfqpoint{0.750000in}{0.500000in}}{\pgfqpoint{4.650000in}{3.020000in}}%
\pgfusepath{clip}%
\pgfsetrectcap%
\pgfsetroundjoin%
\pgfsetlinewidth{0.803000pt}%
\definecolor{currentstroke}{rgb}{0.690196,0.690196,0.690196}%
\pgfsetstrokecolor{currentstroke}%
\pgfsetdash{}{0pt}%
\pgfpathmoveto{\pgfqpoint{0.750000in}{1.592422in}}%
\pgfpathlineto{\pgfqpoint{5.400000in}{1.592422in}}%
\pgfusepath{stroke}%
\end{pgfscope}%
\begin{pgfscope}%
\pgfsetbuttcap%
\pgfsetroundjoin%
\definecolor{currentfill}{rgb}{0.000000,0.000000,0.000000}%
\pgfsetfillcolor{currentfill}%
\pgfsetlinewidth{0.803000pt}%
\definecolor{currentstroke}{rgb}{0.000000,0.000000,0.000000}%
\pgfsetstrokecolor{currentstroke}%
\pgfsetdash{}{0pt}%
\pgfsys@defobject{currentmarker}{\pgfqpoint{-0.048611in}{0.000000in}}{\pgfqpoint{-0.000000in}{0.000000in}}{%
\pgfpathmoveto{\pgfqpoint{-0.000000in}{0.000000in}}%
\pgfpathlineto{\pgfqpoint{-0.048611in}{0.000000in}}%
\pgfusepath{stroke,fill}%
}%
\begin{pgfscope}%
\pgfsys@transformshift{0.750000in}{1.592422in}%
\pgfsys@useobject{currentmarker}{}%
\end{pgfscope}%
\end{pgfscope}%
\begin{pgfscope}%
\definecolor{textcolor}{rgb}{0.000000,0.000000,0.000000}%
\pgfsetstrokecolor{textcolor}%
\pgfsetfillcolor{textcolor}%
\pgftext[x=0.489585in, y=1.534552in, left, base]{\color{textcolor}\rmfamily\fontsize{12.000000}{14.400000}\selectfont \(\displaystyle {40}\)}%
\end{pgfscope}%
\begin{pgfscope}%
\pgfpathrectangle{\pgfqpoint{0.750000in}{0.500000in}}{\pgfqpoint{4.650000in}{3.020000in}}%
\pgfusepath{clip}%
\pgfsetrectcap%
\pgfsetroundjoin%
\pgfsetlinewidth{0.803000pt}%
\definecolor{currentstroke}{rgb}{0.690196,0.690196,0.690196}%
\pgfsetstrokecolor{currentstroke}%
\pgfsetdash{}{0pt}%
\pgfpathmoveto{\pgfqpoint{0.750000in}{2.072397in}}%
\pgfpathlineto{\pgfqpoint{5.400000in}{2.072397in}}%
\pgfusepath{stroke}%
\end{pgfscope}%
\begin{pgfscope}%
\pgfsetbuttcap%
\pgfsetroundjoin%
\definecolor{currentfill}{rgb}{0.000000,0.000000,0.000000}%
\pgfsetfillcolor{currentfill}%
\pgfsetlinewidth{0.803000pt}%
\definecolor{currentstroke}{rgb}{0.000000,0.000000,0.000000}%
\pgfsetstrokecolor{currentstroke}%
\pgfsetdash{}{0pt}%
\pgfsys@defobject{currentmarker}{\pgfqpoint{-0.048611in}{0.000000in}}{\pgfqpoint{-0.000000in}{0.000000in}}{%
\pgfpathmoveto{\pgfqpoint{-0.000000in}{0.000000in}}%
\pgfpathlineto{\pgfqpoint{-0.048611in}{0.000000in}}%
\pgfusepath{stroke,fill}%
}%
\begin{pgfscope}%
\pgfsys@transformshift{0.750000in}{2.072397in}%
\pgfsys@useobject{currentmarker}{}%
\end{pgfscope}%
\end{pgfscope}%
\begin{pgfscope}%
\definecolor{textcolor}{rgb}{0.000000,0.000000,0.000000}%
\pgfsetstrokecolor{textcolor}%
\pgfsetfillcolor{textcolor}%
\pgftext[x=0.489585in, y=2.014526in, left, base]{\color{textcolor}\rmfamily\fontsize{12.000000}{14.400000}\selectfont \(\displaystyle {50}\)}%
\end{pgfscope}%
\begin{pgfscope}%
\pgfpathrectangle{\pgfqpoint{0.750000in}{0.500000in}}{\pgfqpoint{4.650000in}{3.020000in}}%
\pgfusepath{clip}%
\pgfsetrectcap%
\pgfsetroundjoin%
\pgfsetlinewidth{0.803000pt}%
\definecolor{currentstroke}{rgb}{0.690196,0.690196,0.690196}%
\pgfsetstrokecolor{currentstroke}%
\pgfsetdash{}{0pt}%
\pgfpathmoveto{\pgfqpoint{0.750000in}{2.552371in}}%
\pgfpathlineto{\pgfqpoint{5.400000in}{2.552371in}}%
\pgfusepath{stroke}%
\end{pgfscope}%
\begin{pgfscope}%
\pgfsetbuttcap%
\pgfsetroundjoin%
\definecolor{currentfill}{rgb}{0.000000,0.000000,0.000000}%
\pgfsetfillcolor{currentfill}%
\pgfsetlinewidth{0.803000pt}%
\definecolor{currentstroke}{rgb}{0.000000,0.000000,0.000000}%
\pgfsetstrokecolor{currentstroke}%
\pgfsetdash{}{0pt}%
\pgfsys@defobject{currentmarker}{\pgfqpoint{-0.048611in}{0.000000in}}{\pgfqpoint{-0.000000in}{0.000000in}}{%
\pgfpathmoveto{\pgfqpoint{-0.000000in}{0.000000in}}%
\pgfpathlineto{\pgfqpoint{-0.048611in}{0.000000in}}%
\pgfusepath{stroke,fill}%
}%
\begin{pgfscope}%
\pgfsys@transformshift{0.750000in}{2.552371in}%
\pgfsys@useobject{currentmarker}{}%
\end{pgfscope}%
\end{pgfscope}%
\begin{pgfscope}%
\definecolor{textcolor}{rgb}{0.000000,0.000000,0.000000}%
\pgfsetstrokecolor{textcolor}%
\pgfsetfillcolor{textcolor}%
\pgftext[x=0.489585in, y=2.494501in, left, base]{\color{textcolor}\rmfamily\fontsize{12.000000}{14.400000}\selectfont \(\displaystyle {60}\)}%
\end{pgfscope}%
\begin{pgfscope}%
\pgfpathrectangle{\pgfqpoint{0.750000in}{0.500000in}}{\pgfqpoint{4.650000in}{3.020000in}}%
\pgfusepath{clip}%
\pgfsetrectcap%
\pgfsetroundjoin%
\pgfsetlinewidth{0.803000pt}%
\definecolor{currentstroke}{rgb}{0.690196,0.690196,0.690196}%
\pgfsetstrokecolor{currentstroke}%
\pgfsetdash{}{0pt}%
\pgfpathmoveto{\pgfqpoint{0.750000in}{3.032346in}}%
\pgfpathlineto{\pgfqpoint{5.400000in}{3.032346in}}%
\pgfusepath{stroke}%
\end{pgfscope}%
\begin{pgfscope}%
\pgfsetbuttcap%
\pgfsetroundjoin%
\definecolor{currentfill}{rgb}{0.000000,0.000000,0.000000}%
\pgfsetfillcolor{currentfill}%
\pgfsetlinewidth{0.803000pt}%
\definecolor{currentstroke}{rgb}{0.000000,0.000000,0.000000}%
\pgfsetstrokecolor{currentstroke}%
\pgfsetdash{}{0pt}%
\pgfsys@defobject{currentmarker}{\pgfqpoint{-0.048611in}{0.000000in}}{\pgfqpoint{-0.000000in}{0.000000in}}{%
\pgfpathmoveto{\pgfqpoint{-0.000000in}{0.000000in}}%
\pgfpathlineto{\pgfqpoint{-0.048611in}{0.000000in}}%
\pgfusepath{stroke,fill}%
}%
\begin{pgfscope}%
\pgfsys@transformshift{0.750000in}{3.032346in}%
\pgfsys@useobject{currentmarker}{}%
\end{pgfscope}%
\end{pgfscope}%
\begin{pgfscope}%
\definecolor{textcolor}{rgb}{0.000000,0.000000,0.000000}%
\pgfsetstrokecolor{textcolor}%
\pgfsetfillcolor{textcolor}%
\pgftext[x=0.489585in, y=2.974476in, left, base]{\color{textcolor}\rmfamily\fontsize{12.000000}{14.400000}\selectfont \(\displaystyle {70}\)}%
\end{pgfscope}%
\begin{pgfscope}%
\pgfpathrectangle{\pgfqpoint{0.750000in}{0.500000in}}{\pgfqpoint{4.650000in}{3.020000in}}%
\pgfusepath{clip}%
\pgfsetrectcap%
\pgfsetroundjoin%
\pgfsetlinewidth{0.803000pt}%
\definecolor{currentstroke}{rgb}{0.690196,0.690196,0.690196}%
\pgfsetstrokecolor{currentstroke}%
\pgfsetdash{}{0pt}%
\pgfpathmoveto{\pgfqpoint{0.750000in}{3.512320in}}%
\pgfpathlineto{\pgfqpoint{5.400000in}{3.512320in}}%
\pgfusepath{stroke}%
\end{pgfscope}%
\begin{pgfscope}%
\pgfsetbuttcap%
\pgfsetroundjoin%
\definecolor{currentfill}{rgb}{0.000000,0.000000,0.000000}%
\pgfsetfillcolor{currentfill}%
\pgfsetlinewidth{0.803000pt}%
\definecolor{currentstroke}{rgb}{0.000000,0.000000,0.000000}%
\pgfsetstrokecolor{currentstroke}%
\pgfsetdash{}{0pt}%
\pgfsys@defobject{currentmarker}{\pgfqpoint{-0.048611in}{0.000000in}}{\pgfqpoint{-0.000000in}{0.000000in}}{%
\pgfpathmoveto{\pgfqpoint{-0.000000in}{0.000000in}}%
\pgfpathlineto{\pgfqpoint{-0.048611in}{0.000000in}}%
\pgfusepath{stroke,fill}%
}%
\begin{pgfscope}%
\pgfsys@transformshift{0.750000in}{3.512320in}%
\pgfsys@useobject{currentmarker}{}%
\end{pgfscope}%
\end{pgfscope}%
\begin{pgfscope}%
\definecolor{textcolor}{rgb}{0.000000,0.000000,0.000000}%
\pgfsetstrokecolor{textcolor}%
\pgfsetfillcolor{textcolor}%
\pgftext[x=0.489585in, y=3.454450in, left, base]{\color{textcolor}\rmfamily\fontsize{12.000000}{14.400000}\selectfont \(\displaystyle {80}\)}%
\end{pgfscope}%
\begin{pgfscope}%
\definecolor{textcolor}{rgb}{0.000000,0.000000,0.000000}%
\pgfsetstrokecolor{textcolor}%
\pgfsetfillcolor{textcolor}%
\pgftext[x=0.434029in,y=2.010000in,,bottom,rotate=90.000000]{\color{textcolor}\rmfamily\fontsize{12.000000}{14.400000}\selectfont Temperature in \(\displaystyle \mathrm{K}\)}%
\end{pgfscope}%
\begin{pgfscope}%
\pgfpathrectangle{\pgfqpoint{0.750000in}{0.500000in}}{\pgfqpoint{4.650000in}{3.020000in}}%
\pgfusepath{clip}%
\pgfsetrectcap%
\pgfsetroundjoin%
\pgfsetlinewidth{1.505625pt}%
\definecolor{currentstroke}{rgb}{0.121569,0.466667,0.705882}%
\pgfsetstrokecolor{currentstroke}%
\pgfsetdash{}{0pt}%
\pgfpathmoveto{\pgfqpoint{0.961364in}{0.642072in}}%
\pgfpathlineto{\pgfqpoint{0.995182in}{0.642072in}}%
\pgfpathlineto{\pgfqpoint{1.029000in}{0.637273in}}%
\pgfpathlineto{\pgfqpoint{1.062818in}{0.642072in}}%
\pgfpathlineto{\pgfqpoint{1.096636in}{0.651672in}}%
\pgfpathlineto{\pgfqpoint{1.130455in}{0.666071in}}%
\pgfpathlineto{\pgfqpoint{1.164273in}{0.690070in}}%
\pgfpathlineto{\pgfqpoint{1.198091in}{0.718868in}}%
\pgfpathlineto{\pgfqpoint{1.231909in}{0.757266in}}%
\pgfpathlineto{\pgfqpoint{1.265727in}{0.805264in}}%
\pgfpathlineto{\pgfqpoint{1.299545in}{0.848462in}}%
\pgfpathlineto{\pgfqpoint{1.333364in}{0.896459in}}%
\pgfpathlineto{\pgfqpoint{1.367182in}{0.944456in}}%
\pgfpathlineto{\pgfqpoint{1.401000in}{0.997254in}}%
\pgfpathlineto{\pgfqpoint{1.434818in}{1.045251in}}%
\pgfpathlineto{\pgfqpoint{1.468636in}{1.093249in}}%
\pgfpathlineto{\pgfqpoint{1.502455in}{1.136446in}}%
\pgfpathlineto{\pgfqpoint{1.536273in}{1.189243in}}%
\pgfpathlineto{\pgfqpoint{1.570091in}{1.237241in}}%
\pgfpathlineto{\pgfqpoint{1.603909in}{1.285238in}}%
\pgfpathlineto{\pgfqpoint{1.637727in}{1.338036in}}%
\pgfpathlineto{\pgfqpoint{1.671545in}{1.390833in}}%
\pgfpathlineto{\pgfqpoint{1.705364in}{1.438830in}}%
\pgfpathlineto{\pgfqpoint{1.739182in}{1.482028in}}%
\pgfpathlineto{\pgfqpoint{1.773000in}{1.534825in}}%
\pgfpathlineto{\pgfqpoint{1.806818in}{1.582823in}}%
\pgfpathlineto{\pgfqpoint{1.840636in}{1.630820in}}%
\pgfpathlineto{\pgfqpoint{1.874455in}{1.678818in}}%
\pgfpathlineto{\pgfqpoint{1.908273in}{1.726815in}}%
\pgfpathlineto{\pgfqpoint{1.942091in}{1.774812in}}%
\pgfpathlineto{\pgfqpoint{1.975909in}{1.818010in}}%
\pgfpathlineto{\pgfqpoint{2.009727in}{1.866008in}}%
\pgfpathlineto{\pgfqpoint{2.043545in}{1.918805in}}%
\pgfpathlineto{\pgfqpoint{2.077364in}{1.962003in}}%
\pgfpathlineto{\pgfqpoint{2.111182in}{2.010000in}}%
\pgfpathlineto{\pgfqpoint{2.145000in}{2.053198in}}%
\pgfpathlineto{\pgfqpoint{2.178818in}{2.101195in}}%
\pgfpathlineto{\pgfqpoint{2.212636in}{2.139593in}}%
\pgfpathlineto{\pgfqpoint{2.246455in}{2.187591in}}%
\pgfpathlineto{\pgfqpoint{2.280273in}{2.230788in}}%
\pgfpathlineto{\pgfqpoint{2.314091in}{2.273986in}}%
\pgfpathlineto{\pgfqpoint{2.347909in}{2.317184in}}%
\pgfpathlineto{\pgfqpoint{2.381727in}{2.355582in}}%
\pgfpathlineto{\pgfqpoint{2.415545in}{2.398779in}}%
\pgfpathlineto{\pgfqpoint{2.449364in}{2.441977in}}%
\pgfpathlineto{\pgfqpoint{2.483182in}{2.480375in}}%
\pgfpathlineto{\pgfqpoint{2.517000in}{2.518773in}}%
\pgfpathlineto{\pgfqpoint{2.550818in}{2.561971in}}%
\pgfpathlineto{\pgfqpoint{2.584636in}{2.595569in}}%
\pgfpathlineto{\pgfqpoint{2.618455in}{2.633967in}}%
\pgfpathlineto{\pgfqpoint{2.652273in}{2.672365in}}%
\pgfpathlineto{\pgfqpoint{2.686091in}{2.705963in}}%
\pgfpathlineto{\pgfqpoint{2.719909in}{2.739561in}}%
\pgfpathlineto{\pgfqpoint{2.753727in}{2.773160in}}%
\pgfpathlineto{\pgfqpoint{2.787545in}{2.806758in}}%
\pgfpathlineto{\pgfqpoint{2.821364in}{2.840356in}}%
\pgfpathlineto{\pgfqpoint{2.855182in}{2.873954in}}%
\pgfpathlineto{\pgfqpoint{2.889000in}{2.907552in}}%
\pgfpathlineto{\pgfqpoint{2.922818in}{2.936351in}}%
\pgfpathlineto{\pgfqpoint{2.956636in}{2.969949in}}%
\pgfpathlineto{\pgfqpoint{2.990455in}{2.993948in}}%
\pgfpathlineto{\pgfqpoint{3.024273in}{3.027546in}}%
\pgfpathlineto{\pgfqpoint{3.058091in}{3.051545in}}%
\pgfpathlineto{\pgfqpoint{3.091909in}{3.080343in}}%
\pgfpathlineto{\pgfqpoint{3.125727in}{3.104342in}}%
\pgfpathlineto{\pgfqpoint{3.159545in}{3.123541in}}%
\pgfpathlineto{\pgfqpoint{3.193364in}{3.147540in}}%
\pgfpathlineto{\pgfqpoint{3.227182in}{3.161939in}}%
\pgfpathlineto{\pgfqpoint{3.261000in}{3.176338in}}%
\pgfpathlineto{\pgfqpoint{3.294818in}{3.190737in}}%
\pgfpathlineto{\pgfqpoint{3.328636in}{3.200337in}}%
\pgfpathlineto{\pgfqpoint{3.362455in}{2.849955in}}%
\pgfpathlineto{\pgfqpoint{3.396273in}{2.869154in}}%
\pgfpathlineto{\pgfqpoint{3.430091in}{2.883554in}}%
\pgfpathlineto{\pgfqpoint{3.463909in}{2.893153in}}%
\pgfpathlineto{\pgfqpoint{3.497727in}{2.902753in}}%
\pgfpathlineto{\pgfqpoint{3.531545in}{2.912352in}}%
\pgfpathlineto{\pgfqpoint{3.565364in}{2.921952in}}%
\pgfpathlineto{\pgfqpoint{3.599182in}{2.926751in}}%
\pgfpathlineto{\pgfqpoint{3.633000in}{2.931551in}}%
\pgfpathlineto{\pgfqpoint{3.666818in}{2.936351in}}%
\pgfpathlineto{\pgfqpoint{3.700636in}{2.888353in}}%
\pgfpathlineto{\pgfqpoint{3.734455in}{2.720362in}}%
\pgfpathlineto{\pgfqpoint{3.768273in}{2.667565in}}%
\pgfpathlineto{\pgfqpoint{3.802091in}{2.672365in}}%
\pgfpathlineto{\pgfqpoint{3.835909in}{2.681964in}}%
\pgfpathlineto{\pgfqpoint{3.869727in}{2.686764in}}%
\pgfpathlineto{\pgfqpoint{3.903545in}{2.691564in}}%
\pgfpathlineto{\pgfqpoint{3.937364in}{2.691564in}}%
\pgfpathlineto{\pgfqpoint{3.971182in}{2.691564in}}%
\pgfpathlineto{\pgfqpoint{4.005000in}{2.696364in}}%
\pgfpathlineto{\pgfqpoint{4.038818in}{1.870807in}}%
\pgfpathlineto{\pgfqpoint{4.072636in}{1.885207in}}%
\pgfpathlineto{\pgfqpoint{4.106455in}{1.890006in}}%
\pgfpathlineto{\pgfqpoint{4.140273in}{1.899606in}}%
\pgfpathlineto{\pgfqpoint{4.174091in}{1.904406in}}%
\pgfpathlineto{\pgfqpoint{4.207909in}{1.909205in}}%
\pgfpathlineto{\pgfqpoint{4.241727in}{1.918805in}}%
\pgfpathlineto{\pgfqpoint{4.275545in}{1.918805in}}%
\pgfpathlineto{\pgfqpoint{4.309364in}{1.923605in}}%
\pgfpathlineto{\pgfqpoint{4.343182in}{1.928404in}}%
\pgfpathlineto{\pgfqpoint{4.377000in}{1.928404in}}%
\pgfpathlineto{\pgfqpoint{4.410818in}{1.933204in}}%
\pgfpathlineto{\pgfqpoint{4.444636in}{1.938004in}}%
\pgfpathlineto{\pgfqpoint{4.478455in}{1.938004in}}%
\pgfpathlineto{\pgfqpoint{4.512273in}{1.942804in}}%
\pgfpathlineto{\pgfqpoint{4.546091in}{1.942804in}}%
\pgfpathlineto{\pgfqpoint{4.579909in}{1.942804in}}%
\pgfpathlineto{\pgfqpoint{4.613727in}{1.947603in}}%
\pgfpathlineto{\pgfqpoint{4.647545in}{1.947603in}}%
\pgfpathlineto{\pgfqpoint{4.681364in}{1.952403in}}%
\pgfpathlineto{\pgfqpoint{4.715182in}{1.952403in}}%
\pgfpathlineto{\pgfqpoint{4.749000in}{1.952403in}}%
\pgfpathlineto{\pgfqpoint{4.782818in}{1.947603in}}%
\pgfpathlineto{\pgfqpoint{4.816636in}{1.952403in}}%
\pgfpathlineto{\pgfqpoint{4.850455in}{1.952403in}}%
\pgfpathlineto{\pgfqpoint{4.884273in}{1.952403in}}%
\pgfpathlineto{\pgfqpoint{4.918091in}{1.957203in}}%
\pgfpathlineto{\pgfqpoint{4.951909in}{1.952403in}}%
\pgfpathlineto{\pgfqpoint{4.985727in}{1.952403in}}%
\pgfpathlineto{\pgfqpoint{5.019545in}{1.952403in}}%
\pgfpathlineto{\pgfqpoint{5.053364in}{1.952403in}}%
\pgfpathlineto{\pgfqpoint{5.087182in}{1.952403in}}%
\pgfpathlineto{\pgfqpoint{5.121000in}{1.952403in}}%
\pgfpathlineto{\pgfqpoint{5.154818in}{1.952403in}}%
\pgfpathlineto{\pgfqpoint{5.188636in}{1.952403in}}%
\pgfusepath{stroke}%
\end{pgfscope}%
\begin{pgfscope}%
\pgfpathrectangle{\pgfqpoint{0.750000in}{0.500000in}}{\pgfqpoint{4.650000in}{3.020000in}}%
\pgfusepath{clip}%
\pgfsetrectcap%
\pgfsetroundjoin%
\pgfsetlinewidth{1.505625pt}%
\definecolor{currentstroke}{rgb}{1.000000,0.498039,0.054902}%
\pgfsetstrokecolor{currentstroke}%
\pgfsetdash{}{0pt}%
\pgfpathmoveto{\pgfqpoint{0.961364in}{0.728468in}}%
\pgfpathlineto{\pgfqpoint{0.995182in}{0.728468in}}%
\pgfpathlineto{\pgfqpoint{1.029000in}{0.733268in}}%
\pgfpathlineto{\pgfqpoint{1.062818in}{0.738067in}}%
\pgfpathlineto{\pgfqpoint{1.096636in}{0.752467in}}%
\pgfpathlineto{\pgfqpoint{1.130455in}{0.766866in}}%
\pgfpathlineto{\pgfqpoint{1.164273in}{0.795664in}}%
\pgfpathlineto{\pgfqpoint{1.198091in}{0.834062in}}%
\pgfpathlineto{\pgfqpoint{1.231909in}{0.872460in}}%
\pgfpathlineto{\pgfqpoint{1.265727in}{0.925257in}}%
\pgfpathlineto{\pgfqpoint{1.299545in}{0.973255in}}%
\pgfpathlineto{\pgfqpoint{1.333364in}{1.026052in}}%
\pgfpathlineto{\pgfqpoint{1.367182in}{1.078849in}}%
\pgfpathlineto{\pgfqpoint{1.401000in}{1.126847in}}%
\pgfpathlineto{\pgfqpoint{1.434818in}{1.179644in}}%
\pgfpathlineto{\pgfqpoint{1.468636in}{1.232441in}}%
\pgfpathlineto{\pgfqpoint{1.502455in}{1.285238in}}%
\pgfpathlineto{\pgfqpoint{1.536273in}{1.333236in}}%
\pgfpathlineto{\pgfqpoint{1.570091in}{1.386033in}}%
\pgfpathlineto{\pgfqpoint{1.603909in}{1.438830in}}%
\pgfpathlineto{\pgfqpoint{1.637727in}{1.491627in}}%
\pgfpathlineto{\pgfqpoint{1.671545in}{1.544425in}}%
\pgfpathlineto{\pgfqpoint{1.705364in}{1.597222in}}%
\pgfpathlineto{\pgfqpoint{1.739182in}{1.645219in}}%
\pgfpathlineto{\pgfqpoint{1.773000in}{1.698017in}}%
\pgfpathlineto{\pgfqpoint{1.806818in}{1.746014in}}%
\pgfpathlineto{\pgfqpoint{1.840636in}{1.798811in}}%
\pgfpathlineto{\pgfqpoint{1.874455in}{1.842009in}}%
\pgfpathlineto{\pgfqpoint{1.908273in}{1.894806in}}%
\pgfpathlineto{\pgfqpoint{1.942091in}{1.942804in}}%
\pgfpathlineto{\pgfqpoint{1.975909in}{1.990801in}}%
\pgfpathlineto{\pgfqpoint{2.009727in}{2.038798in}}%
\pgfpathlineto{\pgfqpoint{2.043545in}{2.086796in}}%
\pgfpathlineto{\pgfqpoint{2.077364in}{2.134793in}}%
\pgfpathlineto{\pgfqpoint{2.111182in}{2.177991in}}%
\pgfpathlineto{\pgfqpoint{2.145000in}{2.225989in}}%
\pgfpathlineto{\pgfqpoint{2.178818in}{2.273986in}}%
\pgfpathlineto{\pgfqpoint{2.212636in}{2.317184in}}%
\pgfpathlineto{\pgfqpoint{2.246455in}{2.369981in}}%
\pgfpathlineto{\pgfqpoint{2.280273in}{2.398779in}}%
\pgfpathlineto{\pgfqpoint{2.314091in}{2.446777in}}%
\pgfpathlineto{\pgfqpoint{2.347909in}{2.485175in}}%
\pgfpathlineto{\pgfqpoint{2.381727in}{2.533172in}}%
\pgfpathlineto{\pgfqpoint{2.415545in}{2.571570in}}%
\pgfpathlineto{\pgfqpoint{2.449364in}{2.619568in}}%
\pgfpathlineto{\pgfqpoint{2.483182in}{2.657966in}}%
\pgfpathlineto{\pgfqpoint{2.517000in}{2.696364in}}%
\pgfpathlineto{\pgfqpoint{2.550818in}{2.744361in}}%
\pgfpathlineto{\pgfqpoint{2.584636in}{2.777959in}}%
\pgfpathlineto{\pgfqpoint{2.618455in}{2.816357in}}%
\pgfpathlineto{\pgfqpoint{2.652273in}{2.854755in}}%
\pgfpathlineto{\pgfqpoint{2.686091in}{2.883554in}}%
\pgfpathlineto{\pgfqpoint{2.719909in}{2.921952in}}%
\pgfpathlineto{\pgfqpoint{2.753727in}{2.955550in}}%
\pgfpathlineto{\pgfqpoint{2.787545in}{2.989148in}}%
\pgfpathlineto{\pgfqpoint{2.821364in}{3.027546in}}%
\pgfpathlineto{\pgfqpoint{2.855182in}{3.061144in}}%
\pgfpathlineto{\pgfqpoint{2.889000in}{3.099542in}}%
\pgfpathlineto{\pgfqpoint{2.922818in}{3.123541in}}%
\pgfpathlineto{\pgfqpoint{2.956636in}{3.157139in}}%
\pgfpathlineto{\pgfqpoint{2.990455in}{3.185938in}}%
\pgfpathlineto{\pgfqpoint{3.024273in}{3.214736in}}%
\pgfpathlineto{\pgfqpoint{3.058091in}{3.243535in}}%
\pgfpathlineto{\pgfqpoint{3.091909in}{3.277133in}}%
\pgfpathlineto{\pgfqpoint{3.125727in}{3.296332in}}%
\pgfpathlineto{\pgfqpoint{3.159545in}{3.315531in}}%
\pgfpathlineto{\pgfqpoint{3.193364in}{3.334730in}}%
\pgfpathlineto{\pgfqpoint{3.227182in}{3.353929in}}%
\pgfpathlineto{\pgfqpoint{3.261000in}{3.368328in}}%
\pgfpathlineto{\pgfqpoint{3.294818in}{3.382727in}}%
\pgfpathlineto{\pgfqpoint{3.328636in}{3.373128in}}%
\pgfpathlineto{\pgfqpoint{3.362455in}{3.037146in}}%
\pgfpathlineto{\pgfqpoint{3.396273in}{3.051545in}}%
\pgfpathlineto{\pgfqpoint{3.430091in}{3.070744in}}%
\pgfpathlineto{\pgfqpoint{3.463909in}{3.080343in}}%
\pgfpathlineto{\pgfqpoint{3.497727in}{3.094743in}}%
\pgfpathlineto{\pgfqpoint{3.531545in}{3.099542in}}%
\pgfpathlineto{\pgfqpoint{3.565364in}{3.104342in}}%
\pgfpathlineto{\pgfqpoint{3.599182in}{3.109142in}}%
\pgfpathlineto{\pgfqpoint{3.633000in}{3.113942in}}%
\pgfpathlineto{\pgfqpoint{3.666818in}{3.118741in}}%
\pgfpathlineto{\pgfqpoint{3.700636in}{3.065944in}}%
\pgfpathlineto{\pgfqpoint{3.734455in}{2.878754in}}%
\pgfpathlineto{\pgfqpoint{3.768273in}{2.849955in}}%
\pgfpathlineto{\pgfqpoint{3.802091in}{2.859555in}}%
\pgfpathlineto{\pgfqpoint{3.835909in}{2.859555in}}%
\pgfpathlineto{\pgfqpoint{3.869727in}{2.864355in}}%
\pgfpathlineto{\pgfqpoint{3.903545in}{2.869154in}}%
\pgfpathlineto{\pgfqpoint{3.937364in}{2.873954in}}%
\pgfpathlineto{\pgfqpoint{3.971182in}{2.873954in}}%
\pgfpathlineto{\pgfqpoint{4.005000in}{2.441977in}}%
\pgfpathlineto{\pgfqpoint{4.038818in}{2.057997in}}%
\pgfpathlineto{\pgfqpoint{4.072636in}{2.067597in}}%
\pgfpathlineto{\pgfqpoint{4.106455in}{2.072397in}}%
\pgfpathlineto{\pgfqpoint{4.140273in}{2.081996in}}%
\pgfpathlineto{\pgfqpoint{4.174091in}{2.081996in}}%
\pgfpathlineto{\pgfqpoint{4.207909in}{2.091596in}}%
\pgfpathlineto{\pgfqpoint{4.241727in}{2.096395in}}%
\pgfpathlineto{\pgfqpoint{4.275545in}{2.101195in}}%
\pgfpathlineto{\pgfqpoint{4.309364in}{2.101195in}}%
\pgfpathlineto{\pgfqpoint{4.343182in}{2.105995in}}%
\pgfpathlineto{\pgfqpoint{4.377000in}{2.110795in}}%
\pgfpathlineto{\pgfqpoint{4.410818in}{2.110795in}}%
\pgfpathlineto{\pgfqpoint{4.444636in}{2.115594in}}%
\pgfpathlineto{\pgfqpoint{4.478455in}{2.115594in}}%
\pgfpathlineto{\pgfqpoint{4.512273in}{2.120394in}}%
\pgfpathlineto{\pgfqpoint{4.546091in}{2.120394in}}%
\pgfpathlineto{\pgfqpoint{4.579909in}{2.120394in}}%
\pgfpathlineto{\pgfqpoint{4.613727in}{2.125194in}}%
\pgfpathlineto{\pgfqpoint{4.647545in}{2.125194in}}%
\pgfpathlineto{\pgfqpoint{4.681364in}{2.129994in}}%
\pgfpathlineto{\pgfqpoint{4.715182in}{2.129994in}}%
\pgfpathlineto{\pgfqpoint{4.749000in}{2.129994in}}%
\pgfpathlineto{\pgfqpoint{4.782818in}{2.129994in}}%
\pgfpathlineto{\pgfqpoint{4.816636in}{2.129994in}}%
\pgfpathlineto{\pgfqpoint{4.850455in}{2.125194in}}%
\pgfpathlineto{\pgfqpoint{4.884273in}{2.129994in}}%
\pgfpathlineto{\pgfqpoint{4.918091in}{2.129994in}}%
\pgfpathlineto{\pgfqpoint{4.951909in}{2.129994in}}%
\pgfpathlineto{\pgfqpoint{4.985727in}{2.129994in}}%
\pgfpathlineto{\pgfqpoint{5.019545in}{2.129994in}}%
\pgfpathlineto{\pgfqpoint{5.053364in}{2.129994in}}%
\pgfpathlineto{\pgfqpoint{5.087182in}{2.129994in}}%
\pgfpathlineto{\pgfqpoint{5.121000in}{2.129994in}}%
\pgfpathlineto{\pgfqpoint{5.154818in}{2.129994in}}%
\pgfpathlineto{\pgfqpoint{5.188636in}{2.129994in}}%
\pgfusepath{stroke}%
\end{pgfscope}%
\begin{pgfscope}%
\pgfsetrectcap%
\pgfsetmiterjoin%
\pgfsetlinewidth{0.803000pt}%
\definecolor{currentstroke}{rgb}{0.000000,0.000000,0.000000}%
\pgfsetstrokecolor{currentstroke}%
\pgfsetdash{}{0pt}%
\pgfpathmoveto{\pgfqpoint{0.750000in}{0.500000in}}%
\pgfpathlineto{\pgfqpoint{0.750000in}{3.520000in}}%
\pgfusepath{stroke}%
\end{pgfscope}%
\begin{pgfscope}%
\pgfsetrectcap%
\pgfsetmiterjoin%
\pgfsetlinewidth{0.803000pt}%
\definecolor{currentstroke}{rgb}{0.000000,0.000000,0.000000}%
\pgfsetstrokecolor{currentstroke}%
\pgfsetdash{}{0pt}%
\pgfpathmoveto{\pgfqpoint{5.400000in}{0.500000in}}%
\pgfpathlineto{\pgfqpoint{5.400000in}{3.520000in}}%
\pgfusepath{stroke}%
\end{pgfscope}%
\begin{pgfscope}%
\pgfsetrectcap%
\pgfsetmiterjoin%
\pgfsetlinewidth{0.803000pt}%
\definecolor{currentstroke}{rgb}{0.000000,0.000000,0.000000}%
\pgfsetstrokecolor{currentstroke}%
\pgfsetdash{}{0pt}%
\pgfpathmoveto{\pgfqpoint{0.750000in}{0.500000in}}%
\pgfpathlineto{\pgfqpoint{5.400000in}{0.500000in}}%
\pgfusepath{stroke}%
\end{pgfscope}%
\begin{pgfscope}%
\pgfsetrectcap%
\pgfsetmiterjoin%
\pgfsetlinewidth{0.803000pt}%
\definecolor{currentstroke}{rgb}{0.000000,0.000000,0.000000}%
\pgfsetstrokecolor{currentstroke}%
\pgfsetdash{}{0pt}%
\pgfpathmoveto{\pgfqpoint{0.750000in}{3.520000in}}%
\pgfpathlineto{\pgfqpoint{5.400000in}{3.520000in}}%
\pgfusepath{stroke}%
\end{pgfscope}%
\begin{pgfscope}%
\pgfsetbuttcap%
\pgfsetmiterjoin%
\definecolor{currentfill}{rgb}{1.000000,1.000000,1.000000}%
\pgfsetfillcolor{currentfill}%
\pgfsetfillopacity{0.800000}%
\pgfsetlinewidth{1.003750pt}%
\definecolor{currentstroke}{rgb}{0.800000,0.800000,0.800000}%
\pgfsetstrokecolor{currentstroke}%
\pgfsetstrokeopacity{0.800000}%
\pgfsetdash{}{0pt}%
\pgfpathmoveto{\pgfqpoint{3.385527in}{0.583333in}}%
\pgfpathlineto{\pgfqpoint{5.283333in}{0.583333in}}%
\pgfpathquadraticcurveto{\pgfqpoint{5.316667in}{0.583333in}}{\pgfqpoint{5.316667in}{0.616667in}}%
\pgfpathlineto{\pgfqpoint{5.316667in}{1.064814in}}%
\pgfpathquadraticcurveto{\pgfqpoint{5.316667in}{1.098148in}}{\pgfqpoint{5.283333in}{1.098148in}}%
\pgfpathlineto{\pgfqpoint{3.385527in}{1.098148in}}%
\pgfpathquadraticcurveto{\pgfqpoint{3.352194in}{1.098148in}}{\pgfqpoint{3.352194in}{1.064814in}}%
\pgfpathlineto{\pgfqpoint{3.352194in}{0.616667in}}%
\pgfpathquadraticcurveto{\pgfqpoint{3.352194in}{0.583333in}}{\pgfqpoint{3.385527in}{0.583333in}}%
\pgfpathlineto{\pgfqpoint{3.385527in}{0.583333in}}%
\pgfpathclose%
\pgfusepath{stroke,fill}%
\end{pgfscope}%
\begin{pgfscope}%
\pgfsetrectcap%
\pgfsetroundjoin%
\pgfsetlinewidth{1.505625pt}%
\definecolor{currentstroke}{rgb}{0.121569,0.466667,0.705882}%
\pgfsetstrokecolor{currentstroke}%
\pgfsetdash{}{0pt}%
\pgfpathmoveto{\pgfqpoint{3.418861in}{0.973147in}}%
\pgfpathlineto{\pgfqpoint{3.585527in}{0.973147in}}%
\pgfpathlineto{\pgfqpoint{3.752194in}{0.973147in}}%
\pgfusepath{stroke}%
\end{pgfscope}%
\begin{pgfscope}%
\definecolor{textcolor}{rgb}{0.000000,0.000000,0.000000}%
\pgfsetstrokecolor{textcolor}%
\pgfsetfillcolor{textcolor}%
\pgftext[x=3.885527in,y=0.914814in,left,base]{\color{textcolor}\rmfamily\fontsize{12.000000}{14.400000}\selectfont T by thermocouple}%
\end{pgfscope}%
\begin{pgfscope}%
\pgfsetrectcap%
\pgfsetroundjoin%
\pgfsetlinewidth{1.505625pt}%
\definecolor{currentstroke}{rgb}{1.000000,0.498039,0.054902}%
\pgfsetstrokecolor{currentstroke}%
\pgfsetdash{}{0pt}%
\pgfpathmoveto{\pgfqpoint{3.418861in}{0.740740in}}%
\pgfpathlineto{\pgfqpoint{3.585527in}{0.740740in}}%
\pgfpathlineto{\pgfqpoint{3.752194in}{0.740740in}}%
\pgfusepath{stroke}%
\end{pgfscope}%
\begin{pgfscope}%
\definecolor{textcolor}{rgb}{0.000000,0.000000,0.000000}%
\pgfsetstrokecolor{textcolor}%
\pgfsetfillcolor{textcolor}%
\pgftext[x=3.885527in,y=0.682407in,left,base]{\color{textcolor}\rmfamily\fontsize{12.000000}{14.400000}\selectfont T by Raman}%
\end{pgfscope}%
\end{pgfpicture}%
\makeatother%
\endgroup%

    \caption[Temperature comparison between Thermocouple and Raman measurement]{Temperature comparison between measurement with the Thermocouple type K and through the Raman setup; plot of temperature over the measurement time}
    \label{fig:plot-temp-time}
\end{figure}

\mycomment[MK]{Error discussion Temp missing}