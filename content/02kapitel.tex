%!TEX root = ../main.tex

%%%%%%%%%%%%%%%%%%%%%%%%%%%%%%%
%%%%%%%%%%%%%%%%%%%%%%%%%%%%%%%
\chapter{Theoretical basics}
\label{chap:theoretical}

The following theoretical basics are summarized from the standard literature in optics \autocite{bornPrinciplesOpticsElectromagnetic1999,hechtOptik2005,lipsonOptik1997,niedrigOptikWellenUnd2004} and more specifically Raman application \autocite{herzbergMolecularSpectraMolecular2013,schraderInfraredRamanSpectroscopy1995}. 

%%%%%%%%%%%%%%%%%%%%%%%%%%%%%%%
\section{Molecule - light interactions}

%%%%%%%%%%%%%%%%%%%%%%%%%%%%%%%
\section{Scattering effects}

%%%%%%%%%%%%%%%%%%%%%%%%%%%%%%%
\section{Measurement of different phisical properties - RAMAN spectroscopy}

%%%%%%%%%%%%%%%%%%%%%%%%%%%%%%%
%%%%%%%%%%%%%%%%%%%%%%%%%%%%%%%
\chapter{Experimental setup}
\label{chap:experimental}

%%%%%%%%%%%%%%%%%%%%%%%%%%%%%%%
\section{Used equipment}

%%%%%%%%%%%%%%%%%%%%%%%%%%%%%%%
\section{Measurement setup and preparations}

%%%%%%%%%%%%%%%%%%%%%%%%%%%%%%%
\section{Expectations}

%%%%%%%%%%%%%%%%%%%%%%%%%%%%%%%
\section{Execution}

%%%%%%%%%%%%%%%%%%%%%%%%%%%%%%%
%%%%%%%%%%%%%%%%%%%%%%%%%%%%%%%
\chapter{Results}
\label{chap:results}

%%%%%%%%%%%%%%%%%%%%%%%%%%%%%%%
\section{Data presentation and preparation}

\subsection*{Species determination}
\label{subsec:spec-prep}

\subsection*{Temperature calculation}
\label{subsec:temp-prep}

The raw data of the intesity spectrum over the Raman shift can be obtained in \autoref{fig:plot-temp-raw}. 

\begin{figure}[!htb]
    \centering
    %% Creator: Matplotlib, PGF backend
%%
%% To include the figure in your LaTeX document, write
%%   \input{<filename>.pgf}
%%
%% Make sure the required packages are loaded in your preamble
%%   \usepackage{pgf}
%%
%% Also ensure that all the required font packages are loaded; for instance,
%% the lmodern package is sometimes necessary when using math font.
%%   \usepackage{lmodern}
%%
%% Figures using additional raster images can only be included by \input if
%% they are in the same directory as the main LaTeX file. For loading figures
%% from other directories you can use the `import` package
%%   \usepackage{import}
%%
%% and then include the figures with
%%   \import{<path to file>}{<filename>.pgf}
%%
%% Matplotlib used the following preamble
%%   
%%   \usepackage{fontspec}
%%   \setmainfont{Charter.ttc}[Path=\detokenize{/System/Library/Fonts/Supplemental/}]
%%   \setsansfont{DejaVuSans.ttf}[Path=\detokenize{/opt/homebrew/lib/python3.10/site-packages/matplotlib/mpl-data/fonts/ttf/}]
%%   \setmonofont{DejaVuSansMono.ttf}[Path=\detokenize{/opt/homebrew/lib/python3.10/site-packages/matplotlib/mpl-data/fonts/ttf/}]
%%   \makeatletter\@ifpackageloaded{underscore}{}{\usepackage[strings]{underscore}}\makeatother
%%
\begingroup%
\makeatletter%
\begin{pgfpicture}%
\pgfpathrectangle{\pgfpointorigin}{\pgfqpoint{6.400000in}{4.800000in}}%
\pgfusepath{use as bounding box, clip}%
\begin{pgfscope}%
\pgfsetbuttcap%
\pgfsetmiterjoin%
\definecolor{currentfill}{rgb}{1.000000,1.000000,1.000000}%
\pgfsetfillcolor{currentfill}%
\pgfsetlinewidth{0.000000pt}%
\definecolor{currentstroke}{rgb}{1.000000,1.000000,1.000000}%
\pgfsetstrokecolor{currentstroke}%
\pgfsetdash{}{0pt}%
\pgfpathmoveto{\pgfqpoint{0.000000in}{0.000000in}}%
\pgfpathlineto{\pgfqpoint{6.400000in}{0.000000in}}%
\pgfpathlineto{\pgfqpoint{6.400000in}{4.800000in}}%
\pgfpathlineto{\pgfqpoint{0.000000in}{4.800000in}}%
\pgfpathlineto{\pgfqpoint{0.000000in}{0.000000in}}%
\pgfpathclose%
\pgfusepath{fill}%
\end{pgfscope}%
\begin{pgfscope}%
\pgfsetbuttcap%
\pgfsetmiterjoin%
\definecolor{currentfill}{rgb}{1.000000,1.000000,1.000000}%
\pgfsetfillcolor{currentfill}%
\pgfsetlinewidth{0.000000pt}%
\definecolor{currentstroke}{rgb}{0.000000,0.000000,0.000000}%
\pgfsetstrokecolor{currentstroke}%
\pgfsetstrokeopacity{0.000000}%
\pgfsetdash{}{0pt}%
\pgfpathmoveto{\pgfqpoint{0.800000in}{0.528000in}}%
\pgfpathlineto{\pgfqpoint{5.760000in}{0.528000in}}%
\pgfpathlineto{\pgfqpoint{5.760000in}{4.224000in}}%
\pgfpathlineto{\pgfqpoint{0.800000in}{4.224000in}}%
\pgfpathlineto{\pgfqpoint{0.800000in}{0.528000in}}%
\pgfpathclose%
\pgfusepath{fill}%
\end{pgfscope}%
\begin{pgfscope}%
\pgfpathrectangle{\pgfqpoint{0.800000in}{0.528000in}}{\pgfqpoint{4.960000in}{3.696000in}}%
\pgfusepath{clip}%
\pgfsetrectcap%
\pgfsetroundjoin%
\pgfsetlinewidth{0.803000pt}%
\definecolor{currentstroke}{rgb}{0.690196,0.690196,0.690196}%
\pgfsetstrokecolor{currentstroke}%
\pgfsetdash{}{0pt}%
\pgfpathmoveto{\pgfqpoint{1.013437in}{0.528000in}}%
\pgfpathlineto{\pgfqpoint{1.013437in}{4.224000in}}%
\pgfusepath{stroke}%
\end{pgfscope}%
\begin{pgfscope}%
\pgfsetbuttcap%
\pgfsetroundjoin%
\definecolor{currentfill}{rgb}{0.000000,0.000000,0.000000}%
\pgfsetfillcolor{currentfill}%
\pgfsetlinewidth{0.803000pt}%
\definecolor{currentstroke}{rgb}{0.000000,0.000000,0.000000}%
\pgfsetstrokecolor{currentstroke}%
\pgfsetdash{}{0pt}%
\pgfsys@defobject{currentmarker}{\pgfqpoint{0.000000in}{-0.048611in}}{\pgfqpoint{0.000000in}{0.000000in}}{%
\pgfpathmoveto{\pgfqpoint{0.000000in}{0.000000in}}%
\pgfpathlineto{\pgfqpoint{0.000000in}{-0.048611in}}%
\pgfusepath{stroke,fill}%
}%
\begin{pgfscope}%
\pgfsys@transformshift{1.013437in}{0.528000in}%
\pgfsys@useobject{currentmarker}{}%
\end{pgfscope}%
\end{pgfscope}%
\begin{pgfscope}%
\definecolor{textcolor}{rgb}{0.000000,0.000000,0.000000}%
\pgfsetstrokecolor{textcolor}%
\pgfsetfillcolor{textcolor}%
\pgftext[x=1.013437in,y=0.430778in,,top]{\color{textcolor}\rmfamily\fontsize{12.000000}{14.400000}\selectfont \(\displaystyle {2600}\)}%
\end{pgfscope}%
\begin{pgfscope}%
\pgfpathrectangle{\pgfqpoint{0.800000in}{0.528000in}}{\pgfqpoint{4.960000in}{3.696000in}}%
\pgfusepath{clip}%
\pgfsetrectcap%
\pgfsetroundjoin%
\pgfsetlinewidth{0.803000pt}%
\definecolor{currentstroke}{rgb}{0.690196,0.690196,0.690196}%
\pgfsetstrokecolor{currentstroke}%
\pgfsetdash{}{0pt}%
\pgfpathmoveto{\pgfqpoint{1.768092in}{0.528000in}}%
\pgfpathlineto{\pgfqpoint{1.768092in}{4.224000in}}%
\pgfusepath{stroke}%
\end{pgfscope}%
\begin{pgfscope}%
\pgfsetbuttcap%
\pgfsetroundjoin%
\definecolor{currentfill}{rgb}{0.000000,0.000000,0.000000}%
\pgfsetfillcolor{currentfill}%
\pgfsetlinewidth{0.803000pt}%
\definecolor{currentstroke}{rgb}{0.000000,0.000000,0.000000}%
\pgfsetstrokecolor{currentstroke}%
\pgfsetdash{}{0pt}%
\pgfsys@defobject{currentmarker}{\pgfqpoint{0.000000in}{-0.048611in}}{\pgfqpoint{0.000000in}{0.000000in}}{%
\pgfpathmoveto{\pgfqpoint{0.000000in}{0.000000in}}%
\pgfpathlineto{\pgfqpoint{0.000000in}{-0.048611in}}%
\pgfusepath{stroke,fill}%
}%
\begin{pgfscope}%
\pgfsys@transformshift{1.768092in}{0.528000in}%
\pgfsys@useobject{currentmarker}{}%
\end{pgfscope}%
\end{pgfscope}%
\begin{pgfscope}%
\definecolor{textcolor}{rgb}{0.000000,0.000000,0.000000}%
\pgfsetstrokecolor{textcolor}%
\pgfsetfillcolor{textcolor}%
\pgftext[x=1.768092in,y=0.430778in,,top]{\color{textcolor}\rmfamily\fontsize{12.000000}{14.400000}\selectfont \(\displaystyle {2800}\)}%
\end{pgfscope}%
\begin{pgfscope}%
\pgfpathrectangle{\pgfqpoint{0.800000in}{0.528000in}}{\pgfqpoint{4.960000in}{3.696000in}}%
\pgfusepath{clip}%
\pgfsetrectcap%
\pgfsetroundjoin%
\pgfsetlinewidth{0.803000pt}%
\definecolor{currentstroke}{rgb}{0.690196,0.690196,0.690196}%
\pgfsetstrokecolor{currentstroke}%
\pgfsetdash{}{0pt}%
\pgfpathmoveto{\pgfqpoint{2.522746in}{0.528000in}}%
\pgfpathlineto{\pgfqpoint{2.522746in}{4.224000in}}%
\pgfusepath{stroke}%
\end{pgfscope}%
\begin{pgfscope}%
\pgfsetbuttcap%
\pgfsetroundjoin%
\definecolor{currentfill}{rgb}{0.000000,0.000000,0.000000}%
\pgfsetfillcolor{currentfill}%
\pgfsetlinewidth{0.803000pt}%
\definecolor{currentstroke}{rgb}{0.000000,0.000000,0.000000}%
\pgfsetstrokecolor{currentstroke}%
\pgfsetdash{}{0pt}%
\pgfsys@defobject{currentmarker}{\pgfqpoint{0.000000in}{-0.048611in}}{\pgfqpoint{0.000000in}{0.000000in}}{%
\pgfpathmoveto{\pgfqpoint{0.000000in}{0.000000in}}%
\pgfpathlineto{\pgfqpoint{0.000000in}{-0.048611in}}%
\pgfusepath{stroke,fill}%
}%
\begin{pgfscope}%
\pgfsys@transformshift{2.522746in}{0.528000in}%
\pgfsys@useobject{currentmarker}{}%
\end{pgfscope}%
\end{pgfscope}%
\begin{pgfscope}%
\definecolor{textcolor}{rgb}{0.000000,0.000000,0.000000}%
\pgfsetstrokecolor{textcolor}%
\pgfsetfillcolor{textcolor}%
\pgftext[x=2.522746in,y=0.430778in,,top]{\color{textcolor}\rmfamily\fontsize{12.000000}{14.400000}\selectfont \(\displaystyle {3000}\)}%
\end{pgfscope}%
\begin{pgfscope}%
\pgfpathrectangle{\pgfqpoint{0.800000in}{0.528000in}}{\pgfqpoint{4.960000in}{3.696000in}}%
\pgfusepath{clip}%
\pgfsetrectcap%
\pgfsetroundjoin%
\pgfsetlinewidth{0.803000pt}%
\definecolor{currentstroke}{rgb}{0.690196,0.690196,0.690196}%
\pgfsetstrokecolor{currentstroke}%
\pgfsetdash{}{0pt}%
\pgfpathmoveto{\pgfqpoint{3.277401in}{0.528000in}}%
\pgfpathlineto{\pgfqpoint{3.277401in}{4.224000in}}%
\pgfusepath{stroke}%
\end{pgfscope}%
\begin{pgfscope}%
\pgfsetbuttcap%
\pgfsetroundjoin%
\definecolor{currentfill}{rgb}{0.000000,0.000000,0.000000}%
\pgfsetfillcolor{currentfill}%
\pgfsetlinewidth{0.803000pt}%
\definecolor{currentstroke}{rgb}{0.000000,0.000000,0.000000}%
\pgfsetstrokecolor{currentstroke}%
\pgfsetdash{}{0pt}%
\pgfsys@defobject{currentmarker}{\pgfqpoint{0.000000in}{-0.048611in}}{\pgfqpoint{0.000000in}{0.000000in}}{%
\pgfpathmoveto{\pgfqpoint{0.000000in}{0.000000in}}%
\pgfpathlineto{\pgfqpoint{0.000000in}{-0.048611in}}%
\pgfusepath{stroke,fill}%
}%
\begin{pgfscope}%
\pgfsys@transformshift{3.277401in}{0.528000in}%
\pgfsys@useobject{currentmarker}{}%
\end{pgfscope}%
\end{pgfscope}%
\begin{pgfscope}%
\definecolor{textcolor}{rgb}{0.000000,0.000000,0.000000}%
\pgfsetstrokecolor{textcolor}%
\pgfsetfillcolor{textcolor}%
\pgftext[x=3.277401in,y=0.430778in,,top]{\color{textcolor}\rmfamily\fontsize{12.000000}{14.400000}\selectfont \(\displaystyle {3200}\)}%
\end{pgfscope}%
\begin{pgfscope}%
\pgfpathrectangle{\pgfqpoint{0.800000in}{0.528000in}}{\pgfqpoint{4.960000in}{3.696000in}}%
\pgfusepath{clip}%
\pgfsetrectcap%
\pgfsetroundjoin%
\pgfsetlinewidth{0.803000pt}%
\definecolor{currentstroke}{rgb}{0.690196,0.690196,0.690196}%
\pgfsetstrokecolor{currentstroke}%
\pgfsetdash{}{0pt}%
\pgfpathmoveto{\pgfqpoint{4.032056in}{0.528000in}}%
\pgfpathlineto{\pgfqpoint{4.032056in}{4.224000in}}%
\pgfusepath{stroke}%
\end{pgfscope}%
\begin{pgfscope}%
\pgfsetbuttcap%
\pgfsetroundjoin%
\definecolor{currentfill}{rgb}{0.000000,0.000000,0.000000}%
\pgfsetfillcolor{currentfill}%
\pgfsetlinewidth{0.803000pt}%
\definecolor{currentstroke}{rgb}{0.000000,0.000000,0.000000}%
\pgfsetstrokecolor{currentstroke}%
\pgfsetdash{}{0pt}%
\pgfsys@defobject{currentmarker}{\pgfqpoint{0.000000in}{-0.048611in}}{\pgfqpoint{0.000000in}{0.000000in}}{%
\pgfpathmoveto{\pgfqpoint{0.000000in}{0.000000in}}%
\pgfpathlineto{\pgfqpoint{0.000000in}{-0.048611in}}%
\pgfusepath{stroke,fill}%
}%
\begin{pgfscope}%
\pgfsys@transformshift{4.032056in}{0.528000in}%
\pgfsys@useobject{currentmarker}{}%
\end{pgfscope}%
\end{pgfscope}%
\begin{pgfscope}%
\definecolor{textcolor}{rgb}{0.000000,0.000000,0.000000}%
\pgfsetstrokecolor{textcolor}%
\pgfsetfillcolor{textcolor}%
\pgftext[x=4.032056in,y=0.430778in,,top]{\color{textcolor}\rmfamily\fontsize{12.000000}{14.400000}\selectfont \(\displaystyle {3400}\)}%
\end{pgfscope}%
\begin{pgfscope}%
\pgfpathrectangle{\pgfqpoint{0.800000in}{0.528000in}}{\pgfqpoint{4.960000in}{3.696000in}}%
\pgfusepath{clip}%
\pgfsetrectcap%
\pgfsetroundjoin%
\pgfsetlinewidth{0.803000pt}%
\definecolor{currentstroke}{rgb}{0.690196,0.690196,0.690196}%
\pgfsetstrokecolor{currentstroke}%
\pgfsetdash{}{0pt}%
\pgfpathmoveto{\pgfqpoint{4.786711in}{0.528000in}}%
\pgfpathlineto{\pgfqpoint{4.786711in}{4.224000in}}%
\pgfusepath{stroke}%
\end{pgfscope}%
\begin{pgfscope}%
\pgfsetbuttcap%
\pgfsetroundjoin%
\definecolor{currentfill}{rgb}{0.000000,0.000000,0.000000}%
\pgfsetfillcolor{currentfill}%
\pgfsetlinewidth{0.803000pt}%
\definecolor{currentstroke}{rgb}{0.000000,0.000000,0.000000}%
\pgfsetstrokecolor{currentstroke}%
\pgfsetdash{}{0pt}%
\pgfsys@defobject{currentmarker}{\pgfqpoint{0.000000in}{-0.048611in}}{\pgfqpoint{0.000000in}{0.000000in}}{%
\pgfpathmoveto{\pgfqpoint{0.000000in}{0.000000in}}%
\pgfpathlineto{\pgfqpoint{0.000000in}{-0.048611in}}%
\pgfusepath{stroke,fill}%
}%
\begin{pgfscope}%
\pgfsys@transformshift{4.786711in}{0.528000in}%
\pgfsys@useobject{currentmarker}{}%
\end{pgfscope}%
\end{pgfscope}%
\begin{pgfscope}%
\definecolor{textcolor}{rgb}{0.000000,0.000000,0.000000}%
\pgfsetstrokecolor{textcolor}%
\pgfsetfillcolor{textcolor}%
\pgftext[x=4.786711in,y=0.430778in,,top]{\color{textcolor}\rmfamily\fontsize{12.000000}{14.400000}\selectfont \(\displaystyle {3600}\)}%
\end{pgfscope}%
\begin{pgfscope}%
\pgfpathrectangle{\pgfqpoint{0.800000in}{0.528000in}}{\pgfqpoint{4.960000in}{3.696000in}}%
\pgfusepath{clip}%
\pgfsetrectcap%
\pgfsetroundjoin%
\pgfsetlinewidth{0.803000pt}%
\definecolor{currentstroke}{rgb}{0.690196,0.690196,0.690196}%
\pgfsetstrokecolor{currentstroke}%
\pgfsetdash{}{0pt}%
\pgfpathmoveto{\pgfqpoint{5.541366in}{0.528000in}}%
\pgfpathlineto{\pgfqpoint{5.541366in}{4.224000in}}%
\pgfusepath{stroke}%
\end{pgfscope}%
\begin{pgfscope}%
\pgfsetbuttcap%
\pgfsetroundjoin%
\definecolor{currentfill}{rgb}{0.000000,0.000000,0.000000}%
\pgfsetfillcolor{currentfill}%
\pgfsetlinewidth{0.803000pt}%
\definecolor{currentstroke}{rgb}{0.000000,0.000000,0.000000}%
\pgfsetstrokecolor{currentstroke}%
\pgfsetdash{}{0pt}%
\pgfsys@defobject{currentmarker}{\pgfqpoint{0.000000in}{-0.048611in}}{\pgfqpoint{0.000000in}{0.000000in}}{%
\pgfpathmoveto{\pgfqpoint{0.000000in}{0.000000in}}%
\pgfpathlineto{\pgfqpoint{0.000000in}{-0.048611in}}%
\pgfusepath{stroke,fill}%
}%
\begin{pgfscope}%
\pgfsys@transformshift{5.541366in}{0.528000in}%
\pgfsys@useobject{currentmarker}{}%
\end{pgfscope}%
\end{pgfscope}%
\begin{pgfscope}%
\definecolor{textcolor}{rgb}{0.000000,0.000000,0.000000}%
\pgfsetstrokecolor{textcolor}%
\pgfsetfillcolor{textcolor}%
\pgftext[x=5.541366in,y=0.430778in,,top]{\color{textcolor}\rmfamily\fontsize{12.000000}{14.400000}\selectfont \(\displaystyle {3800}\)}%
\end{pgfscope}%
\begin{pgfscope}%
\definecolor{textcolor}{rgb}{0.000000,0.000000,0.000000}%
\pgfsetstrokecolor{textcolor}%
\pgfsetfillcolor{textcolor}%
\pgftext[x=3.280000in,y=0.216287in,,top]{\color{textcolor}\rmfamily\fontsize{12.000000}{14.400000}\selectfont Raman shift \(\displaystyle \Delta v\) in \(\displaystyle \mathrm{cm}^\mathrm{-1}\)}%
\end{pgfscope}%
\begin{pgfscope}%
\pgfpathrectangle{\pgfqpoint{0.800000in}{0.528000in}}{\pgfqpoint{4.960000in}{3.696000in}}%
\pgfusepath{clip}%
\pgfsetrectcap%
\pgfsetroundjoin%
\pgfsetlinewidth{0.803000pt}%
\definecolor{currentstroke}{rgb}{0.690196,0.690196,0.690196}%
\pgfsetstrokecolor{currentstroke}%
\pgfsetdash{}{0pt}%
\pgfpathmoveto{\pgfqpoint{0.800000in}{0.698956in}}%
\pgfpathlineto{\pgfqpoint{5.760000in}{0.698956in}}%
\pgfusepath{stroke}%
\end{pgfscope}%
\begin{pgfscope}%
\pgfsetbuttcap%
\pgfsetroundjoin%
\definecolor{currentfill}{rgb}{0.000000,0.000000,0.000000}%
\pgfsetfillcolor{currentfill}%
\pgfsetlinewidth{0.803000pt}%
\definecolor{currentstroke}{rgb}{0.000000,0.000000,0.000000}%
\pgfsetstrokecolor{currentstroke}%
\pgfsetdash{}{0pt}%
\pgfsys@defobject{currentmarker}{\pgfqpoint{-0.048611in}{0.000000in}}{\pgfqpoint{-0.000000in}{0.000000in}}{%
\pgfpathmoveto{\pgfqpoint{-0.000000in}{0.000000in}}%
\pgfpathlineto{\pgfqpoint{-0.048611in}{0.000000in}}%
\pgfusepath{stroke,fill}%
}%
\begin{pgfscope}%
\pgfsys@transformshift{0.800000in}{0.698956in}%
\pgfsys@useobject{currentmarker}{}%
\end{pgfscope}%
\end{pgfscope}%
\begin{pgfscope}%
\definecolor{textcolor}{rgb}{0.000000,0.000000,0.000000}%
\pgfsetstrokecolor{textcolor}%
\pgfsetfillcolor{textcolor}%
\pgftext[x=0.412657in, y=0.637636in, left, base]{\color{textcolor}\rmfamily\fontsize{12.000000}{14.400000}\selectfont \(\displaystyle {0.00}\)}%
\end{pgfscope}%
\begin{pgfscope}%
\pgfpathrectangle{\pgfqpoint{0.800000in}{0.528000in}}{\pgfqpoint{4.960000in}{3.696000in}}%
\pgfusepath{clip}%
\pgfsetrectcap%
\pgfsetroundjoin%
\pgfsetlinewidth{0.803000pt}%
\definecolor{currentstroke}{rgb}{0.690196,0.690196,0.690196}%
\pgfsetstrokecolor{currentstroke}%
\pgfsetdash{}{0pt}%
\pgfpathmoveto{\pgfqpoint{0.800000in}{1.286196in}}%
\pgfpathlineto{\pgfqpoint{5.760000in}{1.286196in}}%
\pgfusepath{stroke}%
\end{pgfscope}%
\begin{pgfscope}%
\pgfsetbuttcap%
\pgfsetroundjoin%
\definecolor{currentfill}{rgb}{0.000000,0.000000,0.000000}%
\pgfsetfillcolor{currentfill}%
\pgfsetlinewidth{0.803000pt}%
\definecolor{currentstroke}{rgb}{0.000000,0.000000,0.000000}%
\pgfsetstrokecolor{currentstroke}%
\pgfsetdash{}{0pt}%
\pgfsys@defobject{currentmarker}{\pgfqpoint{-0.048611in}{0.000000in}}{\pgfqpoint{-0.000000in}{0.000000in}}{%
\pgfpathmoveto{\pgfqpoint{-0.000000in}{0.000000in}}%
\pgfpathlineto{\pgfqpoint{-0.048611in}{0.000000in}}%
\pgfusepath{stroke,fill}%
}%
\begin{pgfscope}%
\pgfsys@transformshift{0.800000in}{1.286196in}%
\pgfsys@useobject{currentmarker}{}%
\end{pgfscope}%
\end{pgfscope}%
\begin{pgfscope}%
\definecolor{textcolor}{rgb}{0.000000,0.000000,0.000000}%
\pgfsetstrokecolor{textcolor}%
\pgfsetfillcolor{textcolor}%
\pgftext[x=0.412657in, y=1.224876in, left, base]{\color{textcolor}\rmfamily\fontsize{12.000000}{14.400000}\selectfont \(\displaystyle {0.05}\)}%
\end{pgfscope}%
\begin{pgfscope}%
\pgfpathrectangle{\pgfqpoint{0.800000in}{0.528000in}}{\pgfqpoint{4.960000in}{3.696000in}}%
\pgfusepath{clip}%
\pgfsetrectcap%
\pgfsetroundjoin%
\pgfsetlinewidth{0.803000pt}%
\definecolor{currentstroke}{rgb}{0.690196,0.690196,0.690196}%
\pgfsetstrokecolor{currentstroke}%
\pgfsetdash{}{0pt}%
\pgfpathmoveto{\pgfqpoint{0.800000in}{1.873436in}}%
\pgfpathlineto{\pgfqpoint{5.760000in}{1.873436in}}%
\pgfusepath{stroke}%
\end{pgfscope}%
\begin{pgfscope}%
\pgfsetbuttcap%
\pgfsetroundjoin%
\definecolor{currentfill}{rgb}{0.000000,0.000000,0.000000}%
\pgfsetfillcolor{currentfill}%
\pgfsetlinewidth{0.803000pt}%
\definecolor{currentstroke}{rgb}{0.000000,0.000000,0.000000}%
\pgfsetstrokecolor{currentstroke}%
\pgfsetdash{}{0pt}%
\pgfsys@defobject{currentmarker}{\pgfqpoint{-0.048611in}{0.000000in}}{\pgfqpoint{-0.000000in}{0.000000in}}{%
\pgfpathmoveto{\pgfqpoint{-0.000000in}{0.000000in}}%
\pgfpathlineto{\pgfqpoint{-0.048611in}{0.000000in}}%
\pgfusepath{stroke,fill}%
}%
\begin{pgfscope}%
\pgfsys@transformshift{0.800000in}{1.873436in}%
\pgfsys@useobject{currentmarker}{}%
\end{pgfscope}%
\end{pgfscope}%
\begin{pgfscope}%
\definecolor{textcolor}{rgb}{0.000000,0.000000,0.000000}%
\pgfsetstrokecolor{textcolor}%
\pgfsetfillcolor{textcolor}%
\pgftext[x=0.412657in, y=1.812116in, left, base]{\color{textcolor}\rmfamily\fontsize{12.000000}{14.400000}\selectfont \(\displaystyle {0.10}\)}%
\end{pgfscope}%
\begin{pgfscope}%
\pgfpathrectangle{\pgfqpoint{0.800000in}{0.528000in}}{\pgfqpoint{4.960000in}{3.696000in}}%
\pgfusepath{clip}%
\pgfsetrectcap%
\pgfsetroundjoin%
\pgfsetlinewidth{0.803000pt}%
\definecolor{currentstroke}{rgb}{0.690196,0.690196,0.690196}%
\pgfsetstrokecolor{currentstroke}%
\pgfsetdash{}{0pt}%
\pgfpathmoveto{\pgfqpoint{0.800000in}{2.460676in}}%
\pgfpathlineto{\pgfqpoint{5.760000in}{2.460676in}}%
\pgfusepath{stroke}%
\end{pgfscope}%
\begin{pgfscope}%
\pgfsetbuttcap%
\pgfsetroundjoin%
\definecolor{currentfill}{rgb}{0.000000,0.000000,0.000000}%
\pgfsetfillcolor{currentfill}%
\pgfsetlinewidth{0.803000pt}%
\definecolor{currentstroke}{rgb}{0.000000,0.000000,0.000000}%
\pgfsetstrokecolor{currentstroke}%
\pgfsetdash{}{0pt}%
\pgfsys@defobject{currentmarker}{\pgfqpoint{-0.048611in}{0.000000in}}{\pgfqpoint{-0.000000in}{0.000000in}}{%
\pgfpathmoveto{\pgfqpoint{-0.000000in}{0.000000in}}%
\pgfpathlineto{\pgfqpoint{-0.048611in}{0.000000in}}%
\pgfusepath{stroke,fill}%
}%
\begin{pgfscope}%
\pgfsys@transformshift{0.800000in}{2.460676in}%
\pgfsys@useobject{currentmarker}{}%
\end{pgfscope}%
\end{pgfscope}%
\begin{pgfscope}%
\definecolor{textcolor}{rgb}{0.000000,0.000000,0.000000}%
\pgfsetstrokecolor{textcolor}%
\pgfsetfillcolor{textcolor}%
\pgftext[x=0.412657in, y=2.399356in, left, base]{\color{textcolor}\rmfamily\fontsize{12.000000}{14.400000}\selectfont \(\displaystyle {0.15}\)}%
\end{pgfscope}%
\begin{pgfscope}%
\pgfpathrectangle{\pgfqpoint{0.800000in}{0.528000in}}{\pgfqpoint{4.960000in}{3.696000in}}%
\pgfusepath{clip}%
\pgfsetrectcap%
\pgfsetroundjoin%
\pgfsetlinewidth{0.803000pt}%
\definecolor{currentstroke}{rgb}{0.690196,0.690196,0.690196}%
\pgfsetstrokecolor{currentstroke}%
\pgfsetdash{}{0pt}%
\pgfpathmoveto{\pgfqpoint{0.800000in}{3.047916in}}%
\pgfpathlineto{\pgfqpoint{5.760000in}{3.047916in}}%
\pgfusepath{stroke}%
\end{pgfscope}%
\begin{pgfscope}%
\pgfsetbuttcap%
\pgfsetroundjoin%
\definecolor{currentfill}{rgb}{0.000000,0.000000,0.000000}%
\pgfsetfillcolor{currentfill}%
\pgfsetlinewidth{0.803000pt}%
\definecolor{currentstroke}{rgb}{0.000000,0.000000,0.000000}%
\pgfsetstrokecolor{currentstroke}%
\pgfsetdash{}{0pt}%
\pgfsys@defobject{currentmarker}{\pgfqpoint{-0.048611in}{0.000000in}}{\pgfqpoint{-0.000000in}{0.000000in}}{%
\pgfpathmoveto{\pgfqpoint{-0.000000in}{0.000000in}}%
\pgfpathlineto{\pgfqpoint{-0.048611in}{0.000000in}}%
\pgfusepath{stroke,fill}%
}%
\begin{pgfscope}%
\pgfsys@transformshift{0.800000in}{3.047916in}%
\pgfsys@useobject{currentmarker}{}%
\end{pgfscope}%
\end{pgfscope}%
\begin{pgfscope}%
\definecolor{textcolor}{rgb}{0.000000,0.000000,0.000000}%
\pgfsetstrokecolor{textcolor}%
\pgfsetfillcolor{textcolor}%
\pgftext[x=0.412657in, y=2.986596in, left, base]{\color{textcolor}\rmfamily\fontsize{12.000000}{14.400000}\selectfont \(\displaystyle {0.20}\)}%
\end{pgfscope}%
\begin{pgfscope}%
\pgfpathrectangle{\pgfqpoint{0.800000in}{0.528000in}}{\pgfqpoint{4.960000in}{3.696000in}}%
\pgfusepath{clip}%
\pgfsetrectcap%
\pgfsetroundjoin%
\pgfsetlinewidth{0.803000pt}%
\definecolor{currentstroke}{rgb}{0.690196,0.690196,0.690196}%
\pgfsetstrokecolor{currentstroke}%
\pgfsetdash{}{0pt}%
\pgfpathmoveto{\pgfqpoint{0.800000in}{3.635156in}}%
\pgfpathlineto{\pgfqpoint{5.760000in}{3.635156in}}%
\pgfusepath{stroke}%
\end{pgfscope}%
\begin{pgfscope}%
\pgfsetbuttcap%
\pgfsetroundjoin%
\definecolor{currentfill}{rgb}{0.000000,0.000000,0.000000}%
\pgfsetfillcolor{currentfill}%
\pgfsetlinewidth{0.803000pt}%
\definecolor{currentstroke}{rgb}{0.000000,0.000000,0.000000}%
\pgfsetstrokecolor{currentstroke}%
\pgfsetdash{}{0pt}%
\pgfsys@defobject{currentmarker}{\pgfqpoint{-0.048611in}{0.000000in}}{\pgfqpoint{-0.000000in}{0.000000in}}{%
\pgfpathmoveto{\pgfqpoint{-0.000000in}{0.000000in}}%
\pgfpathlineto{\pgfqpoint{-0.048611in}{0.000000in}}%
\pgfusepath{stroke,fill}%
}%
\begin{pgfscope}%
\pgfsys@transformshift{0.800000in}{3.635156in}%
\pgfsys@useobject{currentmarker}{}%
\end{pgfscope}%
\end{pgfscope}%
\begin{pgfscope}%
\definecolor{textcolor}{rgb}{0.000000,0.000000,0.000000}%
\pgfsetstrokecolor{textcolor}%
\pgfsetfillcolor{textcolor}%
\pgftext[x=0.412657in, y=3.573836in, left, base]{\color{textcolor}\rmfamily\fontsize{12.000000}{14.400000}\selectfont \(\displaystyle {0.25}\)}%
\end{pgfscope}%
\begin{pgfscope}%
\pgfpathrectangle{\pgfqpoint{0.800000in}{0.528000in}}{\pgfqpoint{4.960000in}{3.696000in}}%
\pgfusepath{clip}%
\pgfsetrectcap%
\pgfsetroundjoin%
\pgfsetlinewidth{0.803000pt}%
\definecolor{currentstroke}{rgb}{0.690196,0.690196,0.690196}%
\pgfsetstrokecolor{currentstroke}%
\pgfsetdash{}{0pt}%
\pgfpathmoveto{\pgfqpoint{0.800000in}{4.222396in}}%
\pgfpathlineto{\pgfqpoint{5.760000in}{4.222396in}}%
\pgfusepath{stroke}%
\end{pgfscope}%
\begin{pgfscope}%
\pgfsetbuttcap%
\pgfsetroundjoin%
\definecolor{currentfill}{rgb}{0.000000,0.000000,0.000000}%
\pgfsetfillcolor{currentfill}%
\pgfsetlinewidth{0.803000pt}%
\definecolor{currentstroke}{rgb}{0.000000,0.000000,0.000000}%
\pgfsetstrokecolor{currentstroke}%
\pgfsetdash{}{0pt}%
\pgfsys@defobject{currentmarker}{\pgfqpoint{-0.048611in}{0.000000in}}{\pgfqpoint{-0.000000in}{0.000000in}}{%
\pgfpathmoveto{\pgfqpoint{-0.000000in}{0.000000in}}%
\pgfpathlineto{\pgfqpoint{-0.048611in}{0.000000in}}%
\pgfusepath{stroke,fill}%
}%
\begin{pgfscope}%
\pgfsys@transformshift{0.800000in}{4.222396in}%
\pgfsys@useobject{currentmarker}{}%
\end{pgfscope}%
\end{pgfscope}%
\begin{pgfscope}%
\definecolor{textcolor}{rgb}{0.000000,0.000000,0.000000}%
\pgfsetstrokecolor{textcolor}%
\pgfsetfillcolor{textcolor}%
\pgftext[x=0.412657in, y=4.161076in, left, base]{\color{textcolor}\rmfamily\fontsize{12.000000}{14.400000}\selectfont \(\displaystyle {0.30}\)}%
\end{pgfscope}%
\begin{pgfscope}%
\definecolor{textcolor}{rgb}{0.000000,0.000000,0.000000}%
\pgfsetstrokecolor{textcolor}%
\pgfsetfillcolor{textcolor}%
\pgftext[x=0.357102in,y=2.376000in,,bottom,rotate=90.000000]{\color{textcolor}\rmfamily\fontsize{12.000000}{14.400000}\selectfont Signal intensity in \(\displaystyle 10^6\)}%
\end{pgfscope}%
\begin{pgfscope}%
\pgfpathrectangle{\pgfqpoint{0.800000in}{0.528000in}}{\pgfqpoint{4.960000in}{3.696000in}}%
\pgfusepath{clip}%
\pgfsetrectcap%
\pgfsetroundjoin%
\pgfsetlinewidth{1.505625pt}%
\definecolor{currentstroke}{rgb}{0.121569,0.466667,0.705882}%
\pgfsetstrokecolor{currentstroke}%
\pgfsetdash{}{0pt}%
\pgfpathmoveto{\pgfqpoint{1.025455in}{0.706952in}}%
\pgfpathlineto{\pgfqpoint{1.042271in}{0.705071in}}%
\pgfpathlineto{\pgfqpoint{1.059079in}{0.706257in}}%
\pgfpathlineto{\pgfqpoint{1.076865in}{0.704730in}}%
\pgfpathlineto{\pgfqpoint{1.111420in}{0.703853in}}%
\pgfpathlineto{\pgfqpoint{1.299303in}{0.700818in}}%
\pgfpathlineto{\pgfqpoint{1.315970in}{0.698935in}}%
\pgfpathlineto{\pgfqpoint{1.333607in}{0.699942in}}%
\pgfpathlineto{\pgfqpoint{1.366893in}{0.697764in}}%
\pgfpathlineto{\pgfqpoint{1.384500in}{0.698689in}}%
\pgfpathlineto{\pgfqpoint{1.417731in}{0.697862in}}%
\pgfpathlineto{\pgfqpoint{1.435309in}{0.698999in}}%
\pgfpathlineto{\pgfqpoint{1.451901in}{0.697293in}}%
\pgfpathlineto{\pgfqpoint{1.486033in}{0.698274in}}%
\pgfpathlineto{\pgfqpoint{1.502597in}{0.697743in}}%
\pgfpathlineto{\pgfqpoint{1.519152in}{0.698821in}}%
\pgfpathlineto{\pgfqpoint{1.587226in}{0.697567in}}%
\pgfpathlineto{\pgfqpoint{1.603735in}{0.699338in}}%
\pgfpathlineto{\pgfqpoint{1.637696in}{0.699864in}}%
\pgfpathlineto{\pgfqpoint{1.670651in}{0.702600in}}%
\pgfpathlineto{\pgfqpoint{1.704537in}{0.702636in}}%
\pgfpathlineto{\pgfqpoint{1.720982in}{0.703754in}}%
\pgfpathlineto{\pgfqpoint{1.737418in}{0.703157in}}%
\pgfpathlineto{\pgfqpoint{1.754811in}{0.705509in}}%
\pgfpathlineto{\pgfqpoint{1.771229in}{0.705467in}}%
\pgfpathlineto{\pgfqpoint{1.805003in}{0.709800in}}%
\pgfpathlineto{\pgfqpoint{1.821393in}{0.709866in}}%
\pgfpathlineto{\pgfqpoint{1.854148in}{0.714100in}}%
\pgfpathlineto{\pgfqpoint{1.887828in}{0.718962in}}%
\pgfpathlineto{\pgfqpoint{1.904174in}{0.721958in}}%
\pgfpathlineto{\pgfqpoint{1.937799in}{0.725619in}}%
\pgfpathlineto{\pgfqpoint{1.970427in}{0.732773in}}%
\pgfpathlineto{\pgfqpoint{2.020261in}{0.741991in}}%
\pgfpathlineto{\pgfqpoint{2.052800in}{0.749057in}}%
\pgfpathlineto{\pgfqpoint{2.152112in}{0.774814in}}%
\pgfpathlineto{\pgfqpoint{2.184508in}{0.787245in}}%
\pgfpathlineto{\pgfqpoint{2.200693in}{0.790699in}}%
\pgfpathlineto{\pgfqpoint{2.250146in}{0.809494in}}%
\pgfpathlineto{\pgfqpoint{2.266296in}{0.818384in}}%
\pgfpathlineto{\pgfqpoint{2.315641in}{0.841003in}}%
\pgfpathlineto{\pgfqpoint{2.380046in}{0.881351in}}%
\pgfpathlineto{\pgfqpoint{2.429203in}{0.915973in}}%
\pgfpathlineto{\pgfqpoint{2.493364in}{0.972596in}}%
\pgfpathlineto{\pgfqpoint{2.542336in}{1.020468in}}%
\pgfpathlineto{\pgfqpoint{2.558328in}{1.038874in}}%
\pgfpathlineto{\pgfqpoint{2.574312in}{1.060349in}}%
\pgfpathlineto{\pgfqpoint{2.606254in}{1.099537in}}%
\pgfpathlineto{\pgfqpoint{2.655040in}{1.168178in}}%
\pgfpathlineto{\pgfqpoint{2.686896in}{1.220233in}}%
\pgfpathlineto{\pgfqpoint{2.702810in}{1.250917in}}%
\pgfpathlineto{\pgfqpoint{2.718717in}{1.276905in}}%
\pgfpathlineto{\pgfqpoint{2.734615in}{1.307219in}}%
\pgfpathlineto{\pgfqpoint{2.750504in}{1.343044in}}%
\pgfpathlineto{\pgfqpoint{2.782257in}{1.408631in}}%
\pgfpathlineto{\pgfqpoint{2.813977in}{1.480385in}}%
\pgfpathlineto{\pgfqpoint{2.846594in}{1.558963in}}%
\pgfpathlineto{\pgfqpoint{2.878244in}{1.642956in}}%
\pgfpathlineto{\pgfqpoint{2.909861in}{1.734349in}}%
\pgfpathlineto{\pgfqpoint{2.925657in}{1.781318in}}%
\pgfpathlineto{\pgfqpoint{2.988756in}{1.994466in}}%
\pgfpathlineto{\pgfqpoint{3.051720in}{2.225540in}}%
\pgfpathlineto{\pgfqpoint{3.114551in}{2.464943in}}%
\pgfpathlineto{\pgfqpoint{3.130238in}{2.526845in}}%
\pgfpathlineto{\pgfqpoint{3.161586in}{2.641263in}}%
\pgfpathlineto{\pgfqpoint{3.193822in}{2.749809in}}%
\pgfpathlineto{\pgfqpoint{3.209467in}{2.806829in}}%
\pgfpathlineto{\pgfqpoint{3.256351in}{2.951107in}}%
\pgfpathlineto{\pgfqpoint{3.302243in}{3.068305in}}%
\pgfpathlineto{\pgfqpoint{3.317831in}{3.098029in}}%
\pgfpathlineto{\pgfqpoint{3.333410in}{3.132712in}}%
\pgfpathlineto{\pgfqpoint{3.348980in}{3.160938in}}%
\pgfpathlineto{\pgfqpoint{3.395643in}{3.231390in}}%
\pgfpathlineto{\pgfqpoint{3.442231in}{3.278686in}}%
\pgfpathlineto{\pgfqpoint{3.457744in}{3.292629in}}%
\pgfpathlineto{\pgfqpoint{3.473249in}{3.301090in}}%
\pgfpathlineto{\pgfqpoint{3.488746in}{3.312996in}}%
\pgfpathlineto{\pgfqpoint{3.504234in}{3.313090in}}%
\pgfpathlineto{\pgfqpoint{3.519715in}{3.317990in}}%
\pgfpathlineto{\pgfqpoint{3.550651in}{3.323314in}}%
\pgfpathlineto{\pgfqpoint{3.566107in}{3.325852in}}%
\pgfpathlineto{\pgfqpoint{3.581554in}{3.322438in}}%
\pgfpathlineto{\pgfqpoint{3.596994in}{3.326497in}}%
\pgfpathlineto{\pgfqpoint{3.612425in}{3.320693in}}%
\pgfpathlineto{\pgfqpoint{3.626942in}{3.322421in}}%
\pgfpathlineto{\pgfqpoint{3.642357in}{3.319619in}}%
\pgfpathlineto{\pgfqpoint{3.657765in}{3.322610in}}%
\pgfpathlineto{\pgfqpoint{3.673164in}{3.322256in}}%
\pgfpathlineto{\pgfqpoint{3.688555in}{3.323985in}}%
\pgfpathlineto{\pgfqpoint{3.719313in}{3.338117in}}%
\pgfpathlineto{\pgfqpoint{3.734680in}{3.347865in}}%
\pgfpathlineto{\pgfqpoint{3.750039in}{3.359343in}}%
\pgfpathlineto{\pgfqpoint{3.765390in}{3.374817in}}%
\pgfpathlineto{\pgfqpoint{3.779830in}{3.393219in}}%
\pgfpathlineto{\pgfqpoint{3.795166in}{3.409874in}}%
\pgfpathlineto{\pgfqpoint{3.825812in}{3.455760in}}%
\pgfpathlineto{\pgfqpoint{3.841123in}{3.480294in}}%
\pgfpathlineto{\pgfqpoint{3.856426in}{3.502059in}}%
\pgfpathlineto{\pgfqpoint{3.871721in}{3.529382in}}%
\pgfpathlineto{\pgfqpoint{3.901388in}{3.588032in}}%
\pgfpathlineto{\pgfqpoint{3.931922in}{3.653859in}}%
\pgfpathlineto{\pgfqpoint{3.962425in}{3.720544in}}%
\pgfpathlineto{\pgfqpoint{3.977664in}{3.748923in}}%
\pgfpathlineto{\pgfqpoint{4.007224in}{3.817835in}}%
\pgfpathlineto{\pgfqpoint{4.022440in}{3.849841in}}%
\pgfpathlineto{\pgfqpoint{4.037648in}{3.879020in}}%
\pgfpathlineto{\pgfqpoint{4.052848in}{3.912865in}}%
\pgfpathlineto{\pgfqpoint{4.083224in}{3.964720in}}%
\pgfpathlineto{\pgfqpoint{4.097507in}{3.980358in}}%
\pgfpathlineto{\pgfqpoint{4.112676in}{3.999300in}}%
\pgfpathlineto{\pgfqpoint{4.127836in}{4.015038in}}%
\pgfpathlineto{\pgfqpoint{4.142989in}{4.024261in}}%
\pgfpathlineto{\pgfqpoint{4.158134in}{4.035520in}}%
\pgfpathlineto{\pgfqpoint{4.172381in}{4.041320in}}%
\pgfpathlineto{\pgfqpoint{4.187510in}{4.044351in}}%
\pgfpathlineto{\pgfqpoint{4.202632in}{4.045812in}}%
\pgfpathlineto{\pgfqpoint{4.232852in}{4.039947in}}%
\pgfpathlineto{\pgfqpoint{4.247062in}{4.031940in}}%
\pgfpathlineto{\pgfqpoint{4.292310in}{3.997061in}}%
\pgfpathlineto{\pgfqpoint{4.307377in}{3.975137in}}%
\pgfpathlineto{\pgfqpoint{4.321550in}{3.948460in}}%
\pgfpathlineto{\pgfqpoint{4.336602in}{3.924228in}}%
\pgfpathlineto{\pgfqpoint{4.351646in}{3.895972in}}%
\pgfpathlineto{\pgfqpoint{4.366682in}{3.862527in}}%
\pgfpathlineto{\pgfqpoint{4.380827in}{3.825375in}}%
\pgfpathlineto{\pgfqpoint{4.395847in}{3.793299in}}%
\pgfpathlineto{\pgfqpoint{4.410861in}{3.756928in}}%
\pgfpathlineto{\pgfqpoint{4.425866in}{3.716508in}}%
\pgfpathlineto{\pgfqpoint{4.439981in}{3.673367in}}%
\pgfpathlineto{\pgfqpoint{4.454971in}{3.631920in}}%
\pgfpathlineto{\pgfqpoint{4.484928in}{3.537514in}}%
\pgfpathlineto{\pgfqpoint{4.513975in}{3.441653in}}%
\pgfpathlineto{\pgfqpoint{4.572857in}{3.228407in}}%
\pgfpathlineto{\pgfqpoint{4.704248in}{2.780126in}}%
\pgfpathlineto{\pgfqpoint{4.748784in}{2.641289in}}%
\pgfpathlineto{\pgfqpoint{4.777565in}{2.548199in}}%
\pgfpathlineto{\pgfqpoint{4.792381in}{2.504877in}}%
\pgfpathlineto{\pgfqpoint{4.806317in}{2.458293in}}%
\pgfpathlineto{\pgfqpoint{4.821118in}{2.415506in}}%
\pgfpathlineto{\pgfqpoint{4.849826in}{2.326113in}}%
\pgfpathlineto{\pgfqpoint{4.879374in}{2.241408in}}%
\pgfpathlineto{\pgfqpoint{4.893269in}{2.198120in}}%
\pgfpathlineto{\pgfqpoint{4.908024in}{2.157338in}}%
\pgfpathlineto{\pgfqpoint{4.922772in}{2.111602in}}%
\pgfpathlineto{\pgfqpoint{4.966105in}{1.958019in}}%
\pgfpathlineto{\pgfqpoint{5.023203in}{1.719572in}}%
\pgfpathlineto{\pgfqpoint{5.037891in}{1.654531in}}%
\pgfpathlineto{\pgfqpoint{5.051709in}{1.585123in}}%
\pgfpathlineto{\pgfqpoint{5.094846in}{1.389007in}}%
\pgfpathlineto{\pgfqpoint{5.123281in}{1.268415in}}%
\pgfpathlineto{\pgfqpoint{5.137918in}{1.214202in}}%
\pgfpathlineto{\pgfqpoint{5.166310in}{1.118035in}}%
\pgfpathlineto{\pgfqpoint{5.194674in}{1.042176in}}%
\pgfpathlineto{\pgfqpoint{5.209274in}{1.009273in}}%
\pgfpathlineto{\pgfqpoint{5.237595in}{0.952801in}}%
\pgfpathlineto{\pgfqpoint{5.265888in}{0.908087in}}%
\pgfpathlineto{\pgfqpoint{5.308702in}{0.854995in}}%
\pgfpathlineto{\pgfqpoint{5.336924in}{0.827191in}}%
\pgfpathlineto{\pgfqpoint{5.365118in}{0.805755in}}%
\pgfpathlineto{\pgfqpoint{5.393284in}{0.786657in}}%
\pgfpathlineto{\pgfqpoint{5.464003in}{0.754274in}}%
\pgfpathlineto{\pgfqpoint{5.478466in}{0.750091in}}%
\pgfpathlineto{\pgfqpoint{5.492071in}{0.744583in}}%
\pgfpathlineto{\pgfqpoint{5.534545in}{0.732861in}}%
\pgfpathlineto{\pgfqpoint{5.534545in}{0.732861in}}%
\pgfusepath{stroke}%
\end{pgfscope}%
\begin{pgfscope}%
\pgfpathrectangle{\pgfqpoint{0.800000in}{0.528000in}}{\pgfqpoint{4.960000in}{3.696000in}}%
\pgfusepath{clip}%
\pgfsetrectcap%
\pgfsetroundjoin%
\pgfsetlinewidth{1.505625pt}%
\definecolor{currentstroke}{rgb}{1.000000,0.498039,0.054902}%
\pgfsetstrokecolor{currentstroke}%
\pgfsetdash{}{0pt}%
\pgfpathmoveto{\pgfqpoint{1.025455in}{0.707756in}}%
\pgfpathlineto{\pgfqpoint{1.042271in}{0.706640in}}%
\pgfpathlineto{\pgfqpoint{1.076865in}{0.707058in}}%
\pgfpathlineto{\pgfqpoint{1.093654in}{0.705785in}}%
\pgfpathlineto{\pgfqpoint{1.111420in}{0.705697in}}%
\pgfpathlineto{\pgfqpoint{1.128190in}{0.704234in}}%
\pgfpathlineto{\pgfqpoint{1.144950in}{0.705529in}}%
\pgfpathlineto{\pgfqpoint{1.179428in}{0.703663in}}%
\pgfpathlineto{\pgfqpoint{1.213867in}{0.703635in}}%
\pgfpathlineto{\pgfqpoint{1.230580in}{0.702772in}}%
\pgfpathlineto{\pgfqpoint{1.248266in}{0.703130in}}%
\pgfpathlineto{\pgfqpoint{1.264961in}{0.701782in}}%
\pgfpathlineto{\pgfqpoint{1.299303in}{0.701923in}}%
\pgfpathlineto{\pgfqpoint{1.401120in}{0.698021in}}%
\pgfpathlineto{\pgfqpoint{1.468484in}{0.698329in}}%
\pgfpathlineto{\pgfqpoint{1.519152in}{0.696798in}}%
\pgfpathlineto{\pgfqpoint{1.536672in}{0.697751in}}%
\pgfpathlineto{\pgfqpoint{1.553208in}{0.696000in}}%
\pgfpathlineto{\pgfqpoint{1.569736in}{0.697312in}}%
\pgfpathlineto{\pgfqpoint{1.603735in}{0.697368in}}%
\pgfpathlineto{\pgfqpoint{1.688082in}{0.698484in}}%
\pgfpathlineto{\pgfqpoint{1.737418in}{0.698306in}}%
\pgfpathlineto{\pgfqpoint{1.771229in}{0.701056in}}%
\pgfpathlineto{\pgfqpoint{1.787638in}{0.700504in}}%
\pgfpathlineto{\pgfqpoint{1.904174in}{0.710378in}}%
\pgfpathlineto{\pgfqpoint{1.920511in}{0.710530in}}%
\pgfpathlineto{\pgfqpoint{1.954117in}{0.714595in}}%
\pgfpathlineto{\pgfqpoint{2.020261in}{0.723285in}}%
\pgfpathlineto{\pgfqpoint{2.052800in}{0.727241in}}%
\pgfpathlineto{\pgfqpoint{2.070012in}{0.731123in}}%
\pgfpathlineto{\pgfqpoint{2.134947in}{0.740814in}}%
\pgfpathlineto{\pgfqpoint{2.152112in}{0.745665in}}%
\pgfpathlineto{\pgfqpoint{2.168314in}{0.747852in}}%
\pgfpathlineto{\pgfqpoint{2.266296in}{0.770603in}}%
\pgfpathlineto{\pgfqpoint{2.282437in}{0.775856in}}%
\pgfpathlineto{\pgfqpoint{2.315641in}{0.784958in}}%
\pgfpathlineto{\pgfqpoint{2.396126in}{0.815869in}}%
\pgfpathlineto{\pgfqpoint{2.413142in}{0.824534in}}%
\pgfpathlineto{\pgfqpoint{2.429203in}{0.830981in}}%
\pgfpathlineto{\pgfqpoint{2.493364in}{0.863916in}}%
\pgfpathlineto{\pgfqpoint{2.526335in}{0.882996in}}%
\pgfpathlineto{\pgfqpoint{2.574312in}{0.914996in}}%
\pgfpathlineto{\pgfqpoint{2.622212in}{0.949431in}}%
\pgfpathlineto{\pgfqpoint{2.670972in}{0.990823in}}%
\pgfpathlineto{\pgfqpoint{2.734615in}{1.054041in}}%
\pgfpathlineto{\pgfqpoint{2.750504in}{1.071988in}}%
\pgfpathlineto{\pgfqpoint{2.782257in}{1.111005in}}%
\pgfpathlineto{\pgfqpoint{2.830756in}{1.171300in}}%
\pgfpathlineto{\pgfqpoint{2.878244in}{1.240798in}}%
\pgfpathlineto{\pgfqpoint{2.909861in}{1.292542in}}%
\pgfpathlineto{\pgfqpoint{2.941444in}{1.346908in}}%
\pgfpathlineto{\pgfqpoint{2.972994in}{1.409967in}}%
\pgfpathlineto{\pgfqpoint{2.988756in}{1.440650in}}%
\pgfpathlineto{\pgfqpoint{3.035992in}{1.546134in}}%
\pgfpathlineto{\pgfqpoint{3.051720in}{1.582081in}}%
\pgfpathlineto{\pgfqpoint{3.067441in}{1.622906in}}%
\pgfpathlineto{\pgfqpoint{3.083152in}{1.660458in}}%
\pgfpathlineto{\pgfqpoint{3.145916in}{1.827636in}}%
\pgfpathlineto{\pgfqpoint{3.161586in}{1.872241in}}%
\pgfpathlineto{\pgfqpoint{3.178169in}{1.915546in}}%
\pgfpathlineto{\pgfqpoint{3.240731in}{2.097210in}}%
\pgfpathlineto{\pgfqpoint{3.256351in}{2.139488in}}%
\pgfpathlineto{\pgfqpoint{3.286648in}{2.224959in}}%
\pgfpathlineto{\pgfqpoint{3.333410in}{2.349087in}}%
\pgfpathlineto{\pgfqpoint{3.348980in}{2.385949in}}%
\pgfpathlineto{\pgfqpoint{3.380097in}{2.452344in}}%
\pgfpathlineto{\pgfqpoint{3.426710in}{2.541939in}}%
\pgfpathlineto{\pgfqpoint{3.442231in}{2.567313in}}%
\pgfpathlineto{\pgfqpoint{3.457744in}{2.589849in}}%
\pgfpathlineto{\pgfqpoint{3.473249in}{2.615254in}}%
\pgfpathlineto{\pgfqpoint{3.504234in}{2.648882in}}%
\pgfpathlineto{\pgfqpoint{3.535187in}{2.679929in}}%
\pgfpathlineto{\pgfqpoint{3.566107in}{2.704555in}}%
\pgfpathlineto{\pgfqpoint{3.581554in}{2.719322in}}%
\pgfpathlineto{\pgfqpoint{3.612425in}{2.739461in}}%
\pgfpathlineto{\pgfqpoint{3.626942in}{2.751469in}}%
\pgfpathlineto{\pgfqpoint{3.642357in}{2.760461in}}%
\pgfpathlineto{\pgfqpoint{3.657765in}{2.771654in}}%
\pgfpathlineto{\pgfqpoint{3.703938in}{2.817858in}}%
\pgfpathlineto{\pgfqpoint{3.734680in}{2.857003in}}%
\pgfpathlineto{\pgfqpoint{3.765390in}{2.907027in}}%
\pgfpathlineto{\pgfqpoint{3.810493in}{2.998108in}}%
\pgfpathlineto{\pgfqpoint{3.825812in}{3.031153in}}%
\pgfpathlineto{\pgfqpoint{3.887007in}{3.182450in}}%
\pgfpathlineto{\pgfqpoint{3.931922in}{3.318055in}}%
\pgfpathlineto{\pgfqpoint{3.947178in}{3.360270in}}%
\pgfpathlineto{\pgfqpoint{4.052848in}{3.685989in}}%
\pgfpathlineto{\pgfqpoint{4.083224in}{3.767587in}}%
\pgfpathlineto{\pgfqpoint{4.127836in}{3.866295in}}%
\pgfpathlineto{\pgfqpoint{4.142989in}{3.895362in}}%
\pgfpathlineto{\pgfqpoint{4.158134in}{3.917808in}}%
\pgfpathlineto{\pgfqpoint{4.172381in}{3.943019in}}%
\pgfpathlineto{\pgfqpoint{4.187510in}{3.961927in}}%
\pgfpathlineto{\pgfqpoint{4.202632in}{3.983469in}}%
\pgfpathlineto{\pgfqpoint{4.217746in}{4.001152in}}%
\pgfpathlineto{\pgfqpoint{4.232852in}{4.014775in}}%
\pgfpathlineto{\pgfqpoint{4.247062in}{4.029836in}}%
\pgfpathlineto{\pgfqpoint{4.262152in}{4.038863in}}%
\pgfpathlineto{\pgfqpoint{4.277235in}{4.051102in}}%
\pgfpathlineto{\pgfqpoint{4.307377in}{4.056000in}}%
\pgfpathlineto{\pgfqpoint{4.321550in}{4.055355in}}%
\pgfpathlineto{\pgfqpoint{4.336602in}{4.050709in}}%
\pgfpathlineto{\pgfqpoint{4.366682in}{4.037131in}}%
\pgfpathlineto{\pgfqpoint{4.395847in}{4.006646in}}%
\pgfpathlineto{\pgfqpoint{4.410861in}{3.993823in}}%
\pgfpathlineto{\pgfqpoint{4.425866in}{3.975367in}}%
\pgfpathlineto{\pgfqpoint{4.454971in}{3.930728in}}%
\pgfpathlineto{\pgfqpoint{4.484928in}{3.873481in}}%
\pgfpathlineto{\pgfqpoint{4.513975in}{3.807547in}}%
\pgfpathlineto{\pgfqpoint{4.557928in}{3.687345in}}%
\pgfpathlineto{\pgfqpoint{4.587778in}{3.599868in}}%
\pgfpathlineto{\pgfqpoint{4.631619in}{3.467575in}}%
\pgfpathlineto{\pgfqpoint{4.690262in}{3.273792in}}%
\pgfpathlineto{\pgfqpoint{4.719101in}{3.172029in}}%
\pgfpathlineto{\pgfqpoint{4.733947in}{3.122623in}}%
\pgfpathlineto{\pgfqpoint{4.879374in}{2.578462in}}%
\pgfpathlineto{\pgfqpoint{4.908024in}{2.468124in}}%
\pgfpathlineto{\pgfqpoint{4.922772in}{2.409720in}}%
\pgfpathlineto{\pgfqpoint{4.951379in}{2.284516in}}%
\pgfpathlineto{\pgfqpoint{4.994668in}{2.071223in}}%
\pgfpathlineto{\pgfqpoint{5.008507in}{1.995276in}}%
\pgfpathlineto{\pgfqpoint{5.023203in}{1.921495in}}%
\pgfpathlineto{\pgfqpoint{5.066383in}{1.686707in}}%
\pgfpathlineto{\pgfqpoint{5.081049in}{1.613895in}}%
\pgfpathlineto{\pgfqpoint{5.094846in}{1.537318in}}%
\pgfpathlineto{\pgfqpoint{5.123281in}{1.399833in}}%
\pgfpathlineto{\pgfqpoint{5.137918in}{1.334874in}}%
\pgfpathlineto{\pgfqpoint{5.166310in}{1.223819in}}%
\pgfpathlineto{\pgfqpoint{5.194674in}{1.130329in}}%
\pgfpathlineto{\pgfqpoint{5.223868in}{1.053735in}}%
\pgfpathlineto{\pgfqpoint{5.237595in}{1.019721in}}%
\pgfpathlineto{\pgfqpoint{5.265888in}{0.961900in}}%
\pgfpathlineto{\pgfqpoint{5.294153in}{0.915431in}}%
\pgfpathlineto{\pgfqpoint{5.308702in}{0.894171in}}%
\pgfpathlineto{\pgfqpoint{5.336924in}{0.860005in}}%
\pgfpathlineto{\pgfqpoint{5.365118in}{0.832187in}}%
\pgfpathlineto{\pgfqpoint{5.393284in}{0.809181in}}%
\pgfpathlineto{\pgfqpoint{5.421423in}{0.790268in}}%
\pgfpathlineto{\pgfqpoint{5.464003in}{0.767033in}}%
\pgfpathlineto{\pgfqpoint{5.534545in}{0.741295in}}%
\pgfpathlineto{\pgfqpoint{5.534545in}{0.741295in}}%
\pgfusepath{stroke}%
\end{pgfscope}%
\begin{pgfscope}%
\pgfpathrectangle{\pgfqpoint{0.800000in}{0.528000in}}{\pgfqpoint{4.960000in}{3.696000in}}%
\pgfusepath{clip}%
\pgfsetbuttcap%
\pgfsetroundjoin%
\pgfsetlinewidth{1.505625pt}%
\definecolor{currentstroke}{rgb}{0.501961,0.000000,0.501961}%
\pgfsetstrokecolor{currentstroke}%
\pgfsetdash{{5.550000pt}{2.400000pt}}{0.000000pt}%
\pgfpathmoveto{\pgfqpoint{3.994324in}{0.528000in}}%
\pgfpathlineto{\pgfqpoint{3.994324in}{4.224000in}}%
\pgfusepath{stroke}%
\end{pgfscope}%
\begin{pgfscope}%
\pgfsetrectcap%
\pgfsetmiterjoin%
\pgfsetlinewidth{0.803000pt}%
\definecolor{currentstroke}{rgb}{0.000000,0.000000,0.000000}%
\pgfsetstrokecolor{currentstroke}%
\pgfsetdash{}{0pt}%
\pgfpathmoveto{\pgfqpoint{0.800000in}{0.528000in}}%
\pgfpathlineto{\pgfqpoint{0.800000in}{4.224000in}}%
\pgfusepath{stroke}%
\end{pgfscope}%
\begin{pgfscope}%
\pgfsetrectcap%
\pgfsetmiterjoin%
\pgfsetlinewidth{0.803000pt}%
\definecolor{currentstroke}{rgb}{0.000000,0.000000,0.000000}%
\pgfsetstrokecolor{currentstroke}%
\pgfsetdash{}{0pt}%
\pgfpathmoveto{\pgfqpoint{5.760000in}{0.528000in}}%
\pgfpathlineto{\pgfqpoint{5.760000in}{4.224000in}}%
\pgfusepath{stroke}%
\end{pgfscope}%
\begin{pgfscope}%
\pgfsetrectcap%
\pgfsetmiterjoin%
\pgfsetlinewidth{0.803000pt}%
\definecolor{currentstroke}{rgb}{0.000000,0.000000,0.000000}%
\pgfsetstrokecolor{currentstroke}%
\pgfsetdash{}{0pt}%
\pgfpathmoveto{\pgfqpoint{0.800000in}{0.528000in}}%
\pgfpathlineto{\pgfqpoint{5.760000in}{0.528000in}}%
\pgfusepath{stroke}%
\end{pgfscope}%
\begin{pgfscope}%
\pgfsetrectcap%
\pgfsetmiterjoin%
\pgfsetlinewidth{0.803000pt}%
\definecolor{currentstroke}{rgb}{0.000000,0.000000,0.000000}%
\pgfsetstrokecolor{currentstroke}%
\pgfsetdash{}{0pt}%
\pgfpathmoveto{\pgfqpoint{0.800000in}{4.224000in}}%
\pgfpathlineto{\pgfqpoint{5.760000in}{4.224000in}}%
\pgfusepath{stroke}%
\end{pgfscope}%
\begin{pgfscope}%
\pgfsetbuttcap%
\pgfsetmiterjoin%
\definecolor{currentfill}{rgb}{1.000000,1.000000,1.000000}%
\pgfsetfillcolor{currentfill}%
\pgfsetfillopacity{0.800000}%
\pgfsetlinewidth{1.003750pt}%
\definecolor{currentstroke}{rgb}{0.800000,0.800000,0.800000}%
\pgfsetstrokecolor{currentstroke}%
\pgfsetstrokeopacity{0.800000}%
\pgfsetdash{}{0pt}%
\pgfpathmoveto{\pgfqpoint{0.916667in}{3.363860in}}%
\pgfpathlineto{\pgfqpoint{2.559619in}{3.363860in}}%
\pgfpathquadraticcurveto{\pgfqpoint{2.592952in}{3.363860in}}{\pgfqpoint{2.592952in}{3.397194in}}%
\pgfpathlineto{\pgfqpoint{2.592952in}{4.107333in}}%
\pgfpathquadraticcurveto{\pgfqpoint{2.592952in}{4.140667in}}{\pgfqpoint{2.559619in}{4.140667in}}%
\pgfpathlineto{\pgfqpoint{0.916667in}{4.140667in}}%
\pgfpathquadraticcurveto{\pgfqpoint{0.883333in}{4.140667in}}{\pgfqpoint{0.883333in}{4.107333in}}%
\pgfpathlineto{\pgfqpoint{0.883333in}{3.397194in}}%
\pgfpathquadraticcurveto{\pgfqpoint{0.883333in}{3.363860in}}{\pgfqpoint{0.916667in}{3.363860in}}%
\pgfpathlineto{\pgfqpoint{0.916667in}{3.363860in}}%
\pgfpathclose%
\pgfusepath{stroke,fill}%
\end{pgfscope}%
\begin{pgfscope}%
\pgfsetrectcap%
\pgfsetroundjoin%
\pgfsetlinewidth{1.505625pt}%
\definecolor{currentstroke}{rgb}{0.121569,0.466667,0.705882}%
\pgfsetstrokecolor{currentstroke}%
\pgfsetdash{}{0pt}%
\pgfpathmoveto{\pgfqpoint{0.950000in}{4.009693in}}%
\pgfpathlineto{\pgfqpoint{1.116667in}{4.009693in}}%
\pgfpathlineto{\pgfqpoint{1.283333in}{4.009693in}}%
\pgfusepath{stroke}%
\end{pgfscope}%
\begin{pgfscope}%
\definecolor{textcolor}{rgb}{0.000000,0.000000,0.000000}%
\pgfsetstrokecolor{textcolor}%
\pgfsetfillcolor{textcolor}%
\pgftext[x=1.416667in,y=3.951360in,left,base]{\color{textcolor}\rmfamily\fontsize{12.000000}{14.400000}\selectfont \(\displaystyle T_\mathrm{K}=293,35~\mathrm{K}\)}%
\end{pgfscope}%
\begin{pgfscope}%
\pgfsetrectcap%
\pgfsetroundjoin%
\pgfsetlinewidth{1.505625pt}%
\definecolor{currentstroke}{rgb}{1.000000,0.498039,0.054902}%
\pgfsetstrokecolor{currentstroke}%
\pgfsetdash{}{0pt}%
\pgfpathmoveto{\pgfqpoint{0.950000in}{3.767425in}}%
\pgfpathlineto{\pgfqpoint{1.116667in}{3.767425in}}%
\pgfpathlineto{\pgfqpoint{1.283333in}{3.767425in}}%
\pgfusepath{stroke}%
\end{pgfscope}%
\begin{pgfscope}%
\definecolor{textcolor}{rgb}{0.000000,0.000000,0.000000}%
\pgfsetstrokecolor{textcolor}%
\pgfsetfillcolor{textcolor}%
\pgftext[x=1.416667in,y=3.709091in,left,base]{\color{textcolor}\rmfamily\fontsize{12.000000}{14.400000}\selectfont \(\displaystyle T_\mathrm{K}=346,65~\mathrm{K}\)}%
\end{pgfscope}%
\begin{pgfscope}%
\pgfsetbuttcap%
\pgfsetroundjoin%
\pgfsetlinewidth{1.505625pt}%
\definecolor{currentstroke}{rgb}{0.501961,0.000000,0.501961}%
\pgfsetstrokecolor{currentstroke}%
\pgfsetdash{{5.550000pt}{2.400000pt}}{0.000000pt}%
\pgfpathmoveto{\pgfqpoint{0.950000in}{3.525156in}}%
\pgfpathlineto{\pgfqpoint{1.116667in}{3.525156in}}%
\pgfpathlineto{\pgfqpoint{1.283333in}{3.525156in}}%
\pgfusepath{stroke}%
\end{pgfscope}%
\begin{pgfscope}%
\definecolor{textcolor}{rgb}{0.000000,0.000000,0.000000}%
\pgfsetstrokecolor{textcolor}%
\pgfsetfillcolor{textcolor}%
\pgftext[x=1.416667in,y=3.466822in,left,base]{\color{textcolor}\rmfamily\fontsize{12.000000}{14.400000}\selectfont isosbestic point}%
\end{pgfscope}%
\end{pgfpicture}%
\makeatother%
\endgroup%

    \caption[Absolute Raman shift intensities for the lowest and highest temperatures]{Plot of the absolute scattering intensities over the Raman shift $\Delta \tilde{v}$; Plot at the highest recorded temperature and the lowest recorded temperature for simplification purposes}
    \label{fig:plot-temp-raw}
\end{figure}

%%%%%%%%%%%%%%%%%%%%%%%%%%%%%%%
\section{Evaluation and error discussion}
\label{sec:eval}

\subsection*{Species determination}
\label{subsec:spec-eval}

\commenting{
    Text and data from JP.

    Plots need to be integrated.
}

\subsection*{Temperature calculation}
\label{subsec:temp-eval}

The plot of the change of scattered intensities over the Raman shift is illustrated in \autoref{fig:plot-temp}. The isosbestic point can be determined at roughly $3390 \mathrm{cm^{-1}}$, via the lowest standard deviation of all the normalized data sets. The areas can be interpreted as the integration of the curve within the boundaries $[3390~\mathrm{cm^{-1}},~3800~\mathrm{cm^{-1}}]$ for $A_\mathrm{right}$, and $[2600~\mathrm{cm^{-1}},~3390~\mathrm{cm^{-1}}]$ for $A_\mathrm{left}$. Due to the already integrating behavior of the CCD sensor, it is calculated as the sum of all data points within the presented ranges.

\begin{figure}[!htb]
    \centering
    % \includegraphics[width=\textwidth]{02kapitel/temp-shift.png}
    %% Creator: Matplotlib, PGF backend
%%
%% To include the figure in your LaTeX document, write
%%   \input{<filename>.pgf}
%%
%% Make sure the required packages are loaded in your preamble
%%   \usepackage{pgf}
%%
%% Also ensure that all the required font packages are loaded; for instance,
%% the lmodern package is sometimes necessary when using math font.
%%   \usepackage{lmodern}
%%
%% Figures using additional raster images can only be included by \input if
%% they are in the same directory as the main LaTeX file. For loading figures
%% from other directories you can use the `import` package
%%   \usepackage{import}
%%
%% and then include the figures with
%%   \import{<path to file>}{<filename>.pgf}
%%
%% Matplotlib used the following preamble
%%   
%%   \makeatletter\@ifpackageloaded{underscore}{}{\usepackage[strings]{underscore}}\makeatother
%%
\begingroup%
\makeatletter%
\begin{pgfpicture}%
\pgfpathrectangle{\pgfpointorigin}{\pgfqpoint{6.000000in}{4.000000in}}%
\pgfusepath{use as bounding box, clip}%
\begin{pgfscope}%
\pgfsetbuttcap%
\pgfsetmiterjoin%
\definecolor{currentfill}{rgb}{1.000000,1.000000,1.000000}%
\pgfsetfillcolor{currentfill}%
\pgfsetlinewidth{0.000000pt}%
\definecolor{currentstroke}{rgb}{1.000000,1.000000,1.000000}%
\pgfsetstrokecolor{currentstroke}%
\pgfsetdash{}{0pt}%
\pgfpathmoveto{\pgfqpoint{0.000000in}{0.000000in}}%
\pgfpathlineto{\pgfqpoint{6.000000in}{0.000000in}}%
\pgfpathlineto{\pgfqpoint{6.000000in}{4.000000in}}%
\pgfpathlineto{\pgfqpoint{0.000000in}{4.000000in}}%
\pgfpathlineto{\pgfqpoint{0.000000in}{0.000000in}}%
\pgfpathclose%
\pgfusepath{fill}%
\end{pgfscope}%
\begin{pgfscope}%
\pgfsetbuttcap%
\pgfsetmiterjoin%
\definecolor{currentfill}{rgb}{1.000000,1.000000,1.000000}%
\pgfsetfillcolor{currentfill}%
\pgfsetlinewidth{0.000000pt}%
\definecolor{currentstroke}{rgb}{0.000000,0.000000,0.000000}%
\pgfsetstrokecolor{currentstroke}%
\pgfsetstrokeopacity{0.000000}%
\pgfsetdash{}{0pt}%
\pgfpathmoveto{\pgfqpoint{0.750000in}{0.500000in}}%
\pgfpathlineto{\pgfqpoint{5.400000in}{0.500000in}}%
\pgfpathlineto{\pgfqpoint{5.400000in}{3.520000in}}%
\pgfpathlineto{\pgfqpoint{0.750000in}{3.520000in}}%
\pgfpathlineto{\pgfqpoint{0.750000in}{0.500000in}}%
\pgfpathclose%
\pgfusepath{fill}%
\end{pgfscope}%
\begin{pgfscope}%
\pgfpathrectangle{\pgfqpoint{0.750000in}{0.500000in}}{\pgfqpoint{4.650000in}{3.020000in}}%
\pgfusepath{clip}%
\pgfsetrectcap%
\pgfsetroundjoin%
\pgfsetlinewidth{0.803000pt}%
\definecolor{currentstroke}{rgb}{0.690196,0.690196,0.690196}%
\pgfsetstrokecolor{currentstroke}%
\pgfsetdash{}{0pt}%
\pgfpathmoveto{\pgfqpoint{0.950097in}{0.500000in}}%
\pgfpathlineto{\pgfqpoint{0.950097in}{3.520000in}}%
\pgfusepath{stroke}%
\end{pgfscope}%
\begin{pgfscope}%
\pgfsetbuttcap%
\pgfsetroundjoin%
\definecolor{currentfill}{rgb}{0.000000,0.000000,0.000000}%
\pgfsetfillcolor{currentfill}%
\pgfsetlinewidth{0.803000pt}%
\definecolor{currentstroke}{rgb}{0.000000,0.000000,0.000000}%
\pgfsetstrokecolor{currentstroke}%
\pgfsetdash{}{0pt}%
\pgfsys@defobject{currentmarker}{\pgfqpoint{0.000000in}{-0.048611in}}{\pgfqpoint{0.000000in}{0.000000in}}{%
\pgfpathmoveto{\pgfqpoint{0.000000in}{0.000000in}}%
\pgfpathlineto{\pgfqpoint{0.000000in}{-0.048611in}}%
\pgfusepath{stroke,fill}%
}%
\begin{pgfscope}%
\pgfsys@transformshift{0.950097in}{0.500000in}%
\pgfsys@useobject{currentmarker}{}%
\end{pgfscope}%
\end{pgfscope}%
\begin{pgfscope}%
\definecolor{textcolor}{rgb}{0.000000,0.000000,0.000000}%
\pgfsetstrokecolor{textcolor}%
\pgfsetfillcolor{textcolor}%
\pgftext[x=0.950097in,y=0.402778in,,top]{\color{textcolor}\rmfamily\fontsize{12.000000}{14.400000}\selectfont \(\displaystyle {2600}\)}%
\end{pgfscope}%
\begin{pgfscope}%
\pgfpathrectangle{\pgfqpoint{0.750000in}{0.500000in}}{\pgfqpoint{4.650000in}{3.020000in}}%
\pgfusepath{clip}%
\pgfsetrectcap%
\pgfsetroundjoin%
\pgfsetlinewidth{0.803000pt}%
\definecolor{currentstroke}{rgb}{0.690196,0.690196,0.690196}%
\pgfsetstrokecolor{currentstroke}%
\pgfsetdash{}{0pt}%
\pgfpathmoveto{\pgfqpoint{1.657586in}{0.500000in}}%
\pgfpathlineto{\pgfqpoint{1.657586in}{3.520000in}}%
\pgfusepath{stroke}%
\end{pgfscope}%
\begin{pgfscope}%
\pgfsetbuttcap%
\pgfsetroundjoin%
\definecolor{currentfill}{rgb}{0.000000,0.000000,0.000000}%
\pgfsetfillcolor{currentfill}%
\pgfsetlinewidth{0.803000pt}%
\definecolor{currentstroke}{rgb}{0.000000,0.000000,0.000000}%
\pgfsetstrokecolor{currentstroke}%
\pgfsetdash{}{0pt}%
\pgfsys@defobject{currentmarker}{\pgfqpoint{0.000000in}{-0.048611in}}{\pgfqpoint{0.000000in}{0.000000in}}{%
\pgfpathmoveto{\pgfqpoint{0.000000in}{0.000000in}}%
\pgfpathlineto{\pgfqpoint{0.000000in}{-0.048611in}}%
\pgfusepath{stroke,fill}%
}%
\begin{pgfscope}%
\pgfsys@transformshift{1.657586in}{0.500000in}%
\pgfsys@useobject{currentmarker}{}%
\end{pgfscope}%
\end{pgfscope}%
\begin{pgfscope}%
\definecolor{textcolor}{rgb}{0.000000,0.000000,0.000000}%
\pgfsetstrokecolor{textcolor}%
\pgfsetfillcolor{textcolor}%
\pgftext[x=1.657586in,y=0.402778in,,top]{\color{textcolor}\rmfamily\fontsize{12.000000}{14.400000}\selectfont \(\displaystyle {2800}\)}%
\end{pgfscope}%
\begin{pgfscope}%
\pgfpathrectangle{\pgfqpoint{0.750000in}{0.500000in}}{\pgfqpoint{4.650000in}{3.020000in}}%
\pgfusepath{clip}%
\pgfsetrectcap%
\pgfsetroundjoin%
\pgfsetlinewidth{0.803000pt}%
\definecolor{currentstroke}{rgb}{0.690196,0.690196,0.690196}%
\pgfsetstrokecolor{currentstroke}%
\pgfsetdash{}{0pt}%
\pgfpathmoveto{\pgfqpoint{2.365075in}{0.500000in}}%
\pgfpathlineto{\pgfqpoint{2.365075in}{3.520000in}}%
\pgfusepath{stroke}%
\end{pgfscope}%
\begin{pgfscope}%
\pgfsetbuttcap%
\pgfsetroundjoin%
\definecolor{currentfill}{rgb}{0.000000,0.000000,0.000000}%
\pgfsetfillcolor{currentfill}%
\pgfsetlinewidth{0.803000pt}%
\definecolor{currentstroke}{rgb}{0.000000,0.000000,0.000000}%
\pgfsetstrokecolor{currentstroke}%
\pgfsetdash{}{0pt}%
\pgfsys@defobject{currentmarker}{\pgfqpoint{0.000000in}{-0.048611in}}{\pgfqpoint{0.000000in}{0.000000in}}{%
\pgfpathmoveto{\pgfqpoint{0.000000in}{0.000000in}}%
\pgfpathlineto{\pgfqpoint{0.000000in}{-0.048611in}}%
\pgfusepath{stroke,fill}%
}%
\begin{pgfscope}%
\pgfsys@transformshift{2.365075in}{0.500000in}%
\pgfsys@useobject{currentmarker}{}%
\end{pgfscope}%
\end{pgfscope}%
\begin{pgfscope}%
\definecolor{textcolor}{rgb}{0.000000,0.000000,0.000000}%
\pgfsetstrokecolor{textcolor}%
\pgfsetfillcolor{textcolor}%
\pgftext[x=2.365075in,y=0.402778in,,top]{\color{textcolor}\rmfamily\fontsize{12.000000}{14.400000}\selectfont \(\displaystyle {3000}\)}%
\end{pgfscope}%
\begin{pgfscope}%
\pgfpathrectangle{\pgfqpoint{0.750000in}{0.500000in}}{\pgfqpoint{4.650000in}{3.020000in}}%
\pgfusepath{clip}%
\pgfsetrectcap%
\pgfsetroundjoin%
\pgfsetlinewidth{0.803000pt}%
\definecolor{currentstroke}{rgb}{0.690196,0.690196,0.690196}%
\pgfsetstrokecolor{currentstroke}%
\pgfsetdash{}{0pt}%
\pgfpathmoveto{\pgfqpoint{3.072564in}{0.500000in}}%
\pgfpathlineto{\pgfqpoint{3.072564in}{3.520000in}}%
\pgfusepath{stroke}%
\end{pgfscope}%
\begin{pgfscope}%
\pgfsetbuttcap%
\pgfsetroundjoin%
\definecolor{currentfill}{rgb}{0.000000,0.000000,0.000000}%
\pgfsetfillcolor{currentfill}%
\pgfsetlinewidth{0.803000pt}%
\definecolor{currentstroke}{rgb}{0.000000,0.000000,0.000000}%
\pgfsetstrokecolor{currentstroke}%
\pgfsetdash{}{0pt}%
\pgfsys@defobject{currentmarker}{\pgfqpoint{0.000000in}{-0.048611in}}{\pgfqpoint{0.000000in}{0.000000in}}{%
\pgfpathmoveto{\pgfqpoint{0.000000in}{0.000000in}}%
\pgfpathlineto{\pgfqpoint{0.000000in}{-0.048611in}}%
\pgfusepath{stroke,fill}%
}%
\begin{pgfscope}%
\pgfsys@transformshift{3.072564in}{0.500000in}%
\pgfsys@useobject{currentmarker}{}%
\end{pgfscope}%
\end{pgfscope}%
\begin{pgfscope}%
\definecolor{textcolor}{rgb}{0.000000,0.000000,0.000000}%
\pgfsetstrokecolor{textcolor}%
\pgfsetfillcolor{textcolor}%
\pgftext[x=3.072564in,y=0.402778in,,top]{\color{textcolor}\rmfamily\fontsize{12.000000}{14.400000}\selectfont \(\displaystyle {3200}\)}%
\end{pgfscope}%
\begin{pgfscope}%
\pgfpathrectangle{\pgfqpoint{0.750000in}{0.500000in}}{\pgfqpoint{4.650000in}{3.020000in}}%
\pgfusepath{clip}%
\pgfsetrectcap%
\pgfsetroundjoin%
\pgfsetlinewidth{0.803000pt}%
\definecolor{currentstroke}{rgb}{0.690196,0.690196,0.690196}%
\pgfsetstrokecolor{currentstroke}%
\pgfsetdash{}{0pt}%
\pgfpathmoveto{\pgfqpoint{3.780053in}{0.500000in}}%
\pgfpathlineto{\pgfqpoint{3.780053in}{3.520000in}}%
\pgfusepath{stroke}%
\end{pgfscope}%
\begin{pgfscope}%
\pgfsetbuttcap%
\pgfsetroundjoin%
\definecolor{currentfill}{rgb}{0.000000,0.000000,0.000000}%
\pgfsetfillcolor{currentfill}%
\pgfsetlinewidth{0.803000pt}%
\definecolor{currentstroke}{rgb}{0.000000,0.000000,0.000000}%
\pgfsetstrokecolor{currentstroke}%
\pgfsetdash{}{0pt}%
\pgfsys@defobject{currentmarker}{\pgfqpoint{0.000000in}{-0.048611in}}{\pgfqpoint{0.000000in}{0.000000in}}{%
\pgfpathmoveto{\pgfqpoint{0.000000in}{0.000000in}}%
\pgfpathlineto{\pgfqpoint{0.000000in}{-0.048611in}}%
\pgfusepath{stroke,fill}%
}%
\begin{pgfscope}%
\pgfsys@transformshift{3.780053in}{0.500000in}%
\pgfsys@useobject{currentmarker}{}%
\end{pgfscope}%
\end{pgfscope}%
\begin{pgfscope}%
\definecolor{textcolor}{rgb}{0.000000,0.000000,0.000000}%
\pgfsetstrokecolor{textcolor}%
\pgfsetfillcolor{textcolor}%
\pgftext[x=3.780053in,y=0.402778in,,top]{\color{textcolor}\rmfamily\fontsize{12.000000}{14.400000}\selectfont \(\displaystyle {3400}\)}%
\end{pgfscope}%
\begin{pgfscope}%
\pgfpathrectangle{\pgfqpoint{0.750000in}{0.500000in}}{\pgfqpoint{4.650000in}{3.020000in}}%
\pgfusepath{clip}%
\pgfsetrectcap%
\pgfsetroundjoin%
\pgfsetlinewidth{0.803000pt}%
\definecolor{currentstroke}{rgb}{0.690196,0.690196,0.690196}%
\pgfsetstrokecolor{currentstroke}%
\pgfsetdash{}{0pt}%
\pgfpathmoveto{\pgfqpoint{4.487542in}{0.500000in}}%
\pgfpathlineto{\pgfqpoint{4.487542in}{3.520000in}}%
\pgfusepath{stroke}%
\end{pgfscope}%
\begin{pgfscope}%
\pgfsetbuttcap%
\pgfsetroundjoin%
\definecolor{currentfill}{rgb}{0.000000,0.000000,0.000000}%
\pgfsetfillcolor{currentfill}%
\pgfsetlinewidth{0.803000pt}%
\definecolor{currentstroke}{rgb}{0.000000,0.000000,0.000000}%
\pgfsetstrokecolor{currentstroke}%
\pgfsetdash{}{0pt}%
\pgfsys@defobject{currentmarker}{\pgfqpoint{0.000000in}{-0.048611in}}{\pgfqpoint{0.000000in}{0.000000in}}{%
\pgfpathmoveto{\pgfqpoint{0.000000in}{0.000000in}}%
\pgfpathlineto{\pgfqpoint{0.000000in}{-0.048611in}}%
\pgfusepath{stroke,fill}%
}%
\begin{pgfscope}%
\pgfsys@transformshift{4.487542in}{0.500000in}%
\pgfsys@useobject{currentmarker}{}%
\end{pgfscope}%
\end{pgfscope}%
\begin{pgfscope}%
\definecolor{textcolor}{rgb}{0.000000,0.000000,0.000000}%
\pgfsetstrokecolor{textcolor}%
\pgfsetfillcolor{textcolor}%
\pgftext[x=4.487542in,y=0.402778in,,top]{\color{textcolor}\rmfamily\fontsize{12.000000}{14.400000}\selectfont \(\displaystyle {3600}\)}%
\end{pgfscope}%
\begin{pgfscope}%
\pgfpathrectangle{\pgfqpoint{0.750000in}{0.500000in}}{\pgfqpoint{4.650000in}{3.020000in}}%
\pgfusepath{clip}%
\pgfsetrectcap%
\pgfsetroundjoin%
\pgfsetlinewidth{0.803000pt}%
\definecolor{currentstroke}{rgb}{0.690196,0.690196,0.690196}%
\pgfsetstrokecolor{currentstroke}%
\pgfsetdash{}{0pt}%
\pgfpathmoveto{\pgfqpoint{5.195031in}{0.500000in}}%
\pgfpathlineto{\pgfqpoint{5.195031in}{3.520000in}}%
\pgfusepath{stroke}%
\end{pgfscope}%
\begin{pgfscope}%
\pgfsetbuttcap%
\pgfsetroundjoin%
\definecolor{currentfill}{rgb}{0.000000,0.000000,0.000000}%
\pgfsetfillcolor{currentfill}%
\pgfsetlinewidth{0.803000pt}%
\definecolor{currentstroke}{rgb}{0.000000,0.000000,0.000000}%
\pgfsetstrokecolor{currentstroke}%
\pgfsetdash{}{0pt}%
\pgfsys@defobject{currentmarker}{\pgfqpoint{0.000000in}{-0.048611in}}{\pgfqpoint{0.000000in}{0.000000in}}{%
\pgfpathmoveto{\pgfqpoint{0.000000in}{0.000000in}}%
\pgfpathlineto{\pgfqpoint{0.000000in}{-0.048611in}}%
\pgfusepath{stroke,fill}%
}%
\begin{pgfscope}%
\pgfsys@transformshift{5.195031in}{0.500000in}%
\pgfsys@useobject{currentmarker}{}%
\end{pgfscope}%
\end{pgfscope}%
\begin{pgfscope}%
\definecolor{textcolor}{rgb}{0.000000,0.000000,0.000000}%
\pgfsetstrokecolor{textcolor}%
\pgfsetfillcolor{textcolor}%
\pgftext[x=5.195031in,y=0.402778in,,top]{\color{textcolor}\rmfamily\fontsize{12.000000}{14.400000}\selectfont \(\displaystyle {3800}\)}%
\end{pgfscope}%
\begin{pgfscope}%
\definecolor{textcolor}{rgb}{0.000000,0.000000,0.000000}%
\pgfsetstrokecolor{textcolor}%
\pgfsetfillcolor{textcolor}%
\pgftext[x=3.075000in,y=0.199075in,,top]{\color{textcolor}\rmfamily\fontsize{12.000000}{14.400000}\selectfont Raman shift in $\mathrm{cm^{-1}}$}%
\end{pgfscope}%
\begin{pgfscope}%
\pgfpathrectangle{\pgfqpoint{0.750000in}{0.500000in}}{\pgfqpoint{4.650000in}{3.020000in}}%
\pgfusepath{clip}%
\pgfsetrectcap%
\pgfsetroundjoin%
\pgfsetlinewidth{0.803000pt}%
\definecolor{currentstroke}{rgb}{0.690196,0.690196,0.690196}%
\pgfsetstrokecolor{currentstroke}%
\pgfsetdash{}{0pt}%
\pgfpathmoveto{\pgfqpoint{0.750000in}{0.639688in}}%
\pgfpathlineto{\pgfqpoint{5.400000in}{0.639688in}}%
\pgfusepath{stroke}%
\end{pgfscope}%
\begin{pgfscope}%
\pgfsetbuttcap%
\pgfsetroundjoin%
\definecolor{currentfill}{rgb}{0.000000,0.000000,0.000000}%
\pgfsetfillcolor{currentfill}%
\pgfsetlinewidth{0.803000pt}%
\definecolor{currentstroke}{rgb}{0.000000,0.000000,0.000000}%
\pgfsetstrokecolor{currentstroke}%
\pgfsetdash{}{0pt}%
\pgfsys@defobject{currentmarker}{\pgfqpoint{-0.048611in}{0.000000in}}{\pgfqpoint{-0.000000in}{0.000000in}}{%
\pgfpathmoveto{\pgfqpoint{-0.000000in}{0.000000in}}%
\pgfpathlineto{\pgfqpoint{-0.048611in}{0.000000in}}%
\pgfusepath{stroke,fill}%
}%
\begin{pgfscope}%
\pgfsys@transformshift{0.750000in}{0.639688in}%
\pgfsys@useobject{currentmarker}{}%
\end{pgfscope}%
\end{pgfscope}%
\begin{pgfscope}%
\definecolor{textcolor}{rgb}{0.000000,0.000000,0.000000}%
\pgfsetstrokecolor{textcolor}%
\pgfsetfillcolor{textcolor}%
\pgftext[x=0.281061in, y=0.581818in, left, base]{\color{textcolor}\rmfamily\fontsize{12.000000}{14.400000}\selectfont \(\displaystyle {0.000}\)}%
\end{pgfscope}%
\begin{pgfscope}%
\pgfpathrectangle{\pgfqpoint{0.750000in}{0.500000in}}{\pgfqpoint{4.650000in}{3.020000in}}%
\pgfusepath{clip}%
\pgfsetrectcap%
\pgfsetroundjoin%
\pgfsetlinewidth{0.803000pt}%
\definecolor{currentstroke}{rgb}{0.690196,0.690196,0.690196}%
\pgfsetstrokecolor{currentstroke}%
\pgfsetdash{}{0pt}%
\pgfpathmoveto{\pgfqpoint{0.750000in}{1.220148in}}%
\pgfpathlineto{\pgfqpoint{5.400000in}{1.220148in}}%
\pgfusepath{stroke}%
\end{pgfscope}%
\begin{pgfscope}%
\pgfsetbuttcap%
\pgfsetroundjoin%
\definecolor{currentfill}{rgb}{0.000000,0.000000,0.000000}%
\pgfsetfillcolor{currentfill}%
\pgfsetlinewidth{0.803000pt}%
\definecolor{currentstroke}{rgb}{0.000000,0.000000,0.000000}%
\pgfsetstrokecolor{currentstroke}%
\pgfsetdash{}{0pt}%
\pgfsys@defobject{currentmarker}{\pgfqpoint{-0.048611in}{0.000000in}}{\pgfqpoint{-0.000000in}{0.000000in}}{%
\pgfpathmoveto{\pgfqpoint{-0.000000in}{0.000000in}}%
\pgfpathlineto{\pgfqpoint{-0.048611in}{0.000000in}}%
\pgfusepath{stroke,fill}%
}%
\begin{pgfscope}%
\pgfsys@transformshift{0.750000in}{1.220148in}%
\pgfsys@useobject{currentmarker}{}%
\end{pgfscope}%
\end{pgfscope}%
\begin{pgfscope}%
\definecolor{textcolor}{rgb}{0.000000,0.000000,0.000000}%
\pgfsetstrokecolor{textcolor}%
\pgfsetfillcolor{textcolor}%
\pgftext[x=0.281061in, y=1.162278in, left, base]{\color{textcolor}\rmfamily\fontsize{12.000000}{14.400000}\selectfont \(\displaystyle {0.002}\)}%
\end{pgfscope}%
\begin{pgfscope}%
\pgfpathrectangle{\pgfqpoint{0.750000in}{0.500000in}}{\pgfqpoint{4.650000in}{3.020000in}}%
\pgfusepath{clip}%
\pgfsetrectcap%
\pgfsetroundjoin%
\pgfsetlinewidth{0.803000pt}%
\definecolor{currentstroke}{rgb}{0.690196,0.690196,0.690196}%
\pgfsetstrokecolor{currentstroke}%
\pgfsetdash{}{0pt}%
\pgfpathmoveto{\pgfqpoint{0.750000in}{1.800608in}}%
\pgfpathlineto{\pgfqpoint{5.400000in}{1.800608in}}%
\pgfusepath{stroke}%
\end{pgfscope}%
\begin{pgfscope}%
\pgfsetbuttcap%
\pgfsetroundjoin%
\definecolor{currentfill}{rgb}{0.000000,0.000000,0.000000}%
\pgfsetfillcolor{currentfill}%
\pgfsetlinewidth{0.803000pt}%
\definecolor{currentstroke}{rgb}{0.000000,0.000000,0.000000}%
\pgfsetstrokecolor{currentstroke}%
\pgfsetdash{}{0pt}%
\pgfsys@defobject{currentmarker}{\pgfqpoint{-0.048611in}{0.000000in}}{\pgfqpoint{-0.000000in}{0.000000in}}{%
\pgfpathmoveto{\pgfqpoint{-0.000000in}{0.000000in}}%
\pgfpathlineto{\pgfqpoint{-0.048611in}{0.000000in}}%
\pgfusepath{stroke,fill}%
}%
\begin{pgfscope}%
\pgfsys@transformshift{0.750000in}{1.800608in}%
\pgfsys@useobject{currentmarker}{}%
\end{pgfscope}%
\end{pgfscope}%
\begin{pgfscope}%
\definecolor{textcolor}{rgb}{0.000000,0.000000,0.000000}%
\pgfsetstrokecolor{textcolor}%
\pgfsetfillcolor{textcolor}%
\pgftext[x=0.281061in, y=1.742738in, left, base]{\color{textcolor}\rmfamily\fontsize{12.000000}{14.400000}\selectfont \(\displaystyle {0.004}\)}%
\end{pgfscope}%
\begin{pgfscope}%
\pgfpathrectangle{\pgfqpoint{0.750000in}{0.500000in}}{\pgfqpoint{4.650000in}{3.020000in}}%
\pgfusepath{clip}%
\pgfsetrectcap%
\pgfsetroundjoin%
\pgfsetlinewidth{0.803000pt}%
\definecolor{currentstroke}{rgb}{0.690196,0.690196,0.690196}%
\pgfsetstrokecolor{currentstroke}%
\pgfsetdash{}{0pt}%
\pgfpathmoveto{\pgfqpoint{0.750000in}{2.381068in}}%
\pgfpathlineto{\pgfqpoint{5.400000in}{2.381068in}}%
\pgfusepath{stroke}%
\end{pgfscope}%
\begin{pgfscope}%
\pgfsetbuttcap%
\pgfsetroundjoin%
\definecolor{currentfill}{rgb}{0.000000,0.000000,0.000000}%
\pgfsetfillcolor{currentfill}%
\pgfsetlinewidth{0.803000pt}%
\definecolor{currentstroke}{rgb}{0.000000,0.000000,0.000000}%
\pgfsetstrokecolor{currentstroke}%
\pgfsetdash{}{0pt}%
\pgfsys@defobject{currentmarker}{\pgfqpoint{-0.048611in}{0.000000in}}{\pgfqpoint{-0.000000in}{0.000000in}}{%
\pgfpathmoveto{\pgfqpoint{-0.000000in}{0.000000in}}%
\pgfpathlineto{\pgfqpoint{-0.048611in}{0.000000in}}%
\pgfusepath{stroke,fill}%
}%
\begin{pgfscope}%
\pgfsys@transformshift{0.750000in}{2.381068in}%
\pgfsys@useobject{currentmarker}{}%
\end{pgfscope}%
\end{pgfscope}%
\begin{pgfscope}%
\definecolor{textcolor}{rgb}{0.000000,0.000000,0.000000}%
\pgfsetstrokecolor{textcolor}%
\pgfsetfillcolor{textcolor}%
\pgftext[x=0.281061in, y=2.323198in, left, base]{\color{textcolor}\rmfamily\fontsize{12.000000}{14.400000}\selectfont \(\displaystyle {0.006}\)}%
\end{pgfscope}%
\begin{pgfscope}%
\pgfpathrectangle{\pgfqpoint{0.750000in}{0.500000in}}{\pgfqpoint{4.650000in}{3.020000in}}%
\pgfusepath{clip}%
\pgfsetrectcap%
\pgfsetroundjoin%
\pgfsetlinewidth{0.803000pt}%
\definecolor{currentstroke}{rgb}{0.690196,0.690196,0.690196}%
\pgfsetstrokecolor{currentstroke}%
\pgfsetdash{}{0pt}%
\pgfpathmoveto{\pgfqpoint{0.750000in}{2.961528in}}%
\pgfpathlineto{\pgfqpoint{5.400000in}{2.961528in}}%
\pgfusepath{stroke}%
\end{pgfscope}%
\begin{pgfscope}%
\pgfsetbuttcap%
\pgfsetroundjoin%
\definecolor{currentfill}{rgb}{0.000000,0.000000,0.000000}%
\pgfsetfillcolor{currentfill}%
\pgfsetlinewidth{0.803000pt}%
\definecolor{currentstroke}{rgb}{0.000000,0.000000,0.000000}%
\pgfsetstrokecolor{currentstroke}%
\pgfsetdash{}{0pt}%
\pgfsys@defobject{currentmarker}{\pgfqpoint{-0.048611in}{0.000000in}}{\pgfqpoint{-0.000000in}{0.000000in}}{%
\pgfpathmoveto{\pgfqpoint{-0.000000in}{0.000000in}}%
\pgfpathlineto{\pgfqpoint{-0.048611in}{0.000000in}}%
\pgfusepath{stroke,fill}%
}%
\begin{pgfscope}%
\pgfsys@transformshift{0.750000in}{2.961528in}%
\pgfsys@useobject{currentmarker}{}%
\end{pgfscope}%
\end{pgfscope}%
\begin{pgfscope}%
\definecolor{textcolor}{rgb}{0.000000,0.000000,0.000000}%
\pgfsetstrokecolor{textcolor}%
\pgfsetfillcolor{textcolor}%
\pgftext[x=0.281061in, y=2.903658in, left, base]{\color{textcolor}\rmfamily\fontsize{12.000000}{14.400000}\selectfont \(\displaystyle {0.008}\)}%
\end{pgfscope}%
\begin{pgfscope}%
\definecolor{textcolor}{rgb}{0.000000,0.000000,0.000000}%
\pgfsetstrokecolor{textcolor}%
\pgfsetfillcolor{textcolor}%
\pgftext[x=0.225505in,y=2.010000in,,bottom,rotate=90.000000]{\color{textcolor}\rmfamily\fontsize{12.000000}{14.400000}\selectfont Normalized signal intensity}%
\end{pgfscope}%
\begin{pgfscope}%
\pgfpathrectangle{\pgfqpoint{0.750000in}{0.500000in}}{\pgfqpoint{4.650000in}{3.020000in}}%
\pgfusepath{clip}%
\pgfsetrectcap%
\pgfsetroundjoin%
\pgfsetlinewidth{1.505625pt}%
\definecolor{currentstroke}{rgb}{0.121569,0.466667,0.705882}%
\pgfsetstrokecolor{currentstroke}%
\pgfsetdash{}{0pt}%
\pgfpathmoveto{\pgfqpoint{0.961364in}{0.645621in}}%
\pgfpathlineto{\pgfqpoint{0.977129in}{0.644226in}}%
\pgfpathlineto{\pgfqpoint{0.992887in}{0.645105in}}%
\pgfpathlineto{\pgfqpoint{1.025301in}{0.643991in}}%
\pgfpathlineto{\pgfqpoint{1.105714in}{0.641896in}}%
\pgfpathlineto{\pgfqpoint{1.138000in}{0.642323in}}%
\pgfpathlineto{\pgfqpoint{1.202463in}{0.640767in}}%
\pgfpathlineto{\pgfqpoint{1.218097in}{0.641069in}}%
\pgfpathlineto{\pgfqpoint{1.233721in}{0.639673in}}%
\pgfpathlineto{\pgfqpoint{1.250256in}{0.640420in}}%
\pgfpathlineto{\pgfqpoint{1.281462in}{0.638804in}}%
\pgfpathlineto{\pgfqpoint{1.313550in}{0.638861in}}%
\pgfpathlineto{\pgfqpoint{1.535340in}{0.640362in}}%
\pgfpathlineto{\pgfqpoint{1.582577in}{0.642279in}}%
\pgfpathlineto{\pgfqpoint{1.628829in}{0.642805in}}%
\pgfpathlineto{\pgfqpoint{1.645136in}{0.644551in}}%
\pgfpathlineto{\pgfqpoint{1.660527in}{0.644519in}}%
\pgfpathlineto{\pgfqpoint{1.692190in}{0.647734in}}%
\pgfpathlineto{\pgfqpoint{1.707556in}{0.647783in}}%
\pgfpathlineto{\pgfqpoint{1.893994in}{0.671620in}}%
\pgfpathlineto{\pgfqpoint{1.940636in}{0.680162in}}%
\pgfpathlineto{\pgfqpoint{2.032795in}{0.700748in}}%
\pgfpathlineto{\pgfqpoint{2.047976in}{0.705199in}}%
\pgfpathlineto{\pgfqpoint{2.063150in}{0.707762in}}%
\pgfpathlineto{\pgfqpoint{2.124652in}{0.728304in}}%
\pgfpathlineto{\pgfqpoint{2.186020in}{0.752528in}}%
\pgfpathlineto{\pgfqpoint{2.216210in}{0.766657in}}%
\pgfpathlineto{\pgfqpoint{2.277378in}{0.800716in}}%
\pgfpathlineto{\pgfqpoint{2.352547in}{0.854064in}}%
\pgfpathlineto{\pgfqpoint{2.398433in}{0.891910in}}%
\pgfpathlineto{\pgfqpoint{2.428394in}{0.922003in}}%
\pgfpathlineto{\pgfqpoint{2.458324in}{0.953212in}}%
\pgfpathlineto{\pgfqpoint{2.504036in}{1.007445in}}%
\pgfpathlineto{\pgfqpoint{2.518965in}{1.026480in}}%
\pgfpathlineto{\pgfqpoint{2.533885in}{1.049247in}}%
\pgfpathlineto{\pgfqpoint{2.548797in}{1.068531in}}%
\pgfpathlineto{\pgfqpoint{2.563701in}{1.091024in}}%
\pgfpathlineto{\pgfqpoint{2.593486in}{1.141219in}}%
\pgfpathlineto{\pgfqpoint{2.623239in}{1.192461in}}%
\pgfpathlineto{\pgfqpoint{2.668682in}{1.277820in}}%
\pgfpathlineto{\pgfqpoint{2.698354in}{1.340143in}}%
\pgfpathlineto{\pgfqpoint{2.742803in}{1.442809in}}%
\pgfpathlineto{\pgfqpoint{2.801959in}{1.600967in}}%
\pgfpathlineto{\pgfqpoint{2.875725in}{1.816274in}}%
\pgfpathlineto{\pgfqpoint{2.949296in}{2.038418in}}%
\pgfpathlineto{\pgfqpoint{2.979534in}{2.121405in}}%
\pgfpathlineto{\pgfqpoint{3.038185in}{2.278045in}}%
\pgfpathlineto{\pgfqpoint{3.081233in}{2.371257in}}%
\pgfpathlineto{\pgfqpoint{3.095853in}{2.397764in}}%
\pgfpathlineto{\pgfqpoint{3.110466in}{2.419819in}}%
\pgfpathlineto{\pgfqpoint{3.125071in}{2.445554in}}%
\pgfpathlineto{\pgfqpoint{3.139669in}{2.466498in}}%
\pgfpathlineto{\pgfqpoint{3.183415in}{2.518774in}}%
\pgfpathlineto{\pgfqpoint{3.227092in}{2.553867in}}%
\pgfpathlineto{\pgfqpoint{3.241635in}{2.564213in}}%
\pgfpathlineto{\pgfqpoint{3.256171in}{2.570492in}}%
\pgfpathlineto{\pgfqpoint{3.270699in}{2.579326in}}%
\pgfpathlineto{\pgfqpoint{3.285220in}{2.579396in}}%
\pgfpathlineto{\pgfqpoint{3.314238in}{2.585302in}}%
\pgfpathlineto{\pgfqpoint{3.343225in}{2.588865in}}%
\pgfpathlineto{\pgfqpoint{3.357707in}{2.586332in}}%
\pgfpathlineto{\pgfqpoint{3.372182in}{2.589344in}}%
\pgfpathlineto{\pgfqpoint{3.386649in}{2.585037in}}%
\pgfpathlineto{\pgfqpoint{3.400258in}{2.586319in}}%
\pgfpathlineto{\pgfqpoint{3.414710in}{2.584240in}}%
\pgfpathlineto{\pgfqpoint{3.429154in}{2.586459in}}%
\pgfpathlineto{\pgfqpoint{3.443591in}{2.586197in}}%
\pgfpathlineto{\pgfqpoint{3.458021in}{2.587480in}}%
\pgfpathlineto{\pgfqpoint{3.501263in}{2.605199in}}%
\pgfpathlineto{\pgfqpoint{3.515662in}{2.613716in}}%
\pgfpathlineto{\pgfqpoint{3.530053in}{2.625198in}}%
\pgfpathlineto{\pgfqpoint{3.572337in}{2.667945in}}%
\pgfpathlineto{\pgfqpoint{3.615399in}{2.719612in}}%
\pgfpathlineto{\pgfqpoint{3.644069in}{2.761880in}}%
\pgfpathlineto{\pgfqpoint{3.671868in}{2.808819in}}%
\pgfpathlineto{\pgfqpoint{3.729060in}{2.902787in}}%
\pgfpathlineto{\pgfqpoint{3.756772in}{2.953920in}}%
\pgfpathlineto{\pgfqpoint{3.799545in}{3.024433in}}%
\pgfpathlineto{\pgfqpoint{3.828022in}{3.062910in}}%
\pgfpathlineto{\pgfqpoint{3.869847in}{3.100247in}}%
\pgfpathlineto{\pgfqpoint{3.911607in}{3.119748in}}%
\pgfpathlineto{\pgfqpoint{3.939968in}{3.123081in}}%
\pgfpathlineto{\pgfqpoint{3.968298in}{3.118729in}}%
\pgfpathlineto{\pgfqpoint{3.981620in}{3.112788in}}%
\pgfpathlineto{\pgfqpoint{4.024040in}{3.086907in}}%
\pgfpathlineto{\pgfqpoint{4.038166in}{3.070640in}}%
\pgfpathlineto{\pgfqpoint{4.051453in}{3.050845in}}%
\pgfpathlineto{\pgfqpoint{4.065564in}{3.032865in}}%
\pgfpathlineto{\pgfqpoint{4.079668in}{3.011899in}}%
\pgfpathlineto{\pgfqpoint{4.093764in}{2.987083in}}%
\pgfpathlineto{\pgfqpoint{4.107025in}{2.959515in}}%
\pgfpathlineto{\pgfqpoint{4.121107in}{2.935714in}}%
\pgfpathlineto{\pgfqpoint{4.149249in}{2.878735in}}%
\pgfpathlineto{\pgfqpoint{4.217827in}{2.712577in}}%
\pgfpathlineto{\pgfqpoint{4.245868in}{2.634358in}}%
\pgfpathlineto{\pgfqpoint{4.328176in}{2.403542in}}%
\pgfpathlineto{\pgfqpoint{4.424157in}{2.150407in}}%
\pgfpathlineto{\pgfqpoint{4.451985in}{2.080914in}}%
\pgfpathlineto{\pgfqpoint{4.478967in}{2.011841in}}%
\pgfpathlineto{\pgfqpoint{4.492857in}{1.979696in}}%
\pgfpathlineto{\pgfqpoint{4.505922in}{1.945130in}}%
\pgfpathlineto{\pgfqpoint{4.533666in}{1.878533in}}%
\pgfpathlineto{\pgfqpoint{4.587440in}{1.752080in}}%
\pgfpathlineto{\pgfqpoint{4.615099in}{1.687882in}}%
\pgfpathlineto{\pgfqpoint{4.655723in}{1.573922in}}%
\pgfpathlineto{\pgfqpoint{4.709253in}{1.396993in}}%
\pgfpathlineto{\pgfqpoint{4.723023in}{1.348732in}}%
\pgfpathlineto{\pgfqpoint{4.749734in}{1.247136in}}%
\pgfpathlineto{\pgfqpoint{4.803076in}{1.062231in}}%
\pgfpathlineto{\pgfqpoint{4.830514in}{0.985153in}}%
\pgfpathlineto{\pgfqpoint{4.843416in}{0.950648in}}%
\pgfpathlineto{\pgfqpoint{4.870007in}{0.894360in}}%
\pgfpathlineto{\pgfqpoint{4.897376in}{0.848636in}}%
\pgfpathlineto{\pgfqpoint{4.910246in}{0.828043in}}%
\pgfpathlineto{\pgfqpoint{4.936770in}{0.794865in}}%
\pgfpathlineto{\pgfqpoint{4.976908in}{0.755470in}}%
\pgfpathlineto{\pgfqpoint{5.016986in}{0.726606in}}%
\pgfpathlineto{\pgfqpoint{5.056204in}{0.704763in}}%
\pgfpathlineto{\pgfqpoint{5.122503in}{0.680735in}}%
\pgfpathlineto{\pgfqpoint{5.188636in}{0.664846in}}%
\pgfpathlineto{\pgfqpoint{5.188636in}{0.664846in}}%
\pgfusepath{stroke}%
\end{pgfscope}%
\begin{pgfscope}%
\pgfpathrectangle{\pgfqpoint{0.750000in}{0.500000in}}{\pgfqpoint{4.650000in}{3.020000in}}%
\pgfusepath{clip}%
\pgfsetrectcap%
\pgfsetroundjoin%
\pgfsetlinewidth{1.505625pt}%
\definecolor{currentstroke}{rgb}{1.000000,0.498039,0.054902}%
\pgfsetstrokecolor{currentstroke}%
\pgfsetdash{}{0pt}%
\pgfpathmoveto{\pgfqpoint{0.961364in}{0.646878in}}%
\pgfpathlineto{\pgfqpoint{0.992887in}{0.646262in}}%
\pgfpathlineto{\pgfqpoint{1.041956in}{0.645196in}}%
\pgfpathlineto{\pgfqpoint{1.057678in}{0.644000in}}%
\pgfpathlineto{\pgfqpoint{1.073391in}{0.645059in}}%
\pgfpathlineto{\pgfqpoint{1.122322in}{0.643869in}}%
\pgfpathlineto{\pgfqpoint{1.424205in}{0.637925in}}%
\pgfpathlineto{\pgfqpoint{1.440630in}{0.638703in}}%
\pgfpathlineto{\pgfqpoint{1.456133in}{0.637273in}}%
\pgfpathlineto{\pgfqpoint{1.488024in}{0.638129in}}%
\pgfpathlineto{\pgfqpoint{1.707556in}{0.642679in}}%
\pgfpathlineto{\pgfqpoint{1.831985in}{0.652467in}}%
\pgfpathlineto{\pgfqpoint{2.001513in}{0.673890in}}%
\pgfpathlineto{\pgfqpoint{2.017605in}{0.677854in}}%
\pgfpathlineto{\pgfqpoint{2.047976in}{0.682629in}}%
\pgfpathlineto{\pgfqpoint{2.094364in}{0.690928in}}%
\pgfpathlineto{\pgfqpoint{2.139784in}{0.702523in}}%
\pgfpathlineto{\pgfqpoint{2.170913in}{0.709961in}}%
\pgfpathlineto{\pgfqpoint{2.246368in}{0.735217in}}%
\pgfpathlineto{\pgfqpoint{2.368439in}{0.790067in}}%
\pgfpathlineto{\pgfqpoint{2.428394in}{0.825335in}}%
\pgfpathlineto{\pgfqpoint{2.473277in}{0.855350in}}%
\pgfpathlineto{\pgfqpoint{2.489100in}{0.865930in}}%
\pgfpathlineto{\pgfqpoint{2.548797in}{0.916452in}}%
\pgfpathlineto{\pgfqpoint{2.578598in}{0.944492in}}%
\pgfpathlineto{\pgfqpoint{2.638103in}{1.008802in}}%
\pgfpathlineto{\pgfqpoint{2.653833in}{1.025640in}}%
\pgfpathlineto{\pgfqpoint{2.698354in}{1.082427in}}%
\pgfpathlineto{\pgfqpoint{2.742803in}{1.146658in}}%
\pgfpathlineto{\pgfqpoint{2.757604in}{1.169129in}}%
\pgfpathlineto{\pgfqpoint{2.787182in}{1.220655in}}%
\pgfpathlineto{\pgfqpoint{2.801959in}{1.245726in}}%
\pgfpathlineto{\pgfqpoint{2.860988in}{1.361289in}}%
\pgfpathlineto{\pgfqpoint{2.949296in}{1.561932in}}%
\pgfpathlineto{\pgfqpoint{3.081233in}{1.886584in}}%
\pgfpathlineto{\pgfqpoint{3.125071in}{1.988010in}}%
\pgfpathlineto{\pgfqpoint{3.154259in}{2.045853in}}%
\pgfpathlineto{\pgfqpoint{3.183415in}{2.096924in}}%
\pgfpathlineto{\pgfqpoint{3.212541in}{2.145589in}}%
\pgfpathlineto{\pgfqpoint{3.256171in}{2.205495in}}%
\pgfpathlineto{\pgfqpoint{3.299733in}{2.245889in}}%
\pgfpathlineto{\pgfqpoint{3.314238in}{2.258340in}}%
\pgfpathlineto{\pgfqpoint{3.343225in}{2.278462in}}%
\pgfpathlineto{\pgfqpoint{3.357707in}{2.290528in}}%
\pgfpathlineto{\pgfqpoint{3.429154in}{2.333289in}}%
\pgfpathlineto{\pgfqpoint{3.472442in}{2.371042in}}%
\pgfpathlineto{\pgfqpoint{3.501263in}{2.403027in}}%
\pgfpathlineto{\pgfqpoint{3.530053in}{2.443902in}}%
\pgfpathlineto{\pgfqpoint{3.586699in}{2.545326in}}%
\pgfpathlineto{\pgfqpoint{3.644069in}{2.668950in}}%
\pgfpathlineto{\pgfqpoint{3.813787in}{3.114202in}}%
\pgfpathlineto{\pgfqpoint{3.828022in}{3.147065in}}%
\pgfpathlineto{\pgfqpoint{3.869847in}{3.227720in}}%
\pgfpathlineto{\pgfqpoint{3.884052in}{3.251470in}}%
\pgfpathlineto{\pgfqpoint{3.898251in}{3.269811in}}%
\pgfpathlineto{\pgfqpoint{3.911607in}{3.290411in}}%
\pgfpathlineto{\pgfqpoint{3.954137in}{3.337911in}}%
\pgfpathlineto{\pgfqpoint{3.995768in}{3.368725in}}%
\pgfpathlineto{\pgfqpoint{4.009908in}{3.378725in}}%
\pgfpathlineto{\pgfqpoint{4.038166in}{3.382727in}}%
\pgfpathlineto{\pgfqpoint{4.051453in}{3.382200in}}%
\pgfpathlineto{\pgfqpoint{4.065564in}{3.378404in}}%
\pgfpathlineto{\pgfqpoint{4.093764in}{3.367310in}}%
\pgfpathlineto{\pgfqpoint{4.149249in}{3.316842in}}%
\pgfpathlineto{\pgfqpoint{4.176536in}{3.280368in}}%
\pgfpathlineto{\pgfqpoint{4.204620in}{3.233591in}}%
\pgfpathlineto{\pgfqpoint{4.231851in}{3.179717in}}%
\pgfpathlineto{\pgfqpoint{4.273058in}{3.081500in}}%
\pgfpathlineto{\pgfqpoint{4.315023in}{2.973919in}}%
\pgfpathlineto{\pgfqpoint{4.369235in}{2.826328in}}%
\pgfpathlineto{\pgfqpoint{4.424157in}{2.660435in}}%
\pgfpathlineto{\pgfqpoint{4.451985in}{2.575054in}}%
\pgfpathlineto{\pgfqpoint{4.601273in}{2.085274in}}%
\pgfpathlineto{\pgfqpoint{4.628105in}{1.987877in}}%
\pgfpathlineto{\pgfqpoint{4.655723in}{1.877174in}}%
\pgfpathlineto{\pgfqpoint{4.723023in}{1.573092in}}%
\pgfpathlineto{\pgfqpoint{4.749734in}{1.446779in}}%
\pgfpathlineto{\pgfqpoint{4.816798in}{1.159296in}}%
\pgfpathlineto{\pgfqpoint{4.843416in}{1.068553in}}%
\pgfpathlineto{\pgfqpoint{4.870007in}{0.992163in}}%
\pgfpathlineto{\pgfqpoint{4.897376in}{0.929578in}}%
\pgfpathlineto{\pgfqpoint{4.910246in}{0.901785in}}%
\pgfpathlineto{\pgfqpoint{4.936770in}{0.854539in}}%
\pgfpathlineto{\pgfqpoint{4.976908in}{0.799198in}}%
\pgfpathlineto{\pgfqpoint{5.003367in}{0.771281in}}%
\pgfpathlineto{\pgfqpoint{5.029799in}{0.748551in}}%
\pgfpathlineto{\pgfqpoint{5.056204in}{0.729753in}}%
\pgfpathlineto{\pgfqpoint{5.082584in}{0.714300in}}%
\pgfpathlineto{\pgfqpoint{5.122503in}{0.695314in}}%
\pgfpathlineto{\pgfqpoint{5.188636in}{0.674283in}}%
\pgfpathlineto{\pgfqpoint{5.188636in}{0.674283in}}%
\pgfusepath{stroke}%
\end{pgfscope}%
\begin{pgfscope}%
\pgfpathrectangle{\pgfqpoint{0.750000in}{0.500000in}}{\pgfqpoint{4.650000in}{3.020000in}}%
\pgfusepath{clip}%
\pgfsetbuttcap%
\pgfsetroundjoin%
\pgfsetlinewidth{1.505625pt}%
\definecolor{currentstroke}{rgb}{0.501961,0.000000,0.501961}%
\pgfsetstrokecolor{currentstroke}%
\pgfsetdash{{5.550000pt}{2.400000pt}}{0.000000pt}%
\pgfpathmoveto{\pgfqpoint{3.744678in}{0.500000in}}%
\pgfpathlineto{\pgfqpoint{3.744678in}{3.520000in}}%
\pgfusepath{stroke}%
\end{pgfscope}%
\begin{pgfscope}%
\pgfsetrectcap%
\pgfsetmiterjoin%
\pgfsetlinewidth{0.803000pt}%
\definecolor{currentstroke}{rgb}{0.000000,0.000000,0.000000}%
\pgfsetstrokecolor{currentstroke}%
\pgfsetdash{}{0pt}%
\pgfpathmoveto{\pgfqpoint{0.750000in}{0.500000in}}%
\pgfpathlineto{\pgfqpoint{0.750000in}{3.520000in}}%
\pgfusepath{stroke}%
\end{pgfscope}%
\begin{pgfscope}%
\pgfsetrectcap%
\pgfsetmiterjoin%
\pgfsetlinewidth{0.803000pt}%
\definecolor{currentstroke}{rgb}{0.000000,0.000000,0.000000}%
\pgfsetstrokecolor{currentstroke}%
\pgfsetdash{}{0pt}%
\pgfpathmoveto{\pgfqpoint{5.400000in}{0.500000in}}%
\pgfpathlineto{\pgfqpoint{5.400000in}{3.520000in}}%
\pgfusepath{stroke}%
\end{pgfscope}%
\begin{pgfscope}%
\pgfsetrectcap%
\pgfsetmiterjoin%
\pgfsetlinewidth{0.803000pt}%
\definecolor{currentstroke}{rgb}{0.000000,0.000000,0.000000}%
\pgfsetstrokecolor{currentstroke}%
\pgfsetdash{}{0pt}%
\pgfpathmoveto{\pgfqpoint{0.750000in}{0.500000in}}%
\pgfpathlineto{\pgfqpoint{5.400000in}{0.500000in}}%
\pgfusepath{stroke}%
\end{pgfscope}%
\begin{pgfscope}%
\pgfsetrectcap%
\pgfsetmiterjoin%
\pgfsetlinewidth{0.803000pt}%
\definecolor{currentstroke}{rgb}{0.000000,0.000000,0.000000}%
\pgfsetstrokecolor{currentstroke}%
\pgfsetdash{}{0pt}%
\pgfpathmoveto{\pgfqpoint{0.750000in}{3.520000in}}%
\pgfpathlineto{\pgfqpoint{5.400000in}{3.520000in}}%
\pgfusepath{stroke}%
\end{pgfscope}%
\begin{pgfscope}%
\pgfsetbuttcap%
\pgfsetmiterjoin%
\definecolor{currentfill}{rgb}{1.000000,1.000000,1.000000}%
\pgfsetfillcolor{currentfill}%
\pgfsetfillopacity{0.800000}%
\pgfsetlinewidth{1.003750pt}%
\definecolor{currentstroke}{rgb}{0.800000,0.800000,0.800000}%
\pgfsetstrokecolor{currentstroke}%
\pgfsetstrokeopacity{0.800000}%
\pgfsetdash{}{0pt}%
\pgfpathmoveto{\pgfqpoint{0.866667in}{2.689445in}}%
\pgfpathlineto{\pgfqpoint{2.837909in}{2.689445in}}%
\pgfpathquadraticcurveto{\pgfqpoint{2.871243in}{2.689445in}}{\pgfqpoint{2.871243in}{2.722779in}}%
\pgfpathlineto{\pgfqpoint{2.871243in}{3.403333in}}%
\pgfpathquadraticcurveto{\pgfqpoint{2.871243in}{3.436667in}}{\pgfqpoint{2.837909in}{3.436667in}}%
\pgfpathlineto{\pgfqpoint{0.866667in}{3.436667in}}%
\pgfpathquadraticcurveto{\pgfqpoint{0.833333in}{3.436667in}}{\pgfqpoint{0.833333in}{3.403333in}}%
\pgfpathlineto{\pgfqpoint{0.833333in}{2.722779in}}%
\pgfpathquadraticcurveto{\pgfqpoint{0.833333in}{2.689445in}}{\pgfqpoint{0.866667in}{2.689445in}}%
\pgfpathlineto{\pgfqpoint{0.866667in}{2.689445in}}%
\pgfpathclose%
\pgfusepath{stroke,fill}%
\end{pgfscope}%
\begin{pgfscope}%
\pgfsetrectcap%
\pgfsetroundjoin%
\pgfsetlinewidth{1.505625pt}%
\definecolor{currentstroke}{rgb}{0.121569,0.466667,0.705882}%
\pgfsetstrokecolor{currentstroke}%
\pgfsetdash{}{0pt}%
\pgfpathmoveto{\pgfqpoint{0.900000in}{3.311667in}}%
\pgfpathlineto{\pgfqpoint{1.066667in}{3.311667in}}%
\pgfpathlineto{\pgfqpoint{1.233333in}{3.311667in}}%
\pgfusepath{stroke}%
\end{pgfscope}%
\begin{pgfscope}%
\definecolor{textcolor}{rgb}{0.000000,0.000000,0.000000}%
\pgfsetstrokecolor{textcolor}%
\pgfsetfillcolor{textcolor}%
\pgftext[x=1.366667in,y=3.253333in,left,base]{\color{textcolor}\rmfamily\fontsize{12.000000}{14.400000}\selectfont lowest temperature}%
\end{pgfscope}%
\begin{pgfscope}%
\pgfsetrectcap%
\pgfsetroundjoin%
\pgfsetlinewidth{1.505625pt}%
\definecolor{currentstroke}{rgb}{1.000000,0.498039,0.054902}%
\pgfsetstrokecolor{currentstroke}%
\pgfsetdash{}{0pt}%
\pgfpathmoveto{\pgfqpoint{0.900000in}{3.079260in}}%
\pgfpathlineto{\pgfqpoint{1.066667in}{3.079260in}}%
\pgfpathlineto{\pgfqpoint{1.233333in}{3.079260in}}%
\pgfusepath{stroke}%
\end{pgfscope}%
\begin{pgfscope}%
\definecolor{textcolor}{rgb}{0.000000,0.000000,0.000000}%
\pgfsetstrokecolor{textcolor}%
\pgfsetfillcolor{textcolor}%
\pgftext[x=1.366667in,y=3.020926in,left,base]{\color{textcolor}\rmfamily\fontsize{12.000000}{14.400000}\selectfont highest temperature}%
\end{pgfscope}%
\begin{pgfscope}%
\pgfsetbuttcap%
\pgfsetroundjoin%
\pgfsetlinewidth{1.505625pt}%
\definecolor{currentstroke}{rgb}{0.501961,0.000000,0.501961}%
\pgfsetstrokecolor{currentstroke}%
\pgfsetdash{{5.550000pt}{2.400000pt}}{0.000000pt}%
\pgfpathmoveto{\pgfqpoint{0.900000in}{2.846852in}}%
\pgfpathlineto{\pgfqpoint{1.066667in}{2.846852in}}%
\pgfpathlineto{\pgfqpoint{1.233333in}{2.846852in}}%
\pgfusepath{stroke}%
\end{pgfscope}%
\begin{pgfscope}%
\definecolor{textcolor}{rgb}{0.000000,0.000000,0.000000}%
\pgfsetstrokecolor{textcolor}%
\pgfsetfillcolor{textcolor}%
\pgftext[x=1.366667in,y=2.788519in,left,base]{\color{textcolor}\rmfamily\fontsize{12.000000}{14.400000}\selectfont isosbestic point}%
\end{pgfscope}%
\end{pgfpicture}%
\makeatother%
\endgroup%

    \caption[Area-normalized Raman shift intensities for the lowest and highest temperatures]{Plot of the area-normalized scattering intensities over the Raman shift $\Delta \tilde{v}$; Plot at the highest recorded temperature and the lowest recorded temperature for simplification purposes}
    \label{fig:plot-temp}
\end{figure}

With the record of a temperature calibration curve, a function with the form
\begin{align}
    T(x)=42.1 \cdot x^3 - 200.5 \cdot x^2 + 412.4 \cdot x -233.7
\end{align}
can be fitted to calculate the temperature of the liquid through putting in the area ratios $x$. The in between results and the final temperature for the highest and the lowest temperature curve are shown in \autoref{tab:temp}. Compared to the measurement with the thermocouple K, there is a mean error of $3.46~\mathrm{K}$, and a mean squared error of $12.83~\mathrm{K}$ over all measurements.

\begin{table}[!htb]
    \centering
    \small
    \caption[Temperature calculation results comparison]{Calculation results for temperature determination through Raman spectroscopy; comparing with the temperature value of the thermocouple K; only for the highest and the lowest temperature due to simplification}
    \label{tab:temp}
    \vspace{12pt}
    \begin{tabular}{|l|r|r|}
        \hline % \cline{2-3}
        \rowcolor{lightgray}                & highest temperature   & lowest temperature \\ \hline \hline % \multicolumn{1}{l|}{\cellcolor{white}}
        area ratio                          & 1.51281               & 1.01273 \\ \hline
        determined $\mathrm{T}$ Raman       & 77.08 $\mathrm{K}$    & 22.04 $\mathrm{K}$ \\ \hline
        measured $\mathrm{T}$ thermocouple  & 73.5 $\mathrm{K}$     & 20.2 $\mathrm{K}$\\ \hline
        error                               & 3.58 $\mathrm{K}$     & 1.84 $\mathrm{K}$ \\ \hline        
    \end{tabular}
\end{table}

\begin{figure}[!htb]
    \centering
    %% Creator: Matplotlib, PGF backend
%%
%% To include the figure in your LaTeX document, write
%%   \input{<filename>.pgf}
%%
%% Make sure the required packages are loaded in your preamble
%%   \usepackage{pgf}
%%
%% Also ensure that all the required font packages are loaded; for instance,
%% the lmodern package is sometimes necessary when using math font.
%%   \usepackage{lmodern}
%%
%% Figures using additional raster images can only be included by \input if
%% they are in the same directory as the main LaTeX file. For loading figures
%% from other directories you can use the `import` package
%%   \usepackage{import}
%%
%% and then include the figures with
%%   \import{<path to file>}{<filename>.pgf}
%%
%% Matplotlib used the following preamble
%%   
%%   \makeatletter\@ifpackageloaded{underscore}{}{\usepackage[strings]{underscore}}\makeatother
%%
\begingroup%
\makeatletter%
\begin{pgfpicture}%
\pgfpathrectangle{\pgfpointorigin}{\pgfqpoint{6.400000in}{4.800000in}}%
\pgfusepath{use as bounding box, clip}%
\begin{pgfscope}%
\pgfsetbuttcap%
\pgfsetmiterjoin%
\definecolor{currentfill}{rgb}{1.000000,1.000000,1.000000}%
\pgfsetfillcolor{currentfill}%
\pgfsetlinewidth{0.000000pt}%
\definecolor{currentstroke}{rgb}{1.000000,1.000000,1.000000}%
\pgfsetstrokecolor{currentstroke}%
\pgfsetdash{}{0pt}%
\pgfpathmoveto{\pgfqpoint{0.000000in}{0.000000in}}%
\pgfpathlineto{\pgfqpoint{6.400000in}{0.000000in}}%
\pgfpathlineto{\pgfqpoint{6.400000in}{4.800000in}}%
\pgfpathlineto{\pgfqpoint{0.000000in}{4.800000in}}%
\pgfpathlineto{\pgfqpoint{0.000000in}{0.000000in}}%
\pgfpathclose%
\pgfusepath{fill}%
\end{pgfscope}%
\begin{pgfscope}%
\pgfsetbuttcap%
\pgfsetmiterjoin%
\definecolor{currentfill}{rgb}{1.000000,1.000000,1.000000}%
\pgfsetfillcolor{currentfill}%
\pgfsetlinewidth{0.000000pt}%
\definecolor{currentstroke}{rgb}{0.000000,0.000000,0.000000}%
\pgfsetstrokecolor{currentstroke}%
\pgfsetstrokeopacity{0.000000}%
\pgfsetdash{}{0pt}%
\pgfpathmoveto{\pgfqpoint{0.800000in}{0.528000in}}%
\pgfpathlineto{\pgfqpoint{5.760000in}{0.528000in}}%
\pgfpathlineto{\pgfqpoint{5.760000in}{4.224000in}}%
\pgfpathlineto{\pgfqpoint{0.800000in}{4.224000in}}%
\pgfpathlineto{\pgfqpoint{0.800000in}{0.528000in}}%
\pgfpathclose%
\pgfusepath{fill}%
\end{pgfscope}%
\begin{pgfscope}%
\pgfpathrectangle{\pgfqpoint{0.800000in}{0.528000in}}{\pgfqpoint{4.960000in}{3.696000in}}%
\pgfusepath{clip}%
\pgfsetrectcap%
\pgfsetroundjoin%
\pgfsetlinewidth{0.803000pt}%
\definecolor{currentstroke}{rgb}{0.690196,0.690196,0.690196}%
\pgfsetstrokecolor{currentstroke}%
\pgfsetdash{}{0pt}%
\pgfpathmoveto{\pgfqpoint{1.025455in}{0.528000in}}%
\pgfpathlineto{\pgfqpoint{1.025455in}{4.224000in}}%
\pgfusepath{stroke}%
\end{pgfscope}%
\begin{pgfscope}%
\pgfsetbuttcap%
\pgfsetroundjoin%
\definecolor{currentfill}{rgb}{0.000000,0.000000,0.000000}%
\pgfsetfillcolor{currentfill}%
\pgfsetlinewidth{0.803000pt}%
\definecolor{currentstroke}{rgb}{0.000000,0.000000,0.000000}%
\pgfsetstrokecolor{currentstroke}%
\pgfsetdash{}{0pt}%
\pgfsys@defobject{currentmarker}{\pgfqpoint{0.000000in}{-0.048611in}}{\pgfqpoint{0.000000in}{0.000000in}}{%
\pgfpathmoveto{\pgfqpoint{0.000000in}{0.000000in}}%
\pgfpathlineto{\pgfqpoint{0.000000in}{-0.048611in}}%
\pgfusepath{stroke,fill}%
}%
\begin{pgfscope}%
\pgfsys@transformshift{1.025455in}{0.528000in}%
\pgfsys@useobject{currentmarker}{}%
\end{pgfscope}%
\end{pgfscope}%
\begin{pgfscope}%
\definecolor{textcolor}{rgb}{0.000000,0.000000,0.000000}%
\pgfsetstrokecolor{textcolor}%
\pgfsetfillcolor{textcolor}%
\pgftext[x=1.025455in,y=0.430778in,,top]{\color{textcolor}\rmfamily\fontsize{12.000000}{14.400000}\selectfont \(\displaystyle {0}\)}%
\end{pgfscope}%
\begin{pgfscope}%
\pgfpathrectangle{\pgfqpoint{0.800000in}{0.528000in}}{\pgfqpoint{4.960000in}{3.696000in}}%
\pgfusepath{clip}%
\pgfsetrectcap%
\pgfsetroundjoin%
\pgfsetlinewidth{0.803000pt}%
\definecolor{currentstroke}{rgb}{0.690196,0.690196,0.690196}%
\pgfsetstrokecolor{currentstroke}%
\pgfsetdash{}{0pt}%
\pgfpathmoveto{\pgfqpoint{1.746909in}{0.528000in}}%
\pgfpathlineto{\pgfqpoint{1.746909in}{4.224000in}}%
\pgfusepath{stroke}%
\end{pgfscope}%
\begin{pgfscope}%
\pgfsetbuttcap%
\pgfsetroundjoin%
\definecolor{currentfill}{rgb}{0.000000,0.000000,0.000000}%
\pgfsetfillcolor{currentfill}%
\pgfsetlinewidth{0.803000pt}%
\definecolor{currentstroke}{rgb}{0.000000,0.000000,0.000000}%
\pgfsetstrokecolor{currentstroke}%
\pgfsetdash{}{0pt}%
\pgfsys@defobject{currentmarker}{\pgfqpoint{0.000000in}{-0.048611in}}{\pgfqpoint{0.000000in}{0.000000in}}{%
\pgfpathmoveto{\pgfqpoint{0.000000in}{0.000000in}}%
\pgfpathlineto{\pgfqpoint{0.000000in}{-0.048611in}}%
\pgfusepath{stroke,fill}%
}%
\begin{pgfscope}%
\pgfsys@transformshift{1.746909in}{0.528000in}%
\pgfsys@useobject{currentmarker}{}%
\end{pgfscope}%
\end{pgfscope}%
\begin{pgfscope}%
\definecolor{textcolor}{rgb}{0.000000,0.000000,0.000000}%
\pgfsetstrokecolor{textcolor}%
\pgfsetfillcolor{textcolor}%
\pgftext[x=1.746909in,y=0.430778in,,top]{\color{textcolor}\rmfamily\fontsize{12.000000}{14.400000}\selectfont \(\displaystyle {200}\)}%
\end{pgfscope}%
\begin{pgfscope}%
\pgfpathrectangle{\pgfqpoint{0.800000in}{0.528000in}}{\pgfqpoint{4.960000in}{3.696000in}}%
\pgfusepath{clip}%
\pgfsetrectcap%
\pgfsetroundjoin%
\pgfsetlinewidth{0.803000pt}%
\definecolor{currentstroke}{rgb}{0.690196,0.690196,0.690196}%
\pgfsetstrokecolor{currentstroke}%
\pgfsetdash{}{0pt}%
\pgfpathmoveto{\pgfqpoint{2.468364in}{0.528000in}}%
\pgfpathlineto{\pgfqpoint{2.468364in}{4.224000in}}%
\pgfusepath{stroke}%
\end{pgfscope}%
\begin{pgfscope}%
\pgfsetbuttcap%
\pgfsetroundjoin%
\definecolor{currentfill}{rgb}{0.000000,0.000000,0.000000}%
\pgfsetfillcolor{currentfill}%
\pgfsetlinewidth{0.803000pt}%
\definecolor{currentstroke}{rgb}{0.000000,0.000000,0.000000}%
\pgfsetstrokecolor{currentstroke}%
\pgfsetdash{}{0pt}%
\pgfsys@defobject{currentmarker}{\pgfqpoint{0.000000in}{-0.048611in}}{\pgfqpoint{0.000000in}{0.000000in}}{%
\pgfpathmoveto{\pgfqpoint{0.000000in}{0.000000in}}%
\pgfpathlineto{\pgfqpoint{0.000000in}{-0.048611in}}%
\pgfusepath{stroke,fill}%
}%
\begin{pgfscope}%
\pgfsys@transformshift{2.468364in}{0.528000in}%
\pgfsys@useobject{currentmarker}{}%
\end{pgfscope}%
\end{pgfscope}%
\begin{pgfscope}%
\definecolor{textcolor}{rgb}{0.000000,0.000000,0.000000}%
\pgfsetstrokecolor{textcolor}%
\pgfsetfillcolor{textcolor}%
\pgftext[x=2.468364in,y=0.430778in,,top]{\color{textcolor}\rmfamily\fontsize{12.000000}{14.400000}\selectfont \(\displaystyle {400}\)}%
\end{pgfscope}%
\begin{pgfscope}%
\pgfpathrectangle{\pgfqpoint{0.800000in}{0.528000in}}{\pgfqpoint{4.960000in}{3.696000in}}%
\pgfusepath{clip}%
\pgfsetrectcap%
\pgfsetroundjoin%
\pgfsetlinewidth{0.803000pt}%
\definecolor{currentstroke}{rgb}{0.690196,0.690196,0.690196}%
\pgfsetstrokecolor{currentstroke}%
\pgfsetdash{}{0pt}%
\pgfpathmoveto{\pgfqpoint{3.189818in}{0.528000in}}%
\pgfpathlineto{\pgfqpoint{3.189818in}{4.224000in}}%
\pgfusepath{stroke}%
\end{pgfscope}%
\begin{pgfscope}%
\pgfsetbuttcap%
\pgfsetroundjoin%
\definecolor{currentfill}{rgb}{0.000000,0.000000,0.000000}%
\pgfsetfillcolor{currentfill}%
\pgfsetlinewidth{0.803000pt}%
\definecolor{currentstroke}{rgb}{0.000000,0.000000,0.000000}%
\pgfsetstrokecolor{currentstroke}%
\pgfsetdash{}{0pt}%
\pgfsys@defobject{currentmarker}{\pgfqpoint{0.000000in}{-0.048611in}}{\pgfqpoint{0.000000in}{0.000000in}}{%
\pgfpathmoveto{\pgfqpoint{0.000000in}{0.000000in}}%
\pgfpathlineto{\pgfqpoint{0.000000in}{-0.048611in}}%
\pgfusepath{stroke,fill}%
}%
\begin{pgfscope}%
\pgfsys@transformshift{3.189818in}{0.528000in}%
\pgfsys@useobject{currentmarker}{}%
\end{pgfscope}%
\end{pgfscope}%
\begin{pgfscope}%
\definecolor{textcolor}{rgb}{0.000000,0.000000,0.000000}%
\pgfsetstrokecolor{textcolor}%
\pgfsetfillcolor{textcolor}%
\pgftext[x=3.189818in,y=0.430778in,,top]{\color{textcolor}\rmfamily\fontsize{12.000000}{14.400000}\selectfont \(\displaystyle {600}\)}%
\end{pgfscope}%
\begin{pgfscope}%
\pgfpathrectangle{\pgfqpoint{0.800000in}{0.528000in}}{\pgfqpoint{4.960000in}{3.696000in}}%
\pgfusepath{clip}%
\pgfsetrectcap%
\pgfsetroundjoin%
\pgfsetlinewidth{0.803000pt}%
\definecolor{currentstroke}{rgb}{0.690196,0.690196,0.690196}%
\pgfsetstrokecolor{currentstroke}%
\pgfsetdash{}{0pt}%
\pgfpathmoveto{\pgfqpoint{3.911273in}{0.528000in}}%
\pgfpathlineto{\pgfqpoint{3.911273in}{4.224000in}}%
\pgfusepath{stroke}%
\end{pgfscope}%
\begin{pgfscope}%
\pgfsetbuttcap%
\pgfsetroundjoin%
\definecolor{currentfill}{rgb}{0.000000,0.000000,0.000000}%
\pgfsetfillcolor{currentfill}%
\pgfsetlinewidth{0.803000pt}%
\definecolor{currentstroke}{rgb}{0.000000,0.000000,0.000000}%
\pgfsetstrokecolor{currentstroke}%
\pgfsetdash{}{0pt}%
\pgfsys@defobject{currentmarker}{\pgfqpoint{0.000000in}{-0.048611in}}{\pgfqpoint{0.000000in}{0.000000in}}{%
\pgfpathmoveto{\pgfqpoint{0.000000in}{0.000000in}}%
\pgfpathlineto{\pgfqpoint{0.000000in}{-0.048611in}}%
\pgfusepath{stroke,fill}%
}%
\begin{pgfscope}%
\pgfsys@transformshift{3.911273in}{0.528000in}%
\pgfsys@useobject{currentmarker}{}%
\end{pgfscope}%
\end{pgfscope}%
\begin{pgfscope}%
\definecolor{textcolor}{rgb}{0.000000,0.000000,0.000000}%
\pgfsetstrokecolor{textcolor}%
\pgfsetfillcolor{textcolor}%
\pgftext[x=3.911273in,y=0.430778in,,top]{\color{textcolor}\rmfamily\fontsize{12.000000}{14.400000}\selectfont \(\displaystyle {800}\)}%
\end{pgfscope}%
\begin{pgfscope}%
\pgfpathrectangle{\pgfqpoint{0.800000in}{0.528000in}}{\pgfqpoint{4.960000in}{3.696000in}}%
\pgfusepath{clip}%
\pgfsetrectcap%
\pgfsetroundjoin%
\pgfsetlinewidth{0.803000pt}%
\definecolor{currentstroke}{rgb}{0.690196,0.690196,0.690196}%
\pgfsetstrokecolor{currentstroke}%
\pgfsetdash{}{0pt}%
\pgfpathmoveto{\pgfqpoint{4.632727in}{0.528000in}}%
\pgfpathlineto{\pgfqpoint{4.632727in}{4.224000in}}%
\pgfusepath{stroke}%
\end{pgfscope}%
\begin{pgfscope}%
\pgfsetbuttcap%
\pgfsetroundjoin%
\definecolor{currentfill}{rgb}{0.000000,0.000000,0.000000}%
\pgfsetfillcolor{currentfill}%
\pgfsetlinewidth{0.803000pt}%
\definecolor{currentstroke}{rgb}{0.000000,0.000000,0.000000}%
\pgfsetstrokecolor{currentstroke}%
\pgfsetdash{}{0pt}%
\pgfsys@defobject{currentmarker}{\pgfqpoint{0.000000in}{-0.048611in}}{\pgfqpoint{0.000000in}{0.000000in}}{%
\pgfpathmoveto{\pgfqpoint{0.000000in}{0.000000in}}%
\pgfpathlineto{\pgfqpoint{0.000000in}{-0.048611in}}%
\pgfusepath{stroke,fill}%
}%
\begin{pgfscope}%
\pgfsys@transformshift{4.632727in}{0.528000in}%
\pgfsys@useobject{currentmarker}{}%
\end{pgfscope}%
\end{pgfscope}%
\begin{pgfscope}%
\definecolor{textcolor}{rgb}{0.000000,0.000000,0.000000}%
\pgfsetstrokecolor{textcolor}%
\pgfsetfillcolor{textcolor}%
\pgftext[x=4.632727in,y=0.430778in,,top]{\color{textcolor}\rmfamily\fontsize{12.000000}{14.400000}\selectfont \(\displaystyle {1000}\)}%
\end{pgfscope}%
\begin{pgfscope}%
\pgfpathrectangle{\pgfqpoint{0.800000in}{0.528000in}}{\pgfqpoint{4.960000in}{3.696000in}}%
\pgfusepath{clip}%
\pgfsetrectcap%
\pgfsetroundjoin%
\pgfsetlinewidth{0.803000pt}%
\definecolor{currentstroke}{rgb}{0.690196,0.690196,0.690196}%
\pgfsetstrokecolor{currentstroke}%
\pgfsetdash{}{0pt}%
\pgfpathmoveto{\pgfqpoint{5.354182in}{0.528000in}}%
\pgfpathlineto{\pgfqpoint{5.354182in}{4.224000in}}%
\pgfusepath{stroke}%
\end{pgfscope}%
\begin{pgfscope}%
\pgfsetbuttcap%
\pgfsetroundjoin%
\definecolor{currentfill}{rgb}{0.000000,0.000000,0.000000}%
\pgfsetfillcolor{currentfill}%
\pgfsetlinewidth{0.803000pt}%
\definecolor{currentstroke}{rgb}{0.000000,0.000000,0.000000}%
\pgfsetstrokecolor{currentstroke}%
\pgfsetdash{}{0pt}%
\pgfsys@defobject{currentmarker}{\pgfqpoint{0.000000in}{-0.048611in}}{\pgfqpoint{0.000000in}{0.000000in}}{%
\pgfpathmoveto{\pgfqpoint{0.000000in}{0.000000in}}%
\pgfpathlineto{\pgfqpoint{0.000000in}{-0.048611in}}%
\pgfusepath{stroke,fill}%
}%
\begin{pgfscope}%
\pgfsys@transformshift{5.354182in}{0.528000in}%
\pgfsys@useobject{currentmarker}{}%
\end{pgfscope}%
\end{pgfscope}%
\begin{pgfscope}%
\definecolor{textcolor}{rgb}{0.000000,0.000000,0.000000}%
\pgfsetstrokecolor{textcolor}%
\pgfsetfillcolor{textcolor}%
\pgftext[x=5.354182in,y=0.430778in,,top]{\color{textcolor}\rmfamily\fontsize{12.000000}{14.400000}\selectfont \(\displaystyle {1200}\)}%
\end{pgfscope}%
\begin{pgfscope}%
\definecolor{textcolor}{rgb}{0.000000,0.000000,0.000000}%
\pgfsetstrokecolor{textcolor}%
\pgfsetfillcolor{textcolor}%
\pgftext[x=3.280000in,y=0.227075in,,top]{\color{textcolor}\rmfamily\fontsize{12.000000}{14.400000}\selectfont Time in \(\displaystyle \mathrm{s}\)}%
\end{pgfscope}%
\begin{pgfscope}%
\pgfpathrectangle{\pgfqpoint{0.800000in}{0.528000in}}{\pgfqpoint{4.960000in}{3.696000in}}%
\pgfusepath{clip}%
\pgfsetrectcap%
\pgfsetroundjoin%
\pgfsetlinewidth{0.803000pt}%
\definecolor{currentstroke}{rgb}{0.690196,0.690196,0.690196}%
\pgfsetstrokecolor{currentstroke}%
\pgfsetdash{}{0pt}%
\pgfpathmoveto{\pgfqpoint{0.800000in}{0.690126in}}%
\pgfpathlineto{\pgfqpoint{5.760000in}{0.690126in}}%
\pgfusepath{stroke}%
\end{pgfscope}%
\begin{pgfscope}%
\pgfsetbuttcap%
\pgfsetroundjoin%
\definecolor{currentfill}{rgb}{0.000000,0.000000,0.000000}%
\pgfsetfillcolor{currentfill}%
\pgfsetlinewidth{0.803000pt}%
\definecolor{currentstroke}{rgb}{0.000000,0.000000,0.000000}%
\pgfsetstrokecolor{currentstroke}%
\pgfsetdash{}{0pt}%
\pgfsys@defobject{currentmarker}{\pgfqpoint{-0.048611in}{0.000000in}}{\pgfqpoint{-0.000000in}{0.000000in}}{%
\pgfpathmoveto{\pgfqpoint{-0.000000in}{0.000000in}}%
\pgfpathlineto{\pgfqpoint{-0.048611in}{0.000000in}}%
\pgfusepath{stroke,fill}%
}%
\begin{pgfscope}%
\pgfsys@transformshift{0.800000in}{0.690126in}%
\pgfsys@useobject{currentmarker}{}%
\end{pgfscope}%
\end{pgfscope}%
\begin{pgfscope}%
\definecolor{textcolor}{rgb}{0.000000,0.000000,0.000000}%
\pgfsetstrokecolor{textcolor}%
\pgfsetfillcolor{textcolor}%
\pgftext[x=0.539585in, y=0.632256in, left, base]{\color{textcolor}\rmfamily\fontsize{12.000000}{14.400000}\selectfont \(\displaystyle {20}\)}%
\end{pgfscope}%
\begin{pgfscope}%
\pgfpathrectangle{\pgfqpoint{0.800000in}{0.528000in}}{\pgfqpoint{4.960000in}{3.696000in}}%
\pgfusepath{clip}%
\pgfsetrectcap%
\pgfsetroundjoin%
\pgfsetlinewidth{0.803000pt}%
\definecolor{currentstroke}{rgb}{0.690196,0.690196,0.690196}%
\pgfsetstrokecolor{currentstroke}%
\pgfsetdash{}{0pt}%
\pgfpathmoveto{\pgfqpoint{0.800000in}{1.277538in}}%
\pgfpathlineto{\pgfqpoint{5.760000in}{1.277538in}}%
\pgfusepath{stroke}%
\end{pgfscope}%
\begin{pgfscope}%
\pgfsetbuttcap%
\pgfsetroundjoin%
\definecolor{currentfill}{rgb}{0.000000,0.000000,0.000000}%
\pgfsetfillcolor{currentfill}%
\pgfsetlinewidth{0.803000pt}%
\definecolor{currentstroke}{rgb}{0.000000,0.000000,0.000000}%
\pgfsetstrokecolor{currentstroke}%
\pgfsetdash{}{0pt}%
\pgfsys@defobject{currentmarker}{\pgfqpoint{-0.048611in}{0.000000in}}{\pgfqpoint{-0.000000in}{0.000000in}}{%
\pgfpathmoveto{\pgfqpoint{-0.000000in}{0.000000in}}%
\pgfpathlineto{\pgfqpoint{-0.048611in}{0.000000in}}%
\pgfusepath{stroke,fill}%
}%
\begin{pgfscope}%
\pgfsys@transformshift{0.800000in}{1.277538in}%
\pgfsys@useobject{currentmarker}{}%
\end{pgfscope}%
\end{pgfscope}%
\begin{pgfscope}%
\definecolor{textcolor}{rgb}{0.000000,0.000000,0.000000}%
\pgfsetstrokecolor{textcolor}%
\pgfsetfillcolor{textcolor}%
\pgftext[x=0.539585in, y=1.219668in, left, base]{\color{textcolor}\rmfamily\fontsize{12.000000}{14.400000}\selectfont \(\displaystyle {30}\)}%
\end{pgfscope}%
\begin{pgfscope}%
\pgfpathrectangle{\pgfqpoint{0.800000in}{0.528000in}}{\pgfqpoint{4.960000in}{3.696000in}}%
\pgfusepath{clip}%
\pgfsetrectcap%
\pgfsetroundjoin%
\pgfsetlinewidth{0.803000pt}%
\definecolor{currentstroke}{rgb}{0.690196,0.690196,0.690196}%
\pgfsetstrokecolor{currentstroke}%
\pgfsetdash{}{0pt}%
\pgfpathmoveto{\pgfqpoint{0.800000in}{1.864951in}}%
\pgfpathlineto{\pgfqpoint{5.760000in}{1.864951in}}%
\pgfusepath{stroke}%
\end{pgfscope}%
\begin{pgfscope}%
\pgfsetbuttcap%
\pgfsetroundjoin%
\definecolor{currentfill}{rgb}{0.000000,0.000000,0.000000}%
\pgfsetfillcolor{currentfill}%
\pgfsetlinewidth{0.803000pt}%
\definecolor{currentstroke}{rgb}{0.000000,0.000000,0.000000}%
\pgfsetstrokecolor{currentstroke}%
\pgfsetdash{}{0pt}%
\pgfsys@defobject{currentmarker}{\pgfqpoint{-0.048611in}{0.000000in}}{\pgfqpoint{-0.000000in}{0.000000in}}{%
\pgfpathmoveto{\pgfqpoint{-0.000000in}{0.000000in}}%
\pgfpathlineto{\pgfqpoint{-0.048611in}{0.000000in}}%
\pgfusepath{stroke,fill}%
}%
\begin{pgfscope}%
\pgfsys@transformshift{0.800000in}{1.864951in}%
\pgfsys@useobject{currentmarker}{}%
\end{pgfscope}%
\end{pgfscope}%
\begin{pgfscope}%
\definecolor{textcolor}{rgb}{0.000000,0.000000,0.000000}%
\pgfsetstrokecolor{textcolor}%
\pgfsetfillcolor{textcolor}%
\pgftext[x=0.539585in, y=1.807081in, left, base]{\color{textcolor}\rmfamily\fontsize{12.000000}{14.400000}\selectfont \(\displaystyle {40}\)}%
\end{pgfscope}%
\begin{pgfscope}%
\pgfpathrectangle{\pgfqpoint{0.800000in}{0.528000in}}{\pgfqpoint{4.960000in}{3.696000in}}%
\pgfusepath{clip}%
\pgfsetrectcap%
\pgfsetroundjoin%
\pgfsetlinewidth{0.803000pt}%
\definecolor{currentstroke}{rgb}{0.690196,0.690196,0.690196}%
\pgfsetstrokecolor{currentstroke}%
\pgfsetdash{}{0pt}%
\pgfpathmoveto{\pgfqpoint{0.800000in}{2.452364in}}%
\pgfpathlineto{\pgfqpoint{5.760000in}{2.452364in}}%
\pgfusepath{stroke}%
\end{pgfscope}%
\begin{pgfscope}%
\pgfsetbuttcap%
\pgfsetroundjoin%
\definecolor{currentfill}{rgb}{0.000000,0.000000,0.000000}%
\pgfsetfillcolor{currentfill}%
\pgfsetlinewidth{0.803000pt}%
\definecolor{currentstroke}{rgb}{0.000000,0.000000,0.000000}%
\pgfsetstrokecolor{currentstroke}%
\pgfsetdash{}{0pt}%
\pgfsys@defobject{currentmarker}{\pgfqpoint{-0.048611in}{0.000000in}}{\pgfqpoint{-0.000000in}{0.000000in}}{%
\pgfpathmoveto{\pgfqpoint{-0.000000in}{0.000000in}}%
\pgfpathlineto{\pgfqpoint{-0.048611in}{0.000000in}}%
\pgfusepath{stroke,fill}%
}%
\begin{pgfscope}%
\pgfsys@transformshift{0.800000in}{2.452364in}%
\pgfsys@useobject{currentmarker}{}%
\end{pgfscope}%
\end{pgfscope}%
\begin{pgfscope}%
\definecolor{textcolor}{rgb}{0.000000,0.000000,0.000000}%
\pgfsetstrokecolor{textcolor}%
\pgfsetfillcolor{textcolor}%
\pgftext[x=0.539585in, y=2.394493in, left, base]{\color{textcolor}\rmfamily\fontsize{12.000000}{14.400000}\selectfont \(\displaystyle {50}\)}%
\end{pgfscope}%
\begin{pgfscope}%
\pgfpathrectangle{\pgfqpoint{0.800000in}{0.528000in}}{\pgfqpoint{4.960000in}{3.696000in}}%
\pgfusepath{clip}%
\pgfsetrectcap%
\pgfsetroundjoin%
\pgfsetlinewidth{0.803000pt}%
\definecolor{currentstroke}{rgb}{0.690196,0.690196,0.690196}%
\pgfsetstrokecolor{currentstroke}%
\pgfsetdash{}{0pt}%
\pgfpathmoveto{\pgfqpoint{0.800000in}{3.039776in}}%
\pgfpathlineto{\pgfqpoint{5.760000in}{3.039776in}}%
\pgfusepath{stroke}%
\end{pgfscope}%
\begin{pgfscope}%
\pgfsetbuttcap%
\pgfsetroundjoin%
\definecolor{currentfill}{rgb}{0.000000,0.000000,0.000000}%
\pgfsetfillcolor{currentfill}%
\pgfsetlinewidth{0.803000pt}%
\definecolor{currentstroke}{rgb}{0.000000,0.000000,0.000000}%
\pgfsetstrokecolor{currentstroke}%
\pgfsetdash{}{0pt}%
\pgfsys@defobject{currentmarker}{\pgfqpoint{-0.048611in}{0.000000in}}{\pgfqpoint{-0.000000in}{0.000000in}}{%
\pgfpathmoveto{\pgfqpoint{-0.000000in}{0.000000in}}%
\pgfpathlineto{\pgfqpoint{-0.048611in}{0.000000in}}%
\pgfusepath{stroke,fill}%
}%
\begin{pgfscope}%
\pgfsys@transformshift{0.800000in}{3.039776in}%
\pgfsys@useobject{currentmarker}{}%
\end{pgfscope}%
\end{pgfscope}%
\begin{pgfscope}%
\definecolor{textcolor}{rgb}{0.000000,0.000000,0.000000}%
\pgfsetstrokecolor{textcolor}%
\pgfsetfillcolor{textcolor}%
\pgftext[x=0.539585in, y=2.981906in, left, base]{\color{textcolor}\rmfamily\fontsize{12.000000}{14.400000}\selectfont \(\displaystyle {60}\)}%
\end{pgfscope}%
\begin{pgfscope}%
\pgfpathrectangle{\pgfqpoint{0.800000in}{0.528000in}}{\pgfqpoint{4.960000in}{3.696000in}}%
\pgfusepath{clip}%
\pgfsetrectcap%
\pgfsetroundjoin%
\pgfsetlinewidth{0.803000pt}%
\definecolor{currentstroke}{rgb}{0.690196,0.690196,0.690196}%
\pgfsetstrokecolor{currentstroke}%
\pgfsetdash{}{0pt}%
\pgfpathmoveto{\pgfqpoint{0.800000in}{3.627189in}}%
\pgfpathlineto{\pgfqpoint{5.760000in}{3.627189in}}%
\pgfusepath{stroke}%
\end{pgfscope}%
\begin{pgfscope}%
\pgfsetbuttcap%
\pgfsetroundjoin%
\definecolor{currentfill}{rgb}{0.000000,0.000000,0.000000}%
\pgfsetfillcolor{currentfill}%
\pgfsetlinewidth{0.803000pt}%
\definecolor{currentstroke}{rgb}{0.000000,0.000000,0.000000}%
\pgfsetstrokecolor{currentstroke}%
\pgfsetdash{}{0pt}%
\pgfsys@defobject{currentmarker}{\pgfqpoint{-0.048611in}{0.000000in}}{\pgfqpoint{-0.000000in}{0.000000in}}{%
\pgfpathmoveto{\pgfqpoint{-0.000000in}{0.000000in}}%
\pgfpathlineto{\pgfqpoint{-0.048611in}{0.000000in}}%
\pgfusepath{stroke,fill}%
}%
\begin{pgfscope}%
\pgfsys@transformshift{0.800000in}{3.627189in}%
\pgfsys@useobject{currentmarker}{}%
\end{pgfscope}%
\end{pgfscope}%
\begin{pgfscope}%
\definecolor{textcolor}{rgb}{0.000000,0.000000,0.000000}%
\pgfsetstrokecolor{textcolor}%
\pgfsetfillcolor{textcolor}%
\pgftext[x=0.539585in, y=3.569319in, left, base]{\color{textcolor}\rmfamily\fontsize{12.000000}{14.400000}\selectfont \(\displaystyle {70}\)}%
\end{pgfscope}%
\begin{pgfscope}%
\pgfpathrectangle{\pgfqpoint{0.800000in}{0.528000in}}{\pgfqpoint{4.960000in}{3.696000in}}%
\pgfusepath{clip}%
\pgfsetrectcap%
\pgfsetroundjoin%
\pgfsetlinewidth{0.803000pt}%
\definecolor{currentstroke}{rgb}{0.690196,0.690196,0.690196}%
\pgfsetstrokecolor{currentstroke}%
\pgfsetdash{}{0pt}%
\pgfpathmoveto{\pgfqpoint{0.800000in}{4.214601in}}%
\pgfpathlineto{\pgfqpoint{5.760000in}{4.214601in}}%
\pgfusepath{stroke}%
\end{pgfscope}%
\begin{pgfscope}%
\pgfsetbuttcap%
\pgfsetroundjoin%
\definecolor{currentfill}{rgb}{0.000000,0.000000,0.000000}%
\pgfsetfillcolor{currentfill}%
\pgfsetlinewidth{0.803000pt}%
\definecolor{currentstroke}{rgb}{0.000000,0.000000,0.000000}%
\pgfsetstrokecolor{currentstroke}%
\pgfsetdash{}{0pt}%
\pgfsys@defobject{currentmarker}{\pgfqpoint{-0.048611in}{0.000000in}}{\pgfqpoint{-0.000000in}{0.000000in}}{%
\pgfpathmoveto{\pgfqpoint{-0.000000in}{0.000000in}}%
\pgfpathlineto{\pgfqpoint{-0.048611in}{0.000000in}}%
\pgfusepath{stroke,fill}%
}%
\begin{pgfscope}%
\pgfsys@transformshift{0.800000in}{4.214601in}%
\pgfsys@useobject{currentmarker}{}%
\end{pgfscope}%
\end{pgfscope}%
\begin{pgfscope}%
\definecolor{textcolor}{rgb}{0.000000,0.000000,0.000000}%
\pgfsetstrokecolor{textcolor}%
\pgfsetfillcolor{textcolor}%
\pgftext[x=0.539585in, y=4.156731in, left, base]{\color{textcolor}\rmfamily\fontsize{12.000000}{14.400000}\selectfont \(\displaystyle {80}\)}%
\end{pgfscope}%
\begin{pgfscope}%
\definecolor{textcolor}{rgb}{0.000000,0.000000,0.000000}%
\pgfsetstrokecolor{textcolor}%
\pgfsetfillcolor{textcolor}%
\pgftext[x=0.484029in,y=2.376000in,,bottom,rotate=90.000000]{\color{textcolor}\rmfamily\fontsize{12.000000}{14.400000}\selectfont Temperature in \(\displaystyle \mathrm{K}\)}%
\end{pgfscope}%
\begin{pgfscope}%
\pgfpathrectangle{\pgfqpoint{0.800000in}{0.528000in}}{\pgfqpoint{4.960000in}{3.696000in}}%
\pgfusepath{clip}%
\pgfsetrectcap%
\pgfsetroundjoin%
\pgfsetlinewidth{1.505625pt}%
\definecolor{currentstroke}{rgb}{0.121569,0.466667,0.705882}%
\pgfsetstrokecolor{currentstroke}%
\pgfsetdash{}{0pt}%
\pgfpathmoveto{\pgfqpoint{1.025455in}{0.701874in}}%
\pgfpathlineto{\pgfqpoint{1.061527in}{0.701874in}}%
\pgfpathlineto{\pgfqpoint{1.097600in}{0.696000in}}%
\pgfpathlineto{\pgfqpoint{1.133673in}{0.701874in}}%
\pgfpathlineto{\pgfqpoint{1.169745in}{0.713622in}}%
\pgfpathlineto{\pgfqpoint{1.205818in}{0.731245in}}%
\pgfpathlineto{\pgfqpoint{1.241891in}{0.760615in}}%
\pgfpathlineto{\pgfqpoint{1.277964in}{0.795860in}}%
\pgfpathlineto{\pgfqpoint{1.314036in}{0.842853in}}%
\pgfpathlineto{\pgfqpoint{1.350109in}{0.901594in}}%
\pgfpathlineto{\pgfqpoint{1.386182in}{0.954462in}}%
\pgfpathlineto{\pgfqpoint{1.422255in}{1.013203in}}%
\pgfpathlineto{\pgfqpoint{1.458327in}{1.071944in}}%
\pgfpathlineto{\pgfqpoint{1.494400in}{1.136559in}}%
\pgfpathlineto{\pgfqpoint{1.530473in}{1.195301in}}%
\pgfpathlineto{\pgfqpoint{1.566545in}{1.254042in}}%
\pgfpathlineto{\pgfqpoint{1.602618in}{1.306909in}}%
\pgfpathlineto{\pgfqpoint{1.638691in}{1.371524in}}%
\pgfpathlineto{\pgfqpoint{1.674764in}{1.430266in}}%
\pgfpathlineto{\pgfqpoint{1.710836in}{1.489007in}}%
\pgfpathlineto{\pgfqpoint{1.746909in}{1.553622in}}%
\pgfpathlineto{\pgfqpoint{1.782982in}{1.618238in}}%
\pgfpathlineto{\pgfqpoint{1.819055in}{1.676979in}}%
\pgfpathlineto{\pgfqpoint{1.855127in}{1.729846in}}%
\pgfpathlineto{\pgfqpoint{1.891200in}{1.794462in}}%
\pgfpathlineto{\pgfqpoint{1.927273in}{1.853203in}}%
\pgfpathlineto{\pgfqpoint{1.963345in}{1.911944in}}%
\pgfpathlineto{\pgfqpoint{1.999418in}{1.970685in}}%
\pgfpathlineto{\pgfqpoint{2.035491in}{2.029427in}}%
\pgfpathlineto{\pgfqpoint{2.071564in}{2.088168in}}%
\pgfpathlineto{\pgfqpoint{2.107636in}{2.141035in}}%
\pgfpathlineto{\pgfqpoint{2.143709in}{2.199776in}}%
\pgfpathlineto{\pgfqpoint{2.179782in}{2.264392in}}%
\pgfpathlineto{\pgfqpoint{2.215855in}{2.317259in}}%
\pgfpathlineto{\pgfqpoint{2.251927in}{2.376000in}}%
\pgfpathlineto{\pgfqpoint{2.288000in}{2.428867in}}%
\pgfpathlineto{\pgfqpoint{2.324073in}{2.487608in}}%
\pgfpathlineto{\pgfqpoint{2.360145in}{2.534601in}}%
\pgfpathlineto{\pgfqpoint{2.396218in}{2.593343in}}%
\pgfpathlineto{\pgfqpoint{2.432291in}{2.646210in}}%
\pgfpathlineto{\pgfqpoint{2.468364in}{2.699077in}}%
\pgfpathlineto{\pgfqpoint{2.504436in}{2.751944in}}%
\pgfpathlineto{\pgfqpoint{2.540509in}{2.798937in}}%
\pgfpathlineto{\pgfqpoint{2.576582in}{2.851804in}}%
\pgfpathlineto{\pgfqpoint{2.612655in}{2.904671in}}%
\pgfpathlineto{\pgfqpoint{2.648727in}{2.951664in}}%
\pgfpathlineto{\pgfqpoint{2.684800in}{2.998657in}}%
\pgfpathlineto{\pgfqpoint{2.720873in}{3.051524in}}%
\pgfpathlineto{\pgfqpoint{2.756945in}{3.092643in}}%
\pgfpathlineto{\pgfqpoint{2.793018in}{3.139636in}}%
\pgfpathlineto{\pgfqpoint{2.829091in}{3.186629in}}%
\pgfpathlineto{\pgfqpoint{2.865164in}{3.227748in}}%
\pgfpathlineto{\pgfqpoint{2.901236in}{3.268867in}}%
\pgfpathlineto{\pgfqpoint{2.937309in}{3.309986in}}%
\pgfpathlineto{\pgfqpoint{2.973382in}{3.351105in}}%
\pgfpathlineto{\pgfqpoint{3.009455in}{3.392224in}}%
\pgfpathlineto{\pgfqpoint{3.045527in}{3.433343in}}%
\pgfpathlineto{\pgfqpoint{3.081600in}{3.474462in}}%
\pgfpathlineto{\pgfqpoint{3.117673in}{3.509706in}}%
\pgfpathlineto{\pgfqpoint{3.153745in}{3.550825in}}%
\pgfpathlineto{\pgfqpoint{3.189818in}{3.580196in}}%
\pgfpathlineto{\pgfqpoint{3.225891in}{3.621315in}}%
\pgfpathlineto{\pgfqpoint{3.261964in}{3.650685in}}%
\pgfpathlineto{\pgfqpoint{3.298036in}{3.685930in}}%
\pgfpathlineto{\pgfqpoint{3.334109in}{3.715301in}}%
\pgfpathlineto{\pgfqpoint{3.370182in}{3.738797in}}%
\pgfpathlineto{\pgfqpoint{3.406255in}{3.768168in}}%
\pgfpathlineto{\pgfqpoint{3.442327in}{3.785790in}}%
\pgfpathlineto{\pgfqpoint{3.478400in}{3.803413in}}%
\pgfpathlineto{\pgfqpoint{3.514473in}{3.821035in}}%
\pgfpathlineto{\pgfqpoint{3.550545in}{3.832783in}}%
\pgfpathlineto{\pgfqpoint{3.586618in}{3.403972in}}%
\pgfpathlineto{\pgfqpoint{3.622691in}{3.427469in}}%
\pgfpathlineto{\pgfqpoint{3.658764in}{3.445091in}}%
\pgfpathlineto{\pgfqpoint{3.694836in}{3.456839in}}%
\pgfpathlineto{\pgfqpoint{3.730909in}{3.468587in}}%
\pgfpathlineto{\pgfqpoint{3.766982in}{3.480336in}}%
\pgfpathlineto{\pgfqpoint{3.803055in}{3.492084in}}%
\pgfpathlineto{\pgfqpoint{3.839127in}{3.497958in}}%
\pgfpathlineto{\pgfqpoint{3.875200in}{3.503832in}}%
\pgfpathlineto{\pgfqpoint{3.911273in}{3.509706in}}%
\pgfpathlineto{\pgfqpoint{3.947345in}{3.450965in}}%
\pgfpathlineto{\pgfqpoint{3.983418in}{3.245371in}}%
\pgfpathlineto{\pgfqpoint{4.019491in}{3.180755in}}%
\pgfpathlineto{\pgfqpoint{4.055564in}{3.186629in}}%
\pgfpathlineto{\pgfqpoint{4.091636in}{3.198378in}}%
\pgfpathlineto{\pgfqpoint{4.127709in}{3.204252in}}%
\pgfpathlineto{\pgfqpoint{4.163782in}{3.210126in}}%
\pgfpathlineto{\pgfqpoint{4.199855in}{3.210126in}}%
\pgfpathlineto{\pgfqpoint{4.235927in}{3.210126in}}%
\pgfpathlineto{\pgfqpoint{4.272000in}{3.216000in}}%
\pgfpathlineto{\pgfqpoint{4.308073in}{2.205650in}}%
\pgfpathlineto{\pgfqpoint{4.344145in}{2.223273in}}%
\pgfpathlineto{\pgfqpoint{4.380218in}{2.229147in}}%
\pgfpathlineto{\pgfqpoint{4.416291in}{2.240895in}}%
\pgfpathlineto{\pgfqpoint{4.452364in}{2.246769in}}%
\pgfpathlineto{\pgfqpoint{4.488436in}{2.252643in}}%
\pgfpathlineto{\pgfqpoint{4.524509in}{2.264392in}}%
\pgfpathlineto{\pgfqpoint{4.560582in}{2.264392in}}%
\pgfpathlineto{\pgfqpoint{4.596655in}{2.270266in}}%
\pgfpathlineto{\pgfqpoint{4.632727in}{2.276140in}}%
\pgfpathlineto{\pgfqpoint{4.668800in}{2.276140in}}%
\pgfpathlineto{\pgfqpoint{4.704873in}{2.282014in}}%
\pgfpathlineto{\pgfqpoint{4.740945in}{2.287888in}}%
\pgfpathlineto{\pgfqpoint{4.777018in}{2.287888in}}%
\pgfpathlineto{\pgfqpoint{4.813091in}{2.293762in}}%
\pgfpathlineto{\pgfqpoint{4.849164in}{2.293762in}}%
\pgfpathlineto{\pgfqpoint{4.885236in}{2.293762in}}%
\pgfpathlineto{\pgfqpoint{4.921309in}{2.299636in}}%
\pgfpathlineto{\pgfqpoint{4.957382in}{2.299636in}}%
\pgfpathlineto{\pgfqpoint{4.993455in}{2.305510in}}%
\pgfpathlineto{\pgfqpoint{5.029527in}{2.305510in}}%
\pgfpathlineto{\pgfqpoint{5.065600in}{2.305510in}}%
\pgfpathlineto{\pgfqpoint{5.101673in}{2.299636in}}%
\pgfpathlineto{\pgfqpoint{5.137745in}{2.305510in}}%
\pgfpathlineto{\pgfqpoint{5.173818in}{2.305510in}}%
\pgfpathlineto{\pgfqpoint{5.209891in}{2.305510in}}%
\pgfpathlineto{\pgfqpoint{5.245964in}{2.311385in}}%
\pgfpathlineto{\pgfqpoint{5.282036in}{2.305510in}}%
\pgfpathlineto{\pgfqpoint{5.318109in}{2.305510in}}%
\pgfpathlineto{\pgfqpoint{5.354182in}{2.305510in}}%
\pgfpathlineto{\pgfqpoint{5.390255in}{2.305510in}}%
\pgfpathlineto{\pgfqpoint{5.426327in}{2.305510in}}%
\pgfpathlineto{\pgfqpoint{5.462400in}{2.305510in}}%
\pgfpathlineto{\pgfqpoint{5.498473in}{2.305510in}}%
\pgfpathlineto{\pgfqpoint{5.534545in}{2.305510in}}%
\pgfusepath{stroke}%
\end{pgfscope}%
\begin{pgfscope}%
\pgfpathrectangle{\pgfqpoint{0.800000in}{0.528000in}}{\pgfqpoint{4.960000in}{3.696000in}}%
\pgfusepath{clip}%
\pgfsetrectcap%
\pgfsetroundjoin%
\pgfsetlinewidth{1.505625pt}%
\definecolor{currentstroke}{rgb}{1.000000,0.498039,0.054902}%
\pgfsetstrokecolor{currentstroke}%
\pgfsetdash{}{0pt}%
\pgfpathmoveto{\pgfqpoint{1.025455in}{0.807608in}}%
\pgfpathlineto{\pgfqpoint{1.061527in}{0.807608in}}%
\pgfpathlineto{\pgfqpoint{1.097600in}{0.813483in}}%
\pgfpathlineto{\pgfqpoint{1.133673in}{0.819357in}}%
\pgfpathlineto{\pgfqpoint{1.169745in}{0.836979in}}%
\pgfpathlineto{\pgfqpoint{1.205818in}{0.854601in}}%
\pgfpathlineto{\pgfqpoint{1.241891in}{0.889846in}}%
\pgfpathlineto{\pgfqpoint{1.277964in}{0.936839in}}%
\pgfpathlineto{\pgfqpoint{1.314036in}{0.983832in}}%
\pgfpathlineto{\pgfqpoint{1.350109in}{1.048448in}}%
\pgfpathlineto{\pgfqpoint{1.386182in}{1.107189in}}%
\pgfpathlineto{\pgfqpoint{1.422255in}{1.171804in}}%
\pgfpathlineto{\pgfqpoint{1.458327in}{1.236420in}}%
\pgfpathlineto{\pgfqpoint{1.494400in}{1.295161in}}%
\pgfpathlineto{\pgfqpoint{1.530473in}{1.359776in}}%
\pgfpathlineto{\pgfqpoint{1.566545in}{1.424392in}}%
\pgfpathlineto{\pgfqpoint{1.602618in}{1.489007in}}%
\pgfpathlineto{\pgfqpoint{1.638691in}{1.547748in}}%
\pgfpathlineto{\pgfqpoint{1.674764in}{1.612364in}}%
\pgfpathlineto{\pgfqpoint{1.710836in}{1.676979in}}%
\pgfpathlineto{\pgfqpoint{1.746909in}{1.741594in}}%
\pgfpathlineto{\pgfqpoint{1.782982in}{1.806210in}}%
\pgfpathlineto{\pgfqpoint{1.819055in}{1.870825in}}%
\pgfpathlineto{\pgfqpoint{1.855127in}{1.929566in}}%
\pgfpathlineto{\pgfqpoint{1.891200in}{1.994182in}}%
\pgfpathlineto{\pgfqpoint{1.927273in}{2.052923in}}%
\pgfpathlineto{\pgfqpoint{1.963345in}{2.117538in}}%
\pgfpathlineto{\pgfqpoint{1.999418in}{2.170406in}}%
\pgfpathlineto{\pgfqpoint{2.035491in}{2.235021in}}%
\pgfpathlineto{\pgfqpoint{2.071564in}{2.293762in}}%
\pgfpathlineto{\pgfqpoint{2.107636in}{2.352503in}}%
\pgfpathlineto{\pgfqpoint{2.143709in}{2.411245in}}%
\pgfpathlineto{\pgfqpoint{2.179782in}{2.469986in}}%
\pgfpathlineto{\pgfqpoint{2.215855in}{2.528727in}}%
\pgfpathlineto{\pgfqpoint{2.251927in}{2.581594in}}%
\pgfpathlineto{\pgfqpoint{2.288000in}{2.640336in}}%
\pgfpathlineto{\pgfqpoint{2.324073in}{2.699077in}}%
\pgfpathlineto{\pgfqpoint{2.360145in}{2.751944in}}%
\pgfpathlineto{\pgfqpoint{2.396218in}{2.816559in}}%
\pgfpathlineto{\pgfqpoint{2.432291in}{2.851804in}}%
\pgfpathlineto{\pgfqpoint{2.468364in}{2.910545in}}%
\pgfpathlineto{\pgfqpoint{2.504436in}{2.957538in}}%
\pgfpathlineto{\pgfqpoint{2.540509in}{3.016280in}}%
\pgfpathlineto{\pgfqpoint{2.576582in}{3.063273in}}%
\pgfpathlineto{\pgfqpoint{2.612655in}{3.122014in}}%
\pgfpathlineto{\pgfqpoint{2.648727in}{3.169007in}}%
\pgfpathlineto{\pgfqpoint{2.684800in}{3.216000in}}%
\pgfpathlineto{\pgfqpoint{2.720873in}{3.274741in}}%
\pgfpathlineto{\pgfqpoint{2.756945in}{3.315860in}}%
\pgfpathlineto{\pgfqpoint{2.793018in}{3.362853in}}%
\pgfpathlineto{\pgfqpoint{2.829091in}{3.409846in}}%
\pgfpathlineto{\pgfqpoint{2.865164in}{3.445091in}}%
\pgfpathlineto{\pgfqpoint{2.901236in}{3.492084in}}%
\pgfpathlineto{\pgfqpoint{2.937309in}{3.533203in}}%
\pgfpathlineto{\pgfqpoint{2.973382in}{3.574322in}}%
\pgfpathlineto{\pgfqpoint{3.009455in}{3.621315in}}%
\pgfpathlineto{\pgfqpoint{3.045527in}{3.662434in}}%
\pgfpathlineto{\pgfqpoint{3.081600in}{3.709427in}}%
\pgfpathlineto{\pgfqpoint{3.117673in}{3.738797in}}%
\pgfpathlineto{\pgfqpoint{3.153745in}{3.779916in}}%
\pgfpathlineto{\pgfqpoint{3.189818in}{3.815161in}}%
\pgfpathlineto{\pgfqpoint{3.225891in}{3.850406in}}%
\pgfpathlineto{\pgfqpoint{3.261964in}{3.885650in}}%
\pgfpathlineto{\pgfqpoint{3.298036in}{3.926769in}}%
\pgfpathlineto{\pgfqpoint{3.334109in}{3.950266in}}%
\pgfpathlineto{\pgfqpoint{3.370182in}{3.973762in}}%
\pgfpathlineto{\pgfqpoint{3.406255in}{3.997259in}}%
\pgfpathlineto{\pgfqpoint{3.442327in}{4.020755in}}%
\pgfpathlineto{\pgfqpoint{3.478400in}{4.038378in}}%
\pgfpathlineto{\pgfqpoint{3.514473in}{4.056000in}}%
\pgfpathlineto{\pgfqpoint{3.550545in}{4.044252in}}%
\pgfpathlineto{\pgfqpoint{3.586618in}{3.633063in}}%
\pgfpathlineto{\pgfqpoint{3.622691in}{3.650685in}}%
\pgfpathlineto{\pgfqpoint{3.658764in}{3.674182in}}%
\pgfpathlineto{\pgfqpoint{3.694836in}{3.685930in}}%
\pgfpathlineto{\pgfqpoint{3.730909in}{3.703552in}}%
\pgfpathlineto{\pgfqpoint{3.766982in}{3.709427in}}%
\pgfpathlineto{\pgfqpoint{3.803055in}{3.715301in}}%
\pgfpathlineto{\pgfqpoint{3.839127in}{3.721175in}}%
\pgfpathlineto{\pgfqpoint{3.875200in}{3.727049in}}%
\pgfpathlineto{\pgfqpoint{3.911273in}{3.732923in}}%
\pgfpathlineto{\pgfqpoint{3.947345in}{3.668308in}}%
\pgfpathlineto{\pgfqpoint{3.983418in}{3.439217in}}%
\pgfpathlineto{\pgfqpoint{4.019491in}{3.403972in}}%
\pgfpathlineto{\pgfqpoint{4.055564in}{3.415720in}}%
\pgfpathlineto{\pgfqpoint{4.091636in}{3.415720in}}%
\pgfpathlineto{\pgfqpoint{4.127709in}{3.421594in}}%
\pgfpathlineto{\pgfqpoint{4.163782in}{3.427469in}}%
\pgfpathlineto{\pgfqpoint{4.199855in}{3.433343in}}%
\pgfpathlineto{\pgfqpoint{4.235927in}{3.433343in}}%
\pgfpathlineto{\pgfqpoint{4.272000in}{2.904671in}}%
\pgfpathlineto{\pgfqpoint{4.308073in}{2.434741in}}%
\pgfpathlineto{\pgfqpoint{4.344145in}{2.446490in}}%
\pgfpathlineto{\pgfqpoint{4.380218in}{2.452364in}}%
\pgfpathlineto{\pgfqpoint{4.416291in}{2.464112in}}%
\pgfpathlineto{\pgfqpoint{4.452364in}{2.464112in}}%
\pgfpathlineto{\pgfqpoint{4.488436in}{2.475860in}}%
\pgfpathlineto{\pgfqpoint{4.524509in}{2.481734in}}%
\pgfpathlineto{\pgfqpoint{4.560582in}{2.487608in}}%
\pgfpathlineto{\pgfqpoint{4.596655in}{2.487608in}}%
\pgfpathlineto{\pgfqpoint{4.632727in}{2.493483in}}%
\pgfpathlineto{\pgfqpoint{4.668800in}{2.499357in}}%
\pgfpathlineto{\pgfqpoint{4.704873in}{2.499357in}}%
\pgfpathlineto{\pgfqpoint{4.740945in}{2.505231in}}%
\pgfpathlineto{\pgfqpoint{4.777018in}{2.505231in}}%
\pgfpathlineto{\pgfqpoint{4.813091in}{2.511105in}}%
\pgfpathlineto{\pgfqpoint{4.849164in}{2.511105in}}%
\pgfpathlineto{\pgfqpoint{4.885236in}{2.511105in}}%
\pgfpathlineto{\pgfqpoint{4.921309in}{2.516979in}}%
\pgfpathlineto{\pgfqpoint{4.957382in}{2.516979in}}%
\pgfpathlineto{\pgfqpoint{4.993455in}{2.522853in}}%
\pgfpathlineto{\pgfqpoint{5.029527in}{2.522853in}}%
\pgfpathlineto{\pgfqpoint{5.065600in}{2.522853in}}%
\pgfpathlineto{\pgfqpoint{5.101673in}{2.522853in}}%
\pgfpathlineto{\pgfqpoint{5.137745in}{2.522853in}}%
\pgfpathlineto{\pgfqpoint{5.173818in}{2.516979in}}%
\pgfpathlineto{\pgfqpoint{5.209891in}{2.522853in}}%
\pgfpathlineto{\pgfqpoint{5.245964in}{2.522853in}}%
\pgfpathlineto{\pgfqpoint{5.282036in}{2.522853in}}%
\pgfpathlineto{\pgfqpoint{5.318109in}{2.522853in}}%
\pgfpathlineto{\pgfqpoint{5.354182in}{2.522853in}}%
\pgfpathlineto{\pgfqpoint{5.390255in}{2.522853in}}%
\pgfpathlineto{\pgfqpoint{5.426327in}{2.522853in}}%
\pgfpathlineto{\pgfqpoint{5.462400in}{2.522853in}}%
\pgfpathlineto{\pgfqpoint{5.498473in}{2.522853in}}%
\pgfpathlineto{\pgfqpoint{5.534545in}{2.522853in}}%
\pgfusepath{stroke}%
\end{pgfscope}%
\begin{pgfscope}%
\pgfsetrectcap%
\pgfsetmiterjoin%
\pgfsetlinewidth{0.803000pt}%
\definecolor{currentstroke}{rgb}{0.000000,0.000000,0.000000}%
\pgfsetstrokecolor{currentstroke}%
\pgfsetdash{}{0pt}%
\pgfpathmoveto{\pgfqpoint{0.800000in}{0.528000in}}%
\pgfpathlineto{\pgfqpoint{0.800000in}{4.224000in}}%
\pgfusepath{stroke}%
\end{pgfscope}%
\begin{pgfscope}%
\pgfsetrectcap%
\pgfsetmiterjoin%
\pgfsetlinewidth{0.803000pt}%
\definecolor{currentstroke}{rgb}{0.000000,0.000000,0.000000}%
\pgfsetstrokecolor{currentstroke}%
\pgfsetdash{}{0pt}%
\pgfpathmoveto{\pgfqpoint{5.760000in}{0.528000in}}%
\pgfpathlineto{\pgfqpoint{5.760000in}{4.224000in}}%
\pgfusepath{stroke}%
\end{pgfscope}%
\begin{pgfscope}%
\pgfsetrectcap%
\pgfsetmiterjoin%
\pgfsetlinewidth{0.803000pt}%
\definecolor{currentstroke}{rgb}{0.000000,0.000000,0.000000}%
\pgfsetstrokecolor{currentstroke}%
\pgfsetdash{}{0pt}%
\pgfpathmoveto{\pgfqpoint{0.800000in}{0.528000in}}%
\pgfpathlineto{\pgfqpoint{5.760000in}{0.528000in}}%
\pgfusepath{stroke}%
\end{pgfscope}%
\begin{pgfscope}%
\pgfsetrectcap%
\pgfsetmiterjoin%
\pgfsetlinewidth{0.803000pt}%
\definecolor{currentstroke}{rgb}{0.000000,0.000000,0.000000}%
\pgfsetstrokecolor{currentstroke}%
\pgfsetdash{}{0pt}%
\pgfpathmoveto{\pgfqpoint{0.800000in}{4.224000in}}%
\pgfpathlineto{\pgfqpoint{5.760000in}{4.224000in}}%
\pgfusepath{stroke}%
\end{pgfscope}%
\begin{pgfscope}%
\pgfsetbuttcap%
\pgfsetmiterjoin%
\definecolor{currentfill}{rgb}{1.000000,1.000000,1.000000}%
\pgfsetfillcolor{currentfill}%
\pgfsetfillopacity{0.800000}%
\pgfsetlinewidth{1.003750pt}%
\definecolor{currentstroke}{rgb}{0.800000,0.800000,0.800000}%
\pgfsetstrokecolor{currentstroke}%
\pgfsetstrokeopacity{0.800000}%
\pgfsetdash{}{0pt}%
\pgfpathmoveto{\pgfqpoint{0.916667in}{3.625852in}}%
\pgfpathlineto{\pgfqpoint{2.814473in}{3.625852in}}%
\pgfpathquadraticcurveto{\pgfqpoint{2.847806in}{3.625852in}}{\pgfqpoint{2.847806in}{3.659186in}}%
\pgfpathlineto{\pgfqpoint{2.847806in}{4.107333in}}%
\pgfpathquadraticcurveto{\pgfqpoint{2.847806in}{4.140667in}}{\pgfqpoint{2.814473in}{4.140667in}}%
\pgfpathlineto{\pgfqpoint{0.916667in}{4.140667in}}%
\pgfpathquadraticcurveto{\pgfqpoint{0.883333in}{4.140667in}}{\pgfqpoint{0.883333in}{4.107333in}}%
\pgfpathlineto{\pgfqpoint{0.883333in}{3.659186in}}%
\pgfpathquadraticcurveto{\pgfqpoint{0.883333in}{3.625852in}}{\pgfqpoint{0.916667in}{3.625852in}}%
\pgfpathlineto{\pgfqpoint{0.916667in}{3.625852in}}%
\pgfpathclose%
\pgfusepath{stroke,fill}%
\end{pgfscope}%
\begin{pgfscope}%
\pgfsetrectcap%
\pgfsetroundjoin%
\pgfsetlinewidth{1.505625pt}%
\definecolor{currentstroke}{rgb}{0.121569,0.466667,0.705882}%
\pgfsetstrokecolor{currentstroke}%
\pgfsetdash{}{0pt}%
\pgfpathmoveto{\pgfqpoint{0.950000in}{4.015667in}}%
\pgfpathlineto{\pgfqpoint{1.116667in}{4.015667in}}%
\pgfpathlineto{\pgfqpoint{1.283333in}{4.015667in}}%
\pgfusepath{stroke}%
\end{pgfscope}%
\begin{pgfscope}%
\definecolor{textcolor}{rgb}{0.000000,0.000000,0.000000}%
\pgfsetstrokecolor{textcolor}%
\pgfsetfillcolor{textcolor}%
\pgftext[x=1.416667in,y=3.957333in,left,base]{\color{textcolor}\rmfamily\fontsize{12.000000}{14.400000}\selectfont T by thermocouple}%
\end{pgfscope}%
\begin{pgfscope}%
\pgfsetrectcap%
\pgfsetroundjoin%
\pgfsetlinewidth{1.505625pt}%
\definecolor{currentstroke}{rgb}{1.000000,0.498039,0.054902}%
\pgfsetstrokecolor{currentstroke}%
\pgfsetdash{}{0pt}%
\pgfpathmoveto{\pgfqpoint{0.950000in}{3.783260in}}%
\pgfpathlineto{\pgfqpoint{1.116667in}{3.783260in}}%
\pgfpathlineto{\pgfqpoint{1.283333in}{3.783260in}}%
\pgfusepath{stroke}%
\end{pgfscope}%
\begin{pgfscope}%
\definecolor{textcolor}{rgb}{0.000000,0.000000,0.000000}%
\pgfsetstrokecolor{textcolor}%
\pgfsetfillcolor{textcolor}%
\pgftext[x=1.416667in,y=3.724926in,left,base]{\color{textcolor}\rmfamily\fontsize{12.000000}{14.400000}\selectfont T by Raman}%
\end{pgfscope}%
\end{pgfpicture}%
\makeatother%
\endgroup%

    \caption[Temperature comparison between Thermocouple and Raman measurement]{Temperature comparison between measurement with the Thermocouple type K and through the Raman setup; plot of temperature over the measurement time}
    \label{fig:plot-temp-time}
\end{figure}

\mycomment[MK]{Error discussion Temp missing}