%%%%%%%%%%%%%%%%%%%%%%%%%%%%%%%%%%%%%%%%%%%%%%%%%%%%%%%%%%%%%%%%%%%%%%%%%%%%%%%
%                                   Einstellungen
%
% Hier können alle relevanten Einstellungen für diese Arbeit gesetzt werden.
% Dazu gehören Angaben u.a. über den Autor sowie Formatierungen.
%%%%%%%%%%%%%%%%%%%%%%%%%%%%%%%%%%%%%%%%%%%%%%%%%%%%%%%%%%%%%%%%%%%%%%%%%%%%%%%


%%%%%%%%%%%%%%%%%%%%%%%%%%%%%%%%%%%% Sprache %%%%%%%%%%%%%%%%%%%%%%%%%%%%%%%%%%%
%% Aktuell sind Deutsch und Englisch unterstützt.
%% Es werden nicht nur alle vom Dokument erzeugten Texte in
%% der entsprechenden Sprache angezeigt, sondern auch weitere
%% Aspekte angepasst, wie z.B. die Anführungszeichen und
%% Datumsformate.
\setzesprache{en} % de oder en
%%%%%%%%%%%%%%%%%%%%%%%%%%%%%%%%%%%%%%%%%%%%%%%%%%%%%%%%%%%%%%%%%%%%%%%%%%%%%%%%



%%%%%%%%%%%%%%%%%%%%%%%%%%%%%%%%%%% Angaben  %%%%%%%%%%%%%%%%%%%%%%%%%%%%%%%%%%%
%% Die meisten der folgenden Daten werden auf dem
%% Deckblatt angezeigt, einige auch im weiteren Verlauf
%% des Dokuments.
\setzematrikelnr{}
\setzekurs{}
\setzetitel{Optical Diagnostics in Energy and Process Engineering - Application of Raman spectroscopy}
\setzedatumAbgabe{}
\setzefirma{}
\setzefirmenort{}
\setzeabgabeort{Erlangen}
\setzestudiengang{Energy Technology}
\setzebetreuer{M.Sc. Philipp Bräuer}
\setzegutachter{}
\setzezeitraum{27.06.2022\ -\ 19.09.2022}
\setzearbeit{Laboratory course report}
\setzeautor{Maximilian Köhler, Jean-Pascal Lafleur}
\setzeversion{0.1}




%%%%%%%%%%%%%%%%%%%%%%%%%%%% Literaturverzeichnis %%%%%%%%%%%%%%%%%%%%%%%%%%%%%%
%% Bei Fehlern während der Verarbeitung bitte in ads/header.tex bei der
%% Einbindung des Pakets biblatex (ungefähr ab Zeile 110,
%% einmal für jede Sprache), biber in bibtex ändern.
\newcommand{\ladeliteratur}{%
\addbibresource{literatur.bib}
%\addbibresource{weitereDatei.bib}
}
%% Zitierstil
%% siehe: http://ctan.mirrorcatalogs.com/macros/latex/contrib/biblatex/doc/biblatex.pdf (3.3.1 Citation Styles)
%% mögliche Werte z.B numeric-comp, alphabetic, authoryear, numeric
\setzezitierstil{ieee}
%\setzezitierstil{alphabetic}
%\setzezitierstil{authoryear}




%%%%%%%%%%%%%%%%%%%%%%%%%%%%%%%%% Layout %%%%%%%%%%%%%%%%%%%%%%%%%%%%%%%%%%%%%%%
%% Verschiedene Schriftarten
% laut nag Warnung: palatino obsolete, use mathpazo, helvet (option scaled=.95), courier instead
\setzeschriftart{charter} % palatino oder goudysans, lmodern, libertine, mathptmx
% palatino oder lmodern ganz nett

%% Paket um Textteile drehen zu können
%\usepackage{rotating}
%% Paket um Seite im Querformat anzuzeigen
%\usepackage{lscape}

%% Seitenränder
% \setzeseitenrand{2.5cm}

%% Abstand vor Kapitelüberschriften zum oberen Seitenrand
\setzekapitelabstand{0pt}

%% Spaltenabstand
\setzespaltenabstand{6pt}
%% Zeilenabstand innerhalb einer Tabelle
\setzezeilenabstand{1.5}




%%%%%%%%%%%%%%%%%%%%%%%%%%%%% Verschiedenes  %%%%%%%%%%%%%%%%%%%%%%%%%%%%%%%%%%%
%% Farben (Angabe in HTML-Notation mit großen Buchstaben)
\newcommand{\ladefarben}{%
	\definecolor{LinkColor}{HTML}{00007A}
	% \definecolor{ListingBackground}{HTML}{FCF7DE}
}

%%Mathematikpakete benutzen (Pakete aktivieren)
\usepackage{amsmath}
\usepackage{amssymb}

%% Programmiersprachen Highlighting (Listings)
\newcommand{\listingsettings}{%
	\lstset{%
		language=Java,			% Standardsprache des Quellcodes
		numbers=left,			% Zeilennummern links
		stepnumber=1,			% Jede Zeile nummerieren.
		numbersep=5pt,			% 5pt Abstand zum Quellcode
		numberstyle=\tiny,		% Zeichengrösse 'tiny' für die Nummern.
		breaklines=true,		% Zeilen umbrechen wenn notwendig.
		breakautoindent=true,	% Nach dem Zeilenumbruch Zeile einrücken.
		postbreak=\space,		% Bei Leerzeichen umbrechen.
		tabsize=2,				% Tabulatorgrösse 2
		basicstyle=\ttfamily\footnotesize, % Nichtproportionale Schrift, klein für den Quellcode
		showspaces=false,		% Leerzeichen nicht anzeigen.
		showstringspaces=false,	% Leerzeichen auch in Strings ('') nicht anzeigen.
		extendedchars=true,		% Alle Zeichen vom Latin1 Zeichensatz anzeigen.
		captionpos=b,			% sets the caption-position to bottom
		% backgroundcolor=\color{ListingBackground}, % Hintergrundfarbe des Quellcodes setzen.
    backgroundcolor=\color{white!50},
		xleftmargin=0pt,		% Rand links
		xrightmargin=0pt,		% Rand rechts
		frame=single,			% Rahmen an
		frameround=ffff,
		rulecolor=\color{black},	% Rahmenfarbe
		% fillcolor=\color{ListingBackground},
    fillcolor=\color{white!50},
		keywordstyle=\color[rgb]{0.133,0.133,0.6},
		commentstyle=\color[rgb]{0.133,0.545,0.133},
		%stringstyle=\color[rgb]{0.627,0.126,0.941}
		stringstyle=\color{red}
	}
}




%%%%%%%%%%%%%%%%%%%%%%%%%%%%%%%% Eigenes %%%%%%%%%%%%%%%%%%%%%%%%%%%%%%%%%%%%%%%
%% Hier können Ergänzungen zur Präambel vorgenommen werden (eigene Pakete, Einstellungen)

%colorpackage
% \usepackage{color}
\usepackage[table,dvipsnames]{xcolor}


%listing package for defining a new language
\usepackage{listings}

% starting footnotes at 1
% \usepackage{perpage}
% \MakePerPage[1]{footnote}

%Functions
\def\bsq#1{%both single quotes
\lq{#1}\rq}


%%%%%%%%%%%%%%%%%%%%%%%%%%%%%%%%%%%%%%%%%%%%
% selfmade language pakets
% JavaScript language
\definecolor{lightgray}{RGB}{227, 227, 227}
\definecolor{darkgray}{RGB}{100, 100, 100}
\definecolor{purple}{rgb}{0.65, 0.12, 0.82}
\definecolor{schaeffler}{RGB}{0, 137, 61}
\definecolor{dark-green}{RGB}{0, 110, 93}
\definecolor{middle-green}{RGB}{115, 161, 149}
\definecolor{light-green}{RGB}{199, 222, 160}

\definecolor{pie1}{RGB}{115, 161, 149}
\definecolor{pie2}{RGB}{192, 198, 191}
\definecolor{pie3}{RGB}{135, 135, 135}
\definecolor{pie4}{RGB}{29, 155, 178}
\definecolor{pie5}{RGB}{182, 186, 194}
\definecolor{pie6}{RGB}{161, 200, 97}
\definecolor{pie7}{RGB}{67, 99, 91}
\definecolor{pie8}{RGB}{112, 123, 110}
\definecolor{pie9}{RGB}{113, 113, 113}

\lstdefinelanguage{JS}{
  keywords={break, case, catch, continue, debugger, default, delete, do, else, false, finally, for, function, if, in, instanceof, new, null, return, switch, this, throw, true, try, typeof, var, void, while, with},
  morecomment=[l]{//},
  morecomment=[s]{/*}{*/},
  morestring=[b]',
  morestring=[b]",
  ndkeywords={class, export, boolean, throw, implements, import, this},
  keywordstyle=\color{blue}\bfseries,
  ndkeywordstyle=\color{darkgray}\bfseries,
  identifierstyle=\color{black},
  commentstyle=\color[rgb]{0.133,0.545,0.133},
  %commentstyle=\color{purple}\ttfamily,
  stringstyle=\color{red}\ttfamily,
  sensitive=true
}

% Einstellung für die Einbindung von Code - hier: Python-Code - nicht ändern!
\lstdefinestyle{style-python}
{
  language=python,	% Programmiersorache einstellen, Latex erkennt dann automatisch code, kommentare, funktionen,...
  basicstyle=\scriptsize\ttfamily,	% Schriftformatierung, ttfamily: text im Schreibmaschinen-Style
  backgroundcolor=\color{white},		% Hintergrundfarbe
  breaklines=true,	% automatischer Zeilenumbruch bei langen Zeilen, funktioniert nur bei bestimmen Zeichen, z.B. Umbruch nach Leerzeichen
  keywordstyle=\bfseries\ttfamily\color{blue},	% Schlüsselwörter einstellen
  stringstyle=\ttfamily\color{Peach},			% Textsrtrings
  showstringspaces=false,	% Leerzeichen in Strings richtig darstellen
  commentstyle=\color{ForestGreen}\ttfamily,	% Kommentare
  flexiblecolumns=false,	% Spaltenbreite dynamisch/fest
  numbers=left,		% Position der Zeilennummern
  numberstyle=\tiny,	% Größe der Zeilennummern
  numberblanklines=false,		% leere Zeilen werde mit ‚false‘ nicht durchnummeriert
  stepnumber=1,		% Beginn der Nummerierung
  numbersep=10pt,		% Abstand zwischen Zeilennummern und Quellcode
  xleftmargin=20pt,	% Abstand zum linken Rand
  xrightmargin=10pt,	% Abstand zum rechten Rand
  extendedchars=true,	% Sonderzeichen korrekt darstellen
  frame=trbl,			% Rahmen um gesamten Code: Top, right, bottom, left (Großbuchstaben ergeben Doppellinien)
  frameround=ffff,	% Ecken des Rahmens anpassen, t: runde Ecken, f: default (eckig), es müssen 4 Buchstaben da stehen!
  literate=		% ersetzen von Zeichen 1 durch Zeichen 2, hier: korrekte Einbindung der Sonderzeichen
   {Ö}{{\"O}}1 
   {Ä}{{\"A}}1 
   {Ü}{{\"U}}1 
   {ß}{{\ss}}1 
   {ü}{{\"u}}1 
   {ä}{{\"a}}1 
   {ö}{{\"o}}1,
  % mit ‚emph‘ und ‚emphstyle‘ können eigene Styles für Wörter angelegt werden
  emph = [1]{clc, color},
  emphstyle = [1]{\color{blue}},
  emph = [2]{function, endfunction},
  emphstyle = [2]{\color{BrickRed}},
  emph = [3]{gcf},
  emphstyle = [3]{\color{black}},
}

% Einstellung für die Einbindung von Code - hier: C-Code - nicht ändern!
\lstdefinestyle{style-c}
{
  language=c,	% Programmiersorache einstellen, Latex erkennt dann automatisch code, kommentare, funktionen,...
  basicstyle=\scriptsize\ttfamily,	% Schriftformatierung, ttfamily: text im Schreibmaschinen-Style
  backgroundcolor=\color{white},		% Hintergrundfarbe
  breaklines=true,	% automatischer Zeilenumbruch bei langen Zeilen, funktioniert nur bei bestimmen Zeichen, z.B. Umbruch nach Leerzeichen
  keywordstyle=\bfseries\ttfamily\color{blue},	% Schlüsselwörter einstellen
  stringstyle=\ttfamily\color{Peach},			% Textsrtrings
  showstringspaces=false,	% Leerzeichen in Strings richtig darstellen
  commentstyle=\color{ForestGreen}\ttfamily,	% Kommentare
  flexiblecolumns=false,	% Spaltenbreite dynamisch/fest
  numbers=left,		% Position der Zeilennummern
  numberstyle=\tiny,	% Größe der Zeilennummern
  numberblanklines=false,		% leere Zeilen werde mit ‚false‘ nicht durchnummeriert
  stepnumber=1,		% Beginn der Nummerierung
  numbersep=10pt,		% Abstand zwischen Zeilennummern und Quellcode
  xleftmargin=20pt,	% Abstand zum linken Rand
  xrightmargin=10pt,	% Abstand zum rechten Rand
  extendedchars=true,	% Sonderzeichen korrekt darstellen
  frame=trbl,			% Rahmen um gesamten Code: Top, right, bottom, left (Großbuchstaben ergeben Doppellinien)
  frameround=ffff,	% Ecken des Rahmens anpassen, t: runde Ecken, f: default (eckig), es müssen 4 Buchstaben da stehen!
  literate=		% ersetzen von Zeichen 1 durch Zeichen 2, hier: korrekte Einbindung der Sonderzeichen
   {Ö}{{\"O}}1 
   {Ä}{{\"A}}1 
   {Ü}{{\"U}}1 
   {ß}{{\ss}}1 
   {ü}{{\"u}}1 
   {ä}{{\"a}}1 
   {ö}{{\"o}}1,
  % mit ‚emph‘ und ‚emphstyle‘ können eigene Styles für Wörter angelegt werden
  %emph = [1]{clc, color},
  %emphstyle = [1]{\color{blue}},
}

% Einstellung für die Einbindung von Code - hier: C++-Code - nicht ändern!
\lstdefinestyle{style-cpp}
{
  language=C++,	% Programmiersorache einstellen, Latex erkennt dann automatisch code, kommentare, funktionen,...
  basicstyle=\scriptsize\ttfamily,	% Schriftformatierung, ttfamily: text im Schreibmaschinen-Style
  backgroundcolor=\color{white},		% Hintergrundfarbe
  breaklines=true,	% automatischer Zeilenumbruch bei langen Zeilen, funktioniert nur bei bestimmen Zeichen, z.B. Umbruch nach Leerzeichen
  keywordstyle=\bfseries\ttfamily\color{blue},	% Schlüsselwörter einstellen
  stringstyle=\ttfamily\color{Peach},			% Textsrtrings
  showstringspaces=false,	% Leerzeichen in Strings richtig darstellen
  commentstyle=\color{ForestGreen}\ttfamily,	% Kommentare
  flexiblecolumns=false,	% Spaltenbreite dynamisch/fest
  numbers=left,		% Position der Zeilennummern
  numberstyle=\tiny,	% Größe der Zeilennummern
  numberblanklines=false,		% leere Zeilen werde mit ‚false‘ nicht durchnummeriert
  stepnumber=1,		% Beginn der Nummerierung
  numbersep=10pt,		% Abstand zwischen Zeilennummern und Quellcode
  xleftmargin=20pt,	% Abstand zum linken Rand
  xrightmargin=10pt,	% Abstand zum rechten Rand
  extendedchars=true,	% Sonderzeichen korrekt darstellen
  frame=trbl,			% Rahmen um gesamten Code: Top, right, bottom, left (Großbuchstaben ergeben Doppellinien)
  frameround=ffff,	% Ecken des Rahmens anpassen, t: runde Ecken, f: default (eckig), es müssen 4 Buchstaben da stehen!
  literate=		% ersetzen von Zeichen 1 durch Zeichen 2, hier: korrekte Einbindung der Sonderzeichen
   {Ö}{{\"O}}1 
   {Ä}{{\"A}}1 
   {Ü}{{\"U}}1 
   {ß}{{\ss}}1 
   {ü}{{\"u}}1 
   {ä}{{\"a}}1 
   {ö}{{\"o}}1,
  % mit ‚emph‘ und ‚emphstyle‘ können eigene Styles für Wörter angelegt werden
  %emph = [1]{clc, color},
  %emphstyle = [1]{\color{blue}},
}